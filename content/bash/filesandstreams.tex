\section{Files and streams}
\subsection{The Unix world: one file after another}
When you think of a computer and everything that goes with it, you usually come
up with a mental list of all sorts of different things:

\begin{itemize}
\setlength{\leftmargin}{0pt}
\setlength{\itemsep}{0pt}
\setlength{\parsep}{0pt}
\setlength{\parskip}{0pt}
	\item The computer itself
	\item The monitor
	\item The keyboard
	\item The mouse
	\item Your hard drive with your files and directories on it
	\item The network connection leading to the Internet
	\item The printer
	\item The DVD player
	\item et cetera
\end{itemize}

Here's a surprise for you: Unix doesn't have any of these things. Well, almost.
Unix certainly has \emph{files}. Unix has endless reams of files. And since
Unix has files, it also has a concept of ``between files'' (think of it this
way: if your universe consists only of boxes, you automatically know about
spaces where there are no boxes as well). But Unix knows nothing else than
that. Everything in the whole (Unix) universe is a file.

Everything is a file. Even things that are really weird things to think of as
files, are files. Your (data) files are files. Your directories are files. Your
hard drive is a file. Your \emph{keyboard}, \emph{monitor} and \emph{printer}
are files. Yes, really: your keyboard is a read-only file of infinite size.
Your monitor and printer are infinitely sized write-only files. Your network
connection is a read/write file.

At this point you're probably asking: \emph{Why?} Why would the designers of
the Unix system have come up with this madness? Why is everything a file? The
answer is: because if everything is a file, you can \textbf{treat} everything
like a file. Or, put a different way, you can treat everything in the Unix
world the same way. And, as we will see shortly, that means you can also
\emph{combine} virtually everything using file operations.

Before we move on, here's an extra level of weirdness for you: everything in
Unix is a file. Including the \emph{processes} that run programs. Effectively
this means that running programs are also files. Including the interactive
shell session that you've been running to practice scripting in. Yes, really,
that text screen with the blinking cursor is also a file. And we can prove it
too. You might recall that in the chapter on Running Commands we mentioned you
can exit the shell using the Ctrl+d key combination.  Because that combination
produces the Unix character for... that's right, end-of-file!

\subsection{Streams: what goes between files}
As we mentioned in the previous subsection, everything in Unix is a file --
except that which sits \emph{between} files. Between files Unix defines a
mechanism that allows data to move, bit by bit, from one file to another: the
\textbf{stream}. A stream is literally what it sounds like: a little river of
bits pouring from one file into another. Although actually a bridge would
probably have been a better name because unlike a stream (which is a constant
flow of water) the flow of bits between files need not be constant, or even
used at all.

\subsubsection{The standard streams}
Within the Unix world it is a general convention that each file is connected to
at least three streams (that's because that turned out to be the most useful
number for those files that are processes, or running programs). There can be
more and in fact each file can cause itself to be connected to any number of
streams (a program can print and open a network connection, for instance). But
there are three basic streams available to all files, even though they may not
always be useful or used. These streams are called the ``standard'' streams:
\begin{description}
\setlength{\leftmargin}{0pt}
\setlength{\itemsep}{0pt}
\setlength{\parsep}{0pt}
\setlength{\parskip}{0pt}
	\item[Standard in (stdin)] the standard stream for input into a file.
	\item[Standard out (stdout)] the standard stream for output out of a file.
	\item[Standard error (stderr)] the standard stream for error output from a file.
\end{description}

As you can probably tell, these streams are very geared towards those files
that are actually processes of the system. In fact many programming languages
(like C, C++, Java and Pascal) use exactly these conventions for their standard
I/O operations. And since the Unix operating system family includes them in the
core of the system definition, these streams are also central to the Bourne
Shell.

\subsubsection{Getting hold of the standard streams in your scripts}
So now we know that there's a general mechanism for basic input and output in
Unix; but how do you get hold of these streams in a script? What do you have to
do to hook your script up to the standard out, or read from the standard in?
Well, the happy answer is: nothing. Your scripts are automatically connected to
the standard in, out and error stream of the process that is running them. When
you read input, it automatically comes from the standard in. Your output goes
straight to the standard out. And program errors go right to the standard
error. In fact you've already used these streams: every example so far that has
printed anything has done so to the standard output stream of your script.

And what about the shell in interactive mode? Does that use those standard
streams as well? Yes, it does. In interactive mode, the standard in stream is
connected to the keyboard file. And the standard output and standard error are
connected to the monitor file.

\subsubsection{Okay... But what good is it?}
This discussion on files and streams has been very interesting so far and a
nice insight into the depths of Unix. But what good does it do you to know all
this? Ah, glad you asked!

The Bourne Shell has some built-in features that allow you to do neat tricks
involving files and their streams. You see, files don't just have streams --
you can also cross-connect the streams of two files. At the end of the previous
subsection we said that the standard input of the interactive session is
connected to the keyboard file. In fact it is connected to the \emph{standard
output} stream of the keyboard file. And the standard output and error of the
interactive session are connected to the \emph{standard input} of the monitor
file. So you can connect the streams of the interactive session to the streams
of devices.

But wait. Do you remember the remark above that the point of Unix considering
everything to be a file was that everything gets treated like a file? This is
why that was important: you can connect a stream from \emph{any} file to a
stream of \emph{any} other file. You can connect your interactive shell session
to the printer or the network rather than to the monitor (or in addition to the
monitor) using streams. You can run a program and have its output go directly
to the printer by reconnecting the standard output stream of the program. You
can connect the standard output stream of one program directly to the standard
input stream of \textbf{another program} and make chains of programs. And the
Bourne Shell makes it really simple to do all that.

Do you suddenly feel like you've stuck your fingers in the electrical socket?
That's the feeling of the raw power of the shell flowing through your body....

\section{Redirecting: using streams in the shell}
As explained in the previous subsection, the shell process is connected by
standard streams to (by default) the keyboard and the monitor. But very often
you will want to change this linking. Connecting a file to a stream is a very
common operation, so would expect it to be called something like ``connecting''
or ``linking''. But since the Bourne Shell has default connections and
everything you do is always a \emph{change} in the default connections,
connecting a file to a (different) stream using the shell is actually called
\textbf{redirecting}.

There are several operators built in to the Bourne Shell that relate to
redirecting. The most basic and general one is the pipe operator, which we will
examine in some detail further on. The others are related to redirecting to
file.

\subsection{Redirecting to file}
As we explained (or rather: hinted at) in the previous subsection, one of the
enormously powerful features of the Bourne Shell on top of a Unix operating
system is the ability to chain programs together. Execute a program, have it
produce output, then automatically send that output to another program as
input. The possible combinations are endless, as is the power of what you can
achieve.

One of the most common places where you might want to send a program's output
is to a file in the file system. And this time by file we mean a regular,
classic data file and not a Unix ``everything is a file including your
hardware'' file. In order to achieve this you can imagine that we can use the
chaining mechanism described above: let a program generate output through the
standard output stream, then connect that stream (i.e. \emph{redirect the
output}) to the standard input stream of a program that creates a data file in
the file system. And this would absolutely work. However, redirecting to a data
file is such a common operation that you don't need a separate end-of-chain
program for it. Redirecting to file is built straight into the Bourne Shell,
through the following operators:
\begin{description}
\setlength{\leftmargin}{0pt}
\setlength{\itemsep}{0pt}
\setlength{\parsep}{0pt}
\setlength{\parskip}{0pt}
	\item[\emph{process} \textgreater{} \emph{data file}] redirect the output
of \emph{process} to the data file; create the file if necessary, overwrite its
existing contents otherwise.
	\item[\emph{process} \textgreater{}\textgreater{} \emph{data file}]
redirect the output of \emph{process} to the data file; create the file if
necessary, append to its existing contents otherwise.
	\item[\emph{process} \textless{} \emph{data file}] read the contents of the
data file and redirect that contents to \emph{process} as input.
\end{description}

\subsection{Redirecting output}
Let's take a closer look at these operators through some examples. Take the
simple Bourne shell script below called 'hello.sh':
\lstset{basicstyle=\scriptsize, numbers=left, captionpos=b, tabsize=4}
\begin{lstlisting}[caption=A simple shell script that generates some output,language={bash},
breaklines=true,xleftmargin=15pt,label=lst:A simple shell script that generates some output]
!/bin/sh
echo Hello
\end{lstlisting}

This code may be run in any of the ways described in the chapter Running
Commands. When we run the script, it simply outputs the string \textbf{Hello}
to the screen and then returns us to our prompt. But let's say we want to
redirect the output to a file instead. We can use the redirect operators to do
that easily:
\lstset{basicstyle=\scriptsize, numbers=left, captionpos=b, tabsize=4}
\begin{lstlisting}[caption=Redirecting the output to a data file,language={bash},
breaklines=true,xleftmargin=15pt,label=lst:Redirecting the output to a data file]
$ hello.sh > myfile.txt
$
\end{lstlisting}

This time, we don't see the string 'Hello' on the screen. Where's it gone?
Well, exactly where we wanted it to: into the (new) data file called
'myfile.txt'. Let's examine this file using the 'cat' command:
\lstset{basicstyle=\scriptsize, numbers=left, captionpos=b, tabsize=4}
\begin{lstlisting}[caption=Examining the results of redirecting some output,language={bash},
breaklines=true,xleftmargin=15pt,label=lst:Examining the results of redirecting some output]
$ cat myfile.txt
Hello
$
\end{lstlisting}

Let's run the program again, this time using the '\textgreater{}\textgreater{}'
operator instead, and then examine 'myfile.txt' again using the 'cat' command:
\lstset{basicstyle=\scriptsize, numbers=left, captionpos=b, tabsize=4}
\begin{lstlisting}[caption=Redirecting using the append redirect,language={bash},
breaklines=true,xleftmargin=15pt,label=lst:Redirecting using the append redirect]
$ hello.sh >> myfile.txt
$ cat myfile.txt
Hello
Hello
$
\end{lstlisting}

You can see that 'myfile.txt' now consists of two lines --- the output has been
added to the end of the file (or concatenated); this is due to the use of the
'\textgreater{}\textgreater{}' operator. If we run the script again, this time
with the single greater-than operator, we get:
\lstset{basicstyle=\scriptsize, numbers=left, captionpos=b, tabsize=4}
\begin{lstlisting}[caption=Redirecting using the overwrite redirect,language={bash},
breaklines=true,xleftmargin=15pt,label=lst:Redirecting using the overwrite redirect]
$ hello.sh > myfile.txt
$ cat myfile.txt
Hello
$
\end{lstlisting}

Just one 'Hello' again, because the '\textgreater{}' will always overwrite the
contents of an existing file if there is one.

\subsection{Redirecting input}
Okay, so it's clear we can redirect output to a data file. But what about
reading from a data file? That's also pretty common. The Bourne Shell helps us
here as well: the entire process of reading a file and pumping its data into a
stream is captured by the '\textless{}' operator.

By default 'stdin' is fed from your keyboard; run the 'cat' command without any
arguments and it will just sit there, waiting for you type something:
\lstset{basicstyle=\scriptsize, numbers=left, captionpos=b, tabsize=4}
\begin{lstlisting}[caption=cat ???,language={bash},
breaklines=true,xleftmargin=15pt,label=lst:cat]
$ cat
\end{lstlisting}

In fact 'cat' will sit there all day until you type a 'Ctrl+D' (the 'End of
File Character' or 'EOF' for short). To redirect our standard input from
somewhere else use the '\textless{}' (less-than operator):
\lstset{basicstyle=\scriptsize, numbers=left, captionpos=b, tabsize=4}
\begin{lstlisting}[caption=Redirecting into the standard input,language={bash},
breaklines=true,xleftmargin=15pt,label=lst:Redirecting into the standard input]
$ cat < myfile.txt
Hello
$
\end{lstlisting}

So 'cat' will now read from the text file 'myfile.txt'; the 'EOF' character is
also generated at the end of file, so 'cat' will exit as before.

Note that we previously used 'cat' in this format:
\lstset{basicstyle=\scriptsize, numbers=left, captionpos=b, tabsize=4}
\begin{lstlisting}[caption=,language={bash},
breaklines=true,xleftmargin=15pt,label=lst:]
$ cat myfile.txt
\end{lstlisting}

Which is functionally identical to 
\lstset{basicstyle=\scriptsize, numbers=left, captionpos=b, tabsize=4}
\begin{lstlisting}[caption=,language={bash},
breaklines=true,xleftmargin=15pt,label=lst:]
$cat <  myfile.txt
\end{lstlisting}

However, these are two fundamentally different mechanisms: one uses an argument
to the command, the other is more general and redirects 'stdin' -- which is
what we're concerned with here. It's more convenient to use 'cat' with a
filename as argument, which is why the inventors of 'cat' put this in. However,
not all programs and scripts are going to take arguments so this is just an
easy example.

\subsection{Combining file redirects}
It's possible to redirect 'stdin' and 'stdout' in one line:
\lstset{basicstyle=\scriptsize, numbers=left, captionpos=b, tabsize=4}
\begin{lstlisting}[caption=Redirecting input to and output from cat at the same time,language={bash},
breaklines=true,xleftmargin=15pt,label=lst:Redirecting input to and output from cat at the same time]
$ cat < myfile.txt > mynewfile.txt
\end{lstlisting}

The command above will copy the contents of 'myfile.txt' to 'mynewfile.txt'
(and will overwrite any previous contents of 'mynewfile.txt'). Once again this
is just a convenient example as we normally would have achieved this effect
using 'cp myfile.txt mynewfile.txt'.

\subsection{Redirecting standard error (and other streams)}
So far we have looked at redirecting the ``normal'' standard streams associated
with files, i.e.\ the files that you use if everything goes correctly and as
planned. But what about that other stream? The one meant for errors? How do we
go about redirecting that? For example, if we wanted to redirect error data
into a log file.

As an example, consider the ls command. If you run the command 'ls myfile.txt',
it simply lists the filename 'myfile.txt' -- if that file exists. If the file
'myfile.txt' does NOT exist, 'ls' will return an error to the 'stderr' stream,
which by default in Bourne Shell is also connected to your monitor.

So, lets run 'ls' a couple of times, first on a file which does exist and then
on one that doesn't:
\lstset{basicstyle=\scriptsize, numbers=left, captionpos=b, tabsize=4}
\begin{lstlisting}[caption=Listing an existing file,language={bash},
breaklines=true,xleftmargin=15pt,label=lst:Listing an existing file]
$ ls myfile.txt
\end{lstlisting}

\scriptsize
\begin{verbatim}
myfile.txt
$
\end{verbatim}
\normalsize

and then:
\lstset{basicstyle=\scriptsize, numbers=left, captionpos=b, tabsize=4}
\begin{lstlisting}[caption=Listing a non-existent file,language={bash},
breaklines=true,xleftmargin=15pt,label=lst:Listing a non-existent file]
$ ls nosuchfile.txt
\end{lstlisting}

\scriptsize
\begin{verbatim}
ls: no such file or directory 
$
\end{verbatim}
\normalsize

And again, this time with 'stdout' redirected only:
\lstset{basicstyle=\scriptsize, numbers=left, captionpos=b, tabsize=4}
\begin{lstlisting}[caption=Trying to redirect...,language={bash},
breaklines=true,xleftmargin=15pt,label=lst:Trying to redirect...]
$ ls nosuchfile.txt > logfile.txt
\end{lstlisting}

\scriptsize
\begin{verbatim}
ls: no such file or directory
$
\end{verbatim}
\normalsize

We still see the error message; 'logfile.txt' will be created but will be
empty. This is because we have now redirected the stdout stream, while the
error message was written to the error stream. So how do we tell the shell that
we want to redirect the error stream?

In order to understand the answer, we have to cover a little more theory about
Unix files and streams. You see, deep down the reason that we can redirect
stdin and stdout with simple operators is that redirecting those streams is so
common that the shell lets us use a shorthand notation for those streams. But
actually, to be completely correct, we should have told the shell in every case
\emph{which} stream we wanted to redirect. In general you see, the shell cannot
know: there could be tons of streams connected to any file. And in order to
distinguish one from the other each stream connected to a file has a number
associated with it: by convention 0 is the standard in, 1 is the standard out,
2 is standard error and any other streams have numbers counting on from there.
To redirect any particular stream you prepend the redirect operator with the
stream number (called the \emph{file descriptor}. So to redirect the error
message in our example, we prepend the redirect operator with a 2, for the
stderr stream:
\lstset{basicstyle=\scriptsize, numbers=left, captionpos=b, tabsize=4}
\begin{lstlisting}[caption=Redirecting the stderr stream,language={bash},
breaklines=true,xleftmargin=15pt,label=lst:Redirecting the stderr stream]
$ ls nosuchfile.txt 2> logfile.txt
\end{lstlisting}

No output to the screen, but if we examine 'logfile.txt':
\lstset{basicstyle=\scriptsize, numbers=left, captionpos=b, tabsize=4}
\begin{lstlisting}[caption=Checking the logfile,language={bash},
breaklines=true,xleftmargin=15pt,label=lst:Checking the logfile]
$ cat logfile.txt
\end{lstlisting}

\scriptsize
\begin{verbatim}
ls: no such file or directory 
\end{verbatim}
\normalsize

As we mentioned before, the operator without a number is a shorthand notation.
In other words, this:

\scriptsize
\begin{verbatim}
$ cat < inputfile.txt > outputfile.txt
\end{verbatim}
\normalsize

is actually short for
\scriptsize
\begin{verbatim}
$ cat 0< inputfile.txt 1> outputfile.txt
\end{verbatim}
\normalsize

\lstset{basicstyle=\scriptsize, numbers=left, captionpos=b, tabsize=4}
\begin{lstlisting}[caption=We can also redirect both 'stdout' and 'stderr' independently like this:,language={bash},
breaklines=true,xleftmargin=15pt,label=lst:We can also redirect both 'stdout' and 'stderr' independently like this:]
$ ls nosuchfile.txt > stdio.txt 2> logfile.txt 
$
\end{lstlisting}

'stdio.txt' will be blank , 'logfile.txt' will contain the error as before.

If we want to redirect stdout and stderr to the same file, we can use the file
descriptor as well:

\lstset{basicstyle=\scriptsize, numbers=left, captionpos=b, tabsize=4}
\begin{lstlisting}[caption=,language={bash},
breaklines=true,xleftmargin=15pt,label=lst:]
$ ls nosuchfile.txt > alloutput.txt 2>&1
\end{lstlisting}

Here '2\textgreater{}\&1' means something like 'redirect stderr to the same
file stdout has been redirected to'. Be careful with the ordering! If you do it
this way:

\lstset{basicstyle=\scriptsize, numbers=left, captionpos=b, tabsize=4}
\begin{lstlisting}[caption=,language={bash},
breaklines=true,xleftmargin=15pt,label=lst:]
$ ls nosuchfile.txt 2>&1 > alloutput.txt
\end{lstlisting}

you will redirect stderr to the file that stdout points to, then send stdout
somewhere else --- and both streams will end up being redirected to different
locations.

\subsection{Special files}
We said earlier that the redirect operators discussed so far all redirect to
data files. While this is technically true, Unix magic still means that there's
more to it than just that. You see, the Unix file system tends to contain a
number of special files called ``devices'', by convention collected in the /dev
directory. These device files include the files that represent your hard drive,
DVD player, USB stick and so on. They also include some special files, like
/dev/null (also known as the bit bucket; anything you write to this file is
discarded). You can redirect to device files as well as to regular data files.
Be careful here; you really don't want to redirect raw text data to the boot
sector of your hard drive (and you can!). But if you know what you're doing,
you can use the device files by redirecting to them (this is how DVDs are
burned in Linux, for instance).

As an example of how you might actually use a device file, in the 'Solaris'
flavour of Unix the loudspeaker and its microphone can be accessed by the file
'/dev/audio'. So:
\lstset{basicstyle=\scriptsize, numbers=left, captionpos=b, tabsize=4}
\begin{lstlisting}[caption=,language={bash},
breaklines=true,xleftmargin=15pt,label=lst:]
# cat /tmp/audio.au > /dev/audio
\end{lstlisting}

Will play a sound, whereas:

\lstset{basicstyle=\scriptsize, numbers=left, captionpos=b, tabsize=4}
\begin{lstlisting}[caption=,language={bash},
breaklines=true,xleftmargin=15pt,label=lst:]
# cat < /dev/audio > /tmp/mysound.au
\end{lstlisting}

Will record a sound.(you will need to CTRL-C this to finish...)

This is fun:
\lstset{basicstyle=\scriptsize, numbers=left, captionpos=b, tabsize=4}
\begin{lstlisting}[caption=,language={bash},
breaklines=true,xleftmargin=15pt,label=lst:]
# cat < /dev/audio > /dev/audio
\end{lstlisting}
Now wave the microphone around whilst shouting --- Jimmy Hendrix style
feedback. Great stuff.  You will probably need to be logged in as 'root' to try
this by the way.

\subsection{Some redirect warnings}
The astute reader will have noticed one or two things in the discussion above.
First of all, a file can have more than just the standard streams associated
with it. Is it legal to redirect those? Is it even \emph{possible}? The answer
is, technically, yes. You can redirect stream 4 or 5 of a file (if they exist).
Don't try it though. If there's more than a few streams in any direction, you
won't know which stream you're redirecting. Plus, if a program needs more than
the standard streams it's a good bet that program also needs its extra streams
going to a specific location.

Second, you might have noticed that file descriptor 0 is, by convention, the
standard input stream. Does that mean you can redirect a program's standard
input \textbf{away} from the program? Could you do the following?
\lstset{basicstyle=\scriptsize, numbers=left, captionpos=b, tabsize=4}
\begin{lstlisting}[caption=,language={bash},
breaklines=true,xleftmargin=15pt,label=lst:]
$ cat 0> somewhere_else
\end{lstlisting}

The answer is, yes you can. And yes, things will break if you do.

\subsection{Pipes, Tees and Named Pipes}
So, after all this talk about redirecting to file, we finally get to it:
general redirecting by cross-connecting streams. The most general form of
redirecting and the most powerful one to boot. It's called a pipe and is
performed using the pipe operator '\textbar{}'. Pipes allow you to join two
processes together through a ``pipeline'', which directly connects the stdout
of one file to the stdin of another.

As an example let's consider the 'grep' command which returns a matching
string, given a keyword and some text to search. And let's also use the ps
command, which lists running processes on the machine. If you give the command 

\lstset{basicstyle=\scriptsize, numbers=left, captionpos=b, tabsize=4}
\begin{lstlisting}[caption=,language={bash},
breaklines=true,xleftmargin=15pt,label=lst:]
$ ps -eaf
\end{lstlisting}

it will generally list pagefuls of running processes on your machine, which you
would have to sift through manually to find what you want. Let's say you are
looking for a process which you know contains the word 'oracle'; use the output
of 'ps' to pipe into grep, which will only return the matching lines:
\lstset{basicstyle=\scriptsize, numbers=left, captionpos=b, tabsize=4}
\begin{lstlisting}[caption=,language={bash},
breaklines=true,xleftmargin=15pt,label=lst:]
$ ps -eaf | grep oracle
\end{lstlisting}

Now you will only get back the lines you need. What happens if there's still
loads of these ? No problem, pipe the output to the command 'more' (or 'pg'),
which will pause your screen if it fills up:
\lstset{basicstyle=\scriptsize, numbers=left, captionpos=b, tabsize=4}
\begin{lstlisting}[caption=,language={bash},
breaklines=true,xleftmargin=15pt,label=lst:]
$ ps -ef | grep oracle | more
\end{lstlisting}

What about if you want to kill all those processes? You need the 'kill'
program, plus the process number for each process (the second column returned
by the ps command). Easy:
\lstset{basicstyle=\scriptsize, numbers=left, captionpos=b, tabsize=4}
\begin{lstlisting}[caption=,language={bash},
breaklines=true,xleftmargin=15pt,label=lst:]
$ ps -ef | grep oracle | awk '{print $2}' | xargs kill -9
\end{lstlisting}

In this command, 'ps' lists the processes and 'grep' narrows the results down
to oracle. The 'awk' tool pulls out the second column of each line. And 'xargs'
feeds each line, one at a time, to 'kill' as a command line argument.

Pipes can be used to link as many programs as you wish within reasonable limits
(and we don't know what these limits are!)

Don't forget you can still use the redirectors in combination:
\lstset{basicstyle=\scriptsize, numbers=left, captionpos=b, tabsize=4}
\begin{lstlisting}[caption=,language={bash},
breaklines=true,xleftmargin=15pt,label=lst:]
$ ps -ef | grep oracle > /tmp/myprocesses.txt
\end{lstlisting}

There is another useful mechanism that can be used with pipes: the 'tee'. To
understand tee, imagine a pipe shaped like a 'T' --- one input, two outputs:
\lstset{basicstyle=\scriptsize, numbers=left, captionpos=b, tabsize=4}
\begin{lstlisting}[caption=,language={bash},
breaklines=true,xleftmargin=15pt,label=lst:]
$ ps -ef | grep oracle | tee /tmp/myprocesses.txt
\end{lstlisting}

The 'tee' will copy whatever is given to its stdin and redirect this to the
argument given (a file); it will also then send a further copy to its stdout
--- which means you can effectively intercept the pipe, take a copy at this
stage, and carry on piping up other commands; useful maybe for outputting to a
logfile, and copying to the screen.

A note on piped commands: piped processes run in parallel on the Unix
environment. Sometimes one process will be blocked, waiting for input from
another process. But each process in a pipeline is, in principle, running
simultaneously with all the others.

\subsection{Named pipes}
There is a variation on the in-line pipe which we have been discussing called
the 'named pipe'. A named pipe is actually a file with its own 'stdin' and
'stdout' --- which you attach processes to. This is useful for allowing
programs to talk to each other, especially when you don't know exactly when one
program will try and talk to the other (waiting for a backup to finish etc) and
when you don't want to write a complicated network-based listener or do a
clumsy polling loop.

To create a 'named pipe', you use the 'mkfifo' command (fifo=first in, first
out; so data is read out in the same order as it is written into).
\lstset{basicstyle=\scriptsize, numbers=left, captionpos=b, tabsize=4}
\begin{lstlisting}[caption=,language={bash},
breaklines=true,xleftmargin=15pt,label=lst:]
$ mkfifo mypipe 
$
\end{lstlisting}

This creates a named pipe called 'mypipe'; next we can start using it.

This test is best run with two terminals logged in:

1. From 'terminal a'
\lstset{basicstyle=\scriptsize, numbers=left, captionpos=b, tabsize=4}
\begin{lstlisting}[caption=,language={bash},
breaklines=true,xleftmargin=15pt,label=lst:]
$ cat < mypipe
\end{lstlisting}

The 'cat' will sit there waiting for an input.

2. From 'terminal b'
\lstset{basicstyle=\scriptsize, numbers=left, captionpos=b, tabsize=4}
\begin{lstlisting}[caption=,language={bash},
breaklines=true,xleftmargin=15pt,label=lst:]
$ cat myfile.txt > mypipe 
 $
\end{lstlisting}

Now try the other way round:

1. From terminal 'b'
\lstset{basicstyle=\scriptsize, numbers=left, captionpos=b, tabsize=4}
\begin{lstlisting}[caption=,language={bash},
breaklines=true,xleftmargin=15pt,label=lst:]
$ cat myfile.txt > mypipe
\end{lstlisting}

This will now sit there, as there isn't another process on the other end to
'drain' the pipe --- it's blocked.

2. From terminal 'a'
\lstset{basicstyle=\scriptsize, numbers=left, captionpos=b, tabsize=4}
\begin{lstlisting}[caption=,language={bash},
breaklines=true,xleftmargin=15pt,label=lst:]
$ cat < mypipe
\end{lstlisting}

As before, both processes will now finish, the output showing on terminal 'a'.

\subsection{Here documents}
So far we have looked at redirecting from and to data files and
cross-connecting data streams. All of these shell mechanisms are based on
having a ``physical'' source for data --- a process or a data file. Sometimes
though, you want to feed some data into a target without having a source for
it. In these cases you can use an ``on the fly'' document called a \emph{here
document}. A here document means that you open a virtual text document (in
memory), type into it as usual, close it and then treat it like any normal
file.

Creating a here document is done using a variation on the input redirect
operator: the '\textless{}\textless{}' operator. Like the input redirect
operator, the here document operator takes an argument. For the input redirect
operator this operand is the name of the file to be streamed in. For the here
document operator it is the string that will terminate the here document. So
using the here document operator looks like this:
\lstset{basicstyle=\scriptsize, numbers=left, captionpos=b, tabsize=4}
\begin{lstlisting}[caption=,language={bash},
breaklines=true,xleftmargin=15pt,label=lst:]
target << terminator string
here document contents
terminator string
\end{lstlisting}

For example: 
\scriptsize
\begin{verbatim}
This is a test.
This test uses a here document.
Hello world.
This here document will end upon the occurrence 
of the string "%%" on a separate line.
So this document is still open now.
But now it will end....
\end{verbatim}
\normalsize

When using here documents in combination with variable or command substitution,
it is important to realize that substitutions are carried out \emph{before} the
here document is passed on. So for example:
\lstset{basicstyle=\scriptsize, numbers=left, captionpos=b, tabsize=4}
\begin{lstlisting}[caption=Using a here document with substitutions,language={bash},
breaklines=true,xleftmargin=15pt,label=lst:Using a here document with substitutions]
$ COMMAND=cat
$ PARAM='Hello World!!'
$ $COMMAND <<%
> `echo $PARAM`
> %
\end{lstlisting}

Output:
\scriptsize
\begin{verbatim}
Hello World!!
\end{verbatim}
\normalsize
