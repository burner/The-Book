\section{Debugging and signal handling}
In the previous sections we've told you all about the Bourne Shell and how to
write scripts using the shell's language. We've been careful to include all the
details we could think of so you will be able to write the best scripts you
can. But no matter how carefully you've paid attention and no matter how
carefully you write your scripts, the time will come to pass when something
you've written simply will not work --- no matter how sure you are it should.
So how do you proceed from here?

In this module we cover the tools the Bourne Shell provides to deal with the
unexpected. Unexpected behavior of your script (for which you must debug the
script) and unexpected behavior \emph{around} your script (caused by signals
being delivered to your script by the operating system).

\subsection{Debugging Flags}
So here you are, in the middle of the night having just finished a long and
complicated shell script, just poured your heart and soul into it for three
days straight while living on nothing but coffee, cola and pizza... and it just
won't work. Somewhere in there is a bug that is just eluding you. Something is
going wrong, some unexpected behavior has popped up or something else is
driving you crazy. So how are you going to debug this script? Sure, you can
pump it full of 'echo' commands and debug that way, but isn't there an easier
way?

Generally speaking the most insightful way to debug any program is to follow
the execution of the program along statement by statement to see what the
program is doing exactly. The most advanced form of this (offered by modern
IDEs) allows you to trace into a program by stopping the execution at a certain
point and examining its internal state. The Bourne Shell is, unfortunately, not
that advanced. But it does offer the next best thing: command tracing. The
shell can print each and every command as it is being executed.

The tracing functionality (there are two of them) is activated using either the
'set' command or by passing parameters directly to the shell executable when it
is called. In either case you can use the \emph{-x} parameter, the \emph{-v}
parameter or both.

\begin{description}
	\item[-v] Turns on verbose mode; each command is printed by the shell as it is read.
	\item[-x] This turns on command tracing; every command is printed by the shell as it is executed.
\end{description}

\subsubsection{Debugging}
Let's consider the following script:
\lstset{basicstyle=\scriptsize, numbers=left, captionpos=b, tabsize=4}
\begin{lstlisting}[caption=divider.sh: Script with a potential error,language={bash},
breaklines=true,xleftmargin=15pt,label=lst:divider.sh: Script with a potential error]
#!/bin/sh

DIVISOR=${1:-0}
echo $DIVISOR
expr 12 / $DIVISOR
\end{lstlisting}

Let's execute this script and not pass in a command-line argument (so we use
the default value 0 for the DIVISOR variable):
\lstset{basicstyle=\scriptsize, numbers=left, captionpos=b, tabsize=4}
\begin{lstlisting}[caption=Running the script,language={bash},
breaklines=true,xleftmargin=15pt,label=lst:Running the script]
$ sh divider.sh
\end{lstlisting}

\scriptsize
\begin{verbatim}
0
expr: division by zero
\end{verbatim}
\normalsize

Of course it's not too hard to figure out what went wrong in this case, but
let's take a closer look anyway. Let's see what the shell executed, using the
\textbf{-x} parameter:
\lstset{basicstyle=\scriptsize, numbers=left, captionpos=b, tabsize=4}
\begin{lstlisting}[caption=Running the script with tracing on,language={bash},
breaklines=true,xleftmargin=15pt,label=lst:Running the script with tracing on]
$ sh -x divider.sh
\end{lstlisting}

\scriptsize
\begin{verbatim}
+ DIVISOR=0
+ echo 0
0
+ expr 12 / 0
expr: division by zero
\end{verbatim}
\normalsize

So indeed, clearly the shell tried to have a division by zero evaluated. Just
in case we're confused about where the zero came from, let's see which commands
the shell actually read:

\scriptsize
\begin{verbatim}
$ sh -v divider.sh
\end{verbatim}
\normalsize

\lstset{basicstyle=\scriptsize, numbers=left, captionpos=b, tabsize=4}
\begin{lstlisting}[caption=Running the script in verbose mode,language={bash},
breaklines=true,xleftmargin=15pt,label=lst:Running the script in verbose mode]
#!/bin/sh

DIVISOR=${1:-0}
echo $DIVISOR
0
expr 12 / $DIVISOR
expr: division by zero
\end{lstlisting}

So obviously, the script read a command with a variable substitution that
didn't work out very well. If we combine these two parameters the resulting
output tells the whole, sad story:

\scriptsize
\begin{verbatim}
$ sh -xv divider.sh
\end{verbatim}
\normalsize

\lstset{basicstyle=\scriptsize, numbers=left, captionpos=b, tabsize=4}
\begin{lstlisting}[caption=Running the script with maximum debugging,language={bash},
breaklines=true,xleftmargin=15pt,label=lst:Running the script with maximum debugging]
#!/bin/sh

DIVISOR=${1:-0}
+ DIVISOR=0
echo $DIVISOR
+ echo 0
0
expr 12 / $DIVISOR
+ expr 12 / 0
expr: division by zero
\end{lstlisting}

There is another parameter that you can use to debug your script, the
\textbf{-n} parameter. This causes the shell to read the commands but not
execute them. You can use this parameter to do a syntax checking run of your
script.

\subsection{Places to put your parameters}
As you saw in the previous section, we used the shell parameters by passing
them in as command-line parameters to the shell executable. But couldn't we
have put the parameters inside the script itself? After all, there is an
interpreter hint in there... And surely enough, we can do exactly that. Let's
modify the script a little and try it.

\lstset{basicstyle=\scriptsize, numbers=left, captionpos=b, tabsize=4}
\begin{lstlisting}[caption=The same script but now with parameters to the interpreter hint,language={bash},
breaklines=true,xleftmargin=15pt,label=lst:The same script but now with parameters to the interpreter hint]
#!/bin/sh -xv

DIVISOR=${1:-0}
echo $DIVISOR
expr 12 / $DIVISOR
\end{lstlisting}

\lstset{basicstyle=\scriptsize, numbers=left, captionpos=b, tabsize=4}
\begin{lstlisting}[caption=Running the script,language={bash},
breaklines=true,xleftmargin=15pt,label=lst:Running the script]
$ chmod +x divider.sh
$ ./divider.sh
\end{lstlisting}

\scriptsize
\begin{verbatim}
#!/bin/sh

DIVISOR=${1:-0}
+ DIVISOR=0
echo $DIVISOR
+ echo 0
0
expr 12 / $DIVISOR
+ expr 12 / 0
expr: division by zero
\end{verbatim}
\normalsize

Works like a charm.

So there's no problem there. But there is a little gotcha. Let's try running
the script again:

\lstset{basicstyle=\scriptsize, numbers=left, captionpos=b, tabsize=4}
\begin{lstlisting}[caption=Running the script again,language={bash},
breaklines=true,xleftmargin=15pt,label=lst:Running the script again]
$ sh divider.sh
\end{lstlisting}

\scriptsize
\begin{verbatim}
0
expr: division by zero
\end{verbatim}
\normalsize
Where did the debugging go?

So what happened to the debugging that time? Well, you have remember that the
interpreter hint is used when you try to execute the script as an executable in
its own right. But in the last example, we weren't doing that. In the last
example we called the shell ourselves and passed it the script as a parameter.
So the shell executed without any debugging activated. It would have worked if
we'd done a ``sh -xv divider.sh'' though.

What about sourcing the script (i.e. using the dot notation)?

\lstset{basicstyle=\scriptsize, numbers=left, captionpos=b, tabsize=4}
\begin{lstlisting}[caption=Running the script again,language={bash},
breaklines=true,xleftmargin=15pt,label=lst:Running the script again]
$ . divider.sh
\end{lstlisting}

\scriptsize
\begin{verbatim}
0
expr: division by zero
\end{verbatim}
\normalsize
No debugging there either...

This time the script was executed by the same shell process that is running the
interactive shell for us. And the same principle applies: no debugging there
either. Because the interactive shell was not started with debugging flags. But
we can fix that as well; this is where the 'set' command comes in:

\lstset{basicstyle=\scriptsize, numbers=left, captionpos=b, tabsize=4}
\begin{lstlisting}[caption=Running the script again,language={bash},
breaklines=true,xleftmargin=15pt,label=lst:Running the script again]
$ set -xv
$ . divider.sh
\end{lstlisting}

\scriptsize
\begin{verbatim}
. divider.sh
+ . divider.sh
#!/bin/sh -vx

DIVISOR=${1:-0}
++ DIVISOR=0
echo $DIVISOR
++ echo 0
0
expr 12 / $DIVISOR
++ expr 12 / 0
expr: division by zero
\end{verbatim}
\normalsize

And now we have debugging active in the interactive shell and we get a full
trace of the script. In fact, we even get a trace of the interactive shell
\emph{calling} the script! But now what happens if we start a new shell process
with debugging on in the interactive shell? Does it carry over?

\lstset{basicstyle=\scriptsize, numbers=left, captionpos=b, tabsize=4}
\begin{lstlisting}[caption=Running the script again,language={bash},
breaklines=true,xleftmargin=15pt,label=lst:Running the script again]
$ sh divider.sh
\end{lstlisting}

\scriptsize
\begin{verbatim}
sh divider.sh
+ sh divider.sh
0
expr: division by zero
\end{verbatim}
\normalsize
Not quite...

Well, we certainly get a trace of the script being called, but no trace of the
script itself. The moral of the story is: when debugging, make sure you know
which shell you're activating the trace in.

By the way, to turn tracing in the interactive shell off again you can either
do a 'set +xv' or simply a 'set -'.

\subsection{Breaking out of a script}
When writing or debugging a shell script it is sometimes useful to exit out (to
stop the execution of the script) at certain points. You use the 'exit'
built-in command to do this. The command looks simply like this:

\lstset{basicstyle=\scriptsize, numbers=left, captionpos=b, tabsize=4}
\begin{lstlisting}[caption=,language={bash},
breaklines=true,xleftmargin=15pt,label=lst:]
exit [n]
\end{lstlisting}

If you leave off the optional exit status, the exit status of he script will be
the exit status of the last command that executed before the call to 'exit'.

For example:
\lstset{basicstyle=\scriptsize, numbers=left, captionpos=b, tabsize=4}
\begin{lstlisting}[caption=Exiting from a script,language={bash},
breaklines=true,xleftmargin=15pt,label=lst:Exiting from a script]
#!/bin/sh -x
echo hello
exit 1
\end{lstlisting}

If you run this script and then test the output status, you will see (using the
\$? built-in variable):
\lstset{basicstyle=\scriptsize, numbers=left, captionpos=b, tabsize=4}
\begin{lstlisting}[caption=Checking the exit status,language={bash},
breaklines=true,xleftmargin=15pt,label=lst:Checking the exit status]
echo $?
\end{lstlisting}

\scriptsize
\begin{verbatim}
1
\end{verbatim}
\normalsize

There's one thing to look out for when using 'exit': 'exit' actually terminates
the executing process. So if you're executing an executable script with an
interpreter hint or have called the shell explicitly and passed in your script
as an argument that is fine. But if you've sourced the script (used the
dot-notation), then your script is being run by the process running your
interactive shell. So you may inadvertently terminate your shell session and
log yourself out by using the 'exit' command!

There's a variation on 'exit' meant specifically for blocks of code that are
not processes of their own. This is the 'return' command, which is very similar
to the 'exit' command:
\lstset{basicstyle=\scriptsize, numbers=left, captionpos=b, tabsize=4}
\begin{lstlisting}[caption=,language={bash},
breaklines=true,xleftmargin=15pt,label=lst:]
return [n]
\end{lstlisting}

Return has exactly the same semantics as 'exit', but is primarily intended for
use in shell functions (it makes a function return without terminating the
script). Here's an example:
\lstset{basicstyle=\scriptsize, numbers=left, captionpos=b, tabsize=4}
\begin{lstlisting}[caption=exitandreturn.sh,language={bash},
breaklines=true,xleftmargin=15pt,label=lst:exitandreturn.sh]
#!/bin/sh

sayHello() {
  echo 'Hi there!!'
  return 2
}

echo 'Hello World!!'
sayHello
echo $?
echo 'Goodbye!!'
exit
\end{lstlisting}

If we run this script, we see the following:
\lstset{basicstyle=\scriptsize, numbers=left, captionpos=b, tabsize=4}
\begin{lstlisting}[caption=Running the script,language={bash},
breaklines=true,xleftmargin=15pt,label=lst:Running the script]
./exit_and_return.sh
\end{lstlisting}

\scriptsize
\begin{verbatim}
Hello World!!
Hi there!!
2
Goodbye!!
\end{verbatim}
\normalsize

The function returned with a testable exit status of 2. The overall exit status
of the script is zero though, since the last command executed by the script
('echo Goodbye!!') exited without errors.

You can also use a 'return' statement to exit a shell script that you executed
by sourcing it (the script will be run by the process that runs the interactive
shell, so that's not a subprocess). But it's usually not a good idea, since
this will limit your script to being sourced: if you try to run it any other
way, the fact that you used a 'return' statement alone will cause an error.

\subsection{Signal trapping}
A syntax, command error or call to 'exit' is not the only thing that can stop
your script from executing. The process that runs your script might also
suddenly receive a signal from the operating system. Signals are a simple form
of event notification: think of a signal as a little light suddenly popping on
in your room, to let you know that somebody outside the room wants your
attention. Only it's not just one light. The Unix system usually allows for
lots of different signals so it's more like having a wall full of little lamps,
each of which could suddenly start blinking.

On a single-process operating system like MS-DOS, life was simple. The
environment was single-process, meaning your code (once running) had complete
machine control. Any signal arriving was always a hardware interrupt (e.g. the
computer signalling that the floppy disk was ready to read) and you could
safely ignore all those signals if you didn't need external hardware; either it
was some device event you weren't interested in, or something was really wrong
--- in which case the computer was crashing anyway and there was nothing you
could do.

On a Unix system, life is not so easy. On Unix signals can come from all over
the place (including other programs). And you never have complete control of
the system either. A signal may be a hardware interrupt, or another program
signalling, or the user who got fed up with waiting, logged in to a second
shell session and is now ordering your process to die. On the bright side, life
is still not too complicated. Most Unix systems (and certainly the Bourne
Shell) come with default handling for most signals. Usually you can still
safely ignore signals and let the shell or the OS deal with them. In fact, if
the signal in question is number 9 (loosely translated: \textbf{KILL!! KILL!!
DIE!! DIE, RIGHT NOW!!}), you probably \emph{should} ignore it and let the OS
kill your process.

But sometimes you just have to do your own signal handling. That might be
because you've been working with files and want to do some cleanup before your
process dies. Or because the signal is part of your multi-process program
design (e.g. listening for signal 16, which is ``user-defined signal 1'').
Which is why the Bourne Shell gives us the 'trap' command.

\subsubsection{Trap}
The trap command is actually quite simple (especially if you've ever done
event-driven programming of any kind). Essentially the trap command says ``if
one of the following signals is received by this process, do this''. It looks
like this:
\lstset{basicstyle=\scriptsize, numbers=left, captionpos=b, tabsize=4}
\begin{lstlisting}[caption=,language={bash},
breaklines=true,xleftmargin=15pt,label=lst:]
trap [command string] signal0 [signal1] ...
\end{lstlisting}

For example, to trap user-defined signal 1 (commonly referred to as SIGUSR1)
and print ``Hello World'' whenever it comes along, you would do this:

\lstset{basicstyle=\scriptsize, numbers=left, captionpos=b, tabsize=4}
\begin{lstlisting}[caption=Trapping SIGUSR1,language={bash},
breaklines=true,xleftmargin=15pt,label=lstTrapping SIGUSR1:Trapping SIGUSR1]
$ trap "echo Hello World" 16
\end{lstlisting}

Most Unix systems also allow you to use symbolic names (we'll get back to these
a little later). So you can probably also do this:

\lstset{basicstyle=\scriptsize, numbers=left, captionpos=b, tabsize=4}
\begin{lstlisting}[caption=Trapping SIGUSR1 (little easier),language={bash},
breaklines=true,xleftmargin=15pt,label=lst:Trapping SIGUSR1 (little easier)]
$ trap "echo Hello World" SIGUSR1
\end{lstlisting}

And if you can do that, you can usually also do this:

\lstset{basicstyle=\scriptsize, numbers=left, captionpos=b, tabsize=4}
\begin{lstlisting}[caption=Trapping SIGUSR1 (even easier),language={bash},
breaklines=true,xleftmargin=15pt,label=lst:Trapping SIGUSR1 (even easier)]
$ trap "echo Hello World" USR1
\end{lstlisting}

The command string passed to 'trap' is a string that contains a command list.
It's not treated as a command list though; it's just a string and it isn't
interpreted until a signal is caught. The command string can be any of the
following:

\begin{description}
	\item[A string] A string containing a command list. Any and all commands
are allowed and you can use multiple commands separated by semicolons as well
(i.e. a command list).
	\item[The empty string] Actually this
is the same as the previous case, since this is the empty command string. This
causes the shell to execute nothing when a signal is trapped --- in other
words, to ignore the signal.
	\item[] Nothing, the null string. This resets the signal handling to the
default signal action (which is usually kill process).
\end{description}

Following the command list you can list as many signals as you want to be
associated with that command list. The traps that you set up in this manner are
valid for every command that \emph{follows} the 'trap' command.

Right about now it's probably a good idea to look at an example to clear things
up a bit. You can use 'trap' anywhere (as usual) including the interactive
shell. But most of the time you will want to introduce traps into a script
rather than into your interactive shell process. Let's create a simple script
that uses the 'trap' command:


\lstset{basicstyle=\scriptsize, numbers=left, captionpos=b, tabsize=4}
\begin{lstlisting}[caption=A simple signal trap,language={bash},
breaklines=true,xleftmargin=15pt,label=lst:A simple signal trap]
#!/bin/sh

trap 'echo Hello World' SIGUSR1

while [ 1 -gt 0 ]
do
   echo Running....
   sleep 5
done
\end{lstlisting}

This script in and of itself is an endless loop, which prints ``Running...''
and then sleeps for five seconds. But we've added a 'trap' command
(\textbf{before} the loop, otherwise the trap would never be executed and it
wouldn't affect the loop) that prints ``Hello World'' whenever the process
receives the SIGUSR1 signal. So let's start the process by running the script:

\lstset{basicstyle=\scriptsize, numbers=left, captionpos=b, tabsize=4}
\begin{lstlisting}[caption=Infinite loop...,language={bash},
breaklines=true,xleftmargin=15pt,label=lst:Infinite loop...]
$ ./trap_signal.sh
\end{lstlisting}

\scriptsize
\begin{verbatim}
Running....
Running....
Running....
Running....
Running....
Running....
...
\end{verbatim}
\normalsize

This could get boring after a while....
To spring the trap, we must send the running process a signal. To do that, log
into a new shell session and use a process tool (like 'ps') to find the correct
process id (PID):

\lstset{basicstyle=\scriptsize, numbers=left, captionpos=b, tabsize=4}
\begin{lstlisting}[caption=Finding the process ID,language={bash},
breaklines=true,xleftmargin=15pt,label=lst:Finding the process ID]
$ ps -ef | grep signal
\end{lstlisting}

\scriptsize
\begin{verbatim}
bzt 10865 7067 0 15:08 pts/0 00:00:00 /bin/sh 
    ./trap_signal.sh
bzt 10808 10415 0 15:12 pts/1 00:00:00 fgrep signal
\end{verbatim}
\normalsize

Now, to send a signal to that process, we use the 'kill' command which is built
into the Bourne Shell: 
\lstset{basicstyle=\scriptsize, numbers=left, captionpos=b, tabsize=4}
\begin{lstlisting}[caption=,language={bash},
breaklines=true,xleftmargin=15pt,label=lst:]
kill [-signal] ID [ID] ...
\end{lstlisting}

As the name suggests, 'kill' was actually intended to kill processes (this fits
with the default signal being SIGTERM and the default signal handler being
terminate). But in fact what it does is nothing more than send a signal to a
process. So for example, we can send a SIGUSR1 to our process like this:
\lstset{basicstyle=\scriptsize, numbers=left, captionpos=b, tabsize=4}
\begin{lstlisting}[caption=Let's trip the trap...,language={bash},
breaklines=true,xleftmargin=15pt,label=lst:Let's trip the trap...]
kill -SIGUSR1 10865
\end{lstlisting}

\scriptsize
\begin{verbatim}
...
Running....
Running....
Running....
Running....
Running....
Hello World
Running....
Running....
...
\end{verbatim}
\normalsize

You might notice that there's a short pause before ``Hello World!'' appears; it
won't happen until the running 'sleep' command is done. But after that, there
it is. But you might be a little surprised: the signal didn't kill the process.
That's because 'trap' \emph{completely} replaces the signal handler with the
commands you set up. And an 'echo Hello World' alone won't kill a process...
The lesson here is a simple one: if you want your signal trap to terminate your
process, make sure you include an 'exit' command.

Between having multiple commands in your command list and potentially trapping
lots of signals, you might be worried that a 'trap' statement can become messy.
Fortunately, you can also use shell functions as commands in a 'trap'. The
following example illustates that and the difference between an exiting event
handler and a non-exiting event handler:

\lstset{basicstyle=\scriptsize, numbers=left, captionpos=b, tabsize=4}
\begin{lstlisting}[caption=A trap with a shell function as a handler,language={bash},
breaklines=true,xleftmargin=15pt,label=lst:A trap with a shell function as a handler]
#!/bin/sh

exit_with_grace() {
  echo Goodbye World
  exit
}

trap "exit_with_grace" USR1 TERM QUIT
trap "echo Hello World" USR2

while [ 1 -gt 0 ]
do
   echo Running....
   sleep 5
done
\end{lstlisting}

\subsubsection{System signals}
Here's the official definition of a signal from the POSIX-1003 2001 edition
standard:

\scriptsize
\begin{verbatim}
A mechanism by which a process or thread may be 
notified of, or affected by, an event occurring 
in the system. Examples of such events include 
hardware exceptions and specific actions by 
processes. The term signal is also used to refer 
to the event itself.
\end{verbatim}
\normalsize

In other words, a signal is some sort of short message that is send from one
process (possible a system process) to another. But what does that mean
exactly? What does a signal look like? The definition given above is kind of
vague...

If you have any feel for what happens in computing when you give a vague
definition, you already know the answer to the questions above: every Unix
flavor that was developed came up with its own definition of ``signal''. They
pretty much all settled on a message that consists of an integer (because
that's simple), but not exactly the \emph{same} list everywhere. Then there was
some standardization and Unix systems organized themselves into the System V
and BSD flavors and at last everybody agreed on the following definition:

\scriptsize
\begin{verbatim}
The system signals are the signals listed in 
/usr/include/sys/signal.h .
\end{verbatim}
\normalsize

God, that's helpful...

Since then a lot has happened, including the definition of the POSIX-1003
standard. This standard, which standardizes most of the Unix interfaces
(including the shell in part 1 (1003.1)) finally came up with a standard list
of symbolic signal names and default handlers. So usually, nowadays, you can
make use of that list and expect your script to work on most systems. Just be
aware that it's not \emph{completely} fool-proof...

POSIX-1003 defines the signals listed in the table below. The values given are
the \emph{typical} numeric values, but they aren't mandatory and you shouldn't
rely on them (but then again, you use symbolic values in order not to use
actual values). See table \ref{tab:posixsignal}.

\begin{table*}[H]
	\begin{tabular}{|c|c|p{4cm}|c|}
		\hline
		\textbf{ Signal} & \textbf{ Default action} & \textbf{ Description} & \textbf{ Typical value(s)} \\ \hline
		 SIGABRT &  Abort with core dump &  Abort process and generate a core dump &  6 \\ \hline
		 SIGALRM &  Terminate &  Alarm clock. &  14 \\ \hline
		 SIGBUS &  Abort with core dump &  Access to an undefined portion of a memory object. &  7, 10 \\ \hline
		 SIGCHLD &  Ignore &  Child process terminated, stopped &  20, 17, 18 \\ \hline
		 SIGCONT &  Continue process (if stopped) &  Continue executing, if stopped. &  19,18,25 \\ \hline
		 SIGFPE &  Abort with core dump &  Erroneous arithmetic operation. &  8 \\ \hline
		 SIGHUP &  Terminate &  Hangup. &  1 \\ \hline
		 SIGILL &  Abort with core dump &  Illegal instruction. &  4 \\ \hline
		 SIGINT &  Terminate &  Terminal interrupt signal. &  2 \\ \hline
		 SIGKILL &  Terminate &  Kill (cannot be caught or ignored). &  9 \\ \hline
		 SIGPIPE &  Terminate &  Write on a pipe with no one to read it (i.e. broken pipe). &  13 \\ \hline
		 SIGQUIT &  Terminate &  Terminal quit signal. &  3 \\ \hline
		 SIGSEGV &  Abort with core dump &  Invalid memory reference. &  11 \\ \hline
		 SIGSTOP &  Stop process &  Stop executing (cannot be caught or ignored). &  17,19,23 \\ \hline
		 SIGTERM &  Terminate &  Termination signal. &  15 \\ \hline
		 SIGTSTP &  Stop process &  Terminal stop signal. &  18,20,24 \\ \hline
		 SIGTTIN &  Stop process &  Background process attempting read. &  21,21,26 \\ \hline
		 SIGTTOU &  Stop process &  Background process attempting write. &  22,22,27 \\ \hline
		 SIGUSR1 &  Terminate &  User-defined signal 1. &  30,10,16 \\ \hline
		 SIGUSR2 &  Terminate &  User-defined signal 2. &  31,12,17 \\ \hline
		 SIGPOLL &  Terminate &  Pollable event. &  - \\ \hline
		 SIGPROF &  Terminate &  Profiling timer expired. &  27,27,29 \\ \hline
		 SIGSYS &  Abort with core dump &  Bad system call. &  12 \\ \hline
		 SIGTRAP &  Abort with core dump &  Trace/breakpoint trap &  5 \\ \hline
		 SIGURG &  Ignore &  High bandwidth data is available at a socket. &  16,23,21 \\ \hline
		 SIGVTALRM &  Terminate &  Virtual timer expired. &  26,28 \\ \hline
		 SIGXCPU &  Abort with core dump &  CPU time limit exceeded. &  24,30 \\ \hline
		 SIGXFSZ &  Abort with core dump &  File size limit exceeded. &  25,31 \\ \hline
	\end{tabular}
	\caption{POSIX system signals}
	\label{tab:posixsignal}
\end{table*}

Earlier on we talked about job control and suspending and resuming jobs. Job
suspension and resuming is actually completely based on sending signals to
processes, so you can in fact control job stopping and starting completely
using 'kill' and the signal list. To suspend a process, send it the SIGSTOP
signal. To resume, send it the SIGCONT signal.

\subsubsection{Err... ERR?}
If you go online and read about 'trap', you might come across another kind of
``signal'' which is called \textbf{ERR}. It's used with 'trap' the same way
regular signals are, but it isn't really a signal at all. It's used to trap
command errors (i.e. non-zero exit statuses), like this:

\lstset{basicstyle=\scriptsize, numbers=left, captionpos=b, tabsize=4}
\begin{lstlisting}[caption=Error trapping,language={bash},
breaklines=true,xleftmargin=15pt,label=lst:Error trapping]
$ trap 'echo HELLO WORLD' ERR
$ expr 1 / 0
\end{lstlisting}

\scriptsize
\begin{verbatim}
expr: division by zero
HELLO WORLD
\end{verbatim}
\normalsize
The non-zero exit status was trapped as though it was a signal.

So why didn we cover this ``signal'' earlier, when we were discussing
'trap'? Well, we saved it until
the discussion on system and non-system signals for a reason: ERR isn't
standard at all. It was added by the Korn Shell to make life easier, but not
adopted by the POSIX standard and it certainly isn't part of the original
Bourne Shell. So if you use it, remember that your script may not be portable
anymore.
