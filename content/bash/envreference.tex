\section{Environment reference}
In the section on the environment we discussed the concept of environment
variables. We also mentioned that there are usually a large number of
environment variables that are created centrally in \textbf{/etc/profile}.
There are a number of these that have a predefined meaning in the Bourne Shell.
They are not set automatically, mind, but they have meaning when they are set.

On most systems there are far more predefined variables than we list here. And
some of these will mean something to your shell (most shells have more options
than the Bourne Shell). Check your shell's documentation for a listing. The
ones below are meaningful to the Bourne Shell and are usually also recognized
by other shells. See the \ref{table:envvariable} table.

\begin{table*}[H]
	\begin{tabular}{|c|p{10cm}|}
		\hline
		\textbf{ Variable} & \textbf{ Meanings} \\ \hline
		 HOME &  The user's home directory. Set automatically at login from the user's login directory in the password file \\ \hline
		 PATH &  The default search path for executables. \\ \hline
		 CDPATH &  The search path used with the cd builtin, to allow for shortcuts. \\ \hline
		 LANG &  The directory for internationalization files, used by localizable programs. \\ \hline
		 MAIL &  The name of a mail file, that will be checked for the arrival of new mail. \\ \hline
		 MAILCHECK &  The frequency in seconds that the shell checks for the arrival of mail. \\ \hline
		 MAILPATH &  A colon “:” separated list of file names, for the shell to check for incoming mail. \\ \hline
		 PS1 &  The control string for your prompt, which defaults to “\$ ”,
unless you are the superuser, in which case it defaults to “\# ”. \\ \hline
		 PS2 &  The control string for your secondary prompt, which defaults to
“\textgreater{} ”. The secondary prompt is what you see when you break a
command over more than one line. \\ \hline
		 PS4 &  The character string you see before the output of an execution
trace (set -x); defaults to “+ ”. \\ \hline
		 IFS &  Input Field Separators. Basically the characters the shell
considers to be whitespace. Normally set to \textless{} space \textgreater{}, \textless tab \textgreater{}, and
\textless newline \textgreater{}. \\ \hline
		 TERM &  The terminal type, for use by the shell. \\ \hline
	\end{tabular}
	\caption{Bourne Shell environment variables}
	\label{table:envvariable}
\end{table*}
