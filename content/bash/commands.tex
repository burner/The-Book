\section{The easy way: the interactive session}
Before we can start any kind of examination of the abilities of the Bourne
Shell and how you can tap into its power, we have to cover some basic ground
first: we have to discuss how to enter commands into the shell for execution by
that shell.

\subsection{Taking another look at what you've probably already seen}
If you have access to a Unix-based machine (or an emulator on another operating
system), you've probably been using the Bourne Shell -- or one of its
descendants -- already, possibly without realising. Surprise: you've been doing
shell scripting for a while already!

In your Unix environment, go to a terminal; either a textual logon terminal, or
a terminal-in-a-window if you're using the X Window System (look for something
called \textit{xterm} or \textit{rxvt} or just \textit{terminal}, if you have
actually not ever done this yet). You'll probably end up looking at a screen
looking something like this:
\scriptsize
\begin{verbatim}
Ben_Tels:Local_Machine:~>_
\end{verbatim}
\normalsize
or
 
The admin says: everybody, STOP TRYING TO CRASH THE SYSTEM Have a lot of fun!

or even something as simple as
\scriptsize
\begin{verbatim}
$_
\end{verbatim}
\normalsize

That's it. That's your shell: your direct access to everything the system has
to offer.

\subsection{Using the shell in interactive mode}
Specifically, the program you accessed a moment ago is your shell, running in
\textit{interactive mode}: the shell is running in such a way that it displays
a prompt and a cursor (the little, blinking line) and is waiting for you to
enter a command for it to execute. You execute commands in interactive mode by
typing them in, followed by a press of the \textbf{Enter} key. The shell then
translates your command to something the operating system understands and
passes off control to the operating system so that it can actually carry out
the task you have sent it.  You'll notice that your cursor will disappear
momentarily while the command is being carried out, and you cannot type anymore
(at this point, the Bourne Shell program is no longer in control of your
terminal -- the other program that you started  by executing your command is).
At some point the operating system will be finished working on your command and
the shell will bring up a new prompt and the cursor as well and will then start
waiting again for you to enter another command. Give it a try: type the command
\scriptsize
\begin{verbatim}
ls enter
\end{verbatim}
\normalsize

After a short time, you'll see a list of files in the working directory (the
directory that your shell considers the "current" directory), a new prompt and
the cursor.

This is the simplest way of executing shell commands: typing them in one at a
time and waiting for each to complete in order. The shell is used in this way
very often, both to execute commands that belong to the Bourne Shell
programming language and simply to start running other programs (like the ls
program from the example above).

\subsection{A useful tidbit}
Before we move on, we'll mention two useful key combinations when using the
shell: the command to interrupt running programs and shell commands and the
command to quit the shell (although, why you would ever want to \textit{stop}
using the shell is beyond me....).

To interrupt a running program or shell command, hit the Control and C keys at
the same time. We'll get back to what this does exactly in a later chapter, but
for now just remember this is the way to interrupt things.

To quit the shell session, hit Control+d. This key combination produces the
Unix end-of-file character -- we'll talk more later about why this also
terminates your shell session. Some modern shells have disabled the use of
Control+d in favor of the "exit" command (shame on them). If you're using such
a shell, just type the word "exit" (like with any other command) and press
Enter (from here on in, I'll leave the "Enter" out of examples).

\section{The only slightly less easy way: the script}
As we saw in the last section, you can very easily execute shell commands for
all purposes by starting an interactive shell session and typing your commands
in at the prompt. However, sometimes you have a set of commands that you have
to repeat regularly, even at different times and in different shell sessions.
Of course, in the programming-centric environment of a Unix system, you can
write a program to get the same result (in the C language for instance). But
wouldn't it be a lot easier to have the convenience of the shell for this same
task? Wouldn't it be more convenient to have a way to replay a set of commands?
And to be able to compose that set as easily as you can write the single
commands that you type into the shell's interactive sessions?

\subsection{The shell script}
Fortunately, there \textit{is} such a way: the Bourne Shell's
\textit{non-interactive} mode. In this mode, the shell doesn't have a prompt or
wait for your commands. Instead, the shell reads commands from a text file
(which tells the shell what to do, kind of like an actor gets commands from a
script -- hence, shell script). This file contains a sequence of commands, just
as you would enter them into the interactive session at the prompt. The file is
read by the shell from top to bottom and commands are executed in that order.

A shell script is very easy to write; you can use any text-editor you like (or
even any wordprocessor or other editor, as long as you remember to save your
script in plain text format). You write commands just as you would in the
interactive shell. And you can run your script the moment you have saved it; no
need to compile it or anything.

\subsection{Running a shell script}
To run a shell script (to have the shell read it and execute all the commands
in the script), you enter a command at an interactive shell prompt as you would
when doing anything else (if you're using a graphical user interface, you can
probably also execute your scripts with a click of the mouse). In this case,
the program you want to start is the shell program itself. For instance, to run
a script called \textit{MyScript}, you'd enter this command in the interactive
shell (assuming the script is in your working directory):
\lstset{basicstyle=\scriptsize, numbers=left, captionpos=b, tabsize=4}
\begin{lstlisting}[caption=Running a script,language={bash},
xleftmargin=15pt, label=lst:Running a script]
sh MyScript
\end{lstlisting}

Starting the shell program from inside the shell program may sound weird at
first,  but it makes perfect sense if you think about it. After all, you're
typing commands in an \textit{interactive mode} shell session. To run a script,
you want to start a shell in \textit{non-interactive mode}. That's what's
happening in the above command. You'll note that the Bourne Shell executable
takes a single parameter in the example above: the name of the script to
execute.

If you happen to be using a POSIX 1003.1-compliant shell, you can also execute
a single command in this new, non-interactive session. You have to use the -c
command-line switch to tell the shell you're passing in a command instead of
the name of a script:
\lstset{basicstyle=\scriptsize, numbers=left, captionpos=b, tabsize=4}
\begin{lstlisting}[caption=Running a command in a new shell,language={bash},
xleftmargin=15pt, label=lst:Running a command in a new shell]
sh -c ls
\end{lstlisting}

We'll get to why you would want to do this (rather than simply enter your
command directly into the interactive shell) a little further down.

There is also another way to run a script from the interactive shell: you type
the execute command (a single period) followed  by the name of the script:
\lstset{basicstyle=\scriptsize, numbers=left, captionpos=b, tabsize=4}
\begin{lstlisting}[caption=Sourcing a script,language={bash},
xleftmargin=15pt, label=lst:Sourcing a script]
. MyScript
\end{lstlisting}

The difference between that and using the \textit{sh} command is that the
\textit{sh} command starts a new process and the execute command does not.
We'll look into this (and its importance) in the next section. By the way, this
notation with the period is commonly referred to as \textit{sourcing} a script.

\subsection{Running a shell script the other way}
There is also another way to execute a shell script, by making more direct use
of a feature of the Unix operating system: the executable mode.

In Unix, each and every file has three different permissions (read, write and
execute) that can be set for three different entities: the user who owns the
file, the group that the file belongs to and "the world" (everybody else). Give
the command
\lstset{basicstyle=\scriptsize, numbers=left, captionpos=b, tabsize=4}
\begin{lstlisting}[language={bash},
xleftmargin=15pt]
ls -l
\end{lstlisting}

in the interactive shell to see the permissions for all files in the working
directory (the column with up to nine letters, r, w and x for read write and
execute, the first three for the user, the middle ones for the group, the right
ones for the world). Whenever one of those entities has the "execute"
permission, that entity can simply run the file as a program. To make your
scripts executable by everybody, use the command
\lstset{basicstyle=\scriptsize, numbers=left, captionpos=b, tabsize=4}
\begin{lstlisting}[language={bash},
xleftmargin=15pt]
chmod +x scriptname
\end{lstlisting}

as in 
\lstset{basicstyle=\scriptsize, numbers=left, captionpos=b, tabsize=4}
\begin{lstlisting}[caption=Making MyScript executable,language={bash},
xleftmargin=15pt, label=lst:Making MyScript executable]
chmod +x MyScript
\end{lstlisting}

You can then execute the script with a simple command like so (assuming it is
in a directory that is in your PATH, the directories that the shell looks in
for programs when you don't tell it exactly where to find the program):
\lstset{basicstyle=\scriptsize, numbers=left, captionpos=b, tabsize=4}
\begin{lstlisting}[caption=Running a command in a new shell,language={bash},
xleftmargin=15pt, label=lst:Running a command in a new shell]
MyScript
\end{lstlisting}

If this fails then the current directory is probably not in your PATH.  You can
force the execution of the script using
\lstset{basicstyle=\scriptsize, numbers=left, captionpos=b, tabsize=4}
\begin{lstlisting}[caption=Making the shell look for your script in the current directory,language={bash},
xleftmargin=15pt, label=lst:Making the shell look for your script in the current directory]
./MyScript
\end{lstlisting}

At this command, the operating system examines the file, places it in memory
and allows it to run like any other program. Of course, not every file makes
\textit{sense} as a program; a binary file is not necessarily a set of commands
that the computer will recognize and a text file cannot be read by a computer
at all. So to make our scripts run like this, we have to do something extra.

As we mentioned before, the Unix operating system starts by examining the
program. If the program is a text file rather than a binary one (and cannot
simply be executed), the operating system expects the first line of the file to
name the interpreter that the operating system should start to interpret the
rest of the file. The line the Unix operating system expects to find looks like
this:

\lstset{basicstyle=\scriptsize, numbers=left, captionpos=b, tabsize=4}
\begin{lstlisting}[language={bash},
xleftmargin=15pt]
#!full path and name of interpreter
\end{lstlisting}

In our case, the following line should work pretty much everywhere:
\lstset{basicstyle=\scriptsize, numbers=left, captionpos=b, tabsize=4}
\begin{lstlisting}[language={bash},
xleftmargin=15pt]
#!/bin/sh
\end{lstlisting}

The Bourne Shell executable, to be found in the bin directory, which is right
under the top of the filesystem tree. For example:
\lstset{basicstyle=\scriptsize, numbers=left, captionpos=b, tabsize=4}
\begin{lstlisting}[caption=Bourne shell script with an explicit interpreter,language={bash},
xleftmargin=15pt, label=lst:Bourne shell script with an explicit interpreter]
#!/bin/sh
echo Hello World!
\end{lstlisting}

\scriptsize
\begin{verbatim}
Hello World!
\end{verbatim}
\normalsize

Executing shell scripts like this has several advantages. First it's less
cumbersome than the other notations (it requires less typing). Second, it's an
extra safety if you're going to pass your scripts around to others. Instead of
relying on them to have the right shell, you can simply specify which shell
they should use. If Bourne Shell is enough, that's what you ask for. If you
absolutely need \textit{ksh} or \textit{bash}, you specify that instead (mind
you, it's not foolproof - other people can ignore your interpreter
specification by running your script with one of the other commands that we
discussed above, even if the script probably won't work if they do that).

Just as a sidenote, Unix doesn't limit this trick to shell scripts. Any script
interpreter that expects its scripts to be plain-text can be specified in this
way. You can use this same trick to make directly executable Perl scripts or
Python, Ruby, etc. scripts as well as Bourne Shell scripts.

\section{A little bit about Unix and multiprocessing}
\subsection{Why you want to know about multiprocessing}
While this is not directly a book about Unix, there are some aspects of the
Unix operating system that we must cover to fully understand why the Bourne
Shell works the way it does from time to time.
One of the most important aspects of the Unix operating system - in fact, the
main aspect that sets it apart from all other main-stream operating systems -
is that the Unix Operating System is and always has been a multi-user,
multi-processing operating system (this in contrast with other operating
systems like MacOS and Microsoft's DOS/Windows operating systems). The Unix OS
was always meant to run machines that would be used simultaneously by several
users, who would all want to run at least one but possibly several programs at
the same time. The ability of an operating system to divide the time of a
machine's processor among several programs so that it seems to the user that
they are all running at the same time is called \textit{multiprocessing}. The
Unix Operating System was designed from the core up with this possibility in
mind and it has an effect on the way your shell sessions behave.

Whenever you start a new process (by running a program, for instance) on your
Unix machine, the operating system provides that process with its very own
operating environment. That environment includes some memory for the process to
play in and it can also include certain predefined settings for all processes.
Whenever you run the shell program, it is running in its own environment.

Whenever you start a new process from another process (for instance by issuing
a command to your shell program in interactive mode), the new process becomes
what is called a \textit{child process} of the first process (the ls program
runs as a child process of your shell, for instance). This is where it becomes
important to know about multiprocessing and process interaction: a child
process always starts with a \textit{copy} of the environment of the parent
process. This means two things:

A child process can \textit{never} make changes to the operating environment
of its parent - it only has access to a copy of that environment if you
actually do \textit{want} to make changes in the environment of your shell (or
specifically want to avoid it), you have to know when a command runs as a child
process and when it runs within your current shell; you might otherwise pick a
variant that has the opposite effect of that which you want

\subsection{What does what}
We have seen several ways of running a shell command or script. With respect to
multiprocessing, they run in the following way:
Way of running Runs as

\begin{tabular}{|p{3cm}|p{3cm}|}
\hline
Way of running & Runs as\\ \hline
Interactive mode command & \begin{itemize}
\item current environment for a shell command
\item child process for a new program
\end{itemize} \\ \hline
Shell non-interactive mode &child process\\ \hline
Dot-notation run command \textit{(. MyScript)} & current environment\\ \hline
Through Unix executable permission with interpreter selection & child process\\
\hline
\end{tabular}

\subsection{A useful thing to know: background processes}
With the above, it may seem like multiprocessing is just a pain when doing
shell scripting. But if that were so, we wouldn't \textit{have} multiprocessing
-- Unix doesn't tend to keep things that aren't useful. Multiprocessing is a
valuable tool in interacting with the rest of the system and one that you can
use to work more efficiently. There are many books available on the benefits of
multiprocessing in program development, but from the point of view of the
Bourne Shell user and scripter the main one is the ability to hand off control
of a process to the operating system \textit{and still keep on working while
that child process is running}. The way to do this is to run your process as a
\textit{background process}.

Running a process as a background process means telling the operating system
that you want to start a process, but that it should not attach itself to any
of the interactive devices (keyboard, screen, etc.) that its parent process is
using. And more than that, it also tells the operating system that the request
to start this child process should return immediately and that the parent
process should then be allowed to continue working without having to wait for
its child process to end.

This sounds complicated, but you have to keep in mind that this ability is
completely ingrained in the Unix operating system and that the Bourne Shell was
intended as an easy interface to the power of Unix. In other words: the Bourne
Shell includes the ability to start a child process as a simple command of its
own. Let's demonstrate how to do this and how useful the ability is at the same
time, with an example. Give the following (rather pointless but still time
consuming) command at the prompt:
\lstset{basicstyle=\scriptsize, numbers=left, captionpos=b, tabsize=4}
\begin{lstlisting}[caption=,language={bash},
breaklines=true,xleftmargin=15pt,label=lst:]
N=0 &\& while [ $N -lt 10000 ];
do date >> scriptout; N=`expr $N + 1`; done
\end{lstlisting}

We'll get into what this says in later chapters; for now, it's enough to know
that this command asks the system for the date and time and writes the result
to a file named "scriptout". Since it then repeats this process 10000 times, it
may take a little time to complete.

Now give the following command:

\lstset{basicstyle=\scriptsize, numbers=left, captionpos=b, tabsize=4}
\begin{lstlisting}[caption=,language={bash},
breaklines=true,xleftmargin=15pt,label=lst:]
N=0 && while [ $N -lt 10000 ]; do date >>
scriptout; N=`expr $N + 1`; done&
\end{lstlisting}

You'll notice that you can immediately resume using the shell (if you don't see
this happening, hit Control+C and check that you have the extra ampersand at
the end). After a while the background process will be finished and the
scriptout file will contain another 10000 time reads.

The way to start a background process in Bourne Shell is to append an ampersand
(\&) to your command.
