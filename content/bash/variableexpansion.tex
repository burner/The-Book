\section{Variable Expansion}
In the Environment module we introduced the idea of an environment variable as
a general way of storing small pieces of data. In this module we take an
in-depth look at using those variables: 'variable expansion', 'parameter
substitution' or just 'substitution'.

\subsection{Substitution}
The reason that using a variable is called substitution is that the shell
literally replaces each reference to any variable with its value. This is done
while evaluating the command-line, which means that the variable substitution
is made \emph{before} the command is actually executed.

The simplest way of using a variable is the way we've already seen, prepending
the variable name with a '\$'. So for instance:
\lstset{basicstyle=\scriptsize, numbers=left, captionpos=b, tabsize=4}
\begin{lstlisting}[caption=,language={bash},
breaklines=true,xleftmargin=15pt,label=lst:]
$ USER=JoeSixpack
$ echo $USER
\end{lstlisting}
\scriptsize
\begin{verbatim}
JoeSixpack
\end{verbatim}
\normalsize

The value JoeSixpack is substituted for \$USER before the echo command is
executed.

Of course, once the substitution is made the result is still just the text that
was in the variable. The interpretation of that text is still done by whatever
program is run. So for example:
\lstset{basicstyle=\scriptsize, numbers=left, captionpos=b, tabsize=4}
\begin{lstlisting}[caption=Variables do not make magic,language={bash},
breaklines=true,xleftmargin=15pt,label=lst:Variables do not make magic]
$ USER=JoeSixpack
$ ls $USER
\end{lstlisting}
\scriptsize
\begin{verbatim}
ls: cannot access JoeSixpack:
No such file or directory
\end{verbatim}
\normalsize

Just because the text came from a variable, doesn't mean the file exists.

Basic variable expansion is already quite flexible. You can use it as described
above, but you can also use variables to create longer strings. For instance,
if you want to set the log directory for an application to the ``log''
directory in your home directory, you might fill in the setting like this:
\lstset{basicstyle=\scriptsize, numbers=left, captionpos=b, tabsize=4}
\begin{lstlisting}[caption=,language={bash},
breaklines=true,xleftmargin=15pt,label=lst:]
$HOME/log
\end{lstlisting}

And if you're going to use that setting more often, you might want to create
your own variable like this:
\lstset{basicstyle=\scriptsize, numbers=left, captionpos=b, tabsize=4}
\begin{lstlisting}[caption=,language={bash},
breaklines=true,xleftmargin=15pt,label=lst:]
LOGDIR=$HOME/log
\end{lstlisting}

And, of course, if you want specific subdirectories for logs for different
programs, then the logs for the Wyxliz application go into directory
\begin{verbatim}
$LOGDIR/Wyxliz/
\end{verbatim}

\subsection{Substitution forms}
The Bourne Shell has a number of different syntaxes for variable substitution,
each with its own meaning and use. In this section we examine these syntaxes.

\subsubsection{Basic variable substitution}
We've already talked at length about basic variable substitution: you define a
variable, stick a '\$' in front of it, the shell substitutes the value for the
variable. By now you're probably bored of hearing about it.

But we've not talked about one situation that you might run into with basic
variable substitution. Consider the following:
\lstset{basicstyle=\scriptsize, numbers=left, captionpos=b, tabsize=4}
\begin{lstlisting}[caption=Adding some text to a variable's value,language={bash},
breaklines=true,xleftmargin=15pt,label=lst:Adding some text to a variable's value]
$ ANIMAL=duck
$ echo One $ANIMAL, two $ANIMALs
\end{lstlisting}
\scriptsize
\begin{verbatim}
One duck, two
Uhhh.... we're missing something...
\end{verbatim}
\normalsize
So what went wrong here? Well, obviously the shell substituted nothing for the
ANIMAL variable, but why? Because with the extra 's' the shell thought we were
asking for the non-existent ANIMALs variable. But what gives there? We've used
variables in the middle of strings before (as in '/home/ANIMAL/logs'). But an
's' is not a '/': an 's' can be a valid part of a variable name, so the shell
cannot tell the difference. In cases where you explicitly have to separate the
variable from other text, you can use braces:
\lstset{basicstyle=\scriptsize, numbers=left, captionpos=b, tabsize=4}
\begin{lstlisting}[caption=Adding some text to a variable's value take II,language={bash},
breaklines=true,xleftmargin=15pt,label=lst:AddingsometexttoavariablesvaluetakeII]
$ ANIMAL=duck
$ echo One $ANIMAL, two ${ANIMAL}s
\end{lstlisting}
\scriptsize
\begin{verbatim}
One duck, two ducks
That's better!
\end{verbatim}
\normalsize
Both cases (with and without the braces) count as basic variable substitution
and the rules are exactly the same. Just remember not to leave any spaces
between the braces and the variable name.

\subsection{Substitution with a default value}
Since a variable can be empty, you'll often write code in your scripts to check
that mandatory variables actually have a value. But in the case of optional
variables it is usually more convenient not to check, but to use a default
value for the case that a variable is not defined. This case is actually so
common that the Bourne Shell defines a special syntax for it: the dash. Since a
dash can mean other things to the shell as well, you have to combine it with
braces --- the final result looks like this:
\scriptsize
\begin{verbatim}
${varname[:]-default}
\end{verbatim}
\normalsize

Again, don't leave any spaces between the braces and the rest of the text. The
way to use this syntax is as follows:

\lstset{basicstyle=\scriptsize, numbers=left, captionpos=b, tabsize=4}
\begin{lstlisting}[caption=Default values,language={bash},
breaklines=true,xleftmargin=15pt,label=lst:Default values]
$ THIS_ONE_SET=Hello
$ echo $THIS_ONE_SET ${THIS_ONE_NOT:-World}
\end{lstlisting}
\scriptsize
\begin{verbatim}
Hello World
\end{verbatim}
\normalsize

Compare that to this:
\lstset{basicstyle=\scriptsize, numbers=left, captionpos=b, tabsize=4}
\begin{lstlisting}[caption=Default not needed,language={bash},
breaklines=true,xleftmargin=15pt,label=lst:Default not needed]
$ TEXT=aaaaaahhhhhhhh
$ echo Say ${TEXT:-bbbbbbbbbb}
\end{lstlisting}
\scriptsize
\begin{verbatim}
Say aaaaaahhhhhhhh
\end{verbatim}
\normalsize
Interestingly, the colon is optional; so\\ \$\{VAR:-default\} has the same result
as\\ \$\{VAR-default\}.

\subsection{Substitution with default assignment}
As an extension to default values, there's a syntax that not only supplies a
default value but assigns it to the unset variable at the same time. It looks
like this:
\begin{verbatim}
${varname[:]=default}
\end{verbatim}
\normalsize

As usual, avoid spaces in between the braces. Here's an example that
demonstrates how this syntax works:
\lstset{basicstyle=\scriptsize, numbers=left, captionpos=b, tabsize=4}
\begin{lstlisting}[caption=Default value assignment,language={bash},
breaklines=true,xleftmargin=15pt,label=lst:Default value assignment]
$ echo $NEWVAR

$ echo ${NEWVAR:=newval}
newval
$ echo $NEWVAR
newval
\end{lstlisting}

As with the default value syntax, the colon is optional.

\subsection{Substitution for actual value}
This substitution is sort of a quick test to see if a variable is defined (and
that's usually what it's used for). It's sort of the reverse of the default
value syntax and looks like this:
\scriptsize
\begin{verbatim}
${varname[:]+substitute}
\end{verbatim}
\normalsize
This syntax returns the substitute value if the variable \textbf{is} defined.
That sounds counterintuitive at first, especially if you ask what is returned
if the variable is \textbf{not} defined --- and learn that the value is
nothing. Here's an example:
\lstset{basicstyle=\scriptsize, numbers=left, captionpos=b, tabsize=4}
\begin{lstlisting}[caption=Actual value substitution,language={bash},
breaklines=true,xleftmargin=15pt,label=lst:Actual value substitution]
$ echo ${NEWVAR:+newval}

$ NEWVAR=oldval
$ echo ${NEWVAR:+newval}
newval
\end{lstlisting}

So what could possibly be the use of this notation? Well, it's used often in
scripts that have to check whether lots of variables are set or not. In this
case the fact that a variable has a value means that a certain option has been
activated, so you're interested in knowing that the variable \emph{has} a
value, not what that value is. It looks sort of like this (pseudocode, this
won't actually work in the shell):
\lstset{basicstyle=\scriptsize, numbers=left, captionpos=b, tabsize=4}
\begin{lstlisting}[caption=Default value assignment,language={bash},
breaklines=true,xleftmargin=15pt,label=lst:Default value assignment]
if ${SPECIFIC_OPTION_VAR:+optionset} == optionset then ...
\end{lstlisting}

Of course, in this notation the colon is optional as well.

\subsection{Substitution with value \\check}
This final syntax is sort of a debug check to check whether or not a variable
is set. It looks like this:
\scriptsize
\begin{verbatim}
${varname[:]?message}
\end{verbatim}
\normalsize

With this syntax, if the variable is defined everything is okay. Otherwise, the
message is printed and the command or script exits with a non-zero exit status.
Or, if there is no message, the text ``parameter null or not set'' is printed.
As usual the colon is optional and you may not have a space between the colon
and the variable name.

You can use this syntax to check that the mandatory variables for your scripts
have been set and to print an error message if they are not.

\lstset{basicstyle=\scriptsize, numbers=left, captionpos=b, tabsize=4}
\begin{lstlisting}[caption=Default value assignment,language={bash},
breaklines=true,xleftmargin=15pt,label=lst:Default value assignment]
$ echo ${SOMEVAR:?has not been set}
-sh: SOMEVAR: has not been set
$ echo ${SOMEVAR:?}
-sh: SOMEVAR: parameter null or not set
\end{lstlisting}
