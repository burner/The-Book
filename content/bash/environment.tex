\section{The Environments}
When discussing a Unix shell, you often come across the term "environment".
This term is used to describe the context in which a program executes and is
usually meant to mean a set of "environment variables" (we'll get to those
shortly). But in fact there are two different terms that are somehow a
program's environment and which often get mixed up together in "environment".
The simpler one of these really is the collection of environment variables and
actually is called the "environment". The second is a much wider collection of
resources that influence the execution of a program and is called the
\textit{command execution environment}.

\subsection{The command execution environment}
Each running program, either started directly by the user from the shell or
indirectly by another process, operates within a collection of global resources
called its \textbf{command execution environment} (CEE).

A program's CEE contains important information such as the source and
destination of data upon which the program can operate (also known as the
../Files and streams|standard input, ../Files and streams|standard output and
../Files and streams|standard error handles). In addition, variables are
defined that list the identity and home directory of the user or process that
started the program, the hostname of the machine and the kind of terminal used
to start the program. There are other variables too, but that's just some of
the main ones. The environment also provides working space for the program, as
well as a simple way of communicating with other, future programs, that will be
run in the same environment. 

The complete list of resources included in the shell's CEE is:
\begin{itemize}
\item Open files held in the parent process that started the shell. These files
are inherited. This list of files includes the files accessed through
redirection (such as standard input, output and error files).
\item The current working directory: the "current" directory of the shell.
\item The file creation mode:The default set of file permissions set when a new
file is created.
\item The active |traps.
\item Shell parameters and variables set during the call to the shell or
inherited from the parent process.
\item Shell |functions inherited from the parent process.
\item Shell options set by \textit{set} or \textit{shopts}, or as command line
options to the shell executable.
\item Shell aliases (if available in your shell).
\item The process id of the shell and of some processes started by the parent
process.
\end{itemize}

Whenever the shell executes a command that starts a child process, that command
is executed it its own CEE. This CEE inherits a copy of part of the CEE of its
parent, but not the entire parent CEE. The inherited copy includes:

\begin{itemize}
\item Open files.
\item The working directory.
\item The file creation mode mask.
\item Any shell variables and functions that are marked to be exported to child processes.
\item Traps set by the shell
\end{itemize}.

\subsubsection{The 'set' command}
The 'set' command allows you to set or disable a number of options that are
part of the CEE and influence the behavior of the shell. To set an option, set
is called with a command line argument of '-' followed by one or more flags. To
disable the option, set is called with '+' and then the same flag. You probably
won't use these options very often; the most common use of 'set' is the call
without any arguments, which produces a list of all defines names in the
environment (variables and functions). Here are some of the options you might
get some use out of:

\begin{description}
\item[+/-a]:When set, automatically mark all newly created or redefined variables for export.
\item[+/-f]When set, ignore filename \_Shell\_Scripting/ Control\_flow \#Regular\_expressions\_and\_metacharacters | metacharacters.
\item[+/-n]When set, only read commands but do not execute them.
\item[+/-v]When set, causes the shell to print commands as they are read from input (verbose debugging flag).
\item[+/-x]When set, causes the shell to print commands as they will be executed (debugging flag).
\end{description}

Again, you'll probably mostly use set without arguments, to inspect the list of
defined variables.

\subsection{The environment and environment variables}
Part of the CEE is something that is simply called the \textit{environment}.
The environment is a collection of name/value pairs called \textit{environment variables}.
These variables technically also contain the shell |functions, but
we'll discuss those in a |separate module.

An environment variable is a piece of labelled storage in the environment,
where you can store anything you like as long as it fits. These spaces are
called variables because you can vary what you put in them. All you need to
know is the name (the label) that you used for storing the content. The Bourne
shell also makes use of these "environment variables". You can make scripts
that examine these variables, and those scripts can make decisions based on the
values stored in the variables.

An environment variable is a name/value pair of the form

\lstset{basicstyle=\scriptsize, numbers=left, captionpos=b, tabsize=4}
\begin{lstlisting}[language={bash},
xleftmargin=15pt]
name=variable
\end{lstlisting}

which is also the way of creating a variable. There are several ways of using a
variable which we will discuss in the module on |substitution, but for now we
will limit ourselves to the simple way: if you prepend a variable name with a
\$-character, the shell will substitute the value for the variable. So, for
example:

\lstset{basicstyle=\scriptsize, numbers=left, captionpos=b, tabsize=4}
\begin{lstlisting}[caption=Simple use of a variable,language={bash},
xleftmargin=15pt, label=lst:simpleuseofavariable]
VAR=Hello
$ echo $VAR
\end{lstlisting}

\scriptsize
\begin{verbatim}
Hello
\end{verbatim}
\normalsize
As you can see from the example above, an environment variable is sort of like
a bulletin board: anybody can post any kind of value there for everybody to
read (as long as they have access to the board). And whatever is posted there
can be interpreted by any reader in whatever way they like. This makes the
environment variable a very general mechanism for passing data along from one
place to another. And as a result environment variables are used for all sorts
of things. For instance, for setting global parameters that a program can use
in its execution. Or for setting a value from one shell script to be picked up
by another. There are even a number of environment variables that the shell
itself uses in its configuration. Some typical examples:

\begin{description}
\item[IFS]This variable lists the characters that the shell considers to be whitespace characters.
\item[PATH]This variable is interpreted as a list of directories (separated by colons on a Unix system). Whenever you type the name of an executable for the shell to execute but do not include the full path of that executable, the shell will look in all of these directories \textit{in order} to find the executable.
\item[PS1]This variable lists a set of codes. These codes instruct your shell about what the command-line prompt in the interactive shell should look like.
\item[PWD]The value of this variable is always the path of the working directory.
\end{description}

The absolute beauty of environment variables, as mentioned above, is that they
just contain random strings of characters without an immediate meaning. The
meaning of any variable is to be interpreted by whatever program or process
reads the variable. So a variable can hold literally any kind of information
and be used practically anywhere. For instance, consider the following example:
\lstset{basicstyle=\scriptsize, numbers=left, captionpos=b, tabsize=4}
\begin{lstlisting}[caption=Environment variables are more flexible than you thought...,language={bash},
xleftmargin=15pt, label=lst:environmentvariables]
$ echo $CMD

$ CMD=ls
$ echo $CMD
ls
$ $CMD
bin  booktemp  Documents  Mail  mbox  public_html  sent
\end{lstlisting}

There's nothing wrong with setting a variable to the name of an executable,
then executing that executable by calling the variable as a command.

\subsection{Different kinds of environment variables}
Although you use all environment variables the same way, there are a couple of
different kinds of variables. In this section we discuss the differences
between them and their uses.

\subsubsection{Named variables}
The simplest and most straightforward environment variable is the named
variable. We saw it earlier: it's just a name with a value, which can be
retrieved by prepending a \$ to the name. You create and define a named
variable in one go, by typing the name, an equals sign and then something that
results in a string of characters.

Earlier we saw the following, simple example:
\lstset{basicstyle=\scriptsize, numbers=left, captionpos=b, tabsize=4}
\begin{lstlisting}[caption=Assigning a simple value to a variable,language={bash},
xleftmargin=15pt,label=lst:Assigningasimplevaluetoavariable1]
$ VAR=Hello
\end{lstlisting}

This just assigns a simple value. Once a variable has been defined, we can also
redefine it:
\lstset{basicstyle=\scriptsize, numbers=left, captionpos=b, tabsize=4}
\begin{lstlisting}[caption=Assigning a simple value to a variable,language={bash},
xleftmargin=15pt,label=lst:Assigningasimplevaluetoavariable2]
$ VAR=Goodbye
\end{lstlisting}


We aren't limited to straightforward strings either. We can just as easily
assign the value of one variable to another:

\lstset{basicstyle=\scriptsize, numbers=left, captionpos=b, tabsize=4}
\begin{lstlisting}[caption=Assigning a simple value to a variable,language={bash},
xleftmargin=15pt,label=lst:Assigningasimplevaluetoavariable3]
$ VAR=$PATH
\end{lstlisting}


We can even go all-out and combine several commands to come up with a value:

\lstset{basicstyle=\scriptsize, numbers=left, captionpos=b, tabsize=4}
\begin{lstlisting}[caption=Assigning a combined value to a variable,language={bash},
xleftmargin=15pt,label=lst:Assigningacombinedvaluetoavariable4]
$ PS1= "`whoami`@`hostname -s` `pwd` 
$ "
\end{lstlisting}


In this case, we're taking the output of the three commands 'whoami',
'hostname', and 'pwd', then we add the '\$' symbol, and some spacing and other
formatting just to pad things out a bit. Whew. All that, just in one piece of
labeled space. As you can see environment variables can hold quite a bit,
including the output of entire commands.

There are usually lots of named variables defined in your environment, even if
you are not aware of them. Try the 'set' command and have a look.

\subsubsection{Positional variables}
Most of the environment variables in the shell are named variables, but there
are also a couple of "special" variables. Variables that you don't set, but
whose values are automatically arranged and maintained by the shell. Variables
which exist to help you out, to discover information about the shell or from
the environment.

The most common of these are the positional or argument variables. Any command
you execute in the shell (in interactive mode or in a script) can have
command-line arguments. Even if the command doesn't actually use them, they can
still be there. You pass command-line arguments to a command simply by typing
them after the command, like so:

\lstset{basicstyle=\scriptsize, numbers=left, captionpos=b, tabsize=4}
\begin{lstlisting}[language={bash},
xleftmargin=15pt]
\textit{command} \textit{arg0} \textit{arg1} ...
\end{lstlisting}

This is allowed for any command. Even your own shell scripts. But say that you
do this (create a shell script, then execute it with arguments); how do you
access the command-line arguments from your script? This is where the
positional variables come in. When the shell executes a command, it
automatically assigns any command-line arguments, in order, to a set of
positional variables. And these variables have numbers for names: 1 through 9,
accessed through \$1 through \$9. Well, actually zero though nine; \$0 is the
name of the command that was executed. For example, consider a script like
this:

\lstset{basicstyle=\scriptsize, numbers=left, captionpos=b, tabsize=4}
\begin{lstlisting}[caption=WithArgs.sh: A script that uses command-line arguments,language={bash},
xleftmargin=15pt,label=lst:WithArgs.sh:Ascriptthatusescommand-linearguments]
#!/bin/sh

echo \$0
echo \$1
echo \$2
\end{lstlisting}

And a call to this script like this:
\lstset{basicstyle=\scriptsize, numbers=left, captionpos=b, tabsize=4}
\begin{lstlisting}[caption=Calling the script,language={bash},
xleftmargin=15pt,label=lst:Callingthescript]
$ WithArgs.sh Hello World
WithArgs.sh
\end{lstlisting}
\scriptsize
\begin{verbatim}
Hello
World
\end{verbatim}
\normalsize

As you can see, the shell automatically assigned the values 'Hello' and 'World'
to \$1 and \$2 (okay, technically to the variables called 1 and 2, but it's
less confusing in written text to call them \$1 and \$2). What happens if we
call this script with more than two arguments?

\lstset{basicstyle=\scriptsize, numbers=left, captionpos=b, tabsize=4}
\begin{lstlisting}[caption=Calling the script with more arguments,language={bash},
xleftmargin=15pt, label=lst:Calling the script with more arguments]
\$ WithArgs.sh Hello World Mouse Cheese
\end{lstlisting}
\scriptsize
\begin{verbatim}
WithArgs.sh
Hello
World
Did the mouse eat the cheese?
\end{verbatim}
\normalsize
This is no problem whatsoever - the extra arguments get assigned to \$3 and
\$4. But we didn't use those variables in the script, so those command-line
arguments are ignored. What about the opposite case (too few arguments)?

\lstset{basicstyle=\scriptsize, numbers=left, captionpos=b, tabsize=4}
\begin{lstlisting}[caption=Calling the script with too few arguments...,language={bash},
xleftmargin=15pt,label=lst:Callingthescriptwithtoofewarguments...]
$ WithArgs.sh Hello
\end{lstlisting}
\scriptsize
\begin{verbatim}
WithArgs.sh
Hello
\end{verbatim}
\normalsize

Again, no problem. When the script accesses \$2, the shell simply substitutes
the value of \$2 for \$2. That value is nothing in this case, so we print
exactly that. In this case it's not a problem, but if your script has mandatory
arguments you should check whether or not they are actually there.

What about if we want 'Hello' and 'World' to be treated as one command-line
argument to be passed to the script? I.e. 'Hello World' rather than 'Hello' and
'World'? We'll get deeply into that when we start talking about
\_Shell\_Scripting/Control\_flow quoting, but for now just surround the
words with single quotes:


\lstset{basicstyle=\scriptsize, numbers=left, captionpos=b, tabsize=4}
\begin{lstlisting}[caption=Calling the script with multi-word arguments,language={bash},
xleftmargin=15pt,label=lst:Calling the script with multi-word arguments]
$ WithArgs.sh 'Hello World' 'Mouse Cheese'
\end{lstlisting}
\scriptsize
\begin{verbatim}
WithArgs.sh
Hello World
Mouse Cheese
\end{verbatim}
\normalsize

\subsubsection{Shifting}
So what happens if you have more than nine command line arguments? Then your
script is too complicated. No, but seriously: then you have a little problem.
It's allowed to pass more than nine arguments, but there are only nine
positional variables (in Bourne Shell at least). To deal with this situation
the shell includes the \textbf{shift} command:

\lstset{basicstyle=\scriptsize, numbers=left, captionpos=b, tabsize=4}
\begin{lstlisting}[language={bash},xleftmargin=15pt]
shift [n]
\end{lstlisting}

\textbf{Shift} causes the positional arguments to shift left. That is, the
the value of \$1 becomes the old value of \$2, the value of \$2 becomes the old
value of \$3 and so on. Using shift, you can access all the command-line
arguments (even if there are more than nine). The optional integer argument to
shift is the number of positions to shift (so you can shift as many positions
in one go as you like). There are a couple of things to keep in mind though:

\begin{itemize}
\item No matter how often you shift, \$0 always remains the original command.
\item If you shift n positions, n must be lower than the number of arguments.
If n is greater than the number of arguments, no shifting occurs.
\item If you shift n positions, the first n arguments are lost. So make sure
you have them stored elsewhere or you don't need them anymore!
\item You cannot shift back to the right.
\end{itemize}

\section{Other, special variables}
In addition to the positional variables the Bourne Shell includes a number of
other, special variables with special information about the shell. You'll
probably not use these as often, but it's good to know they're there. These
variables are
\begin{description}
	\item[\$\#]The number of command-line arguments to the current command (changes after a use of the \textbf{shift} command!).
	\item[\$-]The shell options currently in effect (see \#The \_ .27set. 27\_ command|the 'set' command).
	\item[\$?]The exit status of the last command executed (0 if it succeeded, non-zero if there was an error).
	\item[\$\$]The process id of the current process.
	\item[\$!]The process id of the last background command.
	\item[\$*]All the command-line arguments. When \_Shell\_ Scripting/Control\_ flow \#Quoting | quoted, expands to all command-line arguments as a single word (i.e. "\$*" = "\$1 \$2 \$3 ...").
	\item[\$@]All the command-line arguments. When \_Shell \_ Scripting /Control \_ flow \#Quoting | quoted, expands to all command-line arguments quoted individually (i.e. "\$@" = "\$1" "\$2" "\$3" ...).
\end{description}

\subsection{Exporting variables to a subprocess}
We've mentioned it a couple of times before: Unix is a multi-user,
multiprocessing operating system. And that fact is very much supported by the
Bourne Shell, which allows you to start up new processes right from inside a
running shell. In fact, you can even run multiple processes simultaneously next
to eachother (but we'll get to that a little later). Here's a simple example of
starting a subprocess: 
\lstset{basicstyle=\scriptsize, numbers=left, captionpos=b, tabsize=4}
\begin{lstlisting}[caption=Starting a new shell from the shell,language={bash},
xleftmargin=15pt, label=lst:Starting a new shell from the shell]
\$ sh
\end{lstlisting}

We've also talked about the Command Execution Environment and the Environment
(the latter being a collection of variables). These environments can affect how
programs run, so it's very important that they cannot inadvertently affect one
another. After all, you wouldn't want the screen in your shell to go blue with
yellow letters simply because somebody started Midnight Commander in another
process, right? 

One of the things that the shell does to avoid processes inadvertently
affecting one another, is environment separation. Basically this means that
whenever a new (sub)process is started, it has its own CEE and environment. Of
course it would be damned inconvenient if the environment of a subprocess of
your shell were \textit{completely} empty; your subprocess wouldn't have a PATH
variable or the settings you chose for the format of your prompt. On the other
hand there is usually a good reason NOT to have certain variables in the
environment of your subprocess, and it usually has something to do with not
handing off too much environment data to a process if it doesn't need that
data. This was particularly true when running copies of MS-DOS and versions of
DOS under Windows. You only HAD a limited amount of environment space, so you
had to use it carefully, or ask for more space on startup. These days in a UNIX
environment the space issues aren't the same, but if all your existing
variables ended up in the environment of your subprocess you might still
adversely affect the running of the program that you started in that subprocess
(there's really something to be said for keeping your environment lean and
clean in the case of subprocesses).

The compromise between the two extremes that Stephen Bourne and others came up
with is this: a subprocess has an environment which contains \textit{copies} of
the variables in the environment of its parent process - but only those
variables that are marked to be \textit{exported} (i.e. copied to
subprocesses). In other words, you can have any variable copied into the
environment of your subprocesses, but you have to let the shell know that's
what you want first. Here's an example of the distinction:


\lstset{basicstyle=\scriptsize, numbers=left, captionpos=b, tabsize=4}
\begin{lstlisting}[caption=Exported and non-exported variables,language={bash},
xleftmargin=15pt,label=lst:Exported and non-exported variables]
<source lang="bash">
$ echo $PATH
/usr/local/bin:/usr/bin:/bin
$ VAR=value
$ echo $VAR
value
$ sh
$ echo $PATH
gusr/local/bin:/usr/bin:/bin
$ echo $VAR

$
\end{lstlisting}

In the example above, the PATH variable (which is marked for export by default)
gets copied into the environment of the shell that is started within the shell.
But the VAR variable is \textit{not} marked for export, so the environment of
the second shell doesn't get a copy.

In order to mark a variable for export you use the \textbf{export} command,
like so:

\lstset{basicstyle=\scriptsize, numbers=left, captionpos=b, tabsize=4}
\begin{lstlisting}[language={bash},
xleftmargin=15pt]
export VAR0 [VAR1 VAR2 ...]
\end{lstlisting}

As you can see, you can export as many variables as you like in one go. You can
also issue the \textbf{export} command without any arguments, which will print
a list of variables in the environment marked for export. Here's an example of
exporting a variable:
\lstset{basicstyle=\scriptsize, numbers=left, captionpos=b, tabsize=4}
\begin{lstlisting}[caption=Exporting a variable,language={bash},
xleftmargin=15pt,label=lst:Exporting a variable]
$ VAR=value $ echo
$VAR
value
$ sh
$ echo $VAR

$ exit #Quitting the inner shell
$ export VAR #This is back in the outer shell
$ sh
$ echo $VAR
value
\end{lstlisting}

More modern shells like Korn Shell and Bash have more extended forms of
\textbf{export}. A common extension is to allow for definition and export of a
variable in one single command. Another is to allow you to \textit{remove} the
export marking from a variable. However, Bourne Shell only supports exporting
as explained above.

\subsection{Your profile} In the previous sections we've discussed the
runtime environment of every program and command you run using the shell. We've
talked about the command execution environment and at some length about the
piece of it simply called "the environment", which contains environment
variables. We've seen that you can define your own variables and that the
system usually already has quite a lot of variables to start out with.

Here's a question about those variables that the system starts out with: where
do they come from? Do they descend like manna from heaven? And on a related
note: what do you do if you want to create some variables automatically every
time your shell starts? Or run a program every time you log in?

Those readers who have done some digging around on other operating systems will
know what I'm getting at: there's usually some way of having a set of commands
executed every time you log in (or every time the system starts at least). In
MS-DOS for instance there is a file called autoexec.bat, which is executed
every time the system boots. In older versions of MS-Windows there was
system.ini. The Bourne Shell has something similar: a file in every user's home
directory called \textit{.profile}. The \$HOME/.profile (HOME is a default
variable whose value is your home directory) file is a shell script like any
other, which is executed automatically right after you login to a new shell
session. You can edit the script to have it execute any login-commands that you
like.

Each specific Unix system has its own default implementation of the .profile
script (including none - it's allowed not to have a .profile script). But all
of them start with some variation of this:
\lstset{basicstyle=\scriptsize, numbers=left, captionpos=b, tabsize=4}
\begin{lstlisting}[caption=A basic (but typical) HOME profile,language={bash},
xleftmargin=15pt,label=lst:A basic (but typical) HOME profile]
#!/bin/sh

if [ -f /etc/profile ]; then
 . /etc/profile
fi
PS1= "`whoami`@`hostname -s` `pwd` \$ "
export PS1
\end{lstlisting}

This .profile might surprise you a bit: where are all those variables that get
set? Most of the variables that get set for you on a typical Unix system, also
get set for all other users. In order to make that possible and easily
maintainable, the common solution is to have each \$HOME/.profile script start
by executing another shell script: /etc/profile. This script is a
systemwide script whose contents are maintained by the system administrator
(the user who logs in with username \textit{root}). This script sets all sorts
of variables and calls scripts that set even more variables and generally does
everything that is necessary to provide each user with a comfortable working
environment.

As you can see from the example above, you can add any personal configuration
you want or need to the .profile script in your directory. The call to execute
the system profile script doesn't have to be first, but you probably don't want
to remove it altogether.

\section{Multitasking and job control}
With the arrival of fast computers, CPUs that can switch between multiple tasks
in a very small amount of time, CPUs that can actually do multiple things at
the same time and networks of multiple CPUs, having the computer perform
multiple tasks at the same time has become common. Fast task switching provides
the illusion that the computer really is running multiple tasks simultaneously,
making it possible to effectively serve multiple users at once. And the ability
to switch to a new CPU task while an old task is waiting for a peripheral
device makes CPU use vastly more efficient.

In order to make use of multitasking abilities as a user, you need a command
environment that supports multitasking. For example, the ability to set one
program to a task, then move on and start a new program \textit{while the old
one is still running}. This kind of ability allows you as a user to do multiple
things at once on the same machine, as long as those programs do not interfere.
Of course, you cannot always treat each program as a "fire and forget" affair;
you might have to input a password, or the program might be finished and want
to tell you its results. A multitasking environment must allow you to switch
between the multiple programs you have running and allow those programs to send
you some sort of message if your attention is needed.

To make things a little more tangible think of something like downloading
files. Usually, while you're downloading files, you want to do other stuff as
well - otherwise you're going to be sitting at the keyboard twiddling your
thumbs a really long time when you want to download a whole CD worth of data.
So, you start up your file downloader and feed it a list of files you want to
grab. Once you've entered them, you can then tell it "Go!" and it will start
off by downloading the first file and continue until it finishes the last one,
or until there's a problem. The smarter ones will even try to work through
common problems themselves, such as files not being available. Once it starts
you get the standard shell prompt back, letting you know that you can start
another program. 

If you want to see how far the file downloader has gotten, simply checking the
files in your system against what you have on your list will tell you. But
another way to notify you is via the environment. The environment can include
the files that you work with, and this can help provide information about the
progress of currently running programs like that file downloader. Did it
download all the files? If you check the status file, you'll see that it's
downloaded 65\% of the files and is just working on the last three now.

Other examples of programs that don't need their hand held are programs that
play music. Quite often, once you start a program that plays music tracks, you
don't WANT to tell the program "Okay, now play the next track". It should be
able to do that for itself, given a list of songs to play. In fact, it should
not even have to hold on to the monitor; it should allow you to start running
other software right after you hit the "play" button.

In this section we will explore multitasking support within the Unix shell. We
will look at enabling support, at working with multiple tasks and at the
utilities that a shell has available to help you.

\subsection{Some terminology} Before we discuss the mechanics of
multitasking in the shell, let's cover some terminology. This will help us
discuss the subject clearly and you'll also know what is meant when you run
across these terms elsewhere.

First of all, when we start a program running on a system in a process of its
own, that process with that one running instance of the program is called a
\textit{job}. You'll also come across terms like process, task, instance or
similar. But the term used in Unix shells is job. Second, the ability of the
shell to influence and use multitasking (starting jobs and so on) is referred
to as \textit{job control}.

\begin{description}
\item[Job] A process that is executing an instance of a computer program.
\item[Job control] The ability to selectively stop (suspend) the execution of jobs and continue (resume) their execution at a later point.
\end{description}

Note that these terms are used this way for Unix shells. Other circumstances
and other contexts might allow for different definitions. Here are some more
terms you'll come across:

\begin{description}
\item[Job ID] An ID (usually an integer) that uniquely identifies a job. Can be used to refer to jobs for different tools and commands.
\item[Process ID (or PID)] An ID (usually an integer) that uniquely identifies a process. Can be used to refer to processes for different tools and commands. Not the same as a Job ID.
\item[Foreground job (or foreground process)] A job that has access to the terminal (i.e. can read from the keyboard and write to the monitor).
\item[Background job (or background process)] A job that does not have access to the terminal (i.e. cannot read from the keyboard or write to the monitor).
\item[Stop (or suspend)] Stop the execution of a job and return terminal control to the shell. A stopped job is not a \textit{terminated} job.
\item[Terminate] Unload a program from memory and destroy the job that was running the program.
\end{description}

\subsection{Job control in the shell: what does it mean?}
A job is a program you start within the shell. By default a new job will
suspend the shell and take control over the input and output: every stroke you
type at the keyboard will go to the job, as will every mouse movement. Nothing
but the job will be able to write to the monitor. This is what we call a
\textit{foreground} job: it's in the foreground, clearly visible to you as a
user and obscuring all other jobs in the system from view.

But sometimes that way of working is very clumsy and irritating. What if you
start a long-running job that doesn't need your input (like a backup of your
harddrive)? If this is a foreground process you have to wait until it's done
before you can do anything else. In this situation you'd much rather start the
program as a \textit{background process}: a process that is running, but that
doesn't listen to the input devices and doesn't write to the monitor. Unix
supports them and the shell (with job control) allows you to start any job as a
background job.

But what about a middle ground? Like that file downloader? You have to start
it, log into a remote server, pick your files and start the download. Only
after all that does it make sense for the job to be in the background. But how
do you accomplish that if you've already started the program as a foreground
job? Or how about this: you're busily writing a document in your favorite
editor and you just want to step out to check your mail for a moment. Do you
have to shut down the editor for that? And then, after you're done with your
mail, restart it, re-open your file and find where you'd left off? That's
inconvenient. No, a much better idea in both cases is simply to
\textit{suspend} the program: just stop it from running any further and return
to the shell. Once you're back in the shell, you can start another program
(mail) and then resume the suspended program (editor) when you're done with
that - and return to the program exactly where you left it. Conversely, you can
also decide to let the suspended process (downloader) continue running, but now
in the background.

When we talk about job control in the shell, we are talking about the abilities
described above: to start programs in the background, to suspend running
programs and to resume suspended programs, either in the foreground or in the
background.

\subsection{Enabling job control}
In order to do all the things we talked about in the previous section, you need
two things:

\begin{itemize}
\item An operating system that supports job control.
\item A shell that supports job control and has job control enabled.
\end{itemize}

Unix systems support multitasking and job control. Unix was designed from the
ground up to support multitasking. If you come across a person claiming to be a
Unix vendor but whose software doesn't support job control, call him a fraud.
Then throw his install CDs away. Then throw \textit{him} away.

Of course you've already guessed what comes next, right? I'm going to tell you
Bourne Shell supports job control. And that you can rely on the same mechanisms
to work in all compatible shells. Guess what: you're not correct. The original
Bourne Shell has no job control support; it was a single-tasking shell. There
was an extended version of the Bourne Shell though, called \textit{jsh} (guess
what the 'j' stands for...) which had job control support. To have job control
in the original Bourne Shell, you had to start this extended shell in
interactive mode like this:
\lstset{basicstyle=\scriptsize, numbers=left, captionpos=b, tabsize=4}
\begin{lstlisting}[language={bash},
xleftmargin=15pt]
jsh -i
\end{lstlisting}

Within that shell you had the job control tools we will discuss in the
following sections.

Pretty much every other shell written since incorporated job control straight
into the basic shell and the POSIX 1003 standard has standardized the job
control utilities. So you can pretty much rely on job control being available
nowadays and usually also enabled by default in interactive mode (some older
shells like Korn shell had support but required you to enable that support
specifically). But just in case, remember that you might have to do some extra
stuff on your system to use job control. There is one gotcha though: in shell
scripts, you usually include an interpreter hint that calls for a Bourne Shell
(i.e. \textit{\#!/bin/sh}). Since the original Bourne Shell doesn't have job
control, several modern shells turn off job control by default in
non-interactive mode as a compatibility feature.

\subsection{Creating a job and moving it around}
We've already talked at length about how to create a foreground job: type a
command or executable name at the prompt, hit enter, there's your job. Been
there, done that, bought the T-shirt.

We've also already ../Running\_Commands \#A\_useful \_thing \_to \_know:
\_background\_processes | mentioned how to start a background job: by adding an
ampersand at the end of the command.
\lstset{basicstyle=\scriptsize, numbers=left, captionpos=b, tabsize=4}
\begin{lstlisting}[caption=Creating a background job,language={bash},
xleftmargin=15pt,label=lst:creatingabackgroundjob]
$ ls * > /dev/null &
[1] 4808
$
\end{lstlisting}

But that suddenly looks different that when we issued commands previously;
there's a "[1]" and some number there. The "[1]" is the \textit{job ID} and the
number is the \textit{process ID}. We can use these numbers to refer to the
process and the job that we just created, which is useful for using tools that
work with jobs. When the task finishes, you will receive a notice similar to
the following:
\lstset{basicstyle=\scriptsize, numbers=left, captionpos=b, tabsize=4}
\begin{lstlisting}[caption=Job done,language={bash},
xleftmargin=15pt, label=lst:Job done]
[1]+  Done     ls * > /dev/null &
\end{lstlisting}

One of the tools that you use to manage jobs is the 'fg' command. This command
takes a background job and places it in the foreground. For instance, consider
a background job that actually takes some time to complete:
\lstset{basicstyle=\scriptsize, numbers=left, captionpos=b, tabsize=4}
\begin{lstlisting}[caption=A heftier job,language={bash},
xleftmargin=15pt, label=lst:A heftier job]
while [ $CNT -lt 200000 ];
do echo $CNT >> outp.txt;
CNT=$(expr $CNT + 1);
done &
\end{lstlisting}

We haven't gotten into ../Control flow|flow control yet, but this writes
200,000 integers to a file and takes some time. It also runs in the background.
Say that we start this job:

\lstset{basicstyle=\scriptsize, numbers=left, captionpos=b, tabsize=4}
\begin{lstlisting}[caption=Starting the job,language={bash},
xleftmargin=15pt, label=lst:Starting the job]
$ CNT=0
$ while [ \$CNT -lt 200000 ];
do echo $CNT >> outp.txt;
CNT=$(expr $CNT + 1);
done &
[1] 11246
\end{lstlisting}


The job is given job ID 1 and process ID 11246. Let's move the process to the
foreground:
\lstset{basicstyle=\scriptsize, numbers=left, captionpos=b, tabsize=4}
\begin{lstlisting}[caption=Moving the job to the front,language={bash},
xleftmargin=15pt, label=lst:Moving the job to the front]
$ fg %1
while [ \$CNT -lt 200000 ]; do
    echo \$CNT >> outp.txt; CNT=\$(expr \$CNT + 1);
done
\end{lstlisting}


The job is now running in the foreground, as you can tell from the fact that we
are not returned a prompt. Now type the \textit{CTRL+Z} keyboard combination:

\lstset{basicstyle=\scriptsize, numbers=left, captionpos=b, tabsize=4}
\begin{lstlisting}[caption=Stopping the job,language={bash},
xleftmargin=15pt, label=lst:Stopping the job]
'CTRL+Z'
[1]+  Stopped 
while [ $CNT -lt 200000 ]; do
    echo $CNT >> outp.txt; CNT=$(expr $CNT + 1);
done
$
\end{lstlisting}

Did you notice the shell reports the job as stopped? Try using the 'cat'
command to inspect the outp.txt file. Try it a couple of times; the contents
won't change. The job is not a background job; it's not running at all! The job
is suspended. Many programs recognize the \textit{CTRL+Z} combination to suspend.
And even those that don't usually have some way of suspending themselves.

\subsection{Moving to the background and stopping in the background}
Once a job is suspended, you can resume it either in the foreground or the
background. To resume in the foreground you use the 'fg' command discussed
earlier. You use 'bg' for the background:

\lstset{basicstyle=\scriptsize, numbers=left, captionpos=b, tabsize=4}
\begin{lstlisting}[language={bash},
xleftmargin=15pt]
\textbf{bg} \textit{jobId}
\end{lstlisting}

To resume our long-lasting job that writes numbers, we do the following:

\lstset{basicstyle=\scriptsize, numbers=left, captionpos=b, tabsize=4}
\begin{lstlisting}[caption=Resuming the job in the background,language={bash},
xleftmargin=15pt, label=lst:Resuming the job in the background]
$ bg %1
[1]+ while [ $CNT -lt 200000 ] do
    echo $CNT >> outp.txt; CNT=`expr $CNT + 1`;
done &
$
\end{lstlisting}


The output indicates that the job is running again. In the background this
time, since we are also returned a prompt.

Can we also stop a process in the background? Sure, we can move it to the
foreground and hit 'CTRL+Z'. But can we also do it directly? Well, there is no
utility or command to do it. Mostly, you wouldn't \textit{want} to do it - the
whole point of putting it in the background was to let it run without bothering
anybody or requiring attention. But if you really want to, you can do it like
this:
\lstset{basicstyle=\scriptsize, numbers=left, captionpos=b, tabsize=4}
\begin{lstlisting}[language={bash},
xleftmargin=15pt]
kill -SIGSTOP jobId
\end{lstlisting}

or 
\lstset{basicstyle=\scriptsize, numbers=left, captionpos=b, tabsize=4}
\begin{lstlisting}[language={bash},
xleftmargin=15pt]
kill -SIGSTOP processId
\end{lstlisting}

We'll get back to what this does exactly
../Debugging \_and \_signal \_handling \#System \_signals | later, when we talk about
signals.

\subsection{Job control tools and job status}
We mentioned before that the POSIX 1003.1 standard has standardized a number of
the job control tools that were included for job control in the jsh shell and
its successors. We've already looked at a couple of these tools; in this
section we will cover the complete list.

The standard list of job control tools consists of the following:

\begin{description}
\item[bg]Moves a job to the background.
\item[fg]Moves a job to the foreground.
\item[jobs]Lists the active jobs.
\item[kill]Terminate a job or send a signal to a process.
\item[CTRL+C]Terminate a job (same as 'kill' using the SIGTERM signal).
\item[CRTL+Z]Suspend a foreground job.
\item[wait]Wait for background jobs to terminate.
\end{description}

All of these commands can take a job specification as an argument. A job
specification starts with a percent sign and can be any of the following:

\begin{description}
\item[\%n]A job ID (n is  number).
\item[\%s]The job whose command-line started with the string s.
\item[\%?s]The jobs whose command-lines \textit{contained} the string s.
\item[\%\%]The current job (i.e. the most recent one that you managed using job control).
\item[\%+]The current job (i.e. the most recent one that you managed using job control).
\item[\%-]The previous job.
\end{description}

We've already looked at bg, fg, and CTRL+Z and we'll cover kill in a
\_and\_signal\_handling\#Trap | later section. That leaves us with jobs and
wait. Let's start with the simplest one:

\lstset{basicstyle=\scriptsize, numbers=left, captionpos=b, tabsize=4}
\begin{lstlisting}[caption=job wait,language={bash},
xleftmargin=15pt,label=lst:jobwait]
wait [job spec] ...
\end{lstlisting}

'Wait' is what you call a \textit{synchronization mechanism}: it causes the
invoking process to suspend until all background jobs terminate. Or, if you
include one or more job specifications, until the jobs you list have
terminated. You use 'wait' if you have fired off multiple jobs (simply to make
use of a system's parallel processing capabilities) and you cannot proceed
safely until they're all done.

The 'wait' command is used in quite advanced scripting. In other words, you
might not use it all that often. Here's a command that you probably will use
regularly though:

Where 
\begin{itemize}
\item -l lists the process IDs as well as normal output
\item -n limits the output to information about jobs whose status has changed since the last status report
\item -p lists only the process ID of the jobs' process group leader
\item -r limits output to data on running jobs
\item -s limits output to data on stopped jobs
\item job spec is a specification as listed above
\end{itemize}

\lstset{basicstyle=\scriptsize, numbers=left, captionpos=b, tabsize=4}
\begin{lstlisting}[caption=another wait,language={bash},
xleftmargin=15pt,label=lst:anotherwait]
$ CNT0=0
$ while [ $CNT0 -lt 200000 ];
do echo $CNT0 >> outtemp0.txt;
CNT0=`expr $CNT0 + 1`; done&
[1] 26859
$ CNT1=0
$ while [ $CNT1 -lt 200000 ];
do echo $CNT1 >> outtemp1.txt; 
CNT1=`expr $CNT1 + 1`; done&
[2] 31331
$ jobs
\end{lstlisting}

\scriptsize
\begin{verbatim}
[1]-  Running                 while [ $CNT0 -lt 200000 ]; do
    echo $CNT0 >> outtemp0.txt; CNT0=`expr $CNT0 + 1`;
done &
[2]+  Running                 while [ $CNT1 -lt 200000 ]; do
    echo $CNT1 >> outtemp1.txt; CNT1=`expr $CNT1 + 1`;
done &
\end{verbatim}
\normalsize
The 'jobs' command reports the state of active commands, including the command
line and job IDs. It also indicated the current job (with a +) and the last job
(with a -).

The \textit{jobs} command reports information and status about active jobs
(don't confuse active with running!). It is important to remember though, that
this command reports on jobs and not processes. Since a job is local to a
shell, the 'jobs' command cannot see across shells. The 'jobs' command is a
primary source of information on jobs that you can apply job control to; for
starters, you'll use this command to retrieve job IDs if you don't remember
them. For example, consider the following:

The 'jobs' command reports the state of active commands, including the command
line and job IDs. It also indicated the current job (with a +) and the last job
(with a -).

Speaking of state (which is reported by the 'jobs' command), this is a good
time to talk about the different states we have. Jobs can be in any of several
states, sometimes even in more than one state at the same time. The 'jobs'
command reports on state directly after the job id and order. We recognize the
following states:

\begin{description}
\item[Running] This is where the job is doing what it's supposed to do. You
probably don't need to interrupt it unless you really want to give the program
your personal attention (for example, to stop the program, or to find out how
far through a file download has proceeded). You'll generally find that anything
in the foreground that's not waiting for your attention is in this state,
unless it's been put to sleep.
\item[Sleeping] When programs need to retrieve input that's not yet available,
there is no need for them to continue using CPU resources. As such, they will
enter a sleep mode until another batch of input arrives. You will see more
sleeping processes, since they are not as likely to be processing data at an
exact moment of time.  
\item[Stopped] The <b>stopped</b> state indicates that the program was stopped
by the operating system.  This usually occurs when the user suspends a
background job (e.g. pressing CTRL-Z) or if it receives SIGSTOP.  At that
point, the job cannot actively consume CPU resources and aside from still being
loaded in memory, won't impact the rest of the system.  It will resume at the
point where it left off once it receives the SIGCONT signal or is otherwise
resumed from the shell. The difference between sleeping and stopped is that
"sleep" is a form of waiting until a planned event happens, whereas "stop" can
be user-initiated and indefinite.
\item[Zombie] A \textbf{zombie} process appears if the parent's program
terminated before the child could provide its return value to the parent.
These processes will get cleaned up by the \textit{init} process but sometimes
a reboot will be required to get rid of them.
\end{description}

\subsection{Other job control related tools}
In the \#Job\_control\_tools\_and\_job\_status|last section we discussed the
standard facilities that are available for job control in the Unix shell.
However, there are also a number of non-standard tools that you might come
across. And even though the focus of this book is Bourne Shell scripting
(particularly as the lingua franca of Unix shell scripting) these tools are so
common that we would be remiss if we did not at least mention them.

\subsubsection{Shell commands you might come across}
In addition to the tools previously discussed, there are two shell commands
that are quite common: 'stop' and 'suspend'.

\lstset{basicstyle=\scriptsize, numbers=left, captionpos=b, tabsize=4}
\begin{lstlisting}[language={bash},xleftmargin=15pt]
stop job ID
\end{lstlisting}

The 'stop' command is a command that occurs in the shells of many System
V-compatible Unix systems. It is used to suspend background processes - in
other words, it is the equivalent of 'CTRL+Z' for background processes. It
usually takes a job ID, like most of these commands. On systems that do not
have a 'stop' command, you should be able to stop background processes by using
the 'kill' command to send a SIGSTOP signal to the background process.

\lstset{basicstyle=\scriptsize, numbers=left, captionpos=b, tabsize=4}
\begin{lstlisting}[language={bash},
xleftmargin=15pt]
suspend job ID 
suspend [-f]
\end{lstlisting}

The other command you might come across is the the 'suspend' command. The
'suspend' command is a little tricky though, since it doesn't always mean the
same thing on all systems and all shells. There are two variations known to the
authors at this time, both of which are shown above. The first, obvious one
takes a job ID argument and suspends the indicated job; really it's just the
same as 'CTRL+Z'. 

The second variant of 'suspend' doesn't take a job ID at all, which is because
it doesn't suspend any random job. Rather, it suspends the execution of the
shell in which the command was issued. In this variant the -f argument
indicates the shell should be suspended even if it is a login shell. To resume
the shell execution, send it a SIGCONT signal using the 'kill' command.

\subsubsection{The process snapshot utility}
The last tool we will discuss is the process snapshot utility, 'ps'. This
utility is not a shell tool at all, but it occurs in some variant on pretty
much every system and you will want to use it often. Possibly more often even
than the 'jobs' tool.

The 'ps' utility is meant to report on running processes in the system.
Processes, not jobs - meaning it can see across shell instances. Here's an
example of the 'ps' utility:

\lstset{basicstyle=\scriptsize, numbers=left, captionpos=b, tabsize=4}
\begin{lstlisting}[caption=process snapshot,language={bash},
xleftmargin=15pt,label=lst:processsnapshot]
$ ps x
\end{lstlisting}

\scriptsize
\begin{verbatim}
PID    TTY     STAT  TIME    COMMAND
32094  tty5    R     3:37:21 /bin/sh
37759  tty5    S     0:00:00 /bin/ps
\end{verbatim}
\normalsize

Typical process output includes the process ID, the ID of the terminal the
process is connected to (or running on), the CPU time the process has taken and
the command issued to start the process. Possibly you also get a process state.
The process state is indicated by a letter code, but by-and-large the same
states are reported as for job reports: \textbf{R'''unning,
'\textit{S}'leeping, s'\textit{T}'opped and '''Z}ombie. Different 'ps'
implementations may use different or more codes though.

The main problem with writing about 'ps' is that it is not exactly
standardized, so there are different command-line option sets available. You'll
have to check the documentation on your system for specific details. Some
options are quite common though, so we will list them here:

\begin{description}
\item[a]List all processes except group leader processes.
\item[d]List all processes except session leaders.
\item[e]List all processes, without taking into account user id and other access limits.
\item[f]Produce a full listing as output (i.e. all reporting options).
\item[-g list Limit] output to processes whose group leader process IDs are mentioned in \textit{list}.
\item[-l]Produce a long listing.
\item[-p list] Limit output to processes whose process IDs are mentioned in \textit{list}.
\item[-s list] Limit output to processes whose session leader process IDs are mentioned in \textit{list}.
\item[-t list] Limit output to processes running on terminals mentioned in \textit{list}.
\item[-u list] Limit output to processes owned by user accounts mentioned in \textit{list}.
\end{description}

The 'ps' tool is useful for monitoring jobs across shell instances and for
discovering process IDs for signal transmission.
