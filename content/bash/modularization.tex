\section{Modularization}
If you've ever done any programming in a different environment than the shell,
you're probably familiar with the following scenario: you're writing your
program, happily typing away at the keyboard, until you notice that

\begin{itemize}
	\item you have to repeat some code you typed earlier because your program
has to perform exactly the same actions in two different locations; or
	\item your program is just too \emph{long} to understand anymore.
\end{itemize}

In other words, you've reached the point where it becomes necessary to divide
your program up into modules that can be run as separate subprograms and called
as often as you like. Working in the Bourne Shell is no different than working
in any other language in this respect. Sooner or later you're going to find
yourself writing a shell script that's just too long to be practical anymore.
And the time will have come to divide your script up into modules.

\subsection{Named functions}
Of course, the easy and obvious way to divide a script into modules is just to
create a couple of different shell scripts --- just a few separate text files
with executable permissions. But using separate files isn't always the most
practical solution either. Spreading your script over multiple files can make
it hard to maintain. Especially if you end up with shell scripts that aren't
really meaningful unless they are called specifically from one other,
particular shell script.

Especially for this situation the Bourne Shell includes the concept of a
\emph{named function}: the possibility to associate a name with a command list
and execute the command list by using the name as a command. This is what it
looks like:
\lstset{basicstyle=\scriptsize, numbers=left, captionpos=b, tabsize=4}
\begin{lstlisting}[caption=,language={bash},
breaklines=true,xleftmargin=15pt,label=lst:]
name () command group
\end{lstlisting}

This functionality is available throughout the shell and is useful in several
situations. First of all, you can use it to break a long shell script up into
multiple modules. But second, you can use it to define your own little macros
in your own environment that you don't want to create a full script for. Many
modern shells include a built-in command for this called \emph{alias}, but
old-fashioned shells like the original Bourne Shell did not; you can use named
functions to accomplish the same result.

\subsection{Creating a named function}
\subsubsection{Functions with a simple command group}
Let's start off simply by creating a function that prints ``Hello World!!''.
And let's call it ``hw''. This is what it looks like:

\lstset{basicstyle=\scriptsize, numbers=left, captionpos=b, tabsize=4}
\begin{lstlisting}[caption=Hello world as a named function,language={bash},
breaklines=true,xleftmargin=15pt,label=lst:Hello world as a named function]
hw() {
>  echo 'Hello World!!';
>}
\end{lstlisting}

We can use exactly the same code in a shell script or in the interactive shell
--- the example above is from the interactive shell. There are several things
to notice about this example. First of all, we didn't need a separate keyword
to define a function, just the parentheses did it. To the shell, function
definitions are like extended variable definitions. They're part of the
environment; you set them just by defining a name and a meaning.

The second thing to note is that, once you're past the parentheses, all the
normal rules hold for the command group. In our case we used a command group
with braces, so we needed the semicolon after the echo command. The string we
want to print contains exclamation points, so we have to quote it (as usual).
And we were allowed to break the command group across multiple lines, even in
interactive mode, just like normal.

Here's how you use the new function:
\lstset{basicstyle=\scriptsize, numbers=left, captionpos=b, tabsize=4}
\begin{lstlisting}[caption=Calling our function,language={bash},
breaklines=true,xleftmargin=15pt,label=lst:Calling our function]
$ hw
\end{lstlisting}

\scriptsize
\begin{verbatim}
Hello World!!
\end{verbatim}
\normalsize

\subsubsection{Functions that execute in a separate process}
The definition of a function takes a command group. \emph{Any} command group.
Including the command group with parentheses rather than braces. So if we want,
we can define a function that runs as a subprocess in its own environment as
well. Here's hello world again, in a subprocess:

\lstset{basicstyle=\scriptsize, numbers=left, captionpos=b, tabsize=4}
\begin{lstlisting}[caption=Hello world as a named function,language={bash},
breaklines=true,xleftmargin=15pt,label=lst:Hello world as a named function]
hw() ( echo 'Hello World!!' )
\end{lstlisting}

It's all on one line this time to keep it short, but the same rules apply as
before. And of course the same environment rules apply as well, so any
variables defined in the function will not be available anymore once the
function ends.

\subsection{Functions with parameters}
If you've done any programming in a different programming language you know
that the most useful functions are those that take parameters. In other words,
ones that don't always rigidly do the same thing but can be influenced by
values passed in when the function is called. So here's an interesting
question: can we pass parameters to a function? Can we create a definition like
\lstset{basicstyle=\scriptsize, numbers=left, captionpos=b, tabsize=4}
\begin{lstlisting}[caption=,language={bash},
breaklines=true,xleftmargin=15pt,label=lst:]
functionWithParams (ARG0, ARG1) { do something with ARG0 and ARG1 } 
\end{lstlisting}

And then make a call like 'functionWithParams(Hello, World)'? Well, the answer
is simple: no. The parenthese are just there as a flag for the shell to let it
know that the previous name is the name of a function rather than a variable
and there is no room for parameters.

Or actually, it's more a case of the above being the simple answer rather than
the answer being simple. You see, when you execute a function you are executing
a command. To the shell there's really very little difference between executing
a named function and executing 'ls'. It's a command like any other. And it may
not be able to have parameters, but like any other command it can certainly
have \emph{command line arguments}. So we may not be able to define a function
with parameters like above, but we can certainly do this:
\lstset{basicstyle=\scriptsize, numbers=left, captionpos=b, tabsize=4}
\begin{lstlisting}[caption=Functions with command-line arguments,language={bash},
breaklines=true,xleftmargin=15pt,label=lst:Functions with command-line arguments]
$ repeatOne () { echo $1; }
$ repeatOne 'Hello World!'
\end{lstlisting}

\scriptsize
\begin{verbatim}
Hello World!
\end{verbatim}
\normalsize

And you can use any other variable from the environment as well. Of course,
that's a nice trick for when you're calling a function from the command line in
the interactive shell. But what about in a shell script? The positional
variables for command-line arguments are already taken by the arguments to the
shell script, right? Ah, but wait! Each command executed in the shell (no
matter how it was executed) has its \textbf{own} set of command-line arguments!
So there's no interference and you can use the same mechanism. For example, if
we define a script like this:


\lstset{basicstyle=\scriptsize, numbers=left, captionpos=b, tabsize=4}
\begin{lstlisting}[caption=function.sh: A function in a shell script,language={bash},
breaklines=true,xleftmargin=15pt,label=lst:function.sh: A function in a shell script]
#!/bin/sh

myFunction() {
  echo $1
}

echo $1
myFunction
myFunction "Hello World"
echo $1
\end{lstlisting}

Then it executes exactly the way we want:

\lstset{basicstyle=\scriptsize, numbers=left, captionpos=b, tabsize=4}
\begin{lstlisting}[caption=Executing the function.sh script,language={bash},
breaklines=true,xleftmargin=15pt,label=lst:Executing the function.sh script]
$ . function.sh 'Goodbye World!!'
\end{lstlisting}

\scriptsize
\begin{verbatim}
Goodbye World!

Hello World
Goodbye World!
\end{verbatim}
\normalsize

\subsubsection{Functions in the environment}
We've mentioned it before, but let's delve a little deeper into it now: what
are functions exactly? We've hinted that they're an alias for a command list or
a macro and that they're part of the environment. But what \emph{is} a function
exactly?

A function, as far as the shell is concerned, is just a very verbose variable
definition. And that's really all it is: a name (a text string) that is
associated with a value (some more text) and can be replaced by that value when
the name is used. Just like a shell variable. And we can prove it too: just
define a function in the interactive shell, then give the 'set' command (to
list all the variable definitions in your current environment). Your function
will be in the list.

Because functions are really a special kind of shell variable definition, they
behave exactly the same way normal variables do:

\begin{itemize}
	\item Functions are defined by listing their name, a definition operator
and then the value of the function. Functions use a different definition
operator though: '()' instead of '='. This tells the shell to add some special
considerations to the function (like not needing the '\$' character when using
the function).
	\item Functions are part of the environment. That means that when commands
are issued from the shell, functions are also copied into the copy of the
environment that is given to the issued command.
	\item Functions can also be passed to new subprocesses if they are marked
for export, using the 'export' command. Some shells will require a special
command-line argument to 'export' for functions (bash, for instance, requires
you to do an 'export -f' to export functions).
	\item You can drop function definitions by using the 'unset' command.
\end{itemize}

Of course, when you use them functions behave just like commands (they are
expanded into a command list, after all). We've already seen that you can use
command-line arguments with functions and the positional variables to match.
But you can also redirect input and output to and from commands and pipe
commands together as well.
