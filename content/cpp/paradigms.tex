\section{Programming paradigms}
A \texttt{programming paradigm} is a model of programming based on distinct
concepts that shapes the way programmers design, organize and write programs.
A multi-paradigm programming language allows programmers to choose a specific
single approach or mix parts of different programming paradigms. C++ as a
multi-paradigm programming language supports single or mixed approaches using
Procedural or Object-oriented programming and mixing in utilizations of Generic
and even Functional programming concepts.

\subsection{Procedural programming}
\texttt{Procedural programming} can be defined as a subtype of 
imperative programming as a programming paradigm based upon the
concept of procedure calls, in which  \texttt{ \textit{statements}} are
structured into procedures (also known as subroutines or functions). Procedure
calls are modular and are bound by scope. A procedural program is composed of
one or more modules. Each module is composed of one or more subprograms.
Modules may consist of procedures, functions, subroutines or methods, depending
on the programming language. Procedural programs may possibly have multiple
levels or scopes, with subprograms defined inside other subprograms. Each scope
can contain names which cannot be seen in outer scopes.

Procedural programming offers many benefits over simple sequential programming since procedural code:
\begin{itemize}
\item is easier to read and more maintainable
\item is more flexible
\item facilitates the practice of good program design
\item allows modules to be reused in the form of  \texttt{ \textit{code libraries}}.
\end{itemize}

 \footnote{Nowadays it is very rare to see C++ strictly using the Procedural
 Programming paradigm, mostly it is used only on small demonstration or test
 programs.}

\subsection{Statically typed}
\texttt{Typing} refers to how a computer language handles its variables, how
they are differentiated by \texttt{ \textit{type}}. Variables are values that the
program uses during execution. These values can change; they are variable,
hence their name.  \textit{Static typing} usually results in compiled code that
executes more quickly. When the compiler knows the exact types that are in use,
it can produce machine code that does the right thing easier. In C++, variables
need to be defined before they are used so that compilers know what type they
are, and hence is statically typed.  Languages that are not statically typed
are called  \textit{dynamically typed}.

Static typing usually finds type errors more reliably at compile time,
increasing the reliability of compiled programs. Simply put, it means that "A
round peg won't fit in a square hole", so the compiler will report it when a
type leads to ambiguity or incompatible usage. However, programmers disagree
over how common type errors are and what proportion of bugs that are written
would be caught by static typing. Static typing advocates believe programs are
more reliable when they have been type checked, while dynamic typing advocates
point to dynamic code that has proved reliable and to small bug databases. The
value of static typing, then, presumably increases as the strength of the type
system is increased.

A statically typed system constrains the use of powerful language constructs
more than it constrains less powerful ones. This makes powerful constructs
harder to use, and thus places the burden of choosing the "right tool for the
problem" on the shoulders of the programmer, who might otherwise be inclined to
use the most powerful tool available. Choosing overly powerful tools may cause
additional performance, reliability or correctness problems, because there are
theoretical limits on the properties that can
be expected from powerful language constructs. For example, indiscriminate use
of recursion or global variables may cause
well-documented adverse effects.

Static typing allows construction of libraries which are less likely to be
accidentally misused by their users. This can be used as an additional
mechanism for communicating the intentions of the library developer.

\subsection{Type checking}
\texttt{Type checking} is the process of verifying and enforcing the
constraints of types, which can occur at either compile-time or run-time.
Compile time checking, also called \textit{static type} checking, is carried out
by the compiler when a program is compiled. Run time checking, also
called \textit{dynamic type checking}, is carried out by the program as it is
running. A programming language is said to be \textit{strongly typed} if the
type system ensures that conversions between types must be either valid or
result in an error. A \textit{weakly typed} language on the other hand makes no
such guarantees and generally allows automatic conversions between types which
may have no useful purpose. C++ falls somewhere in the middle, allowing a mix
of automatic type conversion and programmer defined conversions, allowing for
almost complete flexibility in interpreting one type as being of another type.
Converting variables or expression of one type into another type is
called \texttt{ \textit{type casting}}.

\subsection{Object-oriented programming}
\texttt{Object-oriented programming} can be seen
as an extension of procedural programming in which programs are made up of
collection of individual units called \texttt{objects} that have a distinct
purpose and function with limited or no dependencies on
implementation. For example, a car is like an object; it gets
you from point A to point B with no need to know what type of engine the car
uses or how the engine works. Object-oriented languages usually provide a means
of documenting what an object can and cannot do, like
instructions for driving a car.

\subsubsection{Objects and Classes}
An \texttt{object} is composed of \texttt{members} and \texttt{methods}. The
members (also called \textit{data members}, \textit{characteristics},
\textit{attributes}, or \textit{properties}) describe the object. The methods
generally describe the actions associated with a particular object. Think of an
object as a noun, its members as adjectives describing that noun, and its
methods as the verbs that can be performed by or on that noun.

For example, a sports car is an object. Some of its members might be its
height, weight, acceleration, and speed. An object's members just hold data
about that object. Some of the methods of the sports car could be "drive",
"park", "race", etc. The methods really do not mean much unless associated with
the sports car, and the same goes for the members.

The blueprint that lets us build our sports car object is called a
\texttt{class}. A class does not tell us how fast our sports car goes, or what
color it is, but it does tell us that our sports car will have a member
representing speed and color, and that they will be say, a number and a word,
respectively. The class also lays out the methods for us, telling the car how
to park and drive, but these methods can not take any action with just the
blueprint - they need an object to have an effect.

\paragraph{Encapsulation}
Encapsulation, the principle of information hiding (from
the user), is the process of hiding the data structures of the class and
allowing changes in the data through a public interface where the incoming
values are checked for validity, and so not only it permits the hiding of data
in an object but also of behavior. This prevents clients of an interface from
depending on those parts of the implementation that are likely to change in
future, thereby allowing those changes to be made more easily, that is, without
changes to clients. In modern programming languages, the principle of
information hiding manifests itself in a number of ways, including
encapsulation and polymorphism.

\subsection{Generic programming}
\texttt{Generic programming} or \texttt{polymorphism} is
a programming style that emphasizes techniques that allow one value to take on
different types as long as certain contracts such as subtypes and
signature are kept. In simpler terms generic
programming is based in finding the most abstract representations of efficient
algorithms. Templates popularized the notion of
generics. Templates allow code to be written without consideration of the
type with which it will eventually be used. Templates are defined in
the Standard Template Library (STL), where generic
programming was introduced into C++.

\subsection{Free-form}
\texttt{Free-form} refers to how the programmer crafts the code. Basically,
there are no rules on how you choose to write your program, save for the
semantic rules of C++. Any C++ program should compile as long as it is legal
C++.

The free-form nature of C++ is used (or abused, depending on your point of
view) by some programmers in crafting obfuscated C++ (C++ that is purposefully
written to be difficult to understand). The use of obfuscation is regarded by
some as a security mechanism, ensuring that the source code is more difficult
to analyze by the average user or to prevent the functionality from being
duplicated.
