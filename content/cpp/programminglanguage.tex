\section{What is a programming language?}
In the most basic terms, a "w:Programming language|programming language" is a
means of communication between a human being (programmer) and a computer. A
programmer uses this means of communication in order to give the computer
instructions.  These instructions are called "programs".

Like the many languages we use to communicate with each other, there are many
languages that a programmer can use to communicate with a computer.  Each
language has its own set of words and rules, called semantics.  If you're going
to write a program, you have to follow the semantics of the language you're
writing in, or you won't be understood.

Programming languages can basically be divided in to two categories: Low-level
programming language and High-level programming language, next we will
introduce you to these concepts and their relevance to C++.

\subsection{Low-level}
The lower level in computer "languages" are: 

\textit{Machine code} (also called binary) is the lowest form of a low-level
language. Machine code consists of a string of 0s and 1s, which combine to form
meaningful instructions that computers can take action on. If you look at a
page of binary it becomes apparent why binary is never a practical choice for
writing programs; what kind of person would actually be able to remember what a
bunch of strings of 1 and 0 mean? 

\textit{Assembly language} (also called ASM), is just above machine code on
the scale from low level to high level. It is a human-readable translation of
the machine language instructions the computer executes.  For example, instead
of referring to processor instructions by their binary representation (0s and
1s), the programmer refers to those instructions using a more memorable
(mnemonic) form. These mnemonics are usually short collections of letters that
symbolize the action of the respective instruction, such as "ADD" for addition,
and "MOV" for moving values from one place to another.

You do not have to understand assembly language to program in C++, but it does
help to have an idea of what's going on "behind-the-scenes". Learning about
assembly language will also allow you to have more control as a programmer and
help you in debugging and understanding code.

The advantages of writing in a high-level language format far outweigh any
drawbacks, due to the size and complexity of most programming tasks, those
advantages include:

\begin{itemize}
\item Advanced program structure: loops, functions, and objects all have
limited usability in low-level languages, as their existence is already
considered a "high" level feature; that is, each structure element must be
further translated into low-level language.
\item Portability: high-level programs can run on different kinds of computers
with few or no modifications. Low-level programs often use specialized
functions available on only certain processors, and have to be rewritten to run
on another computer.
\item Ease of use: many tasks that would take many lines of code in assembly
can be simplified to several function calls from libraries in high-level
programming languages. For example, Java, a high-level programming language, is
capable of painting a functional window with about five lines of code, while
the equivalent assembly language would take at least four times that amount.
\end{itemize}

\subsection{High-level}
High-level languages do more with less code, although there is sometimes a loss
in performance and less freedom for the programmer. They also attempt to use
English language words in a form which can be read and generally interpreted by
the average person with little experience in them. A program written in one of
these languages is sometimes referred to as "human-readable code".  In general,
more abstraction makes it easier for a language be learned. 

No programming language is written in what one might call "plain English"
though, (although BASIC comes close). Because of this, the text of a program is
sometimes referred to as "code", or more specifically as "source code."  This
is discussed in more detail in the C++ Programming/Programming
Languages/C++/Code|The Code Section of the book.

Higher-level languages partially solve the problem of abstraction to the
hardware (CPU, co-processors, number of registers etc...) by providing
portability of code.

Keep in mind that this classification scheme is evolving. C++ is still
considered a high-level language, but with the appearance of newer languages
(Java, C\#, Ruby etc...), C++ is beginning to be grouped with lower level
languages like C.

\subsection{Translating programming languages}
Since a computer is only capable of understanding machine code, human-readable
code must be either interpreted or translated into machine code.

An \textit{Interpreter} is a program (often written in a lower level language)
that interprets the instructions of a program one instruction at a time into
commands that are to be carried out by the interpreter as it happens. Typically
each instruction consists of one line of text or provides some other clear
means of telling each instruction apart and the program must be reinterpreted
again each time the program is run.

A \textit{Compiler} is a program used to translate the source code, one
instruction at a time, into machine code. The translation into machine code may
involve splitting one instruction understood by the compiler into multiple
machine instructions. The instructions are only translated once and after that
the machine can understand and follow the instructions directly whenever it is
instructed to do so. A complete examination of the C++ compiler is given in the
Compiler Section of the book.

The words and statements used to instruct the computer may differ, but no
matter what words and statements are used, just about every programming
language will include statements that will accomplish the following:

\textit{Input}: Input is the act of getting information from a device such as a
keyboard or mouse, or sometimes another program.

\textit{Output}: Output is the opposite of input; it gives information to the
computer monitor or another device or program. 

\textit{Math}\textit{Algorithm}: All computer processors (the brain of the
computer), have the ability to perform basic mathematical computation, and
every programming language has some way of telling it to do so. 

\textit{Testing}: Testing involves telling the computer to check for a certain
condition and to do something when that condition is true or false.
Conditionals are one of the most important concepts in programming, and all
languages have some method of testing conditions. 

\textit{Repetition}: Perform some action repeatedly, usually with some variation.

An further examination is provided on the C++ Programming/Programming
Statements Section of the book. 

Believe it or not, that's pretty much all there is to it. Every program you've
ever used, no matter how complicated, is made up of functions that look more or
less like these. Thus, one way to describe programming is the process of
breaking a large, complex task up into smaller and smaller subtasks until
eventually the subtasks are simple enough to be performed with one of these
simple functions.

C++ is mostly \textit{compiled} rather than \textit{interpreted} (there are some
C++ interpreters), and then "executed" later. As complicated as this may seem,
later you will see how easy it really is.

So as we have seen in the C++ Programming/Programming Languages/C++|Introducing
C++ Section, C++ evolved from C by adding some levels of abstraction (so we can
correctly state that C++ is of a higher level than C). We will learn the
particulars of those differences in the C++ Programming/Programming
Languages/Paradigms|Programming Paradigms Section of the book and for some of
you that already know some other languages should look into C++
Programming Languages Comparisons Section. 
