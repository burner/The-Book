\section{Git Hooks}
Hooks are little scripts you can place in \$GIT\_DIR/hooks directory to trigger
action at certain points. When git-init is run, a handful example hooks are
copied in the hooks directory of the new repository, but by default they are
all disabled. To enable a hook, rename it by removing its .sample suffix.

\subsection{applypatch-msg}
\lstset{basicstyle=\scriptsize, numbers=none, captionpos=b, tabsize=4}
\begin{lstlisting}[caption=,language={bash},
breaklines=true,label=lst:]
GIT_DIR/hooks/applypatch-msg
\end{lstlisting}

This hook is invoked by git-am script. It takes a single parameter, the name of
the file that holds the proposed commit log message. Exiting with non-zero
status causes git-am to abort before applying the patch.

The hook is allowed to edit the message file in place, and can be used to
normalize the message into some project standard format (if the project has
one). It can also be used to refuse the commit after inspecting the message
file. The default applypatch-msg hook, when enabled, runs the commit-msg hook,
if the latter is enabled.

\subsection{pre-applypatch}
\lstset{basicstyle=\scriptsize, numbers=none, captionpos=b, tabsize=4}
\begin{lstlisting}[caption=,language={bash},
breaklines=true,label=lst:]
GIT_DIR/hooks/pre-applypatch
\end{lstlisting}

This hook is invoked by git-am. It takes no parameter, and is invoked after the
patch is applied, but before a commit is made. If it exits with non-zero
status, then the working tree will not be committed after applying the patch.

It can be used to inspect the current working tree and refuse to make a commit
if it does not pass certain test. The default pre-applypatch hook, when
enabled, runs the pre-commit hook, if the latter is enabled.

\subsection{post-applypatch}
\lstset{basicstyle=\scriptsize, numbers=none, captionpos=b, tabsize=4}
\begin{lstlisting}[caption=,language={bash},
breaklines=true,label=lst:]
GIT_DIR/hooks/post-applypatch
\end{lstlisting}

This hook is invoked by 'git-am'. It takes no parameter, and is invoked after
the patch is applied and a commit is made.

This hook is meant primarily for notification, and cannot affect the outcome of
'git-am'.

\subsection{pre-commit}
\lstset{basicstyle=\scriptsize, numbers=none, captionpos=b, tabsize=4}
\begin{lstlisting}[caption=,language={bash},
breaklines=true,label=lst:]
GIT_DIR/hooks/pre-commit
\end{lstlisting}

This hook is invoked by 'git-commit', and can be bypassed with \--no-verify
option. It takes no parameter, and is invoked before obtaining the proposed
commit log message and making a commit. Exiting with non-zero status from this
script causes the 'git-commit' to abort.

The default 'pre-commit' hook, when enabled, catches introduction of lines with
trailing whitespaces and aborts the commit when such a line is found.

All the 'git-commit' hooks are invoked with the environment variable
GIT\_EDITOR=: if the command will not bring up an editor to modify the commit
message.

Here is an example of a Ruby script that runs RSpec tests before allowing a
commit.
\lstset{basicstyle=\scriptsize, numbers=none, captionpos=b, tabsize=4}
\begin{lstlisting}[caption=,language={ruby},
breaklines=true,label=lst:]
  
html_path = "spec_results.html"  
`spec -f h:#{html_path} -f p spec` # run the spec. send progress to screen. save html results to html_path  

# find out how many errors were found  
html = open(html_path).read  
examples = html.match(/(\d+) examples/)[0].to_i rescue 0  
failures = html.match(/(\d+) failures/)[0].to_i rescue 0  
pending = html.match(/(\d+) pending/)[0].to_i rescue 0  

if failures.zero?  
  puts "0 failures! #{examples} run, #{pending} pending"  
else  
  puts "\aDID NOT COMMIT YOUR FILES!"  
  puts "View spec results at #{File.expand_path(html_path)}"  
  puts  
  puts "#{failures} failures! #{examples} run, #{pending} pending"  
  exit 1  
end
\end{lstlisting}

\subsection{prepare-commit-msg}
\lstset{basicstyle=\scriptsize, numbers=none, captionpos=b, tabsize=4}
\begin{lstlisting}[caption=,language={bash},
breaklines=true,label=lst:]
GIT_DIR/hooks/prepare-commit-msg
\end{lstlisting}

This hook is invoked by 'git-commit' right after preparing the default log
message, and before the editor is started.

It takes one to three parameters. The first is the name of the file that the
commit log message. The second is the source of the commit message, and can be:
message (if a -m or -F option was given); template (if a -t option was given or
the configuration option commit.template is set); merge (if the commit is a
merge or a .git/MERGE\_MSG file exists); squash (if a .git/SQUASH\_MSG file
exists); or commit, followed by a commit SHA1 (if a -c, -C or \--amend option
was given).

If the exit status is non-zero, 'git-commit' will abort.

The purpose of the hook is to edit the message file in place, and it is not
suppressed by the \--no-verify option. A non-zero exit means a failure of the
hook and aborts the commit. It should not be used as replacement for pre-commit
hook.

The sample prepare-commit-msg hook that comes with git comments out the
Conflicts: part of a merge's commit message.

\subsection{commit-msg}
\lstset{basicstyle=\scriptsize, numbers=none, captionpos=b, tabsize=4}
\begin{lstlisting}[caption=,language={bash},
breaklines=true,label=lst:]
GIT_DIR/hooks/commit-msg
\end{lstlisting}

This hook is invoked by 'git-commit', and can be bypassed with \--no-verify
option. It takes a single parameter, the name of the file that holds the
proposed commit log message. Exiting with non-zero status causes the
'git-commit' to abort.

The hook is allowed to edit the message file in place, and can be used to
normalize the message into some project standard format (if the project has
one). It can also be used to refuse the commit after inspecting the message
file.

The default 'commit-msg' hook, when enabled, detects duplicate "Signed-off-by"
lines, and aborts the commit if one is found.

\subsection{post-commit}
\lstset{basicstyle=\scriptsize, numbers=none, captionpos=b, tabsize=4}
\begin{lstlisting}[caption=,language={bash},
breaklines=true,label=lst:]
GIT_DIR/hooks/post-commit
\end{lstlisting}

This hook is invoked by 'git-commit'. It takes no parameter, and is invoked
after a commit is made.

This hook is meant primarily for notification, and cannot affect the outcome of
'git-commit'.

\subsection{pre-rebase}
\lstset{basicstyle=\scriptsize, numbers=none, captionpos=b, tabsize=4}
\begin{lstlisting}[caption=,language={bash},
breaklines=true,label=lst:]
GIT_DIR/hooks/pre-rebase
\end{lstlisting}

This hook is called by 'git-rebase' and can be used to prevent a branch from
getting rebased.

\subsection{post-checkout}
\lstset{basicstyle=\scriptsize, numbers=none, captionpos=b, tabsize=4}
\begin{lstlisting}[caption=,language={bash},
breaklines=true,label=lst:]
GIT_DIR/hooks/post-checkout
\end{lstlisting}

This hook is invoked when a 'git-checkout' is run after having updated the
worktree. The hook is given three parameters: the ref of the previous HEAD, the
ref of the new HEAD (which may or may not have changed), and a flag indicating
whether the checkout was a branch checkout (changing branches, flag=1) or a
file checkout (retrieving a file from the index, flag=0). This hook cannot
affect the outcome of 'git-checkout'.

This hook can be used to perform repository validity checks, auto-display
differences from the previous HEAD if different, or set working dir metadata
properties.

\subsection{post-merge}
\lstset{basicstyle=\scriptsize, numbers=none, captionpos=b, tabsize=4}
\begin{lstlisting}[caption=,language={bash},
breaklines=true,label=lst:]
GIT_DIR/hooks/post-merge
\end{lstlisting}

This hook is invoked by 'git-merge', which happens when a 'git-pull' is done on
a local repository. The hook takes a single parameter, a status flag specifying
whether or not the merge being done was a squash merge. This hook cannot affect
the outcome of 'git-merge' and is not executed, if the merge failed due to
conflicts.

This hook can be used in conjunction with a corresponding pre-commit hook to
save and restore any form of metadata associated with the working tree (eg:
permissions/ownership, ACLS, etc). See contrib/hooks/setgitperms.perl for an
example of how to do this.

\subsection{pre-receive}
\lstset{basicstyle=\scriptsize, numbers=none, captionpos=b, tabsize=4}
\begin{lstlisting}[caption=,language={bash},
breaklines=true,label=lst:]
GIT_DIR/hooks/pre-receive
\end{lstlisting}

This hook is invoked by 'git-receive-pack' on the remote repository, which
happens when a 'git-push' is done on a local repository. Just before starting
to update refs on the remote repository, the pre-receive hook is invoked. Its
exit status determines the success or failure of the update.

This hook executes once for the receive operation. It takes no arguments, but
for each ref to be updated it receives on standard input a line of the format:

<old-value> SP <new-value> SP <ref-name> LF

where <old-value> is the old object name stored in the ref, <new-value> is the
new object name to be stored in the ref and <ref-name> is the full name of the
ref. When creating a new ref, <old-value> is 40 0.

If the hook exits with non-zero status, none of the refs will be updated. If
the hook exits with zero, updating of individual refs can still be prevented by
the <<update,'update'>> hook.

Both standard output and standard error output are forwarded to 'git-send-pack'
on the other end, so you can simply echo messages for the user.
If you wrote it in Ruby, you might get the args this way:
\lstset{basicstyle=\scriptsize, numbers=none, captionpos=b, tabsize=4}
\begin{lstlisting}[caption=,language={ruby},
breaklines=true,label=lst:]
rev_old, rev_new, ref = STDIN.read.split(" ")
Or in a bash script, something like this would work:

#!/bin/sh
# <oldrev> <newrev> <refname>
# update a blame tree
while read oldrev newrev ref
do
    echo "STARTING [$oldrev $newrev $ref]"
    for path in `git diff-tree -r $oldrev..$newrev | awk '{print $6}'`
    do
      echo "git update-ref refs/blametree/$ref/$path $newrev"
      `git update-ref refs/blametree/$ref/$path $newrev`
    done
done
\end{lstlisting}

\subsection{update}
\lstset{basicstyle=\scriptsize, numbers=none, captionpos=b, tabsize=4}
\begin{lstlisting}[caption=,language={bash},
breaklines=true,label=lst:]
GIT_DIR/hooks/update
\end{lstlisting}

This hook is invoked by 'git-receive-pack' on the remote repository, which
happens when a 'git-push' is done on a local repository. Just before updating
the ref on the remote repository, the update hook is invoked. Its exit status
determines the success or failure of the ref update.

The hook executes once for each ref to be updated, and takes three parameters:

the name of the ref being updated, the old object name stored in the ref, and
the new objectname to be stored in the ref.  A zero exit from the update hook
allows the ref to be updated. Exiting with a non-zero status prevents
'git-receive-pack' from updating that ref.

This hook can be used to prevent 'forced' update on certain refs by making sure
that the object name is a commit object that is a descendant of the commit
object named by the old object name. That is, to enforce a "fast forward only"
policy.

It could also be used to log the old..new status. However, it does not know the
entire set of branches, so it would end up firing one e-mail per ref when used
naively, though. The <<post-receive,'post-receive'>> hook is more suited to
that.

Another use suggested on the mailing list is to use this hook to implement
access control which is finer grained than the one based on filesystem group.

Both standard output and standard error output are forwarded to 'git-send-pack'
on the other end, so you can simply echo messages for the user.

The default 'update' hook, when enabled--and with hooks.allowunannotated config
option turned on--prevents unannotated tags to be pushed.

\subsection{post-receive}
\lstset{basicstyle=\scriptsize, numbers=none, captionpos=b, tabsize=4}
\begin{lstlisting}[caption=,language={bash},
breaklines=true,label=lst:]
GIT_DIR/hooks/post-receive
\end{lstlisting}

This hook is invoked by 'git-receive-pack' on the remote repository, which
happens when a 'git-push' is done on a local repository. It executes on the
remote repository once after all the refs have been updated.

This hook executes once for the receive operation. It takes no arguments, but
gets the same information as the <<pre-receive,'pre-receive'>> hook does on its
standard input.

This hook does not affect the outcome of 'git-receive-pack', as it is called
after the real work is done.

This supersedes the <<post-update,'post-update'>> hook in that it gets both old
and new values of all the refs in addition to their names.

Both standard output and standard error output are forwarded to 'git-send-pack'
on the other end, so you can simply echo messages for the user.

The default 'post-receive' hook is empty, but there is a sample script
post-receive-email provided in the contrib/hooks directory in git distribution,
which implements sending commit emails.

\subsection{post-update}
\lstset{basicstyle=\scriptsize, numbers=none, captionpos=b, tabsize=4}
\begin{lstlisting}[caption=,language={bash},
breaklines=true,label=lst:]
GIT_DIR/hooks/post-update
\end{lstlisting}

This hook is invoked by 'git-receive-pack' on the remote repository, which
happens when a 'git-push' is done on a local repository. It executes on the
remote repository once after all the refs have been updated.

It takes a variable number of parameters, each of which is the name of ref that
was actually updated.

This hook is meant primarily for notification, and cannot affect the outcome of
'git-receive-pack'.

The 'post-update' hook can tell what are the heads that were pushed, but it
does not know what their original and updated values are, so it is a poor place
to do log old..new. The <<post-receive,'post-receive'>> hook does get both
original and updated values of the refs. You might consider it instead if you
need them.

When enabled, the default 'post-update' hook runs 'git-update-server-info' to
keep the information used by dumb transports (e.g., HTTP) up-to-date. If you
are publishing a git repository that is accessible via HTTP, you should
probably enable this hook.

Both standard output and standard error output are forwarded to 'git-send-pack'
on the other end, so you can simply echo messages for the user.

\subsection{pre-auto-gc}
\lstset{basicstyle=\scriptsize, numbers=none, captionpos=b, tabsize=4}
\begin{lstlisting}[caption=,language={bash},
breaklines=true,label=lst:]
GIT_DIR/hooks/pre-auto-gc
\end{lstlisting}

This hook is invoked by 'git-gc --auto'. It takes no parameter, and exiting
with non-zero status from this script causes the 'git-gc --auto' to abort.
