\subsection{Browsing Git Objects}
We can ask git about particular objects with the cat-file command. Note that
you can shorten the shas to only a few characters to save yourself typing all
40 hex digits:
\lstset{basicstyle=\scriptsize, numbers=none, captionpos=b, tabsize=4}
\begin{lstlisting}[caption=,language={bash},
breaklines=true,label=lst:]
$ git-cat-file -t 54196cc2
commit
$ git-cat-file commit 54196cc2
tree 92b8b694ffb1675e5975148e11218100
author J. Bruce Fields <bfields@puzzle.fieldses.org> 1143414668 -0500
committer J. Bruce Fields <bfields@puzzle.fieldses.org> 1143414668 -0500

initial commit
\end{lstlisting}

A tree can refer to one or more "blob" objects, each corresponding to a file.
In addition, a tree can also refer to other tree objects, thus creating a
directory hierarchy. You can examine the contents of any tree using ls-tree
(remember that a long enough initial portion of the SHA1 will also work):
\lstset{basicstyle=\scriptsize, numbers=none, captionpos=b, tabsize=4}
\begin{lstlisting}[caption=,language={bash},
breaklines=true,label=lst:]
$ git ls-tree 92b8b694
100644 blob 3b18e512dba79e4c8300dd08aeb37f8e7    file.txt
\end{lstlisting}

Thus we see that this tree has one file in it. The SHA1 hash is a reference to
that file's data:
\lstset{basicstyle=\scriptsize, numbers=none, captionpos=b, tabsize=4}
\begin{lstlisting}[caption=,language={bash},
breaklines=true,label=lst:]
$ git cat-file -t 3b18e512
blob
\end{lstlisting}

A "blob" is just file data, which we can also examine with cat-file:
\lstset{basicstyle=\scriptsize, numbers=none, captionpos=b, tabsize=4}
\begin{lstlisting}[caption=,language={bash},
breaklines=true,label=lst:]
$ git cat-file blob 3b18e512
hello world
\end{lstlisting}

Note that this is the old file data; so the object that git named in its
response to the initial tree was a tree with a snapshot of the directory state
that was recorded by the first commit.

All of these objects are stored under their SHA1 names inside the git
directory:
\lstset{basicstyle=\scriptsize, numbers=none, captionpos=b, tabsize=4}
\begin{lstlisting}[caption=,language={bash},
breaklines=true,label=lst:]
$ find .git/objects/
.git/objects/
.git/objects/pack
.git/objects/info
.git/objects/3b
.git/objects/3b/18e512dba79e4c8300dd08aeb37f8
.git/objects/92
.git/objects/92/b8b694ffb1675e5975148e1121810
.git/objects/54
.git/objects/54/196cc2703dc165cbd373a65a4dcf2
.git/objects/a0
.git/objects/a0/423896973644771497bdc03eb99d5
.git/objects/d0
.git/objects/d0/492b368b66bdabf2ac1fd8c92b39d
.git/objects/c4
.git/objects/c4/d59f390b9cfd4318117afde11d601
\end{lstlisting}

and the contents of these files is just the compressed data plus a header
identifying their length and their type. The type is either a blob, a tree, a
commit, or a tag.

The simplest commit to find is the HEAD commit, which we can find from
.git/HEAD:
\lstset{basicstyle=\scriptsize, numbers=none, captionpos=b, tabsize=4}
\begin{lstlisting}[caption=,language={bash},
breaklines=true,label=lst:]
$ cat .git/HEAD
ref: refs/heads/master
\end{lstlisting}

As you can see, this tells us which branch we're currently on, and it tells us
this by naming a file under the .git directory, which itself contains a SHA1
name referring to a commit object, which we can examine with cat-file:
\lstset{basicstyle=\scriptsize, numbers=none, captionpos=b, tabsize=4}
\begin{lstlisting}[caption=,language={bash},
breaklines=true,label=lst:]
$ cat .git/refs/heads/master
c4d59f390b9cfd4318117afde11d601c1085f241
$ git cat-file -t c4d59f39
commit
$ git cat-file commit c4d59f39
tree d0492b368b66bdabf2ac1fd8c92b39d3d
parent 54196cc2703dc165cbd373a65a4dcf2
author J. Bruce Fields <bfields@puzzle.fieldses.org> 1143418702 -0500
committer J. Bruce Fields <bfields@puzzle.fieldses.org> 1143418702 -0500

add emphasis
\end{lstlisting}

The "tree" object here refers to the new state of the tree:
\lstset{basicstyle=\scriptsize, numbers=none, captionpos=b, tabsize=4}
\begin{lstlisting}[caption=,language={bash},
breaklines=true,label=lst:]
$ git ls-tree d0492b36
100644 blob a0423896973644771497bdc03eb99d528    file.txt
$ git cat-file blob a0423896
hello world!
\end{lstlisting}

and the "parent" object refers to the previous commit:
\lstset{basicstyle=\scriptsize, numbers=none, captionpos=b, tabsize=4}
\begin{lstlisting}[caption=,language={bash},
breaklines=true,label=lst:]
$ git-cat-file commit 54196cc2
tree 92b8b694ffb1675e5975148e1121810081dbdffe
author J. Bruce Fields <bfields@puzzle.fieldses.org> 1143414668 -0500
committer J. Bruce Fields <bfields@puzzle.fieldses.org> 1143414668 -0500
\end{lstlisting}
