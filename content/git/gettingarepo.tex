\section{Getting a Git Repository}
So now that we're all set up, we need a Git repository. We can do this one of
two ways - we can clone one that already exists, or we can initialize one
either from existing files that aren't in source control yet, or from an empty
directory.

\subsection{Cloning a Repository}
In order to get a copy of a project, you will need to know the project's Git
URL - the location of the repository. Git can operate over many different
protocols, so it may begin with ssh://, http(s)://, git://, or just a username
(in which case git will assume ssh). Some repositories may be accessed over
more than one protocol. For example, the source code to Git itself can be
cloned either over the git:// protocol:
\lstset{basicstyle=\scriptsize, numbers=none, captionpos=b, tabsize=4}
\begin{lstlisting}[caption=,language={bash},
breaklines=true,xleftmargin=15pt, label=lst:]
git clone git://git.kernel.org/pub/scm/git/git.git
\end{lstlisting}

or over http:
\lstset{basicstyle=\scriptsize, numbers=none, captionpos=b, tabsize=4}
\begin{lstlisting}[caption=,language={bash},
breaklines=true,xleftmargin=15pt, label=lst:]
git clone http://www.kernel.org/pub/scm/git/git.git
\end{lstlisting}

The git:// protocol is faster and more efficient, but sometimes it is necessary
to use http when behind corporate firewalls or what have you. In either case
you should then have a new directory named 'git' that contains all the Git
source code and history - it is basically a complete copy of what was on the
server.

By default, Git will name the new directory it has checked out your cloned code
into after whatever comes directly before the '.git' in the path of the cloned
project. (ie. git clone http://git.kernel.org/ linux/ kernel/ git/ torvalds/
linux-2.6.git will result in a new directory named 'linux-2.6')
\scriptsize
\begin{verbatim}
Initializing a New Repository
\end{verbatim}
\normalsize

Assume you have a tarball named project.tar.gz with your initial work. You can
place it under git revision control as follows.

\lstset{basicstyle=\scriptsize, numbers=none, captionpos=b, tabsize=4}
\begin{lstlisting}[caption=,language={bash},
breaklines=true,label=lst:]
$ tar xzf project.tar.gz
$ cd project
$ git init
\end{lstlisting}

Git will reply

\scriptsize
\begin{verbatim}
Initialized empty Git repository in .git/
\end{verbatim}
\normalsize

You've now initialized the working directory--you may notice a new directory
created, named ".git".
