\section{Submodules}
Large projects are often composed of smaller, self-contained modules. For
example, an embedded Linux distribution's source tree would include every piece
of software in the distribution with some local modifications; a movie player
might need to build against a specific, known-working version of a
decompression library; several independent programs might all share the same
build scripts.

With centralized revision control systems this is often accomplished by
including every module in one single repository. Developers can check out all
modules or only the modules they need to work with. They can even modify files
across several modules in a single commit while moving things around or
updating APIs and translations.

Git does not allow partial checkouts, so duplicating this approach in Git would
force developers to keep a local copy of modules they are not interested in
touching. Commits in an enormous checkout would be slower than you'd expect as
Git would have to scan every directory for changes. If modules have a lot of
local history, clones would take forever.

On the plus side, distributed revision control systems can much better
integrate with external sources. In a centralized model, a single arbitrary
snapshot of the external project is exported from its own revision control and
then imported into the local revision control on a vendor branch. All the
history is hidden. With distributed revision control you can clone the entire
external history and much more easily follow development and re-merge local
changes.

Git's submodule support allows a repository to contain, as a subdirectory, a
checkout of an external project. Submodules maintain their own identity; the
submodule support just stores the submodule repository location and commit ID,
so other developers who clone the containing project ("superproject") can
easily clone all the submodules at the same revision. Partial checkouts of the
superproject are possible: you can tell Git to clone none, some or all of the
submodules.

The git submodule command is available since Git 1.5.3. Users with Git 1.5.2
can look up the submodule commits in the repository and manually check them
out; earlier versions won't recognize the submodules at all.

To see how submodule support works, create (for example) four example
repositories that can be used later as a submodule:
\lstset{basicstyle=\scriptsize, numbers=none, captionpos=b, tabsize=4}
\begin{lstlisting}[caption=,language={bash},
breaklines=true,label=lst:]
$ mkdir ~/git
$ cd ~/git
$ for i in a b c d
do
    mkdir $i
    cd $i
    git init
    echo "module $i" > $i.txt
    git add $i.txt
    git commit -m "Initial commit, submodule $i"
    cd ..
done
\end{lstlisting}

Now create the superproject and add all the submodules:
\lstset{basicstyle=\scriptsize, numbers=none, captionpos=b, tabsize=4}
\begin{lstlisting}[caption=,language={bash},
breaklines=true,label=lst:]
$ mkdir super
$ cd super
$ git init
$ for i in a b c d
do
    git submodule add ~/git/$i $i
done
\end{lstlisting}

NOTE: Do not use local URLs here if you plan to publish your superproject!

See what files git-submodule created:
\lstset{basicstyle=\scriptsize, numbers=none, captionpos=b, tabsize=4}
\begin{lstlisting}[caption=,language={bash},
breaklines=true,label=lst:]
$ ls -a
.  ..  .git  .gitmodules  a  b  c  d
\end{lstlisting}

The git-submodule add command does a couple of things:

It clones the submodule under the current directory and by default checks out
the master branch.  It adds the submodule's clone path to the gitmodules file
and adds this file to the index, ready to be committed.  It adds the
submodule's current commit ID to the index, ready to be committed.  Commit the
superproject:
\lstset{basicstyle=\scriptsize, numbers=none, captionpos=b, tabsize=4}
\begin{lstlisting}[caption=,language={bash},
breaklines=true,label=lst:]
$ git commit -m "Add submodules a, b, c and d."
\end{lstlisting}

Now clone the superproject:
\lstset{basicstyle=\scriptsize, numbers=none, captionpos=b, tabsize=4}
\begin{lstlisting}[caption=,language={bash},
breaklines=true,label=lst:]
$ cd ..
$ git clone super cloned
$ cd cloned
\end{lstlisting}

The submodule directories are there, but they're empty:
\lstset{basicstyle=\scriptsize, numbers=none, captionpos=b, tabsize=4}
\begin{lstlisting}[caption=,language={bash},
breaklines=true,label=lst:]
$ ls -a a
.  ..
$ git submodule status
-d266b9873ad50488163457f025db7cdd9683d88b a
-e81d457da15309b4fef4249aba9b50187999670d b
-c1536a972b9affea0f16e0680ba87332dc059146 c
-d96249ff5d57de5de093e6baff9e0aafa5276a74 d
\end{lstlisting}

NOTE: The commit object names shown above would be different for you, but they
should match the HEAD commit object names of your repositories. You can check
it by running git ls-remote ../git/a.

Pulling down the submodules is a two-step process. First run git submodule init
to add the submodule repository URLs to .git/config:
\lstset{basicstyle=\scriptsize, numbers=none, captionpos=b, tabsize=4}
\begin{lstlisting}[caption=,language={bash},
breaklines=true,label=lst:]
$ git submodule init
\end{lstlisting}

Now use git-submodule update to clone the repositories and check out the
commits specified in the superproject:
\lstset{basicstyle=\scriptsize, numbers=none, captionpos=b, tabsize=4}
\begin{lstlisting}[caption=,language={bash},
breaklines=true,label=lst:]
$ git submodule update
$ cd a
$ ls -a
.  ..  .git  a.txt
\end{lstlisting}

One major difference between git-submodule update and git-submodule add is that
git-submodule update checks out a specific commit, rather than the tip of a
branch. It's like checking out a tag: the head is detached, so you're not
working on a branch.
\lstset{basicstyle=\scriptsize, numbers=none, captionpos=b, tabsize=4}
\begin{lstlisting}[caption=,language={bash},
breaklines=true,label=lst:]
$ git branch
* (no branch)
master
\end{lstlisting}

If you want to make a change within a submodule and you have a detached head,
then you should create or checkout a branch, make your changes, publish the
change within the submodule, and then update the superproject to reference the
new commit:
\lstset{basicstyle=\scriptsize, numbers=none, captionpos=b, tabsize=4}
\begin{lstlisting}[caption=,language={bash},
breaklines=true,label=lst:]
$ git checkout master
\end{lstlisting}

or
\lstset{basicstyle=\scriptsize, numbers=none, captionpos=b, tabsize=4}
\begin{lstlisting}[caption=,language={bash},
breaklines=true,label=lst:]
$ git checkout -b fix-up
\end{lstlisting}

then
\lstset{basicstyle=\scriptsize, numbers=none, captionpos=b, tabsize=4}
\begin{lstlisting}[caption=,language={bash},
breaklines=true,label=lst:]
$ echo "adding a line again" >> a.txt
$ git commit -a -m "Updated the submodule from within the superproject."
$ git push
$ cd ..
$ git diff
diff --git a/a b/a
index d266b98..261dfac 160000
--- a/a
+++ b/a
@@ -1 +1 @@
-Subproject commit d266b9873ad50488163457f025db7cdd9683d88b
+Subproject commit 261dfac35cb99d380eb966e102c1197139f7fa24
$ git add a
$ git commit -m "Updated submodule a."
$ git push
\end{lstlisting}

You have to run git submodule update after git pull if you want to update
submodules, too.

\section{Pitfalls with submodules}
Always publish the submodule change before publishing the change to the
superproject that references it. If you forget to publish the submodule change,
others won't be able to clone the repository:
\lstset{basicstyle=\scriptsize, numbers=none, captionpos=b, tabsize=4}
\begin{lstlisting}[caption=,language={bash},
breaklines=true,label=lst:]
$ cd ~/git/super/a
$ echo i added another line to this file >> a.txt
$ git commit -a -m "doing it wrong this time"
$ cd ..
$ git add a
$ git commit -m "Updated submodule a again."
$ git push
$ cd ~/git/cloned
$ git pull
$ git submodule update
error: pathspec '261dfac35cb99d380eb966e102c1197139f7fa24' did not match any file(s) known to git.
Did you forget to 'git add'?
Unable to checkout '261dfac35cb99d380eb966e102c1197139f7fa24' in submodule path 'a'
\end{lstlisting}

If you are staging an updated submodule for commit manually, be careful to not
add a trailing slash when specifying the path. With the slash appended, Git
will assume you are removing the submodule and checking that directory's
contents into the containing repository.
\lstset{basicstyle=\scriptsize, numbers=none, captionpos=b, tabsize=4}
\begin{lstlisting}[caption=,language={bash},
breaklines=true,label=lst:]
$ cd ~/git/super/a
$ echo i added another line to this file >> a.txt
$ git commit -a -m "doing it wrong this time"
$ cd ..
$ git add a/
$ git status
# On branch master
# Changes to be committed:
#   (use "git reset HEAD <file>..." to unstage)
#
#       deleted:    a
#       new file:   a/a.txt
#
# Modified submodules:
#
# * a aa5c351...0000000 (1):
#   < Initial commit, submodule a
#
\end{lstlisting}

To fix the index after performing this operation, reset the changes and then
add the submodule without the trailing slash.
\lstset{basicstyle=\scriptsize, numbers=none, captionpos=b, tabsize=4}
\begin{lstlisting}[caption=,language={bash},
breaklines=true,label=lst:]
$ git reset HEAD A
$ git add a
$ git status
# On branch master
# Changes to be committed:
#   (use "git reset HEAD <file>..." to unstage)
#
#       modified:   a
#
# Modified submodules:
#
# * a aa5c351...8d3ba36 (1):
#   > doing it wrong this time
#
\end{lstlisting}

You also should not rewind branches in a submodule beyond commits that were
ever recorded in any superproject.

It's not safe to run git submodule update if you've made and committed changes
within a submodule without checking out a branch first. They will be silently
overwritten:
\lstset{basicstyle=\scriptsize, numbers=none, captionpos=b, tabsize=4}
\begin{lstlisting}[caption=,language={bash},
breaklines=true,label=lst:]
$ cat a.txt
module a
$ echo line added from private2 >> a.txt
$ git commit -a -m "line added inside private2"
$ cd ..
$ git submodule update
Submodule path 'a': checked out 'd266b9873ad50488163457f025db7cdd9683d88b'
$ cd a
$ cat a.txt
module a
\end{lstlisting}

NOTE: The changes are still visible in the submodule's reflog.

This is not the case if you did not commit your changes.
