\subsection{Transfer Protocols}
Here we will go over how clients and servers talk to each other to transfer Git
data around.

\subsubsection{Fetching Data over HTTP}
Fetching over an http/s URL will make Git use a slightly dumber protocol. In
this case, all of the logic is entirely on the client side. The server requires
no special setup - any static webserver will work fine if the git directory you
are fetching from is in the webserver path.

In order for this to work, you do need to run a single command on the server
repo everytime anything is updated, though - git update-server-info, which
updates the objects/info/packs and info/refs files to list which refs and
packfiles are available, since you can't do a listing over http. When that
command is run, the objects/info/packs file looks something like this:
\lstset{basicstyle=\scriptsize, numbers=none, captionpos=b, tabsize=4}
\begin{lstlisting}[caption=,language={bash},
breaklines=true,label=lst:]
P pack-ce2bd34abc3d8ebc5922dc81b2e1f30bf17c10cc.pack
P pack-7ad5f5d05f5e20025898c95296fe4b9c861246d8.pack
\end{lstlisting}

So that if the fetch can't find a loose file, it can try these packfiles. The
info/refs file will look something like this:
\lstset{basicstyle=\scriptsize, numbers=none, captionpos=b, tabsize=4}
\begin{lstlisting}[caption=,language={bash},
breaklines=true,label=lst:]
184063c9b594f8968d61a686b2f6052779551613    refs/heads/development
32aae7aef7a412d62192f710f2130302997ec883    refs/heads/master
\end{lstlisting}

Then when you fetch from this repo, it will start with these refs and walk the
commit objects until the client has all the objects that it needs.

For instance, if you ask to fetch the master branch, it will see that master is
pointing to 32aae7ae and that your master is pointing to ab04d88, so you need
32aae7ae. You fetch that object
\lstset{basicstyle=\scriptsize, numbers=none, captionpos=b, tabsize=4}
\begin{lstlisting}[caption=,language={bash},
breaklines=true,label=lst:]
CONNECT http://myserver.com
GET /git/myproject.git/objects/32/aae7aef7a412d62192f710f2130302997ec883 - 200
\end{lstlisting}

and it looks like this:
\lstset{basicstyle=\scriptsize, numbers=none, captionpos=b, tabsize=4}
\begin{lstlisting}[caption=,language={bash},
breaklines=true,label=lst:]
tree aa176fb83a47d00386be237b450fb9dfb5be251a
parent bd71cad2d597d0f1827d4a3f67bb96a646f02889
author Scott Chacon <schacon@gmail.com> 1220463037 -0700
committer Scott Chacon <schacon@gmail.com> 1220463037 -0700

added chapters on private repo setup, scm migration, raw git
\end{lstlisting}

So now it fetches the tree aa176fb8:
\lstset{basicstyle=\scriptsize, numbers=none, captionpos=b, tabsize=4}
\begin{lstlisting}[caption=,language={bash},
breaklines=true,label=lst:]
GET /git/myproject.git/objects/aa/176fb83a47d00386be237b450fb9dfb5be251a - 200
\end{lstlisting}

which looks like this:
\lstset{basicstyle=\scriptsize, numbers=none, captionpos=b, tabsize=4}
\begin{lstlisting}[caption=,language={bash},
breaklines=true,label=lst:]
100644 blob 6ff87c4664981e4397625791c8ea3bbb5f2279a3    COPYING
100644 blob 97b51a6d3685b093cfb345c9e79516e5099a13fb    README
100644 blob 9d1b23b8660817e4a74006f15fae86e2a508c573    Rakefile
\end{lstlisting}

So then it fetches those objects:
\lstset{basicstyle=\scriptsize, numbers=none, captionpos=b, tabsize=4}
\begin{lstlisting}[caption=,language={bash},
breaklines=true,label=lst:]
GET /git/myproject.git/objects/6f/f87c4664981e4397625791c8ea3bbb5f2279a3 - 200
GET /git/myproject.git/objects/97/b51a6d3685b093cfb345c9e79516e5099a13fb - 200
GET /git/myproject.git/objects/9d/1b23b8660817e4a74006f15fae86e2a508c573 - 200
\end{lstlisting}

It actually does this with Curl, and can open up multiple parallel threads to
speed up this process. When it's done recursing the tree pointed to by the
commit, it fetches the next parent.
\lstset{basicstyle=\scriptsize, numbers=none, captionpos=b, tabsize=4}
\begin{lstlisting}[caption=,language={bash},
breaklines=true,label=lst:]
GET /git/myproject.git/objects/bd/71cad2d597d0f1827d4a3f67bb96a646f02889 - 200
\end{lstlisting}

Now in this case, the commit that comes back looks like this:
\lstset{basicstyle=\scriptsize, numbers=none, captionpos=b, tabsize=4}
\begin{lstlisting}[caption=,language={bash},
breaklines=true,label=lst:]
tree b4cc00cf8546edd4fcf29defc3aec14de53e6cf8
parent ab04d884140f7b0cf8bbf86d6883869f16a46f65
author Scott Chacon <schacon@gmail.com> 1220421161 -0700
committer Scott Chacon <schacon@gmail.com> 1220421161 -0700

added chapters on the packfile and how git stores objects
\end{lstlisting}

and we can see that the parent, ab04d88 is where our master branch is currently
pointing. So, we recursively fetch this tree and then stop, since we know we
have everything before this point. You can force Git to double check that we
have everything with the '--recover' option. See git http-fetch for more
information.

If one of the loose object fetches fails, Git will download the packfile
indexes looking for the sha that it needs, then download that packfile.

It is important if you are running a git server that serves repos this way to
implement a post-receive hook that runs the 'git update-server-info' command
each time or there will be confusion. 

\subsubsection{Fetching Data with Upload Pack}
For the smarter protocols, fetching objects is much more efficient. A socket is
opened, either over ssh or over port 9418 (in the case of the git:// protocol),
and the git fetch-pack command on the client begins communicating with a forked
git upload-pack process on the server.

Then the server will tell the client which SHAs it has for each ref, and the
client figures out what it needs and responds with a list of SHAs it wants and
already has.

At this point, the server will generate a packfile with all the objects that
the client needs and begin streaming it down to the client.

Let's look at an example.

The client connects and sends the request header. The clone command
\lstset{basicstyle=\scriptsize, numbers=none, captionpos=b, tabsize=4}
\begin{lstlisting}[caption=,language={bash},
breaklines=true,label=lst:]
$ git clone git://myserver.com/project.git
\end{lstlisting}

produces the following request:
\lstset{basicstyle=\scriptsize, numbers=none, captionpos=b, tabsize=4}
\begin{lstlisting}[caption=,language={bash},
breaklines=true,label=lst:]
0032git-upload-pack /project.git\000host=myserver.com\000
\end{lstlisting}

The first four bytes contain the hex length of the line (including 4 byte line
length and trailing newline if present). Following are the command and
arguments. This is followed by a null byte and then the host information. The
request is terminated by a null byte.

The request is processed and turned into a call to git-upload-pack:
\lstset{basicstyle=\scriptsize, numbers=none, captionpos=b, tabsize=4}
\begin{lstlisting}[caption=,language={bash},
breaklines=true,label=lst:]
$ git-upload-pack /path/to/repos/project.git
\end{lstlisting}

This immediately returns information of the repo:
\lstset{basicstyle=\scriptsize, numbers=none, captionpos=b, tabsize=4}
\begin{lstlisting}[caption=,language={bash},
breaklines=true,label=lst:]
007c74730d410fcb6603ace96f1dc55ea6196122532d HEAD\000multi_ack thin-pack side-band side-band-64k ofs-delta shallow no-progress
003e7d1665144a3a975c05f1f43902ddaf084e784dbe refs/heads/debug
003d5a3f6be755bbb7deae50065988cbfa1ffa9ab68a refs/heads/dist
003e7e47fe2bd8d01d481f44d7af0531bd93d3b21c01 refs/heads/local
003f74730d410fcb6603ace96f1dc55ea6196122532d refs/heads/master
0000
\end{lstlisting}

Each line starts with a four byte line length declaration in hex. The section
is terminated by a line length declaration of 0000.

This is sent back to the client verbatim. The client responds with another
request:
\lstset{basicstyle=\scriptsize, numbers=none, captionpos=b, tabsize=4}
\begin{lstlisting}[caption=,language={bash},
breaklines=true,label=lst:]
0054want 410fcb6603ace96f1dc55ea6196122532d multi_ack side-band-64k ofs-delta
0032want 144a3a975c05f1f43902ddaf084e784dbe
0032want e755bbb7deae50065988cbfa1ffa9ab68a
0032want 2bd8d01d481f44d7af0531bd93d3b21c01
0032want 410fcb6603ace96f1dc55ea6196122532d
00000009done
\end{lstlisting}

The is sent to the open git-upload-pack process which then streams out the
final response:
\lstset{basicstyle=\scriptsize, numbers=none, captionpos=b, tabsize=4}
\begin{lstlisting}[caption=,language={bash},
breaklines=true,label=lst:]
"0008NAK\n"
"0023\002Counting objects: 2797, done.\n"
"002b\002Compressing objects:   0% (1/1177)   \r"
"002c\002Compressing objects:   1% (12/1177)   \r"
"002c\002Compressing objects:   2% (24/1177)   \r"
"002c\002Compressing objects:   3% (36/1177)   \r"
"002c\002Compressing objects:   4% (48/1177)   \r"
"002c\002Compressing objects:   5% (59/1177)   \r"
"002c\002Compressing objects:   6% (71/1177)   \r"
"0053\002Compressing objects:   7% (83/1177)   \rCompressing objects:   8% (95/1177)   \r"
...
"005b\002Compressing objects: 100% (1177/1177)   \rCompressing objects: 100% (1177/1177), done.\n"
"2004\001PACK\000\000\000\002\000\000\n\355\225\017x\234\235\216K\n\302"...
"2005\001\360\204{\225\376\330\345]z2673"...
...
"0037\002Total 2797 (delta 1799), reused 2360 (delta 1529)\n"
...
"<\276\255L\273s\005\001w0006\001[0000"
\end{lstlisting}

See the Packfile chapter previously for the actual format of the packfile data
in the response.

\subsubsection{Pushing Data}
Pushing data over the git and ssh protocols is similar, but simpler. Basically
what happens is the client requests a receive-pack instance, which is started
up if the client has access, then the server returns all the ref head shas it
has again and the client generates a packfile of everything the server needs
(generally only if what is on the server is a direct ancestor of what it is
pushing) and sends that packfile upstream, where the server either stores it on
disk and builds an index for it, or unpacks it (if there aren't many objects in
it)

This entire process is accomplished through the git send-pack command on the
client, which is invoked by git push and the git receive-pack command on the
server side, which is invoked by the ssh connect process or git daemon (if it's
an open push server).
