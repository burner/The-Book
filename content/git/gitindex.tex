\subsection{The Git Index}
The index is a binary file (generally kept in .git/index) containing a sorted
list of path names, each with permissions and the SHA1 of a blob object; git
ls-files can show you the contents of the index:
\lstset{basicstyle=\scriptsize, numbers=none, captionpos=b, tabsize=4}
\begin{lstlisting}[caption=,language={bash},
breaklines=true,label=lst:]
$ git ls-files --stage
100644 667fa005ff12ad89437f2fdc80926e21c 0   .gitignore
100644 8e8d14decbe4ad99db3f7fb632de0439d 0   .mailmap
100644 664981e4397625791c8ea3bbb5f2279a3 0   COPYING
100644 2bd26be2c2289e1f57a292534a51a93c7 0   Documentation/.gitignore
100644 45b00a54b58d94d06eca48b03d40a50e0 0   Documentation/Makefile
...
100644 8d89ab52be5ec6a5e46236b4b6bcd07ea 0   xdiff/xtypes.h
100644 2574a9f77e7ae4002a4e07a6a38e46d07 0   xdiff/xutils.c
100644 2e05e7c36c4b68857c1cf9855e3d2f70a 0   xdiff/xutils.h
\end{lstlisting}

Note that in older documentation you may see the index called the "current
directory cache" or just the "cache". It has three important properties:
\begin{enumerate}
\item The index contains all the information necessary to generate a single
(uniquely determined) tree object.  For example, running git commit generates
this tree object from the index, stores it in the object database, and uses it
as the tree object associated with the new commit.

\item The index enables fast comparisons between the tree object it defines and
the working tree.  It does this by storing some additional data for each entry
(such as the last modified time). This data is not displayed above, and is not
stored in the created tree object, but it can be used to determine quickly
which files in the working directory differ from what was stored in the index,
and thus save git from having to read all of the data from such files to look
for changes.

\item It can efficiently represent information about merge conflicts between
different tree objects, allowing each pathname to be associated with sufficient
information about the trees involved that you can create a three-way merge
between them.  During a merge, the index can store multiple versions of a
single file (called "stages"). The third column in the git ls-files output
above is the stage number, and will take on values other than 0 for files with
merge conflicts.
\end{enumerate}

The index is thus a sort of temporary staging area, which is filled with a tree
which you are in the process of working on.
