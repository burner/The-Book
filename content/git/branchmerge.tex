\subsection{Basic Branching and Merging}
A single git repository can maintain multiple branches of development. To
create a new branch named "experimental", use

\lstset{basicstyle=\scriptsize, numbers=none, captionpos=b, tabsize=4}
\begin{lstlisting}[caption=,language={bash},
breaklines=true,label=lst:]
$ git branch experimental
\end{lstlisting}

If you now run

you'll get a list of all existing branches:

\lstset{basicstyle=\scriptsize, numbers=none, captionpos=b, tabsize=4}
\begin{lstlisting}[caption=,language={bash},
breaklines=true,label=lst:]
$ git branch
\end{lstlisting}

\lstset{basicstyle=\scriptsize, numbers=none, captionpos=b, tabsize=4}
\begin{lstlisting}[caption=,language={bash},
breaklines=true,label=lst:]
  experimental
* master
\end{lstlisting}

The "experimental" branch is the one you just created, and the "master" branch
is a default branch that was created for you automatically. The asterisk marks
the branch you are currently on; type

\lstset{basicstyle=\scriptsize, numbers=none, captionpos=b, tabsize=4}
\begin{lstlisting}[caption=,language={bash},
breaklines=true,label=lst:]
$ git checkout experimental
\end{lstlisting}

to switch to the experimental branch. Now edit a file, commit the change, and
switch back to the master branch:

\lstset{basicstyle=\scriptsize, numbers=none, captionpos=b, tabsize=4}
\begin{lstlisting}[caption=,language={bash},
breaklines=true,label=lst:]
(edit file)
$ git commit -a
$ git checkout master
\end{lstlisting}

Check that the change you made is no longer visible, since it was made on the
experimental branch and you're back on the master branch.

You can make a different change on the master branch:

\lstset{basicstyle=\scriptsize, numbers=none, captionpos=b, tabsize=4}
\begin{lstlisting}[caption=,language={bash},
breaklines=true,label=lst:]
(edit file)
$ git commit -a
\end{lstlisting}

at this point the two branches have diverged, with different changes made in
each. To merge the changes made in experimental into master, run

\lstset{basicstyle=\scriptsize, numbers=none, captionpos=b, tabsize=4}
\begin{lstlisting}[caption=,language={bash},
breaklines=true,label=lst:]
$ git merge experimental
\end{lstlisting}

If the changes don't conflict, you're done. If there are conflicts, markers
will be left in the problematic files showing the conflict;

\lstset{basicstyle=\scriptsize, numbers=none, captionpos=b, tabsize=4}
\begin{lstlisting}[caption=,language={bash},
breaklines=true,label=lst:]
$ git diff
\end{lstlisting}

will show this. Once you've edited the files to resolve the conflicts,

\lstset{basicstyle=\scriptsize, numbers=none, captionpos=b, tabsize=4}
\begin{lstlisting}[caption=,language={bash},
breaklines=true,label=lst:]
$ git commit -a
\end{lstlisting}

will commit the result of the merge. Finally,

\lstset{basicstyle=\scriptsize, numbers=none, captionpos=b, tabsize=4}
\begin{lstlisting}[caption=,language={bash},
breaklines=true,label=lst:]
$ gitk
\end{lstlisting}

will show a nice graphical representation of the resulting history.

At this point you could delete the experimental branch with

\lstset{basicstyle=\scriptsize, numbers=none, captionpos=b, tabsize=4}
\begin{lstlisting}[caption=,language={bash},
breaklines=true,label=lst:]
$ git branch -d experimental
\end{lstlisting}

This command ensures that the changes in the experimental branch are already in
the current branch.

If you develop on a branch crazy-idea, then regret it, you can always delete
the branch with

\lstset{basicstyle=\scriptsize, numbers=none, captionpos=b, tabsize=4}
\begin{lstlisting}[caption=,language={bash},
breaklines=true,label=lst:]
$ git branch -D crazy-idea
\end{lstlisting}

Branches are cheap and easy, so this is a good way to try something out.

\subsubsection{How to merge}
You can rejoin two diverging branches of development using git merge:

\lstset{basicstyle=\scriptsize, numbers=none, captionpos=b, tabsize=4}
\begin{lstlisting}[caption=,language={bash},
breaklines=true,label=lst:]
$ git merge branchname
\end{lstlisting}

merges the changes made in the branch "branchname" into the current branch. If
there are conflicts--for example, if the same file is modified in two different
ways in the remote branch and the local branch--then you are warned; the output
may look something like this:

\lstset{basicstyle=\scriptsize, numbers=none, captionpos=b, tabsize=4}
\begin{lstlisting}[caption=,language={bash},
breaklines=true,label=lst:]
$ git merge next
 100% (4/4) done
Auto-merged file.txt
CONFLICT (content): Merge conflict in file.txt
Automatic merge failed; fix conflicts and then commit the result.
\end{lstlisting}

Conflict markers are left in the problematic files, and after you resolve the
conflicts manually, you can update the index with the contents and run git
commit, as you normally would when modifying a file.

If you examine the resulting commit using gitk, you will see that it has two
parents: one pointing to the top of the current branch, and one to the top of
the other branch.

\subsubsection{Resolving a merge}
When a merge isn't resolved automatically, git leaves the index and the working
tree in a special state that gives you all the information you need to help
resolve the merge.

Files with conflicts are marked specially in the index, so until you resolve
the problem and update the index, git commit will fail:

\lstset{basicstyle=\scriptsize, numbers=none, captionpos=b, tabsize=4}
\begin{lstlisting}[caption=,language={bash},
breaklines=true,label=lst:]
$ git commit
file.txt: needs merge
\end{lstlisting}

Also, git status will list those files as "unmerged", and the files with
conflicts will have conflict markers added, like this:

\lstset{basicstyle=\scriptsize, numbers=none, captionpos=b, tabsize=4}
\begin{lstlisting}[caption=,language={bash},
breaklines=true,label=lst:]
<<<<<<< HEAD:file.txt
Hello world
=======
Goodbye
>>>>>>> 77976da35a11db4580b80ae27e8d65caf5208086:file.txt
\end{lstlisting}

All you need to do is edit the files to resolve the conflicts, and then

\lstset{basicstyle=\scriptsize, numbers=none, captionpos=b, tabsize=4}
\begin{lstlisting}[caption=,language={bash},
breaklines=true,label=lst:]
$ git add file.txt
$ git commit
\end{lstlisting}

Note that the commit message will already be filled in for you with some
information about the merge. Normally you can just use this default message
unchanged, but you may add additional commentary of your own if desired.

The above is all you need to know to resolve a simple merge. But git also
provides more information to help resolve conflicts:

\subsubsection{Undoing a merge}
If you get stuck and decide to just give up and throw the whole mess away, you
can always return to the pre-merge state with

\lstset{basicstyle=\scriptsize, numbers=none, captionpos=b, tabsize=4}
\begin{lstlisting}[caption=,language={bash},
breaklines=true,label=lst:]
$ git reset --hard HEAD
\end{lstlisting}

Or, if you've already committed the merge that you want to throw away,

\lstset{basicstyle=\scriptsize, numbers=none, captionpos=b, tabsize=4}
\begin{lstlisting}[caption=,language={bash},
breaklines=true,label=lst:]
$ git reset --hard ORIG_HEAD
\end{lstlisting}

However, this last command can be dangerous in some cases--never throw away a
commit if that commit may itself have been merged into another branch, as doing
so may confuse further merges.

\subsubsection{Fast-forward merges}
There is one special case not mentioned above, which is treated differently.
Normally, a merge results in a merge commit with two parents, one for each of
the two lines of development that were merged.

However, if the current branch has not diverged from the other--so every commit
present in the current branch is already contained in the other--then git just
performs a "fast forward"; the head of the current branch is moved forward to
point at the head of the merged-in branch, without any new commits being
created.
