\section{Undoing in Git - Reset, Checkout and Revert}
Git provides multiple methods for fixing up mistakes as you are developing.
Selecting an appropriate method depends on whether or not you have committed
the mistake, and if you have committed the mistake, whether you have shared the
erroneous commit with anyone else.

\subsection{Fixing un-committed mistakes}
If you've messed up the working tree, but haven't yet committed your mistake,
you can return the entire working tree to the last committed state with
\lstset{basicstyle=\scriptsize, numbers=none, captionpos=b, tabsize=4}
\begin{lstlisting}[caption=,language={bash},
breaklines=true,label=lst:]
$ git reset --hard HEAD
\end{lstlisting}

This will throw away any changes you may have added to the git index and as
well as any outstanding changes you have in your working tree. In other words,
it causes the results of "git diff" and "git diff --cached" to both be empty.

If you just want to restore just one file, say your hello.rb, use git checkout
instead
\lstset{basicstyle=\scriptsize, numbers=none, captionpos=b, tabsize=4}
\begin{lstlisting}[caption=,language={bash},
breaklines=true,label=lst:]
$ git checkout -- hello.rb
$ git checkout HEAD hello.rb
\end{lstlisting}

The first command restores hello.rb to the version in the index, so that "git
diff hello.rb" returns no differences. The second command will restore hello.rb
to the version in the HEAD revision, so that both "git diff hello.rb" and "git
diff --cached hello.rb" return no differences.

\subsection{Fixing committed mistakes}
If you make a commit that you later wish you hadn't, there are two
fundamentally different ways to fix the problem:
\begin{enumerate}
\item You can create a new commit that undoes whatever was done by the old commit.
This is the correct thing if your mistake has already been made public.

\item You can go back and modify the old commit. You should never do this if you have
already made the history public; git does not normally expect the "history" of
a project to change, and cannot correctly perform repeated merges from a branch
that has had its history changed.
\end{enumerate}

\subsection{Fixing a mistake with a new commit}
Creating a new commit that reverts an earlier change is very easy; just pass
the git revert command a reference to the bad commit; for example, to revert
the most recent commit:
\lstset{basicstyle=\scriptsize, numbers=none, captionpos=b, tabsize=4}
\begin{lstlisting}[caption=,language={bash},
breaklines=true,label=lst:]
$ git revert HEAD
\end{lstlisting}

This will create a new commit which undoes the change in HEAD. You will be
given a chance to edit the commit message for the new commit.

You can also revert an earlier change, for example, the next-to-last:
\lstset{basicstyle=\scriptsize, numbers=none, captionpos=b, tabsize=4}
\begin{lstlisting}[caption=,language={bash},
breaklines=true,label=lst:]
$ git revert HEAD^
\end{lstlisting}

In this case git will attempt to undo the old change while leaving intact any
changes made since then. If more recent changes overlap with the changes to be
reverted, then you will be asked to fix conflicts manually, just as in the case
of resolving a merge.

\subsection{Fixing a mistake by modifying a commit}
If you have just committed something but realize you need to fix up that
commit, recent versions of git commit support an --amend flag which instructs
git to replace the HEAD commit with a new one, based on the current contents of
the index. This gives you an opportunity to add files that you forgot to add or
correct typos in a commit message, prior to pushing the change out for the
world to see.

If you find a mistake in an older commit, but still one that you have not yet
published to the world, you use git rebase in interactive mode, with "git
rebase -i" marking the change that requires correction with edit. This will
allow you to amend the commit during the rebasing process.
