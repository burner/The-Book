\section{Distributed Workflows}
Suppose that Alice has started a new project with a git repository in
/home/alice/project, and that Bob, who has a home directory on the same
machine, wants to contribute.

Bob begins with:
\lstset{basicstyle=\scriptsize, numbers=none, captionpos=b, tabsize=4}
\begin{lstlisting}[caption=,language={bash},
breaklines=true,label=lst:]
$ git clone /home/alice/project myrepo
\end{lstlisting}

This creates a new directory "myrepo" containing a clone of Alice's repository.
The clone is on an equal footing with the original project, possessing its own
copy of the original project's history.

Bob then makes some changes and commits them:
\lstset{basicstyle=\scriptsize, numbers=none, captionpos=b, tabsize=4}
\begin{lstlisting}[caption=,language={bash},
breaklines=true,label=lst:]
(edit files)
$ git commit -a
(repeat as necessary)
\end{lstlisting}

When he's ready, he tells Alice to pull changes from the repository at
/home/bob/myrepo. She does this with:
\lstset{basicstyle=\scriptsize, numbers=none, captionpos=b, tabsize=4}
\begin{lstlisting}[caption=,language={bash},
breaklines=true,label=lst:]
$ cd /home/alice/project
$ git pull /home/bob/myrepo master
\end{lstlisting}

This merges the changes from Bob's "master" branch into Alice's current branch.
If Alice has made her own changes in the meantime, then she may need to
manually fix any conflicts. (Note that the "master" argument in the above
command is actually unnecessary, as it is the default.)

The "pull" command thus performs two operations: it fetches changes from a
remote branch, then merges them into the current branch.

When you are working in a small closely knit group, it is not unusual to
interact with the same repository over and over again. By defining 'remote'
repository shorthand, you can make it easier:
\lstset{basicstyle=\scriptsize, numbers=none, captionpos=b, tabsize=4}
\begin{lstlisting}[caption=,language={bash},
breaklines=true,label=lst:]
$ git remote add bob /home/bob/myrepo
\end{lstlisting}

With this, Alice can perform the first operation alone using the "git fetch"
command without merging them with her own branch, using:
\lstset{basicstyle=\scriptsize, numbers=none, captionpos=b, tabsize=4}
\begin{lstlisting}[caption=,language={bash},
breaklines=true,label=lst:]
$ git fetch bob
\end{lstlisting}

Unlike the longhand form, when Alice fetches from Bob using a remote repository
shorthand set up with git remote, what was fetched is stored in a remote
tracking branch, in this case bob/master. So after this:
\lstset{basicstyle=\scriptsize, numbers=none, captionpos=b, tabsize=4}
\begin{lstlisting}[caption=,language={bash},
breaklines=true,label=lst:]
$ git log -p master..bob/master
\end{lstlisting}

shows a list of all the changes that Bob made since he branched from Alice's
master branch.

After examining those changes, Alice could merge the changes into her master
branch:
\lstset{basicstyle=\scriptsize, numbers=none, captionpos=b, tabsize=4}
\begin{lstlisting}[caption=,language={bash},
breaklines=true,label=lst:]
$ git merge bob/master
\end{lstlisting}

This merge can also be done by 'pulling from her own remote tracking branch',
like this:
\lstset{basicstyle=\scriptsize, numbers=none, captionpos=b, tabsize=4}
\begin{lstlisting}[caption=,language={bash},
breaklines=true,label=lst:]
$ git pull . remotes/bob/master
\end{lstlisting}

Note that git pull always merges into the current branch, regardless of what
else is given on the command line.

Later, Bob can update his repo with Alice's latest changes using
\lstset{basicstyle=\scriptsize, numbers=none, captionpos=b, tabsize=4}
\begin{lstlisting}[caption=,language={bash},
breaklines=true,label=lst:]
$ git pull
\end{lstlisting}

Note that he doesn't need to give the path to Alice's repository; when Bob
cloned Alice's repository, git stored the location of her repository in the
repository configuration, and that location is used for pulls:
\lstset{basicstyle=\scriptsize, numbers=none, captionpos=b, tabsize=4}
\begin{lstlisting}[caption=,language={bash},
breaklines=true,label=lst:]
$ git config --get remote.origin.url
/home/alice/project
\end{lstlisting}

(The complete configuration created by git-clone is visible using "git config
-l", and the git config man page explains the meaning of each option.)

Git also keeps a pristine copy of Alice's master branch under the name
"origin/master":
\lstset{basicstyle=\scriptsize, numbers=none, captionpos=b, tabsize=4}
\begin{lstlisting}[caption=,language={bash},
breaklines=true,label=lst:]
$ git branch -r
  origin/master
\end{lstlisting}

If Bob later decides to work from a different host, he can still perform clones
and pulls using the ssh protocol:
\lstset{basicstyle=\scriptsize, numbers=none, captionpos=b, tabsize=4}
\begin{lstlisting}[caption=,language={bash},
breaklines=true,label=lst:]
$ git clone alice.org:/home/alice/project myrepo
\end{lstlisting}

Alternatively, git has a native protocol, or can use rsync or http; see git
pull for details.

Git can also be used in a CVS-like mode, with a central repository that various
users push changes to; see git push and gitcvs-migration.

\subsection{Public git repositories}

Another way to submit changes to a project is to tell the maintainer of that
project to pull the changes from your repository using git pull. This is a way
to get updates from the "main" repository, but it works just as well in the
other direction.

If you and the maintainer both have accounts on the same machine, then you can
just pull changes from each other's repositories directly; commands that accept
repository URLs as arguments will also accept a local directory name:
\lstset{basicstyle=\scriptsize, numbers=none, captionpos=b, tabsize=4}
\begin{lstlisting}[caption=,language={bash},
breaklines=true,label=lst:]
$ git clone /path/to/repository
$ git pull /path/to/other/repository
\end{lstlisting}

or an ssh URL:
\lstset{basicstyle=\scriptsize, numbers=none, captionpos=b, tabsize=4}
\begin{lstlisting}[caption=,language={bash},
breaklines=true,label=lst:]
$ git clone ssh://yourhost/~you/repository
\end{lstlisting}

For projects with few developers, or for synchronizing a few private
repositories, this may be all you need.

However, the more common way to do this is to maintain a separate public
repository (usually on a different host) for others to pull changes from. This
is usually more convenient, and allows you to cleanly separate private work in
progress from publicly visible work.

You will continue to do your day-to-day work in your personal repository, but
periodically "push" changes from your personal repository into your public
repository, allowing other developers to pull from that repository. So the flow
of changes, in a situation where there is one other developer with a public
repository, looks like this:
\lstset{basicstyle=\scriptsize, numbers=none, captionpos=b, tabsize=4}
\begin{lstlisting}[caption=,language={bash},
breaklines=true,label=lst:]

                   you push   your
your personal repo ---------> pub repo
  ^                            |
  |                            |
  | you pull                   | they
  |                            | pull
  |                            |
  |               they push    V
their public repo <---------- their repo
\end{lstlisting}

\subsection{Pushing changes to a public repository}
Note that exporting via http or git allow other maintainers to fetch your
latest changes, but they do not allow write access. For this, you will need to
update the public repository with the latest changes created in your private
repository.

The simplest way to do this is using git push and ssh; to update the remote
branch named "master" with the latest state of your branch named "master", run
\lstset{basicstyle=\scriptsize, numbers=none, captionpos=b, tabsize=4}
\begin{lstlisting}[caption=,language={bash},
breaklines=true,label=lst:]
$ git push ssh://yourserver.com/~you/proj.git master:master
\end{lstlisting}

or just
\lstset{basicstyle=\scriptsize, numbers=none, captionpos=b, tabsize=4}
\begin{lstlisting}[caption=,language={bash},
breaklines=true,label=lst:]
$ git push ssh://yourserver.com/~you/proj.git master
\end{lstlisting}

As with git-fetch, git-push will complain if this does not result in a fast
forward; see the following section for details on handling this case.

Note that the target of a "push" is normally a bare repository. You can also
push to a repository that has a checked-out working tree, but the working tree
will not be updated by the push. This may lead to unexpected results if the
branch you push to is the currently checked-out branch!

As with git-fetch, you may also set up configuration options to save typing;
so, for example, after
\lstset{basicstyle=\scriptsize, numbers=none, captionpos=b, tabsize=4}
\begin{lstlisting}[caption=,language={bash},
breaklines=true,label=lst:]
$ cat >>.git/config <<EOF
[remote "public-repo"]
    url = ssh://yourserver.com/~you/proj.git
EOF
\end{lstlisting}

you should be able to perform the above push with just
\lstset{basicstyle=\scriptsize, numbers=none, captionpos=b, tabsize=4}
\begin{lstlisting}[caption=,language={bash},
breaklines=true,label=lst:]
$ git push public-repo master
\end{lstlisting}

See the explanations of the remote..url, branch..remote, and remote..push
options in git config for details.

\subsection{What to do when a push fails}
If a push would not result in a fast forward of the remote branch, then it will
fail with an error like:
\lstset{basicstyle=\scriptsize, numbers=none, captionpos=b, tabsize=4}
\begin{lstlisting}[caption=,language={bash},
breaklines=true,label=lst:]
error: remote 'refs/heads/master' is not an ancestor of
local  'refs/heads/master'.
Maybe you are not up-to-date and need to pull first?
error: failed to push to 'ssh://yourserver.com/~you/proj.git'
\end{lstlisting}

This can happen, for example, if you:
\lstset{basicstyle=\scriptsize, numbers=none, captionpos=b, tabsize=4}
\begin{lstlisting}[caption=,language={bash},
breaklines=true,label=lst:]
- use `git-reset --hard` to remove already-published commits, or
- use `git-commit --amend` to replace already-published commits, or
- use `git-rebase` to rebase any already-published commits.
\end{lstlisting}

You may force git-push to perform the update anyway by preceding the branch
name with a plus sign:
\lstset{basicstyle=\scriptsize, numbers=none, captionpos=b, tabsize=4}
\begin{lstlisting}[caption=,language={bash},
breaklines=true,label=lst:]
$ git push ssh://yourserver.com/~you/proj.git +master
\end{lstlisting}

Normally whenever a branch head in a public repository is modified, it is
modified to point to a descendant of the commit that it pointed to before. By
forcing a push in this situation, you break that convention.

Nevertheless, this is a common practice for people that need a simple way to
publish a work-in-progress patch series, and it is an acceptable compromise as
long as you warn other developers that this is how you intend to manage the
branch.

It's also possible for a push to fail in this way when other people have the
right to push to the same repository. In that case, the correct solution is to
retry the push after first updating your work: either by a pull, or by a fetch
followed by a rebase; see the next section and gitcvs-migration for more.
