Here we have the meanings of some terms used into Git context.
These terms were entirely copied from Git Glossary.
\definecolor{light-gray}{gray}{0.85}
\subsection*{alternate object database}

\textcolor{light-gray}{Via the alternates mechanism, a repository}

can inherit part of its object database
from another object database, which is called "alternate". 

\subsection*{bare repository}

A bare repository is normally an appropriately

named directory with a `.git` suffix that does not
have a locally checked-out copy of any of the files under
revision control. That is, all of the `git`
administrative and control files that would normally be present in the
hidden `.git` sub-directory are directly present in the
`repository.git` directory instead,
and no other files are present and checked out. Usually publishers of
public repositories make bare repositories available.
blob object

Untyped object, e.g. the contents of a file.

branch

A "branch" is an active line of development. The most recent

commit on a branch is referred to as the tip of
that branch.  The tip of the branch is referenced by a branch
head, which moves forward as additional development
is done on the branch.  A single git
repository can track an arbitrary number of
branches, but your working tree is
associated with just one of them (the "current" or "checked out"
branch), and HEAD points to that branch.
cache

Obsolete for: index.

chain

A list of objects, where each object in the list contains

a reference to its successor (for example, the successor of a
commit could be one of its parents).
changeset

BitKeeper/cvsps speak for "commit". Since git does not

store changes, but states, it really does not make sense to use the term
"changesets" with git.
checkout

The action of updating all or part of the

working tree with a tree object
or blob from the
object database, and updating the
index and HEAD if the whole working tree has
been pointed at a new branch.
cherry-picking

In SCM jargon, "cherry pick" means to choose a subset of

changes out of a series of changes (typically commits) and record them
as a new series of changes on top of a different codebase. In GIT, this is
performed by the "git cherry-pick" command to extract the change introduced
by an existing commit and to record it based on the tip
of the current branch as a new commit.
clean

A working tree is clean, if it

corresponds to the revision referenced by the current
head. Also see "dirty".
commit

As a noun: A single point in the

git history; the entire history of a project is represented as a
set of interrelated commits.  The word "commit" is often
used by git in the same places other revision control systems
use the words "revision" or "version".  Also used as a short
hand for commit object.
As a verb: The action of storing a new snapshot of the project's

state in the git history, by creating a new commit representing the current
state of the index and advancing HEAD
to point at the new commit.
commit object

An object which contains the information about a

particular revision, such as parents, committer,
author, date and the tree object which corresponds
to the top directory of the stored
revision.
core git

Fundamental data structures and utilities of git. Exposes only limited

source code management tools.
DAG

Directed acyclic graph. The commit objects form a

directed acyclic graph, because they have parents (directed), and the
graph of commit objects is acyclic (there is no chain
which begins and ends with the same object).
dangling object

An unreachable object which is not

reachable even from other unreachable objects; a
dangling object has no references to it from any
reference or object in the repository.
detached HEAD

Normally the HEAD stores the name of a

branch.  However, git also allows you to check out
an arbitrary commit that isn't necessarily the tip of any
particular branch.  In this case HEAD is said to be "detached".
dircache

You are waaaaay behind. See index.

directory

The list you get with "ls" :-)

dirty

A working tree is said to be "dirty" if

it contains modifications which have not been committed to the current
branch.
ent

Favorite synonym to "tree-ish" by some total geeks. See

Middle-earth for an in-depth
explanation. Avoid this term, not to confuse people.
evil merge

An evil merge is a merge that introduces changes that

do not appear in any parent.
fast forward

A fast-forward is a special type of merge where you have a

revision and you are "merging" another
branch's changes that happen to be a descendant of what
you have. In such these cases, you do not make a new merge
commit but instead just update to his
revision. This will happen frequently on a
tracking branch of a remote
repository.
fetch

Fetching a branch means to get the

branch's head ref from a remote
repository, to find out which objects are
missing from the local object database,
and to get them, too.  See also git fetch.
file system

Linus Torvalds originally designed git to be a user space file system,

i.e. the infrastructure to hold files and directories. That ensured the
efficiency and speed of git.
git archive

Synonym for repository (for arch people).

grafts

Grafts enables two otherwise different lines of development to be joined

together by recording fake ancestry information for commits. This way
you can make git pretend the set of parents a commit has
is different from what was recorded when the commit was
created. Configured via the `.git/info/grafts` file.
hash

In git's context, synonym to object name.

head

A named reference to the commit at the tip of a

\scriptsize
\begin{verbatim}
$GIT_DIR/refs/heads/, except when using packed refs. (See
git pack-refs.)
\end{verbatim}
\normalsize

HEAD

The current branch. In more detail: Your working tree is normally derived

from the state of the tree referred to by HEAD.  HEAD is a reference to one
of the heads in your repository, except when using a detached HEAD, in which
case it may reference an arbitrary commit.
head ref

A synonym for head.

hook

During the normal execution of several git commands, call-outs are made
\scriptsize
\begin{verbatim}
to optional scripts that allow a developer to add functionality or
checking. Typically, the hooks allow for a command to be pre-verified
and potentially aborted, and allow for a post-notification after the
operation is done. The hook scripts are found in the
`$GIT_DIR/hooks/` directory, and are enabled by simply
removing the `.sample` suffix from the filename. In earlier versions
of git you had to make them executable.
\end{verbatim}
\normalsize

index

A collection of files with stat information, whose contents are stored

as objects. The index is a stored version of your
working tree. Truth be told, it can also contain a second, and even
a third version of a working tree, which are used
when merging.
index entry

The information regarding a particular file, stored in the

index. An index entry can be unmerged, if a
merge was started, but not yet finished (i.e. if
the index contains multiple versions of that file).
master

The default development branch. Whenever you

create a git repository, a branch named
"master" is created, and becomes the active branch. In most
cases, this contains the local development, though that is
purely by convention and is not required.
merge

As a verb: To bring the contents of another

branch (possibly from an external
repository) into the current branch.  In the
case where the merged-in branch is from a different repository,
this is done by first fetching the remote branch
and then merging the result into the current branch.  This
combination of fetch and merge operations is called a
pull.  Merging is performed by an automatic process
that identifies changes made since the branches diverged, and
then applies all those changes together.  In cases where changes
conflict, manual intervention may be required to complete the
merge.
As a noun: unless it is a fast forward, a

successful merge results in the creation of a new commit
representing the result of the merge, and having as
parents the tips of the merged branches.
This commit is referred to as a "merge commit", or sometimes just a
"merge".
object

The unit of storage in git. It is uniquely identified by the

SHA1> of its contents. Consequently, an
object can not be changed.
object database

Stores a set of "objects", and an individual object is
\scriptsize
\begin{verbatim}
identified by its object name. The objects usually
live in `$GIT_DIR/objects/`.
\end{verbatim}
\normalsize

object identifier

Synonym for object name.

object name

The unique identifier of an object. The hash

of the object's contents using the Secure Hash Algorithm
1 and usually represented by the 40 character hexadecimal encoding of
the hash of the object.
object type

One of the identifiers "commit", "tree", "tag" or "blob" describing the

type of an object.
octopus

To merge more than two branches. Also denotes an

intelligent predator.
origin

The default upstream repository. Most projects have

at least one upstream project which they track. By default
'origin' is used for that purpose. New upstream updates
will be fetched into remote tracking branches named
origin/name-of-upstream-branch, which you can see using
"`git branch -r`".
pack

A set of objects which have been compressed into one file (to save space

or to transmit them efficiently).
pack index

The list of identifiers, and other information, of the objects in a

pack, to assist in efficiently accessing the contents of a
pack.
parent

A commit object contains a (possibly empty) list

of the logical predecessor(s) in the line of development, i.e. its
parents.
pickaxe

The term pickaxe refers to an option to the diffcore

routines that help select changes that add or delete a given text
string. With the `--pickaxe-all` option, it can be used to view the full
changeset that introduced or removed, say, a
particular line of text. See git diff.
plumbing

Cute name for core git.

porcelain

Cute name for programs and program suites depending on

core git, presenting a high level access to
core git. Porcelains expose more of a SCM
interface than the plumbing.
pull

Pulling a branch means to fetch it and

merge it.  See also git pull.
push

Pushing a branch means to get the branch's

head ref from a remote repository,
find out if it is a direct ancestor to the branch's local
head ref, and in that case, putting all
objects, which are reachable from the local
head ref, and which are missing from the remote
repository, into the remote
object database, and updating the remote
head ref. If the remote head is not an
ancestor to the local head, the push fails.
reachable

All of the ancestors of a given commit are said to be

"reachable" from that commit. More
generally, one object is reachable from
another if we can reach the one from the other by a chain
that follows tags to whatever they tag,
commits to their parents or trees, and
trees to the trees or blobs
that they contain.
rebase

To reapply a series of changes from a branch to a

different base, and reset the head of that branch
to the result.
ref

A 40-byte hex representation of a SHA1 or a name that
\scriptsize
\begin{verbatim}
denotes a particular object. These may be stored in
`$GIT_DIR/refs/`.
\end{verbatim}
\normalsize

reflog

A reflog shows the local "history" of a ref. In other words,
\scriptsize
\begin{verbatim}
it can tell you what the 3rd last revision in _this_ repository
was, and what was the current state in _this_ repository,
yesterday 9:14pm.  See git reflog for details.
\end{verbatim}
\normalsize

refspec

A "refspec" is used by fetch and

push to describe the mapping between remote
ref and local ref. They are combined with a colon in
the format <src>:<dst>, preceded by an optional plus sign, +.
For example: `git fetch $URL
refs/heads/master:refs/heads/origin` means "grab the master
branch head from the $URL and store
it as my origin branch head". And `git push
$URL refs/heads/master:refs/heads/to-upstream` means "publish my
master branch head as to-upstream branch at $URL". See also
git push.
repository

A collection of refs together with an

object database containing all objects
which are reachable from the refs, possibly
accompanied by meta data from one or more porcelains. A
repository can share an object database with other repositories
via alternates mechanism.
resolve

The action of fixing up manually what a failed automatic

merge left behind.
revision

A particular state of files and directories which was stored in the

object database. It is referenced by a
commit object.
rewind

To throw away part of the development, i.e. to assign the

head to an earlier revision.
SCM

Source code management (tool).

SHA1

Synonym for object name.

shallow repository

A shallow repository has an incomplete

history some of whose commits have parents cauterized away (in other
words, git is told to pretend that these commits do not have the
parents, even though they are recorded in the commit
object). This is sometimes useful when you are interested only in the
recent history of a project even though the real history recorded in the
upstream is much larger. A shallow repository
is created by giving the `--depth` option to git clone, and
its history can be later deepened with git fetch.
symref

Symbolic reference: instead of containing the SHA1

id itself, it is of the format 'ref: refs/some/thing' and when
referenced, it recursively dereferences to this reference.
'HEAD' is a prime example of a symref. Symbolic
references are manipulated with the git symbolic-ref
command.
tag

A ref pointing to a tag or

\scriptsize
\begin{verbatim}commit object. In contrast to a head,
a tag is not changed by a commit. Tags (not
tag objects) are stored in `$GIT_DIR/refs/tags/`. A
git tag has nothing to do with a Lisp tag (which would be
called an object type in git's context). A
tag is most typically used to mark a particular point in the
commit ancestry chain.
\end{verbatim}
\normalsize

tag object

An object containing a ref pointing to

another object, which can contain a message just like a
commit object. It can also contain a (PGP)
signature, in which case it is called a "signed tag object".
topic branch

A regular git branch that is used by a developer to

identify a conceptual line of development. Since branches are very easy
and inexpensive, it is often desirable to have several small branches
that each contain very well defined concepts or small incremental yet
related changes.
tracking branch

A regular git branch that is used to follow changes from

another repository. A tracking
branch should not contain direct modifications or have local commits
made to it. A tracking branch can usually be
identified as the right-hand-side ref in a Pull:
refspec.
tree

Either a working tree, or a tree object together with the dependent

blob and tree objects (i.e. a stored representation of a working tree).
tree object

An object containing a list of file names and modes along

with refs to the associated blob and/or tree objects. A
tree is equivalent to a directory.
tree-ish

A ref pointing to either a commit object, a tree object, or a tag

object pointing to a tag or commit or tree object.
unmerged index

An index which contains unmerged

index entries.
unreachable object

An object which is not reachable from a

branch, tag, or any other reference.
working tree

The tree of actual checked out files. The working tree is

normally equal to the HEAD plus any local changes
that you have made but not yet committed.
