\subsection{Tracking Branches}
A 'tracking branch' in Git is a local branch that is connected to a remote
branch. When you push and pull on that branch, it automatically pushes and
pulls to the remote branch that it is connected with.

Use this if you always pull from the same upstream branch into the new branch,
and if you don't want to use "git pull " explicitly.

The 'git clone' command automatically sets up a 'master' branch that is a
tracking branch for 'origin/master' - the master branch on the cloned
repository.

You can create a tracking branch manually by adding the '--track' option to the
'branch' command in Git.
\lstset{basicstyle=\scriptsize, numbers=none, captionpos=b, tabsize=4}
\begin{lstlisting}[caption=,language={bash},
breaklines=true,label=lst:]
git branch --track experimental origin/experimental
\end{lstlisting}

Then when you run:
\lstset{basicstyle=\scriptsize, numbers=none, captionpos=b, tabsize=4}
\begin{lstlisting}[caption=,language={bash},
breaklines=true,label=lst:]
$ git pull experimental
\end{lstlisting}

It will automatically fetch from 'origin' and merge 'origin/experimental' into
your local 'experimental' branch.

Likewise, when you push to origin, it will push what your 'experimental' points
to to origins 'experimental', without having to specify it.
