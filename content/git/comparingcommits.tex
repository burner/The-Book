\section{Comparing Commits - Git Diff}
You can generate diffs between any two versions of your project using git diff:
\lstset{basicstyle=\scriptsize, numbers=none, captionpos=b, tabsize=4}
\begin{lstlisting}[caption=,language={bash},
breaklines=true,label=lst:]
$ git diff master..test
\end{lstlisting}

That will produce the diff between the tips of the two branches. If you'd
prefer to find the diff from their common ancestor to test, you can use three
dots instead of two:
\lstset{basicstyle=\scriptsize, numbers=none, captionpos=b, tabsize=4}
\begin{lstlisting}[caption=,language={bash},
breaklines=true,label=lst:]
$ git diff master...test
\end{lstlisting}

git diff is an incredibly useful tool for figuring out what has changed between
any two points in your project's history, or to see what people are trying to
introduce in new branches, etc.

\subsection{What you will commit}

You will commonly use git diff for figuring out differences between your last
commit, your index, and your current working directory. A common use is to
simply run
\lstset{basicstyle=\scriptsize, numbers=none, captionpos=b, tabsize=4}
\begin{lstlisting}[caption=,language={bash},
breaklines=true,label=lst:]
$ git diff
\end{lstlisting}

which will show you changes in the working directory that are not yet staged
for the next commit. If you want to see what is staged for the next commit, you
can run
\lstset{basicstyle=\scriptsize, numbers=none, captionpos=b, tabsize=4}
\begin{lstlisting}[caption=,language={bash},
breaklines=true,label=lst:]
$ git diff --cached
\end{lstlisting}

which will show you the difference between the index and your last commit; what
you would be committing if you run "git commit" without the "-a" option.
Lastly, you can run
\lstset{basicstyle=\scriptsize, numbers=none, captionpos=b, tabsize=4}
\begin{lstlisting}[caption=,language={bash},
breaklines=true,label=lst:]
$ git diff HEAD
\end{lstlisting}

which shows changes in the working directory since your last commit; what you
would be committing if you run "git commit -a".

\subsection{More Diff Options}
If you want to see how your current working directory differs from the state of
the project in another branch, you can run something like
\lstset{basicstyle=\scriptsize, numbers=none, captionpos=b, tabsize=4}
\begin{lstlisting}[caption=,language={bash},
breaklines=true,label=lst:]
$ git diff test
\end{lstlisting}

This will show you what is different between your current working directory and
the snapshot on the 'test' branch. You can also limit the comparison to a
specific file or subdirectory by adding a path limiter:
\lstset{basicstyle=\scriptsize, numbers=none, captionpos=b, tabsize=4}
\begin{lstlisting}[caption=,language={bash},
breaklines=true,label=lst:]
$ git diff HEAD -- ./lib 
\end{lstlisting}

That command will show the changes between your current working directory and
the last commit (or, more accurately, the tip of the current branch), limiting
the comparison to files in the 'lib' subdirectory.

If you don't want to see the whole patch, you can add the '--stat' option,
which will limit the output to the files that have changed along with a little
text graph depicting how many lines changed in each file.
\lstset{basicstyle=\scriptsize, numbers=none, captionpos=b, tabsize=4}
\begin{lstlisting}[caption=,language={bash},
breaklines=true,label=lst:]
$>git diff --stat
 layout/book_index_template.html                    |    8 ++-
 text/05_Installing_Git/0_Source.markdown           |   14 ++++++
 text/05_Installing_Git/1_Linux.markdown            |   17 +++++++
 text/05_Installing_Git/2_Mac_104.markdown          |   11 +++++
 text/05_Installing_Git/3_Mac_105.markdown          |    8 ++++
 text/05_Installing_Git/4_Windows.markdown          |    7 +++
 .../1_Getting_a_Git_Repo.markdown                  |    7 +++-
 .../0_ Comparing_Commits_Git_Diff.markdown         |   45 +++++++++++++++++++-
 .../0_ Hosting_Git_gitweb_repoorcz_github.markdown |    4 +-
 9 files changed, 115 insertions(+), 6 deletions(-)
\end{lstlisting}

Sometimes that makes it easier to see overall what has changed, to jog your
memory.
