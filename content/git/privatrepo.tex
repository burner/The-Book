\subsection{Setting Up a Private Repository}
If you need to setup a private repository and want to do so locally, rather
than using a hosted solution, you have a number of options.

\subsubsection{Repo Access over SSH}
Generally, the easiest solution is to simply use Git over SSH. If users already
have ssh accounts on a machine, you can put the git repository anywhere on the
box that they have access to and let them access it over normal ssh logins. For
example, say you have a repository you want to host. You can export it as a
bare repo and then scp it onto your server like so:
\lstset{basicstyle=\scriptsize, numbers=none, captionpos=b, tabsize=4}
\begin{lstlisting}[caption=,language={bash},
breaklines=true,label=lst:]
$ git clone --bare /home/user/myrepo/.git /tmp/myrepo.git
$ scp -r /tmp/myrepo.git myserver.com:/opt/git/myrepo.git
\end{lstlisting}

Then someone else with an ssh account on myserver.com can clone via:
\lstset{basicstyle=\scriptsize, numbers=none, captionpos=b, tabsize=4}
\begin{lstlisting}[caption=,language={bash},
breaklines=true,label=lst:]
$ git clone myserver.com:/opt/git/myrepo.git
\end{lstlisting}

Which will simply prompt them for thier ssh password or use thier public key,
however they have ssh authentication setup.
