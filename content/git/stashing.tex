\subsection{Stashing}
While you are in the middle of working on something complicated, you find an
unrelated but obvious and trivial bug. You would like to fix it before
continuing. You can use git stash to save the current state of your work, and
after fixing the bug (or, optionally after doing so on a different branch and
then coming back), unstash the work-in-progress changes.
\lstset{basicstyle=\scriptsize, numbers=none, captionpos=b, tabsize=4}
\begin{lstlisting}[caption=,language={bash},
breaklines=true,label=lst:]
$ git stash "work in progress for foo feature"
\end{lstlisting}

This command will save your changes away to the stash, and reset your working
tree and the index to match the tip of your current branch. Then you can make
your fix as usual.
\lstset{basicstyle=\scriptsize, numbers=none, captionpos=b, tabsize=4}
\begin{lstlisting}[caption=,language={bash},
breaklines=true,label=lst:]
... edit and test ...
$ git commit -a -m "blorpl: typofix"
\end{lstlisting}

After that, you can go back to what you were working on with git stash apply:
\lstset{basicstyle=\scriptsize, numbers=none, captionpos=b, tabsize=4}
\begin{lstlisting}[caption=,language={bash},
breaklines=true,label=lst:]
$ git stash apply
\end{lstlisting}

\subsubsection{Stash Queue}
You can also use stashing to queue up stashed changes.  If you run 'git stash
list' you can see which stashes you have saved:
\lstset{basicstyle=\scriptsize, numbers=none, captionpos=b, tabsize=4}
\begin{lstlisting}[caption=,language={bash},
breaklines=true,label=lst:]
$>git stash list
stash@{0}: WIP on book: 51bea1d... fixed images
stash@{1}: WIP on master: 9705ae6... changed the browse code to the official repo
\end{lstlisting}

Then you can apply them individually with 'git stash apply stash@{1}'. You can
clear out the list with 'git stash clear'.
