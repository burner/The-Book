\subsection{Everyday GIT With 20 Commands Or So}
Individual Developer (Standalone) commands are essential for anybody who makes
a commit, even for somebody who works alone.  If you work with other people,
you will need commands listed in the [Individual Developer (Participant)
section as well.  People who play the Integrator role need to learn some more
commands in addition to the above.  [Repository Administration] commands are
for system administrators who are responsible for the care and feeding of git
repositories.

\subsubsection{Individual Developer (Standalone)}
A standalone individual developer does not exchange patches with other people,
and works alone in a single repository, using the following commands.

\begin{itemize}
\setlength{\itemsep}{0cm}
\setlength{\parskip}{0cm}
\item git-init(1) to create a new repository.
\item git-show-branch(1) to see where you are.
\item git-log(1) to see what happened.
\item git-checkout(1) and git-branch(1) to switch branches.
\item git-add(1) to manage the index file.
\item git-diff(1) and git-status(1) to see what you are in the middle of doing.
\item git-commit(1) to advance the current branch.
\item git-reset(1) and git-checkout(1) (with pathname parameters) to undo changes.
\item git-merge(1) to merge between local branches.
\item git-rebase(1) to maintain topic branches.
\item git-tag(1) to mark known point.
\end{itemize}

Examples:\\
\lstset{basicstyle=\scriptsize, numbers=none, captionpos=b, tabsize=4}
\begin{lstlisting}[caption=Use a tarball as a starting point for a new repository,language={bash},
breaklines=true,label=lst:useatarballasastartingpointforanewrepository]
$ tar zxf frotz.tar.gz
$ cd frotz
$ git init
$ git add . <1>
$ git commit -m "import of frotz source tree."
$ git tag v2.43 <2>
\end{lstlisting}

\begin{enumerate}
\setlength{\itemsep}{0cm}
\setlength{\parskip}{0cm}
\item add everything under the current directory.
\item make a lightweight, unannotated tag.
\end{enumerate}

\lstset{basicstyle=\scriptsize, numbers=none, captionpos=b, tabsize=4}
\begin{lstlisting}[caption=Create a topic branch and develop,language={bash},
breaklines=true,label=lst:createatopicbranchanddevelop]
$ git checkout -b alsa-audio <1>
$ edit/compile/test
$ git checkout -- curses/ux_audio_oss.c <2>
$ git add curses/ux_audio_alsa.c <3>
$ edit/compile/test
$ git diff HEAD <4>
$ git commit -a -s <5>
$ edit/compile/test
$ git reset --soft HEAD^ <6>
$ edit/compile/test
$ git diff ORIG_HEAD <7>
$ git commit -a -c ORIG_HEAD <8>
$ git checkout master <9>
$ git merge alsa-audio <10>
$ git log --since='3 days ago' <11>
$ git log v2.43.. curses/ <12>
\end{lstlisting}

\begin{enumerate}
\setlength{\itemsep}{0cm}
\setlength{\parskip}{0cm}
\item create a new topic branch.
\item revert your botched changes in curses/ux\_audio\_oss.c.
\item you need to tell git if you added a new file; removal and modification
will be caught if you do git commit -a later.
\item to see what changes you are committing.
\item commit everything as you have tested, with your sign-off.
\item take the last commit back, keeping what is in the working tree.
\item look at the changes since the premature commit we took back.
\item redo the commit undone in the previous step, using the message you
originally wrote.
\item switch to the master branch.
\item merge a topic branch into your master branch.
\item review commit logs; other forms to limit output can be combined and
include --max-count=10 (show 10 commits), --until=2005-12-10, etc.
\item view only the changes that touch what’s in curses/ directory, since v2.43
tag.
\end{enumerate}

\subsubsection{Individual Developer (Participant)}
A developer working as a participant in a group project needs to learn how to
communicate with others, and uses these commands in addition to the ones needed
by a standalone developer.

\begin{itemize}
\setlength{\itemsep}{0cm}
\setlength{\parskip}{0cm}
\item git-clone(1) from the upstream to prime your local repository.
\item git-pull(1) and git-fetch(1) from "origin" to keep up-to-date with the
upstream.
\item git-push(1) to shared repository, if you adopt CVS style shared
repository workflow.
\item git-format-patch(1) to prepare e-mail submission, if you adopt Linux
kernel-style public forum workflow.
\end{itemize}

\paragraph{Clone the upstream and work on it. Feed changes to upstream.}
\lstset{basicstyle=\scriptsize, numbers=none, captionpos=b, tabsize=4}
\begin{lstlisting}[caption=,language={bash},
breaklines=true,label=lst:]
$ git clone git://git.kernel.org/pub/scm/.../torvalds/linux-2.6 my2.6
$ cd my2.6
$ edit/compile/test; git commit -a -s <1>
$ git format-patch origin <2>
$ git pull <3>
$ git log -p ORIG_HEAD.. arch/i386 include/asm-i386 <4>
$ git pull git://git.kernel.org/pub/.../jgarzik/libata-dev.git ALL <5>
$ git reset --hard ORIG_HEAD <6>
$ git gc <7>
$ git fetch --tags <8>
\end{lstlisting}

\begin{enumerate}
\setlength{\itemsep}{0cm}
\setlength{\parskip}{0cm}
\item repeat as needed.
\item extract patches from your branch for e-mail submission.
\item git pull fetches from origin by default and merges into the current
branch.
\item immediately after pulling, look at the changes done upstream since last
time we checked, only in the area we are interested in.
\item fetch from a specific branch from a specific repository and merge.
\item revert the pull.
\item garbage collect leftover objects from reverted pull.
\item from time to time, obtain official tags from the origin and store them under .git/refs/tags/.
\end{enumerate}

\paragraph{Push into another repository.}
\lstset{basicstyle=\scriptsize, numbers=none, captionpos=b, tabsize=4}
\begin{lstlisting}[caption=,language={bash},
breaklines=true,label=lst:]
satellite$ git clone mothership:frotz frotz <1>
satellite$ cd frotz
satellite$ git config --get-regexp '^(remote|branch)\.' <2>
remote.origin.url mothership:frotz
remote.origin.fetch refs/heads/*:refs/remotes/origin/*
branch.master.remote origin
branch.master.merge refs/heads/master
satellite$ git config remote.origin.push \
           master:refs/remotes/satellite/master <3>
satellite$ edit/compile/test/commit
satellite$ git push origin <4>

mothership$ cd frotz
mothership$ git checkout master
mothership$ git merge satellite/master <5>
\end{lstlisting}

\begin{enumerate}
\setlength{\itemsep}{0cm}
\setlength{\parskip}{0cm}
\item mothership machine has a frotz repository under your home directory;
clone from it to start a repository on the satellite machine.
\item clone sets these configuration variables by default. It arranges git pull
to fetch and store the branches of mothership machine to local remotes/origin/*
remote-tracking branches.
\item arrange git push to push local master branch to remotes/satellite/master
branch of the mothership machine.
\item push will stash our work away on remotes/satellite/master remote-tracking
branch on the mothership machine. You could use this as a back-up method.
\item on mothership machine, merge the work done on the satellite machine into
the master branch.
\end{enumerate}

\paragraph{Branch off of a specific tag.}
\lstset{basicstyle=\scriptsize, numbers=none, captionpos=b, tabsize=4}
\begin{lstlisting}[caption=,language={bash},
breaklines=true,label=lst:]
$ git checkout -b private2.6.14 v2.6.14 <1>
$ edit/compile/test; git commit -a
$ git checkout master
$ git format-patch -k -m --stdout v2.6.14..private2.6.14 |
  git am -3 -k <2>
\end{lstlisting}

\begin{enumerate}
\setlength{\itemsep}{0cm}
\setlength{\parskip}{0cm}
\item create a private branch based on a well known (but somewhat behind) tag.
\item forward port all changes in private2.6.14 branch to master branch without a formal "merging".
\end{enumerate}

\subsubsection{Integrator}
A fairly central person acting as the integrator in a group project receives
changes made by others, reviews and integrates them and publishes the result
for others to use, using these commands in addition to the ones needed by
participants.
\begin{itemize}
\setlength{\itemsep}{0cm}
\setlength{\parskip}{0cm}
\item git-am(1) to apply patches e-mailed in from your contributors.
\item git-pull(1) to merge from your trusted lieutenants.
\item git-format-patch(1) to prepare and send suggested alternative to contributors.
\item git-revert(1) to undo botched commits.
\item git-push(1) to publish the bleeding edge.
\end{itemize}

A typical git day.
\lstset{basicstyle=\scriptsize, numbers=none, captionpos=b, tabsize=4}
\begin{lstlisting}[caption=,language={bash},
breaklines=true,label=lst:]
$ git status <1>
$ git show-branch <2>
$ mailx <3>
& s 2 3 4 5 ./+to-apply
& s 7 8 ./+hold-linus
& q
$ git checkout -b topic/one master
$ git am -3 -i -s -u ./+to-apply <4>
$ compile/test
$ git checkout -b hold/linus && git am -3 -i -s -u ./+hold-linus <5>
$ git checkout topic/one && git rebase master <6>
$ git checkout pu && git reset --hard next <7>
$ git merge topic/one topic/two && git merge hold/linus <8>
$ git checkout maint
$ git cherry-pick master~4 <9>
$ compile/test
$ git tag -s -m "GIT 0.99.9x" v0.99.9x <10>
$ git fetch ko && git show-branch master maint 'tags/ko-*' <11>
$ git push ko <12>
$ git push ko v0.99.9x <13>
\end{lstlisting}

\begin{enumerate}
\setlength{\itemsep}{0cm}
\setlength{\parskip}{0cm}
\item see what I was in the middle of doing, if any.
\item see what topic branches I have and think about how ready they are.
\item read mails, save ones that are applicable, and save others that are not
quite ready.
\item apply them, interactively, with my sign-offs.
\item create topic branch as needed and apply, again with my sign-offs.
\item rebase internal topic branch that has not been merged to the master, nor
exposed as a part of a stable branch.
\item restart pu every time from the next.
\item and bundle topic branches still cooking.
\item backport a critical fix.
\item create a signed tag.
\item make sure I did not accidentally rewind master beyond what I already
pushed out. ko shorthand points at the repository I have at kernel.org, and
looks like this:
\item push out the bleeding edge.
\item push the tag out, too.
\end{enumerate}

\lstset{basicstyle=\scriptsize, numbers=none, captionpos=b, tabsize=4}
\begin{lstlisting}[caption=,language={bash},
breaklines=true,label=lst:]
$ cat .git/remotes/ko
URL: kernel.org:/pub/scm/git/git.git
Pull: master:refs/tags/ko-master
Pull: next:refs/tags/ko-next
Pull: maint:refs/tags/ko-maint
Push: master
Push: next
Push: +pu
Push: maint
\end{lstlisting}

In the output from git show-branch, master should have everything ko-master
has, and next should have everything ko-next has.
