\section{Everyday GIT With 20 Commands Or So}
Individual Developer (Standalone) commands are essential for anybody who makes
a commit, even for somebody who works alone.  If you work with other people,
you will need commands listed in the [Individual Developer (Participant)
section as well.  People who play the Integrator role need to learn some more
commands in addition to the above.  [Repository Administration] commands are
for system administrators who are responsible for the care and feeding of git
repositories.

\subsection{Individual Developer (Standalone)}
A standalone individual developer does not exchange patches with other people,
and works alone in a single repository, using the following commands.

\begin{itemize}
\setlength{\itemsep}{0cm}
\setlength{\parskip}{0cm}
\item git-init(1) to create a new repository.
\item git-show-branch(1) to see where you are.
\item git-log(1) to see what happened.
\item git-checkout(1) and git-branch(1) to switch branches.
\item git-add(1) to manage the index file.
\item git-diff(1) and git-status(1) to see what you are in the middle of doing.
\item git-commit(1) to advance the current branch.
\item git-reset(1) and git-checkout(1) (with pathname parameters) to undo changes.
\item git-merge(1) to merge between local branches.
\item git-rebase(1) to maintain topic branches.
\item git-tag(1) to mark known point.
\end{itemize}

Examples:\\
Use a tarball as a starting point for a new repository.
\lstset{basicstyle=\scriptsize, numbers=none, captionpos=b, tabsize=4}
\begin{lstlisting}[caption=,language={bash},
breaklines=true,label=lst:]
$ tar zxf frotz.tar.gz
$ cd frotz
$ git init
$ git add . <1>
$ git commit -m "import of frotz source tree."
$ git tag v2.43 <2>
\end{lstlisting}

\begin{enumerate}
\setlength{\itemsep}{0cm}
\setlength{\parskip}{0cm}
\item add everything under the current directory.
\item make a lightweight, unannotated tag.
\end{enumerate}

\lstset{basicstyle=\scriptsize, numbers=none, captionpos=b, tabsize=4}
\begin{lstlisting}[caption=,language={bash},
breaklines=true,label=lst:]
$ git checkout -b alsa-audio <1>
$ edit/compile/test
$ git checkout -- curses/ux_audio_oss.c <2>
$ git add curses/ux_audio_alsa.c <3>
$ edit/compile/test
$ git diff HEAD <4>
$ git commit -a -s <5>
$ edit/compile/test
$ git reset --soft HEAD^ <6>
$ edit/compile/test
$ git diff ORIG_HEAD <7>
$ git commit -a -c ORIG_HEAD <8>
$ git checkout master <9>
$ git merge alsa-audio <10>
$ git log --since='3 days ago' <11>
$ git log v2.43.. curses/ <12>
\end{lstlisting}

\begin{enumerate}
\setlength{\itemsep}{0cm}
\setlength{\parskip}{0cm}
\item create a new topic branch.
\item revert your botched changes in curses/ux\_audio\_oss.c.
\item you need to tell git if you added a new file; removal and modification
will be caught if you do git commit -a later.
\item to see what changes you are committing.
\item commit everything as you have tested, with your sign-off.
\item take the last commit back, keeping what is in the working tree.
\item look at the changes since the premature commit we took back.
\item redo the commit undone in the previous step, using the message you
originally wrote.
\item switch to the master branch.
\item merge a topic branch into your master branch.
\item review commit logs; other forms to limit output can be combined and
include --max-count=10 (show 10 commits), --until=2005-12-10, etc.
\item view only the changes that touch what’s in curses/ directory, since v2.43
tag.
\end{enumerate}

\subsection{Individual Developer (Participant)}
A developer working as a participant in a group project needs to learn how to
communicate with others, and uses these commands in addition to the ones needed
by a standalone developer.

\begin{itemize}
\setlength{\itemsep}{0cm}
\setlength{\parskip}{0cm}
\item git-clone(1) from the upstream to prime your local repository.
\item git-pull(1) and git-fetch(1) from "origin" to keep up-to-date with the
upstream.
\item git-push(1) to shared repository, if you adopt CVS style shared
repository workflow.
\item git-format-patch(1) to prepare e-mail submission, if you adopt Linux
kernel-style public forum workflow.
\end{itemize}

Examples:\\
\lstset{basicstyle=\scriptsize, numbers=none, captionpos=b, tabsize=4}
\begin{lstlisting}[caption=,language={bash},
breaklines=true,label=lst:]
$ git clone git://git.kernel.org/pub/scm/.../torvalds/linux-2.6 my2.6
$ cd my2.6
$ edit/compile/test; git commit -a -s <1>
$ git format-patch origin <2>
$ git pull <3>
$ git log -p ORIG_HEAD.. arch/i386 include/asm-i386 <4>
$ git pull git://git.kernel.org/pub/.../jgarzik/libata-dev.git ALL <5>
$ git reset --hard ORIG_HEAD <6>
$ git gc <7>
$ git fetch --tags <8>
\end{lstlisting}

\begin{enumerate}
\setlength{\itemsep}{0cm}
\setlength{\parskip}{0cm}
\item repeat as needed.
\item extract patches from your branch for e-mail submission.
\item git pull fetches from origin by default and merges into the current
branch.
\item immediately after pulling, look at the changes done upstream since last
time we checked, only in the area we are interested in.
\item fetch from a specific branch from a specific repository and merge.
\item revert the pull.
\item garbage collect leftover objects from reverted pull.
\item from time to time, obtain official tags from the origin and store them under .git/refs/tags/.
\end{enumerate}
