\section{Maintaining Git}
Ensuring good performance

On large repositories, git depends on compression to keep the history
information from taking up too much space on disk or in memory.

This compression is not performed automatically. Therefore you should
occasionally run git gc:
\lstset{basicstyle=\scriptsize, numbers=none, captionpos=b, tabsize=4}
\begin{lstlisting}[caption=,language={bash},
breaklines=true,label=lst:]
$ git gc
\end{lstlisting}

to recompress the archive. This can be very time-consuming, so you may prefer
to run git-gc when you are not doing other work.

\section{Ensuring reliability}
The git fsck command runs a number of self-consistency checks on the
repository, and reports on any problems. This may take some time. The most
common warning by far is about "dangling" objects:
\lstset{basicstyle=\scriptsize, numbers=none, captionpos=b, tabsize=4}
\begin{lstlisting}[caption=,language={bash},
breaklines=true,label=lst:]
$ git fsck
dangling commit 7281251ddd2a61e38657c...
dangling commit 2706a059f258c6b245f29...
dangling commit 13472b7c4b80851a1bc55...
dangling blob 218761f9d90712d37a9c5e3...
dangling commit bf093535a34a4d35731aa...
dangling commit 8e4bec7f2ddaa268bef99...
dangling tree d50bb86186bf27b681d25af...
dangling tree b24c2473f1fd3d91352a624...
...
\end{lstlisting}

Dangling objects are not a problem. At worst they may take up a little extra
disk space. They can sometimes provide a last-resort method for recovering lost
work.
