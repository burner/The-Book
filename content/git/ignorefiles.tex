\subsection{Ignoring files}
A project will often generate files that you do 'not' want to track with git.
This typically includes files generated by a build process or temporary backup
files made by your editor. Of course, 'not' tracking files with git is just a
matter of 'not' calling "git-add" on them. But it quickly becomes annoying to
have these untracked files lying around; e.g. they make "git add ." and "git
commit -a" practically useless, and they keep showing up in the output of "git
status".

You can tell git to ignore certain files by creating a file called .gitignore
in the top level of your working directory, with contents such as:
\lstset{basicstyle=\scriptsize, numbers=none, captionpos=b, tabsize=4}
\begin{lstlisting}[caption=,language={bash},
breaklines=true,label=lst:]
# Lines starting with '#' are considered comments.
# Ignore any file named foo.txt.
foo.txt
# Ignore (generated) html files,
*.html
# except foo.html which is maintained by hand.
!foo.html
# Ignore objects and archives.
*.[oa]
\end{lstlisting}

See gitignore for a detailed explanation of the syntax. You can also place
.gitignore files in other directories in your working tree, and they will apply
to those directories and their subdirectories. The .gitignore files can be
added to your repository like any other files (just run git add .gitignore and
git commit, as usual), which is convenient when the exclude patterns (such as
patterns matching build output files) would also make sense for other users who
clone your repository.

If you wish the exclude patterns to affect only certain repositories (instead
of every repository for a given project), you may instead put them in a file in
your repository named .git/info/exclude, or in any file specified by the
core.excludesfile configuration variable. Some git commands can also take
exclude patterns directly on the command line. See gitignore for the details.
