\subsection{Git and Email}
\subsubsection{Submitting patches to a project}
If you just have a few changes, the simplest way to submit them may just be to
send them as patches in email:

First, use git format-patch; for example:
\lstset{basicstyle=\scriptsize, numbers=none, captionpos=b, tabsize=4}
\begin{lstlisting}[caption=,language={bash},
breaklines=true,label=lst:]
$ git format-patch origin
\end{lstlisting}

will produce a numbered series of files in the current directory, one for each
patch in the current branch but not in origin/HEAD.

You can then import these into your mail client and send them by hand. However,
if you have a lot to send at once, you may prefer to use the git send-email
script to automate the process. Consult the mailing list for your project first
to determine how they prefer such patches be handled.

\subsubsection{Importing patches to a project}
Git also provides a tool called git am (am stands for "apply mailbox"), for
importing such an emailed series of patches. Just save all of the
patch-containing messages, in order, into a single mailbox file, say
"patches.mbox", then run
\lstset{basicstyle=\scriptsize, numbers=none, captionpos=b, tabsize=4}
\begin{lstlisting}[caption=,language={bash},
breaklines=true,label=lst:]
$ git am -3 patches.mbox
\end{lstlisting}

Git will apply each patch in order; if any conflicts are found, it will stop,
and you can manually fix the conflicts and resolve the merge. (The "-3" option
tells git to perform a merge; if you would prefer it just to abort and leave
your tree and index untouched, you may omit that option.)

Once the index is updated with the results of the conflict resolution, instead
of creating a new commit, just run
\lstset{basicstyle=\scriptsize, numbers=none, captionpos=b, tabsize=4}
\begin{lstlisting}[caption=,language={bash},
breaklines=true,label=lst:]
$ git am --resolved
\end{lstlisting}

and git will create the commit for you and continue applying the remaining patches from the mailbox.

The final result will be a series of commits, one for each patch in the original mailbox, with authorship and commit log message each taken from the message containing each patch.
