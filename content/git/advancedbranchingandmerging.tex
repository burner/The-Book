\subsection{Advanced Branching And Merging}
\subsubsection{Getting conflict-resolution help during a merge}

All of the changes that git was able to merge automatically are already added
to the index file, so git diff shows only the conflicts. It uses an unusual
syntax:
\lstset{basicstyle=\scriptsize, numbers=none, captionpos=b, tabsize=4}
\begin{lstlisting}[caption=,language={bash},
breaklines=true,label=lst:]
$ git diff
diff --cc file.txt
index 802992c,2b60207..0000000
--- a/file.txt
+++ b/file.txt
@@@ -1,1 -1,1 +1,5 @@@
++<<<<<<< HEAD:file.txt
 +Hello world
++=======
+ Goodbye
++>>>>>>> 77976da35a11db4580b80ae27e8d65caf5208086:file.txt
\end{lstlisting}

Recall that the commit which will be committed after we resolve this conflict
will have two parents instead of the usual one: one parent will be HEAD, the
tip of the current branch; the other will be the tip of the other branch, which
is stored temporarily in MERGE\_HEAD.

During the merge, the index holds three versions of each file. Each of these
three "file stages" represents a different version of the file:
\lstset{basicstyle=\scriptsize, numbers=none, captionpos=b, tabsize=4}
\begin{lstlisting}[caption=,language={bash},
breaklines=true,label=lst:]
$ git show :1:file.txt  # the file in a common ancestor of both branches
$ git show :2:file.txt  # the version from HEAD.
$ git show :3:file.txt  # the version from MERGE_HEAD.
\end{lstlisting}

When you ask git diff to show the conflicts, it runs a three-way diff between
the conflicted merge results in the work tree with stages 2 and 3 to show only
hunks whose contents come from both sides, mixed (in other words, when a hunk's
merge results come only from stage 2, that part is not conflicting and is not
shown. Same for stage 3).

The diff above shows the differences between the working-tree version of
file.txt and the stage 2 and stage 3 versions. So instead of preceding each
line by a single "+" or "-", it now uses two columns: the first column is used
for differences between the first parent and the working directory copy, and
the second for differences between the second parent and the working directory
copy. (See the "COMBINED DIFF FORMAT" section of git diff-files for a details
of the format.)

After resolving the conflict in the obvious way (but before updating the
index), the diff will look like:
\lstset{basicstyle=\scriptsize, numbers=none, captionpos=b, tabsize=4}
\begin{lstlisting}[caption=,language={bash},
breaklines=true,label=lst:]
$ git diff
diff --cc file.txt
index 802992c,2b60207..0000000
--- a/file.txt
+++ b/file.txt
@@@ -1,1 -1,1 +1,1 @@@
- Hello world
-Goodbye
++Goodbye world
\end{lstlisting}

This shows that our resolved version deleted "Hello world" from the first
parent, deleted "Goodbye" from the second parent, and added "Goodbye world",
which was previously absent from both.

Some special diff options allow diffing the working directory against any of
these stages:
\lstset{basicstyle=\scriptsize, numbers=none, captionpos=b, tabsize=4}
\begin{lstlisting}[caption=,language={bash},
breaklines=true,label=lst:]
$ git diff -1 file.txt      # diff against stage 1
$ git diff --base file.txt  # same as the above
$ git diff -2 file.txt      # diff against stage 2
$ git diff --ours file.txt  # same as the above
$ git diff -3 file.txt      # diff against stage 3
$ git diff --theirs file.txt    # same as the above.
\end{lstlisting}

The git log and gitk commands also provide special help for merges:
\lstset{basicstyle=\scriptsize, numbers=none, captionpos=b, tabsize=4}
\begin{lstlisting}[caption=,language={bash},
breaklines=true,label=lst:]
$ git log --merge
$ gitk --merge
\end{lstlisting}

These will display all commits which exist only on HEAD or on MERGE\_HEAD, and
which touch an unmerged file.

You may also use git mergetool, which lets you merge the unmerged files using
external tools such as emacs or kdiff3.

Each time you resolve the conflicts in a file and update the index:
\lstset{basicstyle=\scriptsize, numbers=none, captionpos=b, tabsize=4}
\begin{lstlisting}[caption=,language={bash},
breaklines=true,label=lst:]
$ git add file.txt
\end{lstlisting}

the different stages of that file will be "collapsed", after which git-diff
will (by default) no longer show diffs for that file.

\subsubsection{Multiway Merge}
You can merge several heads at one time by simply listing them on the same git
merge command. For instance,
\lstset{basicstyle=\scriptsize, numbers=none, captionpos=b, tabsize=4}
\begin{lstlisting}[caption=,language={bash},
breaklines=true,label=lst:]
$ git merge scott/master rick/master tom/master
\end{lstlisting}

is the equivalent of:
\lstset{basicstyle=\scriptsize, numbers=none, captionpos=b, tabsize=4}
\begin{lstlisting}[caption=,language={bash},
breaklines=true,label=lst:]
$ git merge scott/master
$ git merge rick/master
$ git merge tom/master
\end{lstlisting}

\subsubsection{Subtree}
There are situations where you want to include contents in your project from an
independently developed project. You can just pull from the other project as
long as there are no conflicting paths.

The problematic case is when there are conflicting files. Potential candidates
are Makefiles and other standard filenames. You could merge these files but
probably you do not want to. A better solution for this problem can be to merge
the project as its own subdirectory. This is not supported by the recursive
merge strategy, so just pulling won't work.

What you want is the subtree merge strategy, which helps you in such a
situation.

In this example, let's say you have the repository at /path/to/B (but it can be
an URL as well, if you want). You want to merge the master branch of that
repository to the dir-B subdirectory in your current branch.

Here is the command sequence you need:
\lstset{basicstyle=\scriptsize, numbers=none, captionpos=b, tabsize=4}
\begin{lstlisting}[caption=,language={bash},
breaklines=true,label=lst:]
$ git remote add -f Bproject /path/to/B (1)
$ git merge -s ours --no-commit Bproject/master (2)
$ git read-tree --prefix=dir-B/ -u Bproject/master (3)
$ git commit -m "Merge B project as our subdirectory" (4)
$ git pull -s subtree Bproject master (5)
\end{lstlisting}

The benefit of using subtree merge is that it requires less administrative
burden from the users of your repository. It works with older (before Git
v1.5.2) clients and you have the code right after clone.

However if you use submodules then you can choose not to transfer the submodule
objects. This may be a problem with the subtree merge.

Also, in case you make changes to the other project, it is easier to submit
changes if you just use submodules.
