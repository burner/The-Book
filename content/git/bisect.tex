\section{Finding Issues - Git Bisect}
Suppose version 2.6.18 of your project worked, but the version at "master"
crashes. Sometimes the best way to find the cause of such a regression is to
perform a brute-force search through the project's history to find the
particular commit that caused the problem. The git bisect command can help you
do this:
\lstset{basicstyle=\scriptsize, numbers=none, captionpos=b, tabsize=4}
\begin{lstlisting}[caption=,language={bash},
breaklines=true,label=lst:]
$ git bisect start
$ git bisect good v2.6.18
$ git bisect bad master
Bisecting: 3537 revisions left to test after this
[65934a9a028b88e83e2b0f8b36618fe503349f8e] BLOCK: Make USB storage depend on SCSI rather than selecting it [try #6]
\end{lstlisting}

If you run "git branch" at this point, you'll see that git has temporarily
moved you to a new branch named "bisect". This branch points to a commit (with
commit id 65934...) that is reachable from "master" but not from v2.6.18.
Compile and test it, and see whether it crashes. Assume it does crash. Then:
\lstset{basicstyle=\scriptsize, numbers=none, captionpos=b, tabsize=4}
\begin{lstlisting}[caption=,language={bash},
breaklines=true,label=lst:]
$ git bisect bad
Bisecting: 1769 revisions left to test after this
[7eff82c8b1511017ae605f0c99ac275a7e21b867] i2c-core: Drop useless bitmaskings
\end{lstlisting}

checks out an older version. Continue like this, telling git at each stage
whether the version it gives you is good or bad, and notice that the number of
revisions left to test is cut approximately in half each time.

After about 13 tests (in this case), it will output the commit id of the guilty
commit. You can then examine the commit with git show, find out who wrote it,
and mail them your bug report with the commit id. Finally, run
\lstset{basicstyle=\scriptsize, numbers=none, captionpos=b, tabsize=4}
\begin{lstlisting}[caption=,language={bash},
breaklines=true,label=lst:]
$ git bisect reset
\end{lstlisting}

to return you to the branch you were on before and delete the temporary
"bisect" branch.

Note that the version which git-bisect checks out for you at each point is just
a suggestion, and you're free to try a different version if you think it would
be a good idea. For example, occasionally you may land on a commit that broke
something unrelated; run
\lstset{basicstyle=\scriptsize, numbers=none, captionpos=b, tabsize=4}
\begin{lstlisting}[caption=,language={bash},
breaklines=true,label=lst:]
$ git bisect visualize
\end{lstlisting}

which will run gitk and label the commit it chose with a marker that says
"bisect". Choose a safe-looking commit nearby, note its commit id, and check it
out with:
\lstset{basicstyle=\scriptsize, numbers=none, captionpos=b, tabsize=4}
\begin{lstlisting}[caption=,language={bash},
breaklines=true,label=lst:]
$ git reset --hard fb47ddb2db...
\end{lstlisting}

then test, run "bisect good" or "bisect bad" as appropriate, and continue.
