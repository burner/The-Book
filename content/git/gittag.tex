\subsection{Git Tag}
\subsubsection{Lightweight Tags}
We can create a tag to refer to a particular commit by running git tag with no
arguments.
\lstset{basicstyle=\scriptsize, numbers=none, captionpos=b, tabsize=4}
\begin{lstlisting}[caption=,language={bash},
breaklines=true,label=lst:]
$ git tag stable-1 1b2e1d63ff
\end{lstlisting}

After that, we can use stable-1 to refer to the commit 1b2e1d63ff.

This creates a "lightweight" tag, basically a branch that never moves. If you
would also like to include a comment with the tag, and possibly sign it
cryptographically, then we can create a tag object instead.

\subsubsection{Tag Objects}
If one of -a, -s, or -u <key-id> is passed, the command creates a tag object,
and requires the tag message. Unless -m or -F is given, an editor is started
for the user to type in the tag message.

When this happens, a new object is added to the Git object database and the tag
ref points to that tag object, rather than the commit itself. The strength of
this is that you can sign the tag, so you can verify that it is the correct
commit later. You can create a tag object like this:
\lstset{basicstyle=\scriptsize, numbers=none, captionpos=b, tabsize=4}
\begin{lstlisting}[caption=,language={bash},
breaklines=true,label=lst:]
$ git tag -a stable-1 1b2e1d63ff
\end{lstlisting}

It is actually possible to tag any object, but tagging commit objects is the
most common. (In the Linux kernel source, the first tag object references a
tree, rather than a commit)

\subsubsection{Signed Tags}
If you have a GPG key setup, you can create signed tags fairly easily. First,
you will probably want to setup your key id in your .git/config or ~.gitconfig
file.
\lstset{basicstyle=\scriptsize, numbers=none, captionpos=b, tabsize=4}
\begin{lstlisting}[caption=,language={bash},
breaklines=true,label=lst:]
[user]
    signingkey = <gpg-key-id>
\end{lstlisting}

You can also set that with
\lstset{basicstyle=\scriptsize, numbers=none, captionpos=b, tabsize=4}
\begin{lstlisting}[caption=,language={bash},
breaklines=true,label=lst:]
$ git config (--global) user.signingkey <gpg-key-id>
\end{lstlisting}

Now you can create a signed tag simply by replacing the -a with a -s.
\lstset{basicstyle=\scriptsize, numbers=none, captionpos=b, tabsize=4}
\begin{lstlisting}[caption=,language={bash},
breaklines=true,label=lst:]
$ git tag -s stable-1 1b2e1d63ff
\end{lstlisting}

If you don't have your GPG key in your config file, you can accomplish the same
thing this way:
\lstset{basicstyle=\scriptsize, numbers=none, captionpos=b, tabsize=4}
\begin{lstlisting}[caption=,language={bash},
breaklines=true,label=lst:]
$ git tag -u <gpg-key-id> stable-1 1b2e1d63ff
\end{lstlisting}
