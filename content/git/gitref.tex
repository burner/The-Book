\subsection{Git References}
Branches, remote-tracking branches, and tags are all references to commits. All
references are named with a slash-separated path name starting with "refs"; the
names we've been using so far are actually shorthand:
\lstset{basicstyle=\scriptsize, numbers=none, captionpos=b, tabsize=4}
\begin{lstlisting}[caption=,language={bash},
breaklines=true,label=lst:]
- The branch "test" is short for "refs/heads/test".
- The tag "v2.6.18" is short for "refs/tags/v2.6.18".
- "origin/master" is short for "refs/remotes/origin/master".
\end{lstlisting}

The full name is occasionally useful if, for example, there ever exists a tag
and a branch with the same name.

(Newly created refs are actually stored in the .git/refs directory, under the
path given by their name. However, for efficiency reasons they may also be
packed together in a single file; see git pack-refs).

As another useful shortcut, the "HEAD" of a repository can be referred to just
using the name of that repository. So, for example, "origin" is usually a
shortcut for the HEAD branch in the repository "origin".

For the complete list of paths which git checks for references, and the order
it uses to decide which to choose when there are multiple references with the
same shorthand name, see the "SPECIFYING REVISIONS" section of git rev-parse.

\subsubsection{Showing commits unique to a given branch}
Suppose you would like to see all the commits reachable from the branch head
named "master" but not from any other head in your repository.

We can list all the heads in this repository with git show-ref:
\lstset{basicstyle=\scriptsize, numbers=none, captionpos=b, tabsize=4}
\begin{lstlisting}[caption=,language={bash},
breaklines=true,label=lst:]
$ git show-ref --heads
e363d73353a9dcf094c59595f3153b7 refs/heads/core-tutorial
4c1bb46f63ee9d6e1772bd047e05bf9 refs/heads/maint
24b2524a059a3414e99f6f44bebc1e7 refs/heads/master
a14dc1aebe09f14c8ecf32010690627 refs/heads/tutorial-2
06626c2f31eaa63d26fc0fd646c8af2 refs/heads/tutorial-fixes
\end{lstlisting}

We can get just the branch-head names, and remove "master", with the help of
the standard utilities cut and grep:
\lstset{basicstyle=\scriptsize, numbers=none, captionpos=b, tabsize=4}
\begin{lstlisting}[caption=,language={bash},
breaklines=true,label=lst:]
$ git show-ref --heads | cut -d' ' -f2 | grep -v '^refs/heads/master'
refs/heads/core-tutorial
refs/heads/maint
refs/heads/tutorial-2
refs/heads/tutorial-fixes
\end{lstlisting}

And then we can ask to see all the commits reachable from master but not from
these other heads:
\lstset{basicstyle=\scriptsize, numbers=none, captionpos=b, tabsize=4}
\begin{lstlisting}[caption=,language={bash},
breaklines=true,label=lst:]
$ gitk master --not $( git show-ref --heads | cut -d' ' -f2 |
                grep -v '^refs/heads/master' )
\end{lstlisting}

Obviously, endless variations are possible; for example, to see all commits
reachable from some head but not from any tag in the repository:
\lstset{basicstyle=\scriptsize, numbers=none, captionpos=b, tabsize=4}
\begin{lstlisting}[caption=,language={bash},
breaklines=true,label=lst:]
$ gitk $( git show-ref --heads ) --not  $( git show-ref --tags )
\end{lstlisting}
