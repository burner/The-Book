\section{Setting Up A Public Repository}
Assume your personal repository is in the directory ~/proj. We first create a
new clone of the repository and tell git-daemon that it is meant to be public:
\lstset{basicstyle=\scriptsize, numbers=none, captionpos=b, tabsize=4}
\begin{lstlisting}[caption=,language={bash},
breaklines=true,label=lst:]
$ git clone --bare ~/proj proj.git
$ touch proj.git/git-daemon-export-ok
\end{lstlisting}

The resulting directory proj.git contains a "bare" git repository--it is just
the contents of the ".git" directory, without any files checked out around it.

Next, copy proj.git to the server where you plan to host the public repository.
You can use scp, rsync, or whatever is most convenient.

\subsection{Exporting a git repository via the git protocol}
This is the preferred method.

If someone else administers the server, they should tell you what directory to
put the repository in, and what git:// URL it will appear at.

Otherwise, all you need to do is start git daemon; it will listen on port 9418.
By default, it will allow access to any directory that looks like a git
directory and contains the magic file git-daemon-export-ok. Passing some
directory paths as git-daemon arguments will further restrict the exports to
those paths.

You can also run git-daemon as an inetd service; see the git daemon man page
for details. (See especially the examples section.)

\subsection{Exporting a git repository via http}
The git protocol gives better performance and reliability, but on a host with a
web server set up, http exports may be simpler to set up.

All you need to do is place the newly created bare git repository in a
directory that is exported by the web server, and make some adjustments to give
web clients some extra information they need:
\lstset{basicstyle=\scriptsize, numbers=none, captionpos=b, tabsize=4}
\begin{lstlisting}[caption=,language={bash},
breaklines=true,label=lst:]
$ mv proj.git /home/you/public_html/proj.git
$ cd proj.git
$ git --bare update-server-info
$ chmod a+x hooks/post-update
\end{lstlisting}

(For an explanation of the last two lines, see git update-server-info and
githooks.)

Advertise the URL of proj.git. Anybody else should then be able to clone or
pull from that URL, for example with a command line like:
\lstset{basicstyle=\scriptsize, numbers=none, captionpos=b, tabsize=4}
\begin{lstlisting}[caption=,language={bash},
breaklines=true,label=lst:]
$ git clone http://yourserver.com/~you/proj.git
\end{lstlisting}
