\section{Normal Workflow}
Modify some files, then add their updated contents to the index:
\lstset{basicstyle=\scriptsize, numbers=none, captionpos=b, tabsize=4}
\begin{lstlisting}[caption=,language={bash},
breaklines=true,label=lst:]
$ git add file1 file2 file3
You are now ready to commit. You can see what is about to be committed using
git diff with the --cached option:
\end{lstlisting}

\lstset{basicstyle=\scriptsize, numbers=none, captionpos=b, tabsize=4}
\begin{lstlisting}[caption=,language={bash},
breaklines=true,label=lst:]
$ git diff --cached
(Without --cached, git diff will show you any changes that you've made but not
yet added to the index.) You can also get a brief summary of the situation with
git status:
\end{lstlisting}

\lstset{basicstyle=\scriptsize, numbers=none, captionpos=b, tabsize=4}
\begin{lstlisting}[caption=,language={bash},
breaklines=true,label=lst:]
$ git status
# On branch master
# Changes to be committed:
#   (use "git reset HEAD <file>..." to unstage)
#
#   modified:   file1
#   modified:   file2
#   modified:   file3
#
\end{lstlisting}

If you need to make any further adjustments, do so now, and then add any newly
modified content to the index. Finally, commit your changes with:
\lstset{basicstyle=\scriptsize, numbers=none, captionpos=b, tabsize=4}
\begin{lstlisting}[caption=,language={bash},
breaklines=true,label=lst:]
$ git commit
\end{lstlisting}

This will again prompt you for a message describing the change, and then record
a new version of the project.

Alternatively, instead of running git add beforehand, you can use
\lstset{basicstyle=\scriptsize, numbers=none, captionpos=b, tabsize=4}
\begin{lstlisting}[caption=,language={bash},
breaklines=true,label=lst:]
$ git commit -a
\end{lstlisting}

which will automatically notice any modified (but not new) files, add them to
the index, and commit, all in one step.

A note on commit messages: Though not required, it's a good idea to begin the
commit message with a single short (less than 50 character) line summarizing
the change, followed by a blank line and then a more thorough description.
Tools that turn commits into email, for example, use the first line on the
Subject: line and the rest of the commit message in the body.

\section{Git tracks content not files}
Many revision control systems provide an "add" command that tells the system to
start tracking changes to a new file. Git's "add" command does something
simpler and more powerful: git add is used both for new and newly modified
files, and in both cases it takes a snapshot of the given files and stages that
content in the index, ready for inclusion in the next commit.
