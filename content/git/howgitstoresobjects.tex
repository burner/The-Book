\subsection{How Git Stores Objects}
This chapter goes into detail about how Git physically stores objects.

All objects are stored as compressed contents by their sha values. They contain
the object type, size and contents in a gzipped format.

There are two formats that Git keeps objects in - loose objects and packed
objects.

\subsubsection{Loose Objects}
Loose objects are the simpler format. It is simply the compressed data stored
in a single file on disk. Every object written to a seperate file.

If the sha of your object is ab04d884140f7b0cf8bbf86d6883869f16a46f65, then the
file will be stored in the following path:
\lstset{basicstyle=\scriptsize, numbers=none, captionpos=b, tabsize=4}
\begin{lstlisting}[caption=,language={bash},
breaklines=true,label=lst:]
GIT_DIR/objects/ab/04d884140f7b0cf8bbf86d6883869f16a46f65
\end{lstlisting}

It pulls the first two characters off and uses that as the subdirectory, so
that there are never too many objects in one directory. The actual file name is
the remaining 38 characters.

The easiest way to describe exactly how the object data is stored is this Ruby
implementation of object storage:
\lstset{basicstyle=\scriptsize, numbers=none, captionpos=b, tabsize=4}
\begin{lstlisting}[caption=,language={ruby},
breaklines=true,label=lst:]
def put_raw_object(content, type)
  size = content.length.to_s

  header = "#{type} #{size}\0" # type(space)size(null byte)
  store = header + content

  sha1 = Digest::SHA1.hexdigest(store)
  path = @git_dir + '/' + sha1[0...2] + '/' + sha1[2..40]

  if !File.exists?(path)
    content = Zlib::Deflate.deflate(store)

    FileUtils.mkdir_p(@directory+'/'+sha1[0...2])
    File.open(path, 'w') do |f|
      f.write content
    end
  end
  return sha1
end
\end{lstlisting}

\subsubsection{Packed Objects}
The other format for object storage is the packfile. Since Git stores each
version of each file as a seperate object, it can get pretty inefficient.
Imagine having a file several thousand lines long and changing a single line.
Git will store the second file in it's entirety, which is a great big waste of
space.

In order to save that space, Git utilizes the packfile. This is a format where
Git will only save the part that has changed in the second file, with a pointer
to the file it is similar to.  When objects are written to disk, it is often in
the loose format, since that format is less expensive to access. However,
eventually you'll want to save the space by packing up the objects - this is
done with the git gc command. It will use a rather complicated heuristic to
determine which files are likely most similar and base the deltas off that
analysis. There can be multiple packfiles, they can be repacked if neccesary
(git repack) or unpacked back into loose files (git unpack-objects) relatively
easily.

Git will also write out an index file for each packfile that is much smaller
and contains offsets into the packfile to more quickly find specific objects by
sha.

The actual details of the packfile implementation are found in the Packfile
chapter a little later on.
