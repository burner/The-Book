\section{Help}
Vim has such a diverse list of commands, keyboard shortcuts, buffers, and so
on. It's impossible to remember how all of them work. In fact, it is not even
useful to know all of them. The best situation is that you know how to look for
certain functionality in Vim whenever you need it. For example, you want to
avoid having to type a long name every time, you suddenly remember there is an
abbreviations feature in Vim that'll help you do just that, but don't remember
how to use it. What do you do? Let's look at the various ways of finding help
about how to use Vim. 

\subsection{The :help command} 
The first and most important place to try to look for help is the built-in
documentation and Vim has one of the most comprehensive user manuals that I've
ever seen. In our case, just run \texttt{:help abbreviation} and you'll be
taken to the help for \texttt{abbreviations} and you can read about how to use
the \texttt{:ab} and \texttt{:iab} commands. Sometimes, it can be as simple as
that. If you don't know what you're looking for, then you can run \texttt{:help
user-manual} and browse through the list of contents of the entire user manual
and read the
chapter that you feel is relevant to what you're trying to do. 

\subsection{How to read the :help topic} 
Let us take some sample text from \texttt{:help abbreviate}:
:ab[breviate] [<expr>] {lhs} {rhs} add abbreviation for {lhs} to {rhs}. If
{lhs} already existed it is replaced with the new {rhs}. {rhs} may contain
spaces. See :map-<expr> for the optional <expr> argument. Notice that there
is a standard way of writing help in Vim to make it easy for us to figure out
the parts that are needed for us instead of trying to understand the whole
command. The first line explains the syntax i.e. how to use this command. The
square brackets in \texttt{:ab[breviate]} indicate that the latter part of the
full name is optional. The minimum you have to type is \texttt{:ab} so that Vim
recognizes the command. You can also use \texttt{:abb} or \texttt{:abbr} or
\texttt{:abbre} and so on till the full name \texttt{:abbreviate}. Most people
tend to use the shortest form possible. The square brackets in
\texttt{[<expr>]} again indicate that the 'expression' is optional. The curly
brackets in \texttt{{lhs} {rhs}} indicate that these are placeholders for
actual arguments to be supplied. The names are short for 'left hand side' and
'right hand side' respectively. Following the first line is an indented
paragraph that briefly explains what this command does. Notice the second
paragraph which points you to further information. You can position the cursor
on the text between the two pipe symbols and press \texttt{ctrl-]} to follow
the "link" to the corresponding \texttt{:help} topic. To jump back, press
\texttt{ctrl-o}. 

\subsection{The :helpgrep command} 
If you do not know what the name of the topic is, then you can search the
entire documentation for a phrase by using \texttt{:helpgrep}. Suppose you want
to know how to look for the beginning of a word, then just run
\texttt{:helpgrep beginning of a word}. You can use \texttt{:cnext} and
\texttt{:cprev} to move to the next and previous part of the documentation
where that phrase occurs. Use \texttt{:clist} to see
the whole list of all the occurrences of the phrase. 

\subsection{Quick help}
Copy the following text into a file in Vim and then also run it: \begin{lstlisting} :let
keywordprg=':help' \end{lstlisting} Now, position your cursor anywhere on the
word \texttt{keywordprg} and just press \texttt{K}. You'll be taken to the help
immediately for that word. This shortcut avoids having to type \texttt{:help
keywordprg}. 

\subsection{Online forum and IRC} 
If you are still not able to figure out
what you want to do, then the next best thing is to approach other Vim users to
help you out. Don't worry, this is actually very easy and it is amazing how
other Vimmers who are willing to help you out. First step is to search the Vim
mailing list to see if someone has already answered your question. Just go to
the [http://tech.groups.yahoo.com/group/vim/msearch\_adv Vim Group search page]
and then enter the keywords of your question. Most of the times, many common
questions will be already answered since this is such a high-traffic mailing
list i.e. lots and lots of people ask questions and give answers in this group.
If you cannot find any relevant answer, then you can visit the Vim IRC forum.
Open an IRC application such as [http://www.silverex.org/download/ XChat]
(available for Windows, Linux, BSD) or [http://colloquy.info/ Colloquy] (for
Mac OS X), connect to the "FreeNode" network, join the \texttt{\#vim} channel and
politely ask your question and wait for a response. Most of the times, someone
will reply in a few minutes. If nobody answers your question, probably they're
all busy, so try again later or try to rephrase your question such that it
makes it easy for someone to help you out. Otherwise, post a message in the
mailing list mentioned above.
