\section{Modes}
We had our first encounter with modes in the \_en:First
Stepsprevious chapter. Now, let us explore this concept further regarding types
of modes available and what we can do in each mode. 

\subsection{Types of modes}
There are three basic modes in Vim  normal, insert and visual. 
\begin{itemize}
\item Normal mode is where you can run commands. This is the default mode in which Vim starts up. 
\item Insert mode is where you insert i.e. write the text. 
\item Visual mode is where you visually select a bunch of text so that you can run a command/operation only on
that part of the text. 
\end{itemize}

\subsection{Normal mode}
By default, you're in normal mode.
Let's see what we can do in this mode. Type \texttt{:echo "hello world"} and
press enter. You should see the famous words \texttt{hello world} echoed back
to you. What you just did was run a Vim command called \texttt{:echo} and you
supplied some text to it which was promptly printed back. Type \texttt{/hello}
and press the enter key. Vim will search for that phrase and will jump to the
first occurrence. This was just two simple examples of the kind of commands
available in the normal mode. We will see many more such commands in later
chapters. 

\subsubsection{How to use the help}
Almost as important as knowing the normal
mode, is knowing how to use the \texttt{:help} command. This is where you learn
more about the commands available in Vim. Remember that you do not need to know
every command available in Vim, it's better to simply know where to find them
when you need them. For example, see \texttt{:help usr\_toc} takes us to the
table of contents of the reference manual. You can see \texttt{:help index} to
search for the particular topic you are interested in, for example, run
\texttt{/insert mode} to see the relevant information regarding insert mode. If
you can't remember these two help topics at first, just press \texttt{F1} or
simply run \texttt{:help}. 

\subsection{Insert mode}
When Vim starts up in normal mode,
we have seen how to use \texttt{i} to get into insert mode. There are other
ways of switching from normal mode to insert mode as well: 
\begin{itemize}
\item Run \texttt{:e dapping.txt} 
\item Press \texttt{i} 
\item Type the following paragraph (including all
the typos and mistakes, we'll correct them later): <blockquote> means being
determined about being determined and being passionate about being passionate
\item Press \texttt{<Esc>} key to switch back to normal mode. 
\item Run \texttt{:w} 
\end{itemize}
Oops, we seem to have missed a word at the beginning of the
line, and our cursor is at the end of the line, what do we do now? What would
be the most efficient way of going to the start of the line and insert the
missing word? Should we use the mouse to move the cursor to the start of the
line? Should we use arrow keys to travel all the way to the start of the line.
Should we press home key and then press \texttt{i} to switch back to insert
mode again? It turns out that the most efficient way to be press \texttt{I}
(upper case I): 
\begin{itemize}
\item Press \texttt{I} 
\item Type \texttt{Dappin} 
\item Press \texttt{<Esc>} key to switch back to the normal mode. 
\end{itemize}
Notice that we used a different key to switch to insert mode, its specialty is
that it moves the cursor to the start of the line and then switches to the
insert mode. Also notice how important it is to ''switch back to the normal
mode as soon as you're done typing the text''. Making this a habit will be
beneficial because most of your work (after the initial writing phase) will be
in the normal mode - that's where the all-important rewriting/editing/polishing
happens. Now, let's take a different variation of the \texttt{i} command.
Notice that pressing \texttt{i} will place your cursor before the current
position and then switch to insert mode. To place the cursor 'a'fter the
current position, press \texttt{a}. 
\begin{itemize}
\item Press \texttt{a} 
\item Type \texttt{g} (to complete the word as "Dapping") 
\item Press \texttt{<Esc>} to switch back to normal mode 
\end{itemize}
Similar to the relationship between \texttt{i} and \texttt{I} keys, there is a
relationship between the \texttt{a} and \texttt{A} keys - if you want to append
text at the end of the line, press the \texttt{A} key. 
\begin{itemize}
\item Press \texttt{A} 
\item Type \texttt{.} (put a dot to complete the sentence properly) 
\item Press \texttt{<Esc>} to switch back to the normal mode 
\end{itemize}

To summarize the four keys we have learnt so far: 
\begin{tabular}{l l} 
\hline 
\texttt{Command} & \texttt{Action} \\  
i & insert text just before the cursor \\ 
I & insert text at the start of the line \\ 
a & append text just after the cursor \\ 
A & append text at the end of the line \\
\hline
\end{tabular}
Notice how the upper case commands are 'bigger' versions of the lower case
commands. Now that we are proficient in quickly moving in the current line,
let's see how to move to new lines. If you want to 'o'pen a new line to start
writing, press the \texttt{o} key. 
\begin{itemize}
\item Press \texttt{o} 
\item Type \texttt{I'm a rapper.} 
\item Press \texttt{<Esc>} to switch
back to the normal mode. 
\end{itemize}

Hmmm, it would be more appealing if that new sentence
we wrote was in a paragraph by itself. 
\begin{itemize}
\item Press \texttt{O} (upper case 'O') 
\item Press \texttt{<Esc>} to switch back to the normal mode. 
\end{itemize}
To summarize the two new keys we just learnt: 

\begin{tabular}{l p{4.5cm}}
\hline 
\texttt{Command} & \texttt{Action} \\  
o & open a new line below \\ 
O & open a new line above \\
\hline
\end{tabular} 
Notice how the upper and lower case 'o' commands are opposite in the direction
in which they open the line. Was there something wrong in the text that we just
wrote? Aah,
it should be 'dapper', not 'rapper'! It's a single character that we have to
change, what's the most efficient way to make this change? We \textit{could} press
\texttt{i} to switch to insert mode, press \texttt{<Del>} key to delete
the \texttt{r}, type \texttt{d} and then press \texttt{<Esc>} to switch
back to the insert mode. But that is four steps for such a simple change! Is
there something better? You can use the \texttt{s} key - s for 's'ubstitute. 
\begin{itemize}
\item Move the cursor to the character \texttt{r} (or simply press \texttt{b} to move
'b'ack to the start of the word) 
\item Press \texttt{s} 
\item Type \texttt{d} 
\item Press \texttt{<Esc>} to switch back to the normal mode 
\end{itemize}
Well, okay, it may not have saved us much right now, but imagine repeating such a process over and
over again throughout the day! Making such a mundane operation as fast as
possible is beneficial because it helps us focus our energies to more creative
and interesting aspects. As Linus Torvalds says, ''"it's not just doing things
faster, but because it is so fast, the way you work dramatically changes."''
Again, there is a bigger version of the \texttt{s} key, \texttt{S} which
substitutes the whole line instead of the current character. 
\begin{itemize}
\item Press \texttt{S}
\item Type \texttt{Be a sinner.} 
\item Press \texttt{<Esc>} to switch back to normal mode. 
\end{itemize}
\begin{tabular}{l l}
\hline 
\texttt{Command} & \texttt{Action} \\ 
s & substitute the current character \\ 
S & substitute the current line \\
\hline
\end{tabular} 
Let's go back our last action... Can't we make it more efficient since we want to
'r'eplace just a single character? Yes, we can use the \texttt{r} key. 
\begin{itemize}
\item Move the cursor to the first character of the word \texttt{sinner}. 
\item Press \texttt{r} 
\item Type \texttt{d} Notice we're already back in the normal mode and
didn't need to press \texttt{<Esc>}. 
\end{itemize}
There's a bigger version of
\texttt{r} called \texttt{R} which will replace continuous characters. 
\begin{itemize}
\item Move the cursor to the 'i' in \texttt{sinner}. 
\item Press \texttt{R} 
\item Type \texttt{app} (the word now becomes 'dapper') 
\item Press \texttt{<Esc>} to switch back to normal mode. 
\end{itemize}
\begin{tabular}{l l}
\hline 
\texttt{Command} & \texttt{Action} \\  
r & replace the current character \\ 
R & replace continuous characters \\
\hline
\end{tabular}
The text should now look like this: <blockquote> Dapping means being determined
about being determined and being passionate about being passionate. Be a
dapper. </blockquote> Phew. We have covered a lot in this chapter, but I
guarantee that this is the only step that is the hardest. Once you understand
this, you've pretty much understood the heart and soul of how Vim works, and
all other functionality in Vim, is just icing on the cake. To repeat,
understanding how modes work and how switching between modes work is the key to
becoming a Vimmer, so if you haven't digested the above examples yet, please
feel free to read them again. Take all the time you need. If you want to read
more specific details about these commands, see \texttt{:help inserting} and
\texttt{:help replacing}. 

\subsection{Visual mode} Suppose that you want to select a
bunch of words and replace them completely with some new text that you want to
write. What do you do? One way would be to use the mouse to click at the start
of the text that you are interested in, hold down the left mouse button, drag
the mouse till the end of the relevant text and then release the left mouse
button. This seems like an awful lot of distraction. We could use the
\texttt{<Del>} or \texttt{<Backspace>} keys to delete all the
characters, but this seems even worse in efficiency. The most efficient way
would be to position the cursor at the start of the text, press \texttt{v} to
start the visual mode, use arrow keys or any text movement commands to the move
to the end of the relevant text (for example, press \texttt{5e} to move to the
end of the 5th word counted from the current cursor position) and then press
\texttt{c} to 'c'hange the text. Notice the improvement in efficiency. In this
particular operation (the \texttt{c} command), you'll be put into insert mode
after it is over, so press \texttt{<Esc>} to return to normal mode. The
\texttt{v} command works on a character basis. If you want to operate in terms
of lines, use the upper case \texttt{V}. 
