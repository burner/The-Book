\section{Introduction} 
Once you've written the initial text, editing and rewriting
requires a lot of movement between the various parts of the document. For
example, you're writing a story and you suddenly get an idea for a new plot,
but to develop this plot you need to go back to the part where the protagonist
enters the new city (or something like that)... how do you quickly move around
the text so that you don't lose your train of thought? Let's see a few examples
of how Vim makes this fast. * Want to move the cursor to the next word? Press
\texttt{w}. 
\begin{itemize}
\item Want to move to to the next paragraph? Press \texttt{\}}. 
\item Want
to move to the 3rd occurrence of the letter 'h'? Press \texttt{3fh}. 
\item Want to
move 35 lines downwards? Press \texttt{35j}. 
\item After one of the above
movements, want to jump back to the previous location? Press \texttt{ctrl-o}. 
\item Want to learn how all these work? Let's dive in. 
\end{itemize}
First, open a file called
\texttt{chandrayaan.txt} and type the following text
([http://en.wikipedia.org/wiki/Chandrayaan-1 taken from Wikipedia]):
<blockquote> Chandrayaan-1 is India's first mission to the moon. Launched by
India's national space agency the Indian Space Research Organisation (ISRO).
The unmanned lunar exploration mission includes a lunar orbiter and an
impactor. The spacecraft was launched by a modified version of the PSLV XL on
22 October 2008 from Satish Dhawan Space Centre, Sriharikota, Andhra Pradesh at
06:23 IST (00:52 UTC). The vehicle was successfully inserted into lunar orbit
on 8 November 2008. The Moon Impact Probe was successfully impacted at the
lunar south pole at 20:31 hours on 14 November 2008. The remote sensing
satellite had a mass of 1,380 kilograms (3,042 lb) at launch and 675 kilograms
(1,488 lb) at lunar orbit and carries high resolution remote sensing equipment
for visible, near infrared, and soft and hard X-ray frequencies. Over a
two-year period, it is intended to survey the lunar surface to produce a
complete map of its chemical characteristics and 3-dimensional topography. The
polar regions are of special interest, as they might contain ice. The lunar
mission carries five ISRO payloads and six payloads from other international
space agencies including NASA, ESA, and the Bulgarian Aerospace Agency, which
were carried free of cost.

\subsection{Move your cursor, the Vim way}
The most basic keys that you should use are the 'hjkl' keys. These 4 keys
correspond to the left, down, up and right arrow keys respectively. Notice
these keys are situated directly under your right hand when they are placed on
the \_en:Typing Skills\#Home Row Techniquehome row. But why not use the arrow
keys themselves? The problem is that they are located in a separate location in
the keyboard and it requires as much hand movement as it requires to use a
mouse. Remember, that the right hand fingers should always be placed on
\texttt{jkl;} keys (and the thumb on the space bar). Now, let's see how to use
these 4 keys: :Hjkl.png|thumb|Using h,j,k,l instead of arrow keys
\begin{tabular}{l p{5.8cm}} \hline
h & You have to stretch your index finger (which is on 'j') to the left to press the 'h'. This is the left-most key and signifies going left. \\  
j & The drooping 'j' key signifies going down. \\ 
k & The upward pointing 'k' key signifies going up. \\ 
l & The right-most 'l' key signifies going right. \\ \hline
\end{tabular} 
Note that we can repeat the operation
by prefixing a count. For example, \texttt{2j} will repeat the \texttt{j}
operation \texttt{2} times. Open up the \texttt{chandrayaan.txt} text document
and start practicing these keys: * Position your cursor at the first letter 'C'
of the document. * Press \texttt{2j} and it should skip the current long line,
the blank line and go to the second line i.e. second paragraph. * Press
\texttt{2k} to get back to where we were. Or alternatively, press
\texttt{ctrl-o} to jump back. * Press \texttt{5l} to move 5 characters to the
right. * Press \texttt{5h} to move left by 5 characters. Or alternatively,
press \texttt{ctrl-o} to jump back. Make it a habit to use the \texttt{'hjkl'}
keys instead of the arrow keys. Within a few tries, you'll notice how much
faster you can be using these keys. Similarly, there are more simple keys that
replace the following special movements. Notice that this again is intended to
reduce hand movement. In this particular case, people are prone to searching
and hunting for these special keys, so we can avoid that altogether.
\begin{tabular}{p{3cm} p{3cm}} \hline 
Traditional & Vim \\  
'home' key moves to the start of the line & \texttt{\^} key (think 'anchored to the start') \\ 
'end' key moves to the end of the line & \texttt{\$} key (think 'the buck stops here') \\ 
'pgup' key moves one screen up & \texttt{ctrl-b} which means move one screen 'b'ackward \\  
'pgdn' key moves one screen down & \texttt{ctrl-f} which means move one screen 'f'orward \\ \hline
\end{tabular} 
If you know the absolute line number
that you want to jump to, say line 50, press \texttt{50G} and Vim will jump to
the 50th line. If no number is specified, \texttt{G} will take you to the last
line of the file. How do you get to the top of the file? Simple, press
\texttt{1G}. Notice how a single key can do so much. 
\begin{itemize}
\item Move the cursor to the first line by pressing \texttt{1G}. 
\item Move 20 characters to the right by pressing \texttt{20l}. 
\item Move back to the first character by pressing \texttt{\^}. 
\item Jump to the last character by pressing \texttt{\$}. 
\item Press \texttt{G} to jump to the last line. 
\end{itemize}
What if you wanted to the middle of the
text that is currently being shown in the window? 
\begin{itemize}
\item Press H to jump as 'h'igh as possible (first line of the window) 
\item Press M to jump to the 'm'iddle of the window 
\item Press L to jump as 'l'ow as possible (last line being displayed) 
\end{itemize}
You must have started to notice the emphasis on touch-typing and never having to
move your hands off the main area. That's a good thing. 

\subsection{Words, sentences, paragraphs} 
We have seen how to move by characters and lines. But we tend to
think of our text as words and how we put them together - sentences,
paragraphs, sections, and so on. So, why not move across such text parts i.e.
"text objects"? Let's take the first few words from our sample text:
<blockquote> The polar regions are of special interest, as they might contain
ice. </blockquote> First, let's position the cursor on the first character by
pressing \texttt{\^}. <blockquote> [T]he polar regions are of special interest,
as they might contain ice. </blockquote> Note that we are using the square
brackets to mark the cursor position. Want to move to the next 'w'ord? Press
\texttt{w}. The cursor should now be at the 'p' in 'polar'. The
[p]olar regions are of special interest, as they might contain ice.
</blockquote> How about moving 2 words forward? Just add the prefix count to
'w': \texttt{2w}. <blockquote> The polar regions [a]re of special interest, as
they might contain ice. </blockquote> Similarly, to move to the 'e'nd of the
next word, press \texttt{e}. <blockquote> The polar regions ar[e] of special
interest, as they might contain ice. </blockquote> To move one word 'b'ackward,
press \texttt{b}. By prefixing a count, \texttt{2b} will go back by 2 words.
<blockquote> The polar [r]egions are of special interest, as they might contain
ice. See \texttt{:help word-motions} for details. We have seen
character motions and word motions, let's move on to sentences. <blockquote>
[C]handrayaan-1 is India's first mission to the moon. Launched by India's
national space agency the Indian Space Research Organisation (ISRO). The
unmanned lunar exploration mission includes a lunar orbiter and an impactor.
The spacecraft was launched by a modified version of the PSLV XL on 22 October
2008 from Satish Dhawan Space Centre, Sriharikota, Andhra Pradesh at 06:23 IST
(00:52 UTC). The vehicle was successfully inserted into lunar orbit on 8
November 2008. The Moon Impact Probe was successfully impacted at the lunar
south pole at 20:31 hours on 14 November 2008. </blockquote> Position the
cursor at the first character (\texttt{\^}). To move to the next sentence, press
\texttt{)}. <blockquote> Chandrayaan-1 is India's first mission to the moon.
[L]aunched by India's national space agency the Indian Space Research
Organisation (ISRO). The unmanned lunar exploration mission includes a lunar
orbiter and an impactor. The spacecraft was launched by a modified version of
the PSLV XL on 22 October 2008 from Satish Dhawan Space Centre, Sriharikota,
Andhra Pradesh at 06:23 IST (00:52 UTC). The vehicle was successfully inserted
into lunar orbit on 8 November 2008. The Moon Impact Probe was successfully
impacted at the lunar south pole at 20:31 hours on 14 November 2008.
</blockquote> Isn't that cool? To move to the previous sentence, press
\texttt{(}. Go ahead, try it out and see how fast you can move. Again, you can
prefix a count such as \texttt{3)} to move forward by 3 sentences. Now, use the
whole text and try out moving by paragraphs. Press \texttt{\}} to move to the
next paragraph and \texttt{\{} to move to the previous paragraph. Notice that
the 'bigger' brackets is for the bigger text object. If you had already noticed
this, then congratulations, you have already started to think like a winner,
err, "think like a Vimmer". Again, don't try to \textit{remember} these keys, try to
make it a \textit{habit} such that your fingers naturally use these keys. See
\texttt{:help cursor-motions} for more details. 

\subsection{Make your mark} 
You are writing some text but you suddenly remember that you have to update a
related section in the same document, but you do not want to forget where you
are
currently so that you can come back to this later. What do you do? Normally,
this would mean scrolling to that section, update it, and then scroll back to
where you were. This is a lot of overhead and we may tend to forget where we
were last at. We can do things a bit smarter in Vim. Move the cursor to the 5th
line in the following text (the words by John Lennon). Use \texttt{ma} to
create a mark named 'a'. Move the cursor to wherever you want, for example
\texttt{4j}. I am eagerly awaiting my next disappointment.
Ashleigh Brilliant\\ Every man's memory is his
private literature. Aldous Huxley\\ Life is what happens to you while you're busy making other plans. John Lennon 
\\ Life is really simple, but we insist on making it complicated.
Confucius\\  Do not dwell in the past, do not dream of the
future, concentrate the mind on the present moment. Buddha \\
The more decisions that you are forced to make alone, the more you are aware
of your freedom to choose. Thornton Wilder\\ Press
\texttt{'a} (i.e. single quote followed by the name of the mark) and voila, Vim
jumps (back) to the line where that mark was located. You can use any alphabet
(a-zA-Z) to name a mark which means you can have up to 52 named marks for each
file. 

\subsection{Jump around} 
In the various movements that we have learned, we might
often want to jump back to the previous location or to the next location after
a movement. To do this, simply press \texttt{ctrl-o} to jump to the previous
location and \texttt{ctrl-i} to jump forward to the next location again. 

\subsection{Parts of the text} 
There are various ways you can specify text objects in Vim
so that you can pass them to a command. For example, you want to visually
select a part of the text and then convert the case (from upper to lower or
from lower to upper case) of the text using the \texttt{~} key. Open the
\texttt{dapping.txt} file that we created in previous chapters. Use the various
keys to move to the first letter of the word 'dapper' in the second paragraph.
Hint: Use \texttt{\}}, \texttt{j}, \texttt{w}. Dapping means being
determined about being determined and being passionate about being passionate.
Be a dapper. Press \texttt{v} to start the visual
mode, and press \texttt{ap} to select 'a' 'p'aragraph. Press \texttt{~} to flip
the case of the text. If you want to cancel the selection, simply press
\texttt{<Esc>}. Dapping means being determined about being
determined and being passionate about being passionate. bE A
DAPPER. Other text object mnemonics are \texttt{aw} which means
'a' 'w'ord, \texttt{a"} means a quoted string (like "this is a quoted string"),
\texttt{ab} means 'a' 'b'lock which means anything within a pair of
parentheses, and so on. See \texttt{:help object-motions} and \texttt{:help
text-objects} for more details. 
