\section{First Steps}
\subsection{Graphical or Terminal?} 
The graphical version of Vim has menus at the top of the application as well as
various options accessible via the mouse, but note that this is completely
optional. You can still access all the features of Vim using ''only'' the
keyboard. Why is this important? Because once a person becomes efficient at
typing, using only the keyboard makes the person much faster and less
error-prone, as opposed to using the mouse. This is because the hand movement
required to switch between the keyboard and the mouse is slow and there is a
context switch required in the mind of the person when shifting the hand
between the keyboard and the mouse. If we make it a habit to use the keyboard
as much as possible, you're saving valuable hand movement. Of course, this is
subjective. Some people prefer the mouse and some prefer the keyboard.  I
encourage you to use the keyboard as much as possible to experience the real
power of Vim. 
\subsection{Introduction to Modes} 
Imagine it's a Saturday evening and you're bored of the shows on television.
You want to watch an old favorite
movie instead. So, you ''switch the TV to video mode'' so that it shows what
the DVD player is displaying instead of the cable channels. Note that the
television is still displaying video, but you switch the context on whether you
want to watch a DVD or a live television channel. Similarly, Vim has modes. For
example, Vim has a mode for writing text, a mode for running commands, etc.
They are all related to the main purpose of editing text, but you switch
context on whether you want to simply type stuff or you want to run some
commands on the text. Isn't that simple? Traditionally, the concept of modes is
the most oft-cited reason by beginners on why they find Vim "confusing". I
compare it to riding a bicycle - you'll trip a few times, but once you've got
the hang of it, you'll wonder what the fuss was all about. So why does Vim have
modes? To make things as simple as possible, even though it's usage may seem
"strange" at first. What do I mean by that? Let's take an example - one of the
key goals in Vim is to make it fully accessible from the keyboard without ever
having to need to use a mouse (you can still use the mouse if you want to but
it is strictly optional). In such a scenario, how would you distinguish between
the text you want to write, and the commands that you want to run? Vim's
solution is to have a "normal" mode where you can execute commands and an
"insert" mode where you are simply writing text. You can switch between the two
modes any time. For example, pressing \texttt{i} switches Vim to insert mode,
and pressing \texttt{<Esc>} switches Vim back to normal mode. How do
traditional editors achieve this distinction between commands and writing text?
By using graphical menus and keyboard shortcuts. The problem is that this does
not scale. First of all, if you have hundreds of commands, creating menus for
each of these commands would be insane and confusing. Secondly, customizing how
to use each of these commands would be even more difficult. Let's take a
specific example. Suppose you want to change all occurrences of the word "from"
to the word "to" in a document. In a traditional editor, you can run a menu
command like Edit -> Replace (or use a keyboard shortcut like Ctrl-R) and then
enter the 'from' word and the 'to' word and then click on 'Replace'. Then,
check the 'Replace All' option. In Vim, you simply run \texttt{:\%s/from/to/g}
in the normal mode. The \texttt{:s} is the "substitute" command. See how
simpler this is? What if you want to now run this substitution only in the
first 10 lines of the text and you want to have a yes/no confirmation for each
replacement? In traditional text editors, achieving the yes/no confirmation is
easy by unchecking the 'Replace All' option, but notice that you have to first
search for this option and then use the mouse to click on the option (or use a
long series of keys using the keyboard). But how will you run the Replace for
only the first 10 lines? In Vim, you can simply run \texttt{:0,10s/from/to/gc}.
The new \texttt{c} option we are using means we want a 'c'onfirmation message
for every replace. By separating the writing (insert) and command (normal)
modes, Vim makes it easy for us to switch the two contexts easily. Notice how
the first steps to using Vim seem a little "weird", a little "strange", but
once you have seen it in action, it starts to make sense. The best part is that
these core concepts will help you to understand all you need to know on how to
use Vim. Since you now understand the difference between normal mode and insert
mode, you can look up the various commands you can run in the normal mode, and
you can immediately start using them. Compare that to learning new commands in
traditional editors which generally means having to read a lot of
documentation, searching a lot of menus, a lot of trial and error or plain
asking someone for help. Personally, I find the names of the modes not
intuitive to beginners. I prefer calling the insert mode as "writing" mode and
the normal mode as "rewriting" mode, but we will stick to the standard Vim
terminology to avoid confusion. ; Note : All commands in the normal mode should
end with the enter key to signal Vim that we have written the full command. So,
when we say run \texttt{:help vim-modes-intro}, it means you should type
\texttt{:help vim-modes-intro} and then press the enter key at the end of the
command. 

\subsection{Writing a file} 
We have seen how to open and close files in Vim,
now let's do something in between, which is, write. 
\#Open Vim.  :firststeps\_open.pngthumb \# Type \texttt{:edit hello.txt} and press the enter
key. :firststeps\_edit.png|thumb \# Press \texttt{i}.
:firststeps\_insert.png|thumb \# Type the text \texttt{Hello World}.
:firststeps\_type.png|thumb \# Press the \texttt{<Esc>} key.
:firststeps\_normal.png|thumb \# Type \texttt{:write} and press the enter key.
:firststeps\_write.png|thumb \# Close Vim by running \texttt{:q}.
:firststeps\_quit.png|thumb Congratulations! You just wrote your first file
using Vim : ). Does this seem like a lot of steps? Yes, it does, ''at
first''. That is because this is the first time we are getting used to opening
Vim, writing a file and closing Vim. You have to keep in mind that this will
only be a minor part of your time compared to the actual time that goes into in
editing the content of the document. Let us see what the above commands do. *
\texttt{:edit hello.txt} or simply \texttt{:e hello.txt} ** This opens a file
for ''e''diting. If the file with the specified name does not exist, it will be
created the first time we "save" the file. * Press \texttt{i} ** This switches
Vim to the insert mode * Type the text \texttt{Hello World} ** This is where
you type the actual text that you want to write. * Press \texttt{<Esc>}
** This escapes Vim back to normal mode * \texttt{:write} or simply \texttt{:w}
** This tells Vim to ''w''rite the text (which is currently stored in the
computer's memory) to the file on the hard disk. This means that whatever we
wrote so far is now permanently stored. * \texttt{:quit} or simply \texttt{:q}
to quit the file in the "window" that we are editing. If there was only one
"window" open, this will also close Vim (Concept of windows will be discussed
in a \_en:Multiplicitylater chapter). Try to repeat this process a few times
with different file names, different text, etc. so that you get used to the
basic set of steps in using Vim. Notice that when you are in insert mode, Vim
displays \texttt{-- INSERT --} at the bottom left corner. When you switch to
normal mode, it will not display anything. This is because normal mode is the
''default'' mode in which Vim runs. Take some time to soak in this information,
this is probably the hardest lesson there is to learn about Vim, the rest is
easy :) And don't worry, help is not too far away. Actually, it's just a
\texttt{:help} command away. For example, run \texttt{:help :edit} and you'll
see the documentation open up. Go ahead, try it. 

\subsection{Summary} 
We have now discussed the basic concepts and usage of Vim. See \texttt{:help
notation} and \texttt{:help keycodes} also. Be sure to understand these
concepts well. Once you start "thinking in Vim", understanding the rest of
Vim's features is easy.
