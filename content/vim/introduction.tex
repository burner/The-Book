\section{What is Vim?} 
Vim is a computer program used for writing any kind of text, whether it is your
shopping list, a book, or software code. What makes Vim special is that it is
one of those few software which is both \textbf{simple and powerful}. Simple means
it is easy to get started with. Simple means that it
has a minimalistic interface that helps you to concentrate on your main task -
writing. Simple means it is built around few core concepts that helps you learn
deeper functionality easily. Powerful means getting things done faster, better
and easier. Powerful means making not-so-simple things possible. Powerful does
not mean it has to be complicated. Powerful means following the paradigm of
\textbf{"Minimal effort. Maximal effect."} 

\subsection{What can Vim do?} 
I can hear you say, "So it's a text editor. What's the big deal anyway?" Well,
a lot. Let's see some random examples to compare Vim with your current choice
of editor. The point of this exercise is for you to answer the question \textit{"How
would I do this in the editor I currently use?"} for each example. Note:
Don't worry too much about the details of the Vim commands here, the point here
is to enlighten you with the possibilities, not to start explaining how these
things work. That is what the rest of the book is for. See table
\ref{tab:vimusageexample1} and \ref{tab:vimusageexample2} for a taste what is
possible with vim.

\begin{table*}[H]
	\begin{tabular}{p{6cm}| p{7.5cm}}
Edit & In Vim \\ \hline
How do you move the cursor down by 7 lines? & Press \texttt{7j} \\ 
How do you delete a word? Yes, a "word". &Press \texttt{dw} \\ 
How do you search the file for the current word that the cursor is currently 
placed on? & Press \texttt{*} \\ 
How to do a find and replace only in lines 50-100? & Run 
\texttt{:50,100s/old/new/g} \\ 
What if you wanted to view two different parts of the same file simultaneously? 
& Run \texttt{:sp} to 'split' the view \\ 
What if you wanted to open a file whose name is in the current document and 
the cursor is placed on that name? & Press \texttt{gf} (which means 'g'o to 
this 'f'ile) \\
What if you wanted to choose a better color scheme of the display? & Run
\texttt{:colorscheme desert} to choose the 'desert' color scheme (my favorite)\\ 
What if you wanted to map the keyboard shortcut \texttt{ctrl-s} to save the
file? & Run \texttt{:nmap <c-s> w:<CR>}. Note that \texttt{<CR>} means a 
'c'arriage 'r'eturn i.e. the enter key. \\ 
What if you wanted to save the current set of open files as well as any settings 
you have changed, so that you can continue editing later? & Run \texttt{:mksession
~/latest\_session.vim}, and open Vim next time as 
\texttt{vim -S ~/latest\_session.vim}. \\ 
What if you wanted to see colors for different parts of your code? & Run 
\texttt{:syntax on}. If it doesn't recognize the language properly, use 
\texttt{:set filetype=Wikipedia}, for example. \\ 
	\caption{Vim usage examples 1}
	\label{tab:vimusageexample1}
\end{tabular}
\end{table*}
\begin{table*}[H]
	\begin{tabular}{p{6cm}| p{7.5cm}}
What if you wanted different parts of your file to be folded so that you can 
concentrate on only one part at a time? & Run \texttt{:set foldmethod=indent} 
assuming your file is properly indented. There are other methods of folding as 
well. \\ 
What if you wanted to open multiple files in tabs?  & Use \texttt{:tabedit
<file>} to open multiple files in "tabs" (just like browser tabs), and
use \texttt{ctrl-pgup}/\texttt{ctrl-pgdn} to switch between the tabs. \\ 
You use some words frequently in your document and wish there was a way that it
could be quickly filled in the next time you use the same word? & Press
\texttt{ctrl-n} and see the list of "completions" for the current word, based
on all the words that you have used in the current document. Alternatively, use
\texttt{:ab mas Maslow's hierarchy of needs} to expand the abbreviation
automatically when you type \texttt{m a s <space>}. \\ 
You have some data where only the first 10 characters in each line are useful
and the rest is no longer useful for you. How do you get only that data? &
Press \texttt{ctrl-v}, select the text and press \texttt{y} to copy the
selected rows and columns of text. \\ 
What if you received a document from someone which is in all caps, find it
irritating and want to convert it to lower case? & In Vim, run the following:
<pre> :for i in range(0, line('\$')) : call setline(i, tolower(getline(i)))
:endfor </pre> Don't worry, other details will be explored in later chapters. A
more succinct way of doing this would be to run
\texttt{:\%s\#\textbackslash \textbackslash (. \textbackslash \textbackslash ) \# \textbackslash \textbackslash l
\textbackslash \textbackslash 1\#g}
but then again, it's easier to think of the above way of doing it. There is an
even simpler method of selecting all the text (\texttt{1GVG}) and using the
\texttt{u} operator to convert to lowercase, but then again that's too easy,
and isn't as cool as showing off the above way of making Vim do steps
of actions. \\
	\caption{Vim usage examples 2}
	\label{tab:vimusageexample2}
\end{tabular}
\end{table*}
Phew. Are you convinced yet? In these examples, you can see the power of Vim in
action. Any other editor would make it insanely hard to achieve the same level
of functionality. And yet, amazingly, all this power is made as understandable
as possible. Notice that we didn't use the mouse even once during these
examples! This is a good thing.  Count how many times you shift your hand
between the keyboard and the mouse in a single day, and you'll realize why it
is good to avoid it when possible.  Don't be overwhelmed by the features here.
The best part of Vim is that you don't need to know all of these features to be
productive with it, you just need to know a few basic concepts. After learning
those basic concepts, all the other features can be easily learned when you
need them.
