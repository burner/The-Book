\section{Basics}
The nasm assembler is a assembler for intel x86 as well as x86-64. Nasm stands
for netwide assambler. To comment the code we use the C programming language.

\section{The first program}
\lstset{basicstyle=\scriptsize, numbers=left, captionpos=b, tabsize=4}
\begin{lstlisting}[caption=First nasm program,language={[x86masm]Assembler},
xleftmargin=20pt, label=lst:firstNamsProgram]
section .data
section .bss
section .text

global _start

_start:				; void main() {

end:mov eax, 1		; exit(0);
	mov ebx, 9
	int 80h			; }
\end{lstlisting}
All this program does is to exit with a return value of 9. To compile and run
it to the following.
\begin{lstlisting}[language=bash,numbers=none]
nasm -f elf32 -g FILENAME
ld -o OUTPUTFILE FILENAME.o
./OUTPUTFILE
echo $?
\end{lstlisting}
The shell should print 9. You can ignore the lines $1 \dots 3$ right now. Line
5 makes the label \_start visible to the outside so it can be used as startup
function for the program. The interesting parts are $9 \dots 11$. These three
lines move the integer 1 to the eax register. The interger 9 into ebx and than
creates a interrupt. The value in register eax defines which function is
called, value 1 means exit. The value in register ebx is the exit code, this
code is returned to the shell. That means the syntax of nasm is OPERATION
DESTINATION SOURCE. Everything that comes before the : is a label name. So
\textbf{\_start} as well is \textbf{end} mark a label. Labels are point you can
jump to. Everything after a semicolon is a comment. Comments are ignored. 
