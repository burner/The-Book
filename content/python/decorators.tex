\section{Decorators}
Decorator in Python is a syntax sugar for high-level function.  Minimal Example
of property decorator:
\lstset{basicstyle=\scriptsize, numbers=left, captionpos=b, tabsize=4}
\begin{lstlisting}[caption=Minimal Example,language={Python},
xleftmargin=15pt, label=lst:minimalexample]
>>> class Foo(object):
...     @property
...     def bar(self):
...         return 'baz'
...
>>> F = Foo()
>>> print(F.bar)
baz
\end{lstlisting}

The above example is really just a syntax sugar for codes like this:
\lstset{basicstyle=\scriptsize, numbers=left, captionpos=b, tabsize=4}
\begin{lstlisting}[caption=Different Style,language={Python},
xleftmargin=15pt, label=lst:differentstyle]
>>> class Foo(object):
...     def bar(self):
...         return 'baz'
...     bar = property(bar)
...
>>> F = Foo()
>>> print(F.bar)
baz
\end{lstlisting}

Minimal Example of generic decorator:
\lstset{basicstyle=\scriptsize, numbers=left, captionpos=b, tabsize=4}
\begin{lstlisting}[caption=Generic Decorator,language={Python},
xleftmargin=15pt, label=lst:genericdecorator]
>>> def decorator(f):
...     def called(*args, **kargs):
...         print('A function is called somewhere')
...         return f(*args, **kargs)
...     return called
...
>>> class Foo(object):
...     @decorator
...     def bar(self):
...         return 'baz'
...
>>> F = Foo()
>>> print(F.bar())
\end{lstlisting}

\scriptsize
\begin{verbatim}
A function is called somewhere
baz
\end{verbatim}
\normalsize
A good use for the decorators is to allow you to refactor your code so that
common features can be moved into decorators. Consider for example, that you
would like to trace all calls to some functions and print out the values of all
the parameters of the functions for each invocation. Now you can implement this
in a decorator as follows
\lstset{basicstyle=\scriptsize, numbers=left, captionpos=b, tabsize=4}
\begin{lstlisting}[caption=Refactoring,language={Python},
xleftmargin=15pt, label=lst:refactoring]
#define the Trace class that will be 
#invoked using decorators
class Trace(object):
    def __init__(self, f):
        self.f =f
 
    def __call__(self, *args, **kwds):
        print("entering function " + self.f.__name__)
        i=0
        for arg in args:
            print "arg {0}: {1}".format(i, arg)
            i =i+1
 
        return self.f(*args, **kwds)
\end{lstlisting}

Then you can use the decorator on any function that you defined by
\lstset{basicstyle=\scriptsize, numbers=left, captionpos=b, tabsize=4}
\begin{lstlisting}[caption=Trace,language={Python},
xleftmargin=15pt, label=lst:Trace]
@Trace
def sum(a, b):
    print("inside sum")
    return a + b
\end{lstlisting}

On running this code you would see output like
\scriptsize
\begin{verbatim}
>>> sum(3,2)
entering function sum
arg 0: 3
arg 1: 2
inside sum
\end{verbatim}
\normalsize

Alternately instead of creating the decorator as a class you could have use a function as well.
\lstset{basicstyle=\scriptsize, numbers=left, captionpos=b, tabsize=4}
\begin{lstlisting}[caption=Different usage,language={Python},
xleftmargin=15pt, label=lst:differentusage]
def Trace(f):
    def my_f(*args, **kwds):
        print("entering " +  f.__name__)
        result= f(*args, **kwds)
        print("exiting " +  f.__name__)
        return result
    my_f.__name = f.__name__
    my_f.__doc__ = f.__doc__
    return my_f
 
#An example of the trace decorator
@Trace
def sum(a, b):
    print("inside sum")
    return a + b
\end{lstlisting}

if you run this you should see
\scriptsize
\begin{verbatim}
>>> sum(3,2)
entering sum
inside sum
exiting sum
10: 5
\end{verbatim}
\normalsize
remember it is good practice to return the function or a sensible decorated
replacement for the function. So that decorators can be chained.
