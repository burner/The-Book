\section{Internet}
The urllib module which is bundled with python can be used for web interaction.
This module provides a file-like interface for web urls.
\subsection{Getting page text as a string}
An example of reading the contents of a webpage
\lstset{basicstyle=\scriptsize, numbers=left, captionpos=b, tabsize=4}
\begin{lstlisting}[caption=Read Webpage,language={Python},
xleftmargin=15pt, label=lst:readwebpage]
import urllib
pageText = urllib.urlopen("http://www.spam.org/eggs.html").read()
print pageText
\end{lstlisting}

Get and post methods can be used, too.
\lstset{basicstyle=\scriptsize, numbers=left, captionpos=b, tabsize=4}
\begin{lstlisting}[caption=Get and Post,language={Python},
xleftmargin=15pt, label=lst:getandpost]
import urllib
params = urllib.urlencode({"plato":1, "socrates":10, "sophokles":4, "arkhimedes":11})
 
# Using GET method
pageText = urllib.urlopen("http://international-philosophy.com/greece?%s" % params).read()
print pageText
 
# Using POST method
pageText = urllib.urlopen("http://international-philosophy.com/greece", params).read()
print pageText
\end{lstlisting}

\subsection{Downloading files}
To save the content of a page on the internet directly to a file, you can read()
it and save it as a string to a file object, or you can use the urlretrieve
function:
\lstset{basicstyle=\scriptsize, numbers=left, captionpos=b, tabsize=4}
\begin{lstlisting}[caption=Download File,language={Python},
xleftmargin=15pt, label=lst:downloadfile]
import urllib
urllib.urlretrieve("http://upload.wikimedia.org/wikibooks/en/9/91/Python_Programming.pdf", "pythonbook.pdf")
\end{lstlisting}

This will download the file from here and save it to a file "pythonbook.pdf" on
your hard drive.

\subsection{Other functions}
The urllib module includes other functions that may be helpful when writing
programs that use the internet:
\lstset{basicstyle=\scriptsize, numbers=left, captionpos=b, tabsize=4}
\begin{lstlisting}[caption=Urllib Example,language={Python},
xleftmargin=15pt, label=lst:urllibexample]
>>> plain_text = "This isn't suitable for putting in a URL"
>>> print urllib.quote(plain_text)
This%20isn%27t%20suitable%20for%20putting%20in%20a%20URL
>>> print urllib.quote_plus(plain_text)
This+isn%27t+suitable+for+putting+in+a+URL
\end{lstlisting}

The urlencode function, described above converts a dictionary of key-value pairs
into a query string to pass to a URL, the quote and quote\_plus functions encode
normal strings. The quote\_plus function uses plus signs for spaces, for use in
submitting data for form fields. The unquote and unquote\_plus functions do the
reverse, converting urlencoded text to plain text.

\subsection{Email}
With Python, MIME compatible emails can be sent. This requires an installed SMTP
server.
\lstset{basicstyle=\scriptsize, numbers=left, captionpos=b, tabsize=4}
\begin{lstlisting}[caption=Email Example,language={Python},
xleftmargin=15pt, label=lst:emailexample]
import smtplib
from email.mime.text import MIMEText
 
msg = MIMEText( 
"""Hi there,
 
This is a test email message.
 
Greetings""")
 
me  = 'sender@example.com'
you = 'receiver@example.com'
msg['Subject'] = 'Hello!'
msg['From'] =  me
msg['To'] =  you
s = smtplib.SMTP()
s.connect()
s.sendmail(me, [you], msg.as_string())
s.quit()
\end{lstlisting}

This sends the sample message from 'sender@example.com' to
'receiver@example.com'.

\subsection{HTTP Client}
Make a very simple HTTP client
\lstset{basicstyle=\scriptsize, numbers=left, captionpos=b, tabsize=4}
\begin{lstlisting}[caption=HTTP Client,language={Python},
xleftmargin=15pt, label=lst:httpclient]
import socket
s = socket.socket()
s.connect(('localhost', 80))
s.send('GET / HTTP/1.1\nHost:localhost\n\n')
s.recv(40000) # receive 40000 bytes
\end{lstlisting}
