\section{Variables}
A variable is something with a value that may change. In simplest terms, a
variable is just a box that you can put stuff in. You can use variables to store
all kinds of stuff, but for now, we are just going to look at storing numbers in
variables. Here is a program that uses a variable that holds a number:
\lstset{basicstyle=\scriptsize, numbers=left, captionpos=b, tabsize=4}
\begin{lstlisting}[caption=Variable 1,language={Python},
xleftmargin=15pt, label=lst:variable1]
lucky = 7
print (lucky)
\end{lstlisting}
In this program, we created a variable called lucky, and assigned to it the
value of the integer number 7. Then, we just asked Python to tell us what was
stored in the variable lucky, and it did.  In addition to putting things into
variables, we can also change what is inside a variable. For example:

\lstset{basicstyle=\scriptsize, numbers=left, captionpos=b, tabsize=4}
\begin{lstlisting}[caption=Variable 2,language={Python},
xleftmargin=15pt, label=lst:variable2]
changing = 3
print (changing)
 
changing = 9
print (changing)
 
different = 12
print (different)
print (changing)
 
changing = 15
print (changing)
\end{lstlisting}

In this program, we declared a variable called changing, put (using the
assignment symbol =) the number 3 in it, and asked Python to tell us what was in
the changing variable. Then, we put the number 9 into the changing variable, and
asked Python what was in changing now. Python throws away the 3, and replaces it
with the 9, and so it tells us that 9 is currently in changing. Next, we created
a second variable, called different, and put 12 into it. We then ask Python what
is in different and changing, and it tells us 12 and 9, respectively. As you can
see, changing one variable does not make any changes in a different variable.
However, when we change changing once again, from 9 to 15, it once again gets
rid of the old number and puts in the new one, and it tells us that changing is
now holding 15.  You can also assign the value of a variable to be the value of
another variable. For example:
\lstset{basicstyle=\scriptsize, numbers=left, captionpos=b, tabsize=4}
\begin{lstlisting}[caption=Variable 3,language={Python},
xleftmargin=15pt, label=lst:variable3]
red = 5
blue = 10
print (red, blue)
 
yellow = red
print (yellow, red, blue)
 
red = blue
print (yellow, red, blue)
\end{lstlisting}

This program is more complicated. Just keep in mind that we always have the name
of the variable on the left side of the equals sign (the assignment operator),
and the value of the variable on the right side of the equals sign. First the
name, then the value.  We start out declaring that red is 5, and blue is 10. As
you can see, you can use commas in the print statement to tell it to print
multiple things at the same time, separated by a space. When we ask it to,
Python confirms that red is equal to 5, and blue is equal to 10.  Now we create
a third variable, called yellow. To set its value, we tell Python that we want
yellow to be whatever red is. (Remember: name to the left, value to the right.)
Python knows that red is 5, so it also sets yellow to be 5. When we ask Python
to tell us what the variables are, it confirms that yellow is 5, red is 5, and
blue is still 10.
Now we're going to take the red variable, and set it to the value of the blue
variable. Don't get confused - name on the left, value on the right.
Python looks up to see what the value of the blue variable is, and remembers
that it's 10. So, Python throws away red's old value (5), and replaces it with
blue's value. When we ask Python what's going on, it says that yellow is 5, red
is 10, and blue is 10.  But didn't we say that yellow should be whatever value
red is? The reason that yellow is still 5 when red is 10, is because we only
said that yellow should be whatever red is at that instant that we told it to.
Once we have figured out what red is and set that value to yellow, then yellow
doesn't care about red any more. yellow has a value now, and that value is going
to stay the same no matter what happens to red.  That's all you need to know
about variables right now, but we'll get back to them in a minute. Just keep in
mind: whenever you're declaring variables or changing their values, you always
have the name of the variable you're working with to the left of the equals sign
(the assignment operator), and the value you're setting it to on the right.
