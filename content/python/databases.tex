\section{Databases}
Python has some support for working with databases. Modules included with Python
include modules for SQLite and Berkeley DB. Modules for MySQL and PostgreSQL and
others are available as third-party modules. The latter have to be downloaded
and installed before use. The package MySQLdb can be installed, for example,
using the debain package "python-mysqldb".

\subsection{MySQL}
An Example with MySQL would look like this:
\lstset{basicstyle=\scriptsize, numbers=left, captionpos=b, tabsize=4}
\begin{lstlisting}[caption=MySQL example,language={Python},
xleftmargin=15pt, label=lst:mysqlexample]
import MySQLdb
db = MySQLdb.connect("host machine", "dbuser", "password", "dbname")
cursor = db.cursor()
query = """SELECT * FROM sampletable"""
lines = cursor.execute(query)
data = cursor.fetchall()
db.close()
\end{lstlisting}

On the first line, the Module MySQLdb is imported. Then a connection to the
database is set up and on line 4, we save the actual SQL statement to be
executed in the variable query. On line 5 we execute the query and on line 6 we
fetch all the data. After the execution of this piece of code, lines contains
the number of lines fetched (e.g. the number of rows in the table sampletable).
The variable data contains all the actual data, e.g. the content of sampletable.
In the end, the connection to the database would be closed again. If the number
of lines are large, it is better to use row = cursor.fetchone() and process the
rows individually:
\lstset{basicstyle=\scriptsize, numbers=left, captionpos=b, tabsize=4}
\begin{lstlisting}[caption=retrive row,language={Python},
xleftmargin=15pt, label=lst:retriverow]
  #first 5 lines are the same as above
  while True:
    row = cursor.fetchone()
    if row == None: break
    #do something with this row of data
  db.close()
\end{lstlisting}

Obviously, some kind of data processing has to be used on row, otherwise the
data will not be stored. The result of the fetchone() command is a Tuple.  In
order to make the initialization of the connection easier, a configuration file
can be used:
\lstset{basicstyle=\scriptsize, numbers=left, captionpos=b, tabsize=4}
\begin{lstlisting}[caption=Connection file,language={Python},
xleftmargin=15pt, label=lst:connectionfile]
import MySQLdb
db = MySQLdb.connect(read_default_file="~/.my.cnf")
...
Here, the file .my.cnf in the home directory contains the necessary configuration information for MySQL.
\end{lstlisting}

\subsection{Postgres}
\lstset{basicstyle=\scriptsize, numbers=left, captionpos=b, tabsize=4}
\begin{lstlisting}[caption=Postgres Example,language={Python},
xleftmargin=15pt, label=lst:postgresexample]
import psycopg2
conn = psycopg2.connect("dbname=test")
cursor = conn.cursor()
cursor.execute("select * from test");
for i in cursor.next():
    print i
conn.close()
\end{lstlisting}
