\section{Meta-Classes}
In python, classes are themselves objects. Just as other objects are instances
of a particular class, classes themselves are instances of a metaclass.

\subsection{Class Factories}
The simplest use of python metaclasses is a class factory. This concept makes
use of the fact that class definitions in python are first-class objects. Such a
function can create or modify a class definition, using the same syntax one
would normally use in declaring a class definition. Once again, it is useful to
use the model of classes as dictionaries. First, let's look at a basic class
factory:
\lstset{basicstyle=\scriptsize, numbers=left, captionpos=b, tabsize=4}
\begin{lstlisting}[caption=Class Factory,language={Python},
xleftmargin=15pt, label=lst:]
>>> def StringContainer():
...     # define a class
...     class String:
...             content_string = ""
...             def len(self):
...                     return len(self.content_string)
...     # return the class definition
...     return String
...
>>> # create the class definition
... container_class = StringContainer()
>>>
>>> # create an instance of the class
... wrapped_string = container_class()
>>>
>>> # take it for a test drive
... wrapped_string.content_string = 'emu emissary'
>>> wrapped_string.len()
12
\end{lstlisting}

Of course, just like any other data in python, class definitions can also be
modified. Any modifications to attributes in a class definition will be seen in
any instances of that definition, so long as that instance hasn't overridden the
attribute that you're modifying.
\lstset{basicstyle=\scriptsize, numbers=left, captionpos=b, tabsize=4}
\begin{lstlisting}[caption=Attribute Constructor,language={Python},
xleftmargin=15pt, label=lst:attributeconstructor]
>>> def DeAbbreviate(sequence_container):
...     setattr(sequence_container, 'length', sequence_container.len)
...     delattr(sequence_container, 'len')
...
>>> DeAbbreviate(container_class)
>>> wrapped_string.length()
12
>>> wrapped_string.len()
 Traceback (most recent call last):
   File "<stdin>", line 1, in ?
 AttributeError: String instance has no attribute 'len'
\end{lstlisting}

You can also delete class definitions, but that will not affect instances of the
class.
\lstset{basicstyle=\scriptsize, numbers=left, captionpos=b, tabsize=4}
\begin{lstlisting}[caption=Container Class Deletion,language={Python},
xleftmargin=15pt, label=lst:containerclassdeletion]
>>> del container_class
>>> wrapped_string2 = container_class()
Traceback (most recent call last):
  File "<stdin>", line 1, in ?
NameError: name 'container_class' is not defined
>>> wrapped_string.length()
12
\end{lstlisting}

\subsection{The type Metaclass}
The metaclass for all standard python types is the "type" object.
\lstset{basicstyle=\scriptsize, numbers=left, captionpos=b, tabsize=4}
\begin{lstlisting}[caption=Meta Class,language={Python},
xleftmargin=15pt, label=lst:metaclass]
>>> type(object)
<type 'type'>
>>> type(int)
<type 'type'>
>>> type(list)
<type 'type'>
\end{lstlisting}

Just like list, int and object, "type" is itself a normal python object, and is
itself an instance of a class. In this case, it is in fact an instance of
itself.
\lstset{basicstyle=\scriptsize, numbers=left, captionpos=b, tabsize=4}
\begin{lstlisting}[caption=Type of Type,language={Python},
xleftmargin=15pt, label=lst:typeoftype]
>>> type(type)
<type 'type'>
\end{lstlisting}

It can be instantiated to create new class objects similarly to the class
factory example above by passing the name of the new class, the base classes to
inherit from, and a dictionary defining the namespace to use.  For instance, the
code:
\lstset{basicstyle=\scriptsize, numbers=left, captionpos=b, tabsize=4}
\begin{lstlisting}[caption=Meta Class Inheritance,language={Python},
xleftmargin=15pt, label=lst:metaclassinheritance]
>>> class MyClass(BaseClass):
...     attribute = 42
Could also be written as:
>>> MyClass = type("MyClass", (BaseClass,), {'attribute' : 42})
\end{lstlisting}

\subsection{Metaclasses}
It is possible to create a class with a different metaclass than type by setting
its \_\_metaclass\_\_ attribute when defining. When this is done, the class, and its
subclass will be created using your custom metaclass. For example
\lstset{basicstyle=\scriptsize, numbers=left, captionpos=b, tabsize=4}
\begin{lstlisting}[caption=Custom Meta Class,language={Python},
xleftmargin=15pt, label=lst:custommetaclass]
class CustomMetaclass(type):
    def __init__(cls, name, bases, dct):
        print "Creating class %s using CustomMetaclass" % name
        super(CustomMetaclass, cls).__init__(name, bases, dct)
 
class BaseClass(object):
    __metaclass__ = CustomMetaclass
 
class Subclass1(BaseClass):
    pass
\end{lstlisting}
This will print
\scriptsize
\begin{verbatim}
Creating class BaseClass using CustomMetaclass
Creating class Subclass1 using CustomMetaclass
\end{verbatim}
\normalsize
By creating a custom metaclass in this way, it is possible to change how the
class is constructed. This allows you to add or remove attributes and methods,
register creation of classes and subclasses creation and various other
manipulations when the class is created.
