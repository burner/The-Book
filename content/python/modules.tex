	\section{Modules}
	Modules are libraries that can be called from other scripts. For example, a
	popular module is the time module. You can call it using:
	\lstset{basicstyle=\scriptsize, numbers=left, captionpos=b, tabsize=4}
	\begin{lstlisting}[caption=Basic Import,language={Python},
	xleftmargin=15pt, label=lst:basicimport]
	import time
	\end{lstlisting}

	Then, create a new python file, you can name it anything (except time.py, since
	it'd mess up python's module importing, you'll see why later):
	\lstset{basicstyle=\scriptsize, numbers=left, captionpos=b, tabsize=4}
	\begin{lstlisting}[caption=Name imported Module 1,language={Python},
	xleftmargin=15pt, label=lst:nameimportedmodule1]
	import time
	 
	def main():
		#define the variable 'current_time' as a tuple of time.localtime()
		current_time = time.localtime() 
		print(current_time) # print the tuple
		# if the year is 2009 (first value in the current_time tuple)
		if current_time[0] == 2009: 
			print('The year is 2009') # print the year
	 
	if __name__ == '__main__': # if the function is the main function ...
		main() # ...call it
	\end{lstlisting}

	Modules can be called in a various number of ways. For example, we could import
	the time module as t: import time as t \# import the time module and call it 't'
	\lstset{basicstyle=\scriptsize, numbers=left, captionpos=b, tabsize=4}
	\begin{lstlisting}[caption=Name imported Module 2,language={Python},
	xleftmargin=15pt, label=lst:nameimportedmodule2]
	def main():
		current_time = t.localtime() 
		print(current_time)
		if current_time[0] == 2009: 
			print('The year is 2009')
	 
	if __name__ == '__main__': 
		main()
	\end{lstlisting}

	It is not necessary to import the whole module, if you only need a certain
	function or class. To do this, you can do a from-import. Note that a from-import
	would import the name directly into the global namespace, so when invoking the
	imported function, it is unnecessary (and wrong) to call the module again:
\lstset{basicstyle=\scriptsize, numbers=left, captionpos=b, tabsize=4}
\begin{lstlisting}[caption=Select Import,language={Python},
xleftmargin=15pt, label=lst:selectimport]
from time import localtime #1
 
def main():
    current_time = localtime() #2
    print(current_time)
    if current_time[0] == 2009: 
        print 'The year is 2009'
 
if __name__ == '__main__': 
    main()
\end{lstlisting}
it is possible to alias a name imported through from-import

\lstset{basicstyle=\scriptsize, numbers=left, captionpos=b, tabsize=4}
\begin{lstlisting}[caption=Alias Module,language={Python},
xleftmargin=15pt, label=lst:aliasmodule]
from time import localtime as lt 
 
def main():
    current_time = lt()
    print(current_time)
    if current_time[0] == 2009: 
        print('The year is 2009')
 
if __name__ == '__main__': 
    main()
\end{lstlisting}

\subsection{Creating a Module}
\subsubsection{From a File}
The easiest way to create a module by having a file called mymod.py either in a
directory recognized by the PYTHONPATH variable or (even easier) in the same
directory where you are working. If you have the following file mymod.py
\lstset{basicstyle=\scriptsize, numbers=left, captionpos=b, tabsize=4}
\begin{lstlisting}[caption=Module File,language={Python},
xleftmargin=15pt, label=lst:modulefile]
class Object1:
        def __init__(self):
                self.name = 'object 1'
\end{lstlisting}

you can already import this "module" and create instances of the object Object1.
\lstset{basicstyle=\scriptsize, numbers=left, captionpos=b, tabsize=4}
\begin{lstlisting}[caption=Use of selfmade Module,language={Python},
xleftmargin=15pt, label=lst:useofselfmademodule]
import mymod
myobject = mymod.Object1()
from mymod import *
myobject = Object1()
\end{lstlisting}

\subsubsection{From a Directory}
It is not feasible for larger projects to keep all classes in a single file. It
is often easier to store all files in directories and load all files with one
command. Each directory needs to have a \_\_init\_\_.py file which contains python
commands that are executed upon loading the directory.  Suppose we have two more
objects called Object2 and Object3 and we want to load all three objects with
one command. We then create a directory called mymod and we store three files
called Object1.py, Object2.py and Object3.py in it. These files would then
contain one object per file but this not required (although it adds clarity). We
would then write the following \_\_init\_\_.py file:
\lstset{basicstyle=\scriptsize, numbers=left, captionpos=b, tabsize=4}
\begin{lstlisting}[caption=Module from Dir,language={Python},
xleftmargin=15pt, label=lst:modulefromdir]
from Object1 import *
from Object2 import *
from Object3 import *
 
__all__ = ["Object1", "Object2", "Object3"]
\end{lstlisting}

The first three commands tell python what to do when somebody loads the module.
The last statement defining \_\_all\_\_ tells python what to do when somebody
executes from mymod import *. Usually we want to use parts of a module in other
parts of a module, e.g. we want to use Object1 in Object2. We can do this easily
with an from . import * command as the following file Object2.py shows:
\lstset{basicstyle=\scriptsize, numbers=left, captionpos=b, tabsize=4}
\begin{lstlisting}[caption=Import current Dir,language={Python},
xleftmargin=15pt, label=lst:importcurrentdir]
from . import *
 
class Object2:
        def __init__(self):
                self.name = 'object 2'
                self.otherObject = Object1()
\end{lstlisting}
We can now start python and import mymod as we have in the previous section.
