\section{Modules}
Modules are libraries that can be called from other scripts. For example, a
popular module is the time module. You can call it using:
\lstset{basicstyle=\scriptsize, numbers=left, captionpos=b, tabsize=4}
\begin{lstlisting}[caption=Basic Import,language={Python},
xleftmargin=15pt, label=lst:basicimport]
import time
\end{lstlisting}

Then, create a new python file, you can name it anything (except time.py, since
it'd mess up python's module importing, you'll see why later):
\lstset{basicstyle=\scriptsize, numbers=left, captionpos=b, tabsize=4}
\begin{lstlisting}[caption=Name imported Module 1,language={Python},
xleftmargin=15pt, label=lst:nameimportedmodule1]
import time
 
def main():
    #define the variable 'current_time' as a tuple of time.localtime()
    current_time = time.localtime() 
    print(current_time) # print the tuple
    # if the year is 2009 (first value in the current_time tuple)
    if current_time[0] == 2009: 
        print('The year is 2009') # print the year
 
if __name__ == '__main__': # if the function is the main function ...
    main() # ...call it
\end{lstlisting}

Modules can be called in a various number of ways. For example, we could import
the time module as t: import time as t \# import the time module and call it 't'
\lstset{basicstyle=\scriptsize, numbers=left, captionpos=b, tabsize=4}
\begin{lstlisting}[caption=Name imported Module 2,language={Python},
xleftmargin=15pt, label=lst:nameimportedmodule2]
def main():
    current_time = t.localtime() 
    print(current_time)
    if current_time[0] == 2009: 
        print('The year is 2009')
 
if __name__ == '__main__': 
    main()
\end{lstlisting}

It is not necessary to import the whole module, if you only need a certain
function or class. To do this, you can do a from-import. Note that a from-import
would import the name directly into the global namespace, so when invoking the
imported function, it is unnecessary (and wrong) to call the module again:
\lstset{basicstyle=\scriptsize, numbers=left, captionpos=b, tabsize=4}
\begin{lstlisting}[caption=Select Import,language={Python},
xleftmargin=15pt, label=lst:selectimport]
from time import localtime #1
 
def main():
    current_time = localtime() #2
    print(current_time)
    if current_time[0] == 2009: 
        print 'The year is 2009'
 
if __name__ == '__main__': 
    main()
\end{lstlisting}
it is possible to alias a name imported through from-import

\lstset{basicstyle=\scriptsize, numbers=left, captionpos=b, tabsize=4}
\begin{lstlisting}[caption=Alias Module,language={Python},
xleftmargin=15pt, label=lst:aliasmodule]
from time import localtime as lt 
 
def main():
    current_time = lt()
    print(current_time)
    if current_time[0] == 2009: 
        print('The year is 2009')
 
if __name__ == '__main__': 
    main()
\end{lstlisting}
