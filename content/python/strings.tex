\section{Strings}
A string is simply a list of characters in order. A character is anything you
can type on the keyboard in one keystroke, like a letter, a number, or a
backslash. For example, "hello" is a string. It is five characters long
- h, e, l, l, o. Strings can also have spaces: "hello world" contains 11
characters, including the space between "hello" and "world". There are no limits
to the number of characters you can have in a string - you can have
anywhere from one to a million or more. You can even have a string that has 0
characters, which is usually called "the empty string." There are three ways you
can declare a string in Python: single quotes ('), double quotes ("), and triple
quotes ("""). In all cases, you start and end the string with your chosen string
\lstset{basicstyle=\scriptsize, numbers=left, captionpos=b, tabsize=4}
\begin{lstlisting}[caption=String 1,language={Python},
xleftmargin=15pt, label=lst:string1]
declaration. For example:
>>> print ('I am a single quoted string')
I am a single quoted string
>>> print ("I am a double quoted string")
I am a double quoted string
>>> print ("""I am a triple quoted string""")
\end{lstlisting}

I am a triple quoted string You can use quotation marks within strings by
placing a backslash directly before them, so that Python knows you want to
include the quotation marks in the string, instead of ending the string there.
Placing a backslash directly before another symbol like this is known as
escaping the symbol. Note that if you want to put a backslash into the string,
you also have to escape the backslash, to tell Python that you want to include
the backslash, rather than using it as an escape character.
\lstset{basicstyle=\scriptsize, numbers=left,breaklines=true, captionpos=b, tabsize=4}
\begin{lstlisting}[caption=String 2,language={Python},
xleftmargin=15pt, label=lst:string2]
>>> print ("So I said, "You don't know me! You'll never understand me!"")
So I said, "You don't know me! You'll never understand me!"
>>> print ('So I said, "You don't know me! You'll never understand me!"')
So I said, "You don't know me! You'll never understand me!"
>>> print ("This will result in only three backslashes: \\ \\ \\")
This will result in only three backslashes: \ \ \
>>> print ("""The double quotation mark (") is used to indicate direct quotations.""")
\end{lstlisting}

The double quotation mark (") is used to indicate direct quotations.  As you can
see from the above examples, only the specific character used to quote the
string needs to be escaped. This makes for more readable code.  To see how to
use strings, let's go back for a moment to an old, familiar program:
\lstset{basicstyle=\scriptsize, numbers=left, captionpos=b,breaklines=true, tabsize=4}
\begin{lstlisting}[caption=String 3,language={Python},
xleftmargin=15pt, label=lst:string3]
>>> print("Hello, world!")
Hello, world!
\end{lstlisting}

Look at that! You've been using strings since the beginning! You can also add
two strings together using the + operator: this is called concatenating them.
\lstset{basicstyle=\scriptsize, numbers=left, captionpos=b,breaklines=true, tabsize=4}
\begin{lstlisting}[caption=String 4,language={Python},
xleftmargin=15pt, label=lst:string4]
>>> print ("Hello, " + "world!")
Hello, world! 
\end{lstlisting}

Notice that there is a space at the end of the first string. If you don't put
that in, the two words will run together, and you'll end up with Hello,world!
You can also repeat strings by using the * operator, like so:
\lstset{basicstyle=\scriptsize, numbers=left, captionpos=b,breaklines=true, tabsize=4}
\begin{lstlisting}[caption=String 5,language={Python},
xleftmargin=15pt, label=lst:string5]
>>> print ("bouncy, " * 10)
\end{lstlisting}
bouncy, bouncy, bouncy, bouncy, bouncy, bouncy, bouncy, bouncy, bouncy, bouncy,

If you want to find out how long a string is, we use the len() function, which
simply takes a string and counts the number of characters in it. (len stands for
"length.") Just put the string that you want to find the length of, inside the
parentheses of the function. For example:
\lstset{basicstyle=\scriptsize, numbers=left, captionpos=b,breaklines=true, tabsize=4}
\begin{lstlisting}[caption=String 6,language={Python},
xleftmargin=15pt, label=lst:string6]
>>> print (len("Hello, world!"))
13
\end{lstlisting}

\section{Strings and Variables}
Now that you've learned about variables and strings separately, lets see how
they work together.  Variables can store much more than just numbers. You can
also use them to store strings! Here's how:
\lstset{basicstyle=\scriptsize, numbers=left, captionpos=b, tabsize=4}
\begin{lstlisting}[caption=String 7,language={Python},
xleftmargin=15pt, label=lst:string7]
question = "What did you have for lunch?"
print (question)
\end{lstlisting}
In this program, we are creating a variable called question, and storing the
string "What did you have for lunch?" in it. Then, we just tell Python to print
out whatever is inside the question variable. Notice that when we tell Python to
print out question, there are no quotation marks around the word question: this
is to signify that we are using a variable, instead of a string. If we put in
quotation marks around question, Python would treat it as a string, and simply
print out question instead of What did you have for lunch?.  Let's try something
different. Sure, it's all fine and dandy to ask the user what they had for
lunch, but it doesn't make much difference if they can't respond! Let's edit
this program so that the user can type in what they ate.
\lstset{basicstyle=\scriptsize, numbers=left, captionpos=b, tabsize=4}
\begin{lstlisting}[caption=String 8,language={Python},
xleftmargin=15pt, label=lst:string8]
question = "What did you have for lunch?"
print (question)
answer = raw_input()
 
print ("You had " + answer + "! That sounds delicious!")
\end{lstlisting}

To ask the user to write something, we used a function called input(), which
waits until the user writes something and presses enter, and then returns what
the user wrote. Don't forget the parentheses! Even though there's nothing inside
of them, they're still important, and Python will give you an error if you don't
put them in. You can also use a different function called input(), which works
in nearly the same way. We will learn the differences between these two
functions later.

\section{Combining Numbers and Strings}
Take a look at this program, and see if you can figure out what it's supposed to do.
\lstset{basicstyle=\scriptsize, numbers=left, captionpos=b, tabsize=4}
\begin{lstlisting}[caption=String 9,language={Python},
xleftmargin=15pt, label=lst:string9]

print ("Please give me a number: ",)
number = raw_input()
 
plusTen = number + 10
print ("If we add 10 to your number, we get " + plusTen)
\end{lstlisting}

This program should take a number from the user, add 10 to it, and print out the
result. But if you try running it, it won't work! You'll get an error that looks
like this:
\scriptsize
\begin{verbatim}
Traceback (most recent call last):
  File "test.py", line 7, in <module>
    print "If we add 10 to your number, we get " + plusTen
TypeError: cannot concatenate 'str' and 'int' objects
\end{verbatim}
\normalsize

What's going on here? Python is telling us that there is a TypeError, which
means there is a problem with the types of information being used. Specifically,
Python can't figure out how to reconcile the two types of data that are being
used simultaneously: integers and strings. For example, Python thinks that the
number variable is holding a string, instead of a number. If the user enters 15,
then number will contain a string that is two characters long: a 1, followed by
a 5. So how can we tell Python that 15 should be a number, instead of a string?
Also, when printing out the answer, we are telling Python to concatenate
together a string ("If we add 10 to your number, we get ") and a number
(plusTen). Python doesn't know how to do that -- it can only concatenate strings
together. How do we tell Python to treat a number as a string, so that we can
print it out with another string?  Luckily, there are two functions that are
perfect solutions for these problems. The int() function will take a string and
turn it into an integer, while the str() function will take an integer and turn
it into a string. In both cases, we put what we want to change inside the
parentheses. Therefore, our modified program will look like this:
\lstset{basicstyle=\scriptsize, numbers=left, captionpos=b, tabsize=4}
\begin{lstlisting}[caption=String 10,language={Python},
xleftmargin=15pt, label=lst:string10]

print ("Please give me a number:",)
response = raw_input()
 
number = int(response) 
plusTen = number + 10
\end{lstlisting}
 
Another way of doing the same is to add a comma after the string part and then
the number variable, like this:
\lstset{basicstyle=\scriptsize, numbers=left, captionpos=b, tabsize=4}
\begin{lstlisting}[caption=String 11,language={Python},
xleftmargin=15pt, label=lst:string11]
print ("If we add 10 to your number, we get " + str(plusTen))
print ("If we add 10 to your number, we get ", plusTen)
\end{lstlisting}
