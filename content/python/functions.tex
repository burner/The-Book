\section{Function calls}
A callable object is an object that can accept some arguments (also called
parameters) and possibly return an object (often a tuple containing multiple
objects).  A function is the simplest callable object in Python, but there are
others, such as classes or certain class instances.

\subsection{Defining functions}
A function is defined in Python by the following format:
\lstset{basicstyle=\scriptsize, numbers=left, captionpos=b, tabsize=4}
\begin{lstlisting}[caption=Basic Function Usage,language={Python},
xleftmargin=15pt, label=lst:basicfunctionusage]
def functionname(arg1, arg2, ...):
    statement1
    statement2
    ...
>>> def functionname(arg1,arg2):
...     return arg1+arg2
...
>>> t = functionname(24,24) # Result: 48
\end{lstlisting}

If a function takes no arguments, it must still include the parentheses, but
without anything in them:
\lstset{basicstyle=\scriptsize, numbers=left, captionpos=b, tabsize=4}
\begin{lstlisting}[caption=Function Template,language={Python},
xleftmargin=15pt, label=lst:functiontemplate]
def functionname():
    statement1
    statement2
    ...
\end{lstlisting}

The arguments in the function definition bind the arguments passed at function
invocation (i.e. when the function is called), which are called actual
parameters, to the names given when the function is defined, which are called
formal parameters. The interior of the function has no knowledge of the names
given to the actual parameters; the names of the actual parameters may not even
be accessible (they could be inside another function).  A function can 'return'
a value, like so
\lstset{basicstyle=\scriptsize, numbers=left, captionpos=b, tabsize=4}
\begin{lstlisting}[caption=Return values,language={Python},
xleftmargin=15pt, label=lst:returnvalues]
def square(x):
    return x*x
\end{lstlisting}

A function can define variables within the function body, which are considered
'local' to the function. The locals together with the arguments comprise all the
variables within the scope of the function. Any names within the function are
unbound when the function returns or reaches the end of the function body.

\subsection{Declaring Arguments}
\subsubsection{Default Argument Values}
If any of the formal parameters in the function definition are declared with the
format "arg = value," then you will have the option of not specifying a value
for those arguments when calling the function. If you do not specify a value,
then that parameter will have the default value given when the function
executes.
\lstset{basicstyle=\scriptsize, numbers=left, captionpos=b, tabsize=4}
\begin{lstlisting}[caption=Default Arguments,language={Python},
xleftmargin=15pt, label=lst:]
>>> def display_message(message, truncate_after = 4):
...     print(message[:truncate_after])
...
>>> display_message("message")
mess
>>> display_message("message", 6)
messag
\end{lstlisting}

\subsubsection{Variable-Length Argument Lists}
Python allows you to declare two special arguments which allow you to create
arbitrary-length argument lists. This means that each time you call the
function, you can specify any number of arguments above a certain number.
\lstset{basicstyle=\scriptsize, numbers=left, captionpos=b, tabsize=4}
\begin{lstlisting}[caption=Var Args Function,language={Python},
xleftmargin=15pt, label=lst:varargsfunction]
def function(first,second,*remaining):
    statement1
    statement2
    ...
\end{lstlisting}

When calling the above function, you must provide value for each of the first
two arguments. However, since the third parameter is marked with an asterisk,
any actual parameters after the first two will be packed into a tuple and bound
to "remaining."
\lstset{basicstyle=\scriptsize, numbers=left, captionpos=b, tabsize=4}
\begin{lstlisting}[caption=Var Args Usage,language={Python},
xleftmargin=15pt, label=lst:varargsusage]
>>> def print_tail(first,*tail):
...     print tail
...
>>> print_tail(1, 5, 2, "omega")
(5, 2, 'omega')
\end{lstlisting}

If we declare a formal parameter prefixed with two asterisks, then it will be
bound to a dictionary containing any keyword arguments in the actual parameters
which do not correspond to any formal parameters. For example, consider the
function:
\lstset{basicstyle=\scriptsize, numbers=left, captionpos=b, tabsize=4}
\begin{lstlisting}[caption=Formal Parameter,language={Python},
xleftmargin=15pt, label=lst:formalparameter]
def make_dictionary(max_length = 10, **entries):
    return dict([(key, entries[key]) for i, key in enumerate(entries.keys()) if i < max_length])
\end{lstlisting}

If we call this function with any keyword arguments other than max\_length, they
will be placed in the dictionary "entries." If we include the keyword argument
of max\_length, it will be bound to the formal parameter max\_length, as usual.
\lstset{basicstyle=\scriptsize, numbers=left, captionpos=b, tabsize=4}
\begin{lstlisting}[caption=Formal Parameter usage,language={Python},
xleftmargin=15pt, label=lst:formalparameterusage]
>>> make_dictionary(max_length = 2, key1 = 5, key2 = 7, key3 = 9)
{'key3': 9, 'key2': 7}
\end{lstlisting}

\subsubsection{Calling functions}
A function can be called by appending the arguments in parentheses to the
function name, or an empty matched set of parentheses if the function takes no
arguments.
\lstset{basicstyle=\scriptsize, numbers=left, captionpos=b, tabsize=4}
\begin{lstlisting}[caption=Calling Function,language={Python},
xleftmargin=15pt, label=lst:callingfunction]
foo()
square(3)
bar(5, x)
A function's return value can be used by assigning it to a variable, like so:
x = foo()
y = bar(5,x)
\end{lstlisting}

As shown above, when calling a function you can specify the parameters by name
and you can do so in any order
\lstset{basicstyle=\scriptsize, numbers=left, captionpos=b, tabsize=4}
\begin{lstlisting}[caption=Var Args Specific,language={Python},
xleftmargin=15pt, label=lst:varargsspecific]
def display_message(message, start=0, end=4):
   print(message[start:end])
 
display_message("message", end=3)
\end{lstlisting}

This above is valid and start will be the default value of 0. A restriction
placed on this is after the first named argument then all arguments after it
must also be named. The following is not valid
\lstset{basicstyle=\scriptsize, numbers=left, captionpos=b, tabsize=4}
\begin{lstlisting}[caption=Invalid use,language={Python},
xleftmargin=15pt, label=lst:invaliduse]
display_message(end=5, start=1, "my message")
\end{lstlisting}
because the third argument ("my message") is an unnamed argument.

\subsubsection{Closure}
Closure, also known as nested function definition, is a function defined inside
another function. Perhaps best described with an example:
\lstset{basicstyle=\scriptsize, numbers=left, captionpos=b, tabsize=4}
\begin{lstlisting}[caption=Closures,language={Python},
xleftmargin=15pt, label=lst:]
>>> def outer(outer_argument):
...     def inner(inner_argument):
...         return outer_argument + inner_argument
...     return inner
...
>>> f = outer(5)
>>> f(3)
8
>>> f(4)
9
\end{lstlisting}

Closure is possible in python because function is a first-class object, that
means a function is merely an object of type function. Being an object means it
is possible to pass function object (an uncalled function) around as argument or
as return value or to assign another name to the function object. A unique
feature that makes closure useful is that the enclosed function may use the
names defined in the parent function's scope.

\subsubsection{lambda}
lambda is an anonymous (unnamed) function, it is used primarily to write very
short functions that is a hassle to define in the normal way. A function like
this:
\lstset{basicstyle=\scriptsize, numbers=left, captionpos=b, tabsize=4}
\begin{lstlisting}[caption=lambda usecase,language={Python},
xleftmargin=15pt, label=lst:lambdausecase]
>>> def add(a, b):
...    return a + b
...
>>> add(4, 3)
7
\end{lstlisting}

may also be defined using lambda
\lstset{basicstyle=\scriptsize, numbers=left, captionpos=b, tabsize=4}
\begin{lstlisting}[caption=lamdba example,language={Python},
xleftmargin=15pt, label=lst:lamdbaexample]
>>> print( (lambda a, b: a + b)(4, 3) )
7
\end{lstlisting}

Lambda is often used as an argument to other functions that expects a function
object, such as sorted()'s 'key' argument.
\lstset{basicstyle=\scriptsize, numbers=left, captionpos=b, tabsize=4}
\begin{lstlisting}[caption=Sort with lambda,language={Python},
xleftmargin=15pt, label=lst:sortwithlambda]
>>> sorted([[3, 4], [3, 5], [1, 2], [7, 3]], key=lambda x: x[1])
[[1, 2], [7, 3], [3, 4], [3, 5]]
\end{lstlisting}

The lambda form is often useful to be used as closure, such as illustrated in
the following example:
\lstset{basicstyle=\scriptsize, numbers=left, captionpos=b, tabsize=4}
\begin{lstlisting}[caption=lambda Closure,language={Python},
xleftmargin=15pt, label=lst:lambdaclosure]
>>> def attribution(name):
...    return lambda x: x + ' -- ' + name
...
>>> pp = attribution('John')
>>> pp('Dinner is in the fridge')
\end{lstlisting}

'Dinner is in the fridge -- John' note that the lambda function can use the
values of variables from the scope in which it was created (like pre and post).
This is the essence of closure.
