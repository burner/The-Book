\section{Erros}
In python there are two types of errors; syntax errors and exceptions.

\subsection{Syntax errors}
Syntax errors are the most basic type of error. They arise when the Python
parser is unable to understand a line of code. Syntax errors are always fatal,
i.e. there is no way to successfully execute a piece of code containing syntax
errors.

\subsection{Exceptions}
Exceptions arise when the python parser knows what to do with a piece of code
but is unable to perform the action. An example would be trying to access the
internet with python without an internet connection; the python interpreter
knows what to do with that command but is unable to perform it.

\subsection{Dealing with exceptions}
Unlike syntax errors, exceptions are not always fatal. Exceptions can be handled
with the use of a try statement.  Consider the following code to display the
HTML of the website 'goat.com'. When the execution of the program reaches the
try statement it will attempt to perform the indented code following, if for
some reason there is an error (the computer is not connected to the internet or
something) the python interpreter will jump to the indented code below the
'except:' command.
\lstset{basicstyle=\scriptsize, numbers=left, captionpos=b, tabsize=4}
\begin{lstlisting}[caption=try except 1,language={Python},
xleftmargin=15pt, label=lst:tryexcept1]
import urllib2
url = 'http://www.goat.com'
try:
    req = urllib2.Request(url)
    response = urllib2.urlopen(req)
    the_page = response.read()
    print the_page
except:
    print "We have a problem."
\end{lstlisting}

Another way to handle an error is to except a specific error.
\lstset{basicstyle=\scriptsize, numbers=left, captionpos=b, tabsize=4}
\begin{lstlisting}[caption=expect type,language={Python},
xleftmargin=15pt, label=lst:expecttype]
try:
    age = int(raw_input("Enter your age: "))
    print "You must be {0} years old.".format(age)
except ValueError:
    print "Your age must be numeric."
\end{lstlisting}

If the user enters a numeric value as his/her age, the output should look like
this: Enter your age: 5 You must be 5 years old.  However, if the user enters a
non-numeric value as his/her age, a ValueError is thrown when trying to execute
the int() method on a non-numeric string, and the code under the except clause
is executed: Enter your age: five Your age must be a number.  You can also use a
try block with a while loop to validate input:
\lstset{basicstyle=\scriptsize, numbers=left, captionpos=b, tabsize=4}
\begin{lstlisting}[caption=expect type 2,language={Python},
xleftmargin=15pt, label=lst:expecttype2]
valid = False
while valid == False:
    try:
        age = int(raw_input("Enter your age: "))
        valid = True     # This statement will only execute if the above statement executes without error.
        print "You must be {0} years old.".format(age)
    except ValueError:
        print "Your age must be numeric."
\end{lstlisting}
The program will prompt you for your age until you enter a valid age:
\scriptsize
\begin{verbatim}
Enter your age: five
Your age must be numeric.
Enter your age: abc10
Your age must be numeric.
Enter your age: 15
You must be 15 years old.
\end{verbatim}
\normalsize
