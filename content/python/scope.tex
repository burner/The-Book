\section{Scope}
Variables in Python are automatically declared by assignment. Variables are
always references to objects, and are never typed. Variables exist only in the
current scope or global scope. When they go out of scope, the variables are
destroyed, but the objects to which they refer are not (unless the number of
references to the object drops to zero).  Scope is delineated by function and
class blocks. Both functions and their scopes can be nested. So therefore
\lstset{basicstyle=\scriptsize, numbers=left, captionpos=b, tabsize=4}
\begin{lstlisting}[caption=Scope Example,language={Python},
xleftmargin=15pt, label=lst:scopeexample]
def foo():
    def bar():
        x = 5 # x is now in scope
        return x + y # y is defined in the enclosing scope later
    y = 10
    return bar() # now that y is defined, bar's scope includes y
\end{lstlisting}

Now when this code is tested,
\lstset{basicstyle=\scriptsize, numbers=left, captionpos=b, tabsize=4}
\begin{lstlisting}[caption=Scope Error,language={Python},
xleftmargin=15pt, label=lst:scopeerror]
>>> foo()
15
>>> bar()
Traceback (most recent call last):
  File "<pyshell#26>", line 1, in -toplevel-
    bar()
NameError: name 'bar' is not defined
\end{lstlisting}

The name 'bar' is not found because a higher scope does not have access to the
names lower in the hierarchy.  It is a common pitfall to fail to lookup an
attribute (such as a method) of an object (such as a container) referenced by a
variable before the variable is assigned the object. In its most common form:
\lstset{basicstyle=\scriptsize, numbers=left, captionpos=b, tabsize=4}
\begin{lstlisting}[caption=Another Scope Error,language={Python},
xleftmargin=15pt, label=lst:anotherscopeerror]
>>> for x in range(10):
         y.append(x) # append is an attribute of lists
 
Traceback (most recent call last):
  File "<pyshell#46>", line 2, in -toplevel-
    y.append(x)
NameError: name 'y' is not defined
\end{lstlisting}
Here, to correct this problem, one must add y = [] before the for loop.
