\section{Python XML Tools}
Python includes several modules for manipulating xml.

\subsection{xml.sax.handler}
\lstset{basicstyle=\scriptsize, numbers=left, captionpos=b, tabsize=4}
\begin{lstlisting}[caption=sax,language={Python},
xleftmargin=15pt, label=lst:sax]
import xml.sax.handler as saxhandler
import xml.sax as saxparser
 
class MyReport:
    def __init__(self):
        self.Y = 1
 
 
class MyCH(saxhandler.ContentHandler):
    def __init__(self, report):
        self.X = 1
        self.report = report
 
    def startDocument(self):
        print 'startDocument'
 
    def startElement(self, name, attrs):
        print 'Element:', name
 
report = MyReport()          #for future use
ch = MyCH(report)
 
xml = """\
<collection>
  <comic title=\"Sandman\" number='62'>
     <writer>Neil Gaiman</writer>
     <penciller pages='1-9,18-24'>Glyn Dillon</penciller>
     <penciller pages="10-17">Charles Vess</penciller>
  </comic>
</collection>
"""
 
print xml
 
saxparser.parseString(xml, ch)
\end{lstlisting}

\subsection{xml.dom.minidom}
An example of doing RSS feed parsing with DOM from xml.dom import minidom as dom
\lstset{basicstyle=\scriptsize, numbers=left, captionpos=b, tabsize=4}
\begin{lstlisting}[caption=minidom,language={Python},
xleftmargin=15pt, label=lst:minidom]
import urllib2
 
def fetchPage(url):
    a = urllib2.urlopen(url)
    return ''.join(a.readlines())
 
def extract(page):
    a = dom.parseString(page)
    item = a.getElementsByTagName('item')
    for i in item:
        if i.hasChildNodes() == True:
            t = i.getElementsByTagName('title')[0].firstChild.wholeText
            l = i.getElementsByTagName('link')[0].firstChild.wholeText
            d = i.getElementsByTagName('description')[0].firstChild.wholeText
            print t, l, d
 
if __name__=='__main__':
    page = fetchPage("http://rss.slashdot.org/Slashdot/slashdot")
    extract(page)
\end{lstlisting}
