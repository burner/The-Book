\section{Introduction} 
Python is a high-level, structured, open-source
programming language that can be used for a wide variety of programming tasks.
It is good for simple quick-and-dirty scripts, as well as complex and intricate
applications.  It is an interpreted programming language that is automatically
compiled into bytecode before execution (the bytecode is then normally saved to
disk, just as automatically, so that compilation need not happen again until
and unless the source gets changed). It is also a dynamically typed language
that includes (but does not require one to use) object oriented features and
constructs.  The most unusual aspect of Python is that whitespace is
significant; instead of block delimiters (braces $\rightarrow \{\}$ in the C family of
languages), indentation is used to indicate where blocks begin and end.  For
example, the following Python code can be interactively typed at an interpreter
prompt, to display the beginning values in the Fibonacci series:
\lstset{basicstyle=\scriptsize, numbers=left, captionpos=b, tabsize=4}
\begin{lstlisting}[caption=Fibonacci,language={Python},
xleftmargin=15pt, label=lst:fibonacci]
>>> a,b = 0,1
>>> while b < 100:
...   a,b = b,(a+b)
...   print(a, end=" ")
...  1 1 2 3 5 8
13 21 34 55 89 144
\end{lstlisting}
Another interesting aspect in Python is reflection and
introspection. The dir() function returns the list of the names of objects in
the current scope. However, dir(object) will return the names of the attributes
of the specified object. The locals() routine returns a dictionary in which the
names in the local namespace are the keys and their values are the objects to
which the names refer. Combined with the interactive interpreter, this provides
a useful environment for exploration and prototyping.  Python provides a
powerful assortment of built-in types (e.g., lists, dictionaries and strings),
a number of built-in functions, and a few constructs, mostly statements. For
example, loop constructs that can iterate over items in a collection instead of
being limited to a simple range of integer values. Python also comes with a
powerful standard library, which includes hundreds of modules to provide
routines for a wide variety of services including regular expressions and
TCP/IP sessions.  Python is used and supported by a large Python Community that
exists on the Internet. The mailing lists and news groups like the tutor list
actively support and help new python programmers. While they discourage doing
homework for you, they are quite helpful and are populated by the authors of
many of the Python textbooks currently available on the market. It is named
after Monty Python's Flying Circus comedy program, and created by Guido Van
Rossum.

In order to program in Python you need the Python interpreter. If it is not
already installed or if the version you are using is obsolete, you will need to
obtain and install Python
