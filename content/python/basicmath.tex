\section{Basic Math}
Now that we know how to work with numbers and strings, let's write a program
that might actually be useful! Let's say you want to find out how much you weigh
in stone. A concise program can make short work of this task. Since a stone is
14 pounds, and there are about 2.2 pounds in a kilogram, the following formula
should do the trick: $m_{stone} = \frac{m_{kg} \cdot 2.2}{14}$
So, let's turn this formula into a program!
\lstset{basicstyle=\scriptsize, numbers=left, captionpos=b, tabsize=4}
\begin{lstlisting}[caption=Wight in Stone,language={Python},
xleftmargin=15pt, label=lst:wightinstone]
mass_kg = int(raw_input("What is your mass in kilograms?" ))
mass_stone = mass_kg * 2.2 / 14
print("You weigh", mass_stone, "stone.")
\end{lstlisting}
Run this program and get your weight in stone! 

\subsection{Mathematical Operators}
Here are some commonly used mathematical operators.
\begin{tabular}{c c c}
Syntax&	Math& Operation Name\\
\hline
a+b & $a+b$ &addition\\
a-b & $a-b$ &subtraction\\
a*b & $a \cdot b$ & multiplication\\
a/b & $a \div b$ & division (see note below)\\
a//b& $a \div b$ & floor division (e.g. 5/2=2)\\
a\%b& $a\ mod\ b$ & modulo\\
-a& $-a$ &	negation\\
abs(a) &$|a|$ &	absolute value\\
a**b & $a^b$ & exponent\\
math.sqrt(a)& $\sqrt{a}$ &square root \\
\end{tabular}

Beware that due to the limitations of floating point arithmetic, rounding errors
can cause unexpected results. For example:
\lstset{basicstyle=\scriptsize, numbers=left, captionpos=b, tabsize=4}
\begin{lstlisting}[caption=Basic Math 1,language={Python},
xleftmargin=15pt, label=lst:basicmath1]
>>> print(0.6/0.2)
3.0
>>> print(0.6//0.2)
2.0
\end{lstlisting}

\subsection{Order of Operations}
 Python uses the standard order of operations as taught in Algebra and Geometry
classes at high school or secondary school. That is, mathematical expressions
are evaluated in the following order (memorized by many as PEMDAS), which is
also applied to parentheticals.
Parentheses, Exponents, Multiplication, Division, Addition, Subtraction

\subsection{Formatting Output}
Wouldn't it be nice if we always worked with nice round numbers while doing
math? Unfortunately, the real world is not quite so neat and tidy as we would
like it to be. Sometimes, we end up with long, ugly numbers like the following:
What is your mass in kilograms? 65
You weigh 10.2142857143 stone.
By default, Python's print statement prints numbers to 10 decimal places. But
what if you only want one or two? We can use the round() function, which rounds
a number to the number of decimal points you choose. round() takes two
arguments: the number you want to round, and the number of decimal places to
round it to. For example:
\lstset{basicstyle=\scriptsize, numbers=left, captionpos=b, tabsize=4}
\begin{lstlisting}[caption=Basic Math 2,language={Python},
xleftmargin=15pt, label=lst:basicmath2]
>>> print (round(3.14159265, 2))
3.14
\end{lstlisting}

Now, let's change our program to only print the result to two decimal places.
print ("You weigh", round(mass\_stone, 2), "stone.")
This also demonstrates the concept of nesting functions. As you can see, you can
place one function inside another function, and everything will still work
exactly the way you would expect. If you don't like this, you can always use
multiple variables, instead:
\lstset{basicstyle=\scriptsize, numbers=left, captionpos=b, tabsize=4}
\begin{lstlisting}[caption=Basic Math 3,language={Python},
xleftmargin=15pt, label=lst:basicmath3]
twoSigFigs = round(mass_stone, 2)
numToString = str(twoSigFigs)
print ("You weigh " + numToString + " stone.")
\end{lstlisting}
