\section{Object Oriented Coding}
OOP is a programming approach where objects are defined with methods (functions,
actions or events) and properties (values, characteristics), resulting in more
readable, more reusable code.  Lets say you're writing a program where you need
to keep track of multiple cars. Each car has different characteristics like
mileage, color, and top speed, but lucky for us they all can perform some common
actions like braking, accelerating, and turning.  Instead of writing code
separately for each car we could create a class called 'Car' that will be the
blueprint for each particular car.  [edit]Constructing a Class

Class is the name given to a generic description of an object. In python you
define a class method (an action, event, or function) using the following
structure:
\lstset{basicstyle=\scriptsize, numbers=left, captionpos=b, tabsize=4}
\begin{lstlisting}[caption=Basic Class Definition,language={Python},
xleftmargin=15pt, label=lst:basicclassdefinition]
class <<name>>:
    def <<method>> (self [, <<optional arguments>>]):
        <<Function codes>>
\end{lstlisting}

Let's take a detailed look. We define our object using the 'class' keyword, the
name we want, and a colon. We define its methods as we would a normal function:
only one indent with 'self' as its first argument (we get to this later). So our
example car class may look like this:
\lstset{basicstyle=\scriptsize, numbers=left, captionpos=b, tabsize=4}
\begin{lstlisting}[caption=Example Class Car,language={Python},
xleftmargin=15pt, label=lst:exampleclasscar]
class Car:
    def brake(self):
        print("Brakes")
 
    def accelerate(self):
        print("Accelerating")
\end{lstlisting}

\textit{This is an example of encapsulation, where processing instructions are defined
as part of another structure for future reuse.
The first letter of the class name should be capitalized (Car instead of car).
That is a very common convention, but is not required by the language.}

\subsection{But how do I use it?}
Once you have created the class, you actually need to create an object for each
instance of that class. In python we create a new variable to create an instance
of a class. Example:
\lstset{basicstyle=\scriptsize, numbers=left, captionpos=b, tabsize=4}
\begin{lstlisting}[caption=Class usage,language={Python},
xleftmargin=15pt, label=lst:classusage]
car1 = Car() # car 1 is my instance for the first car
car2 = Car()
 
# And use the object methods like
car1.brake()
\end{lstlisting}

Using the parentheses ("calling" the class) tells Python that you want to create
an instance and not just copy the class definition. You would need to create a
variable for each car. However, now each car object can take advantage of the
class methods and attributes, so you don't need to write a brake and accelerate
function for each car independently.

\subsection{Properties}
Right now all the cars look the same, but let's give them some properties to
differentiate them.  A property is just a variable that is specific to a given
object. To assign a property we write it like:
\lstset{basicstyle=\scriptsize, numbers=left, captionpos=b, tabsize=4}
\begin{lstlisting}[caption=Property Example Set,language={Python},
xleftmargin=15pt, label=lst:propertyexampleset]
car1.color = "Red"
\end{lstlisting}

And retrieve its value like:
\lstset{basicstyle=\scriptsize, numbers=left, captionpos=b, tabsize=4}
\begin{lstlisting}[caption=Property Example Read,language={Python},
xleftmargin=15pt, label=lst:propertyexampleread]
print(car1.color)
\end{lstlisting}

It is good programming practice to write functions to get (or retrieve) and set
(or assign) properties that are not 'read-only'. For example:
\lstset{basicstyle=\scriptsize, numbers=left, captionpos=b, tabsize=4}
\begin{lstlisting}[caption=Getter Setter,language={Python},
xleftmargin=15pt, label=lst:gettersetter]
class Car:
    ... previous methods ...
 
    def set_owner(self,Owners_Name): # This will set the owner property
        self._owner = Owners_Name
 
    def get_owner(self): # This will retrieve the owner property
        return self._owner
\end{lstlisting}

Notice the single underscore before the property name; this is a way of hiding
variable names from users.  Beginning from Python 2.2, you may also define the
above example in a way that looks like a normal variable:
\lstset{basicstyle=\scriptsize, numbers=left, captionpos=b, tabsize=4}
\begin{lstlisting}[caption=Variable hiding,language={Python},
xleftmargin=15pt, label=lst:variablehiding]
class Car:
    ... previous methods ...
    owner = property(get_owner, set_owner)
\end{lstlisting}

When code such as mycar.owner = "John Smith" is used, the set\_owner function is
called transparently.
\textit{Think of a property that needs to be validated before it can be
assigned. If we don't hide such a variable from the user, it can create a
security risk in our program.}

\subsection{Extending a Class}
Let's say we want to add more functions to our class, but we are reluctant to
change its code for fear that this might mess up programs that depend on its
current version. The solution is to 'extend' our class. When you extend a class
you inherit all the parent methods and properties and can add new ones. For
example, we can add a start\_car method to our car class. To extend a class we
supply the name of the parent class in parentheses after the new class name, for
example:
\lstset{basicstyle=\scriptsize, numbers=left, captionpos=b, tabsize=4}
\begin{lstlisting}[caption=Extending Classes,language={Python},
xleftmargin=15pt, label=lst:extendingclasses]
class New_car(Car):
   def start_car(self):
      self.on = true
\end{lstlisting}

This new class extends the parent class.
\textit{When methods and attributes are passed down to new classes in
hierarchies, it is called Inheritance.}

\subsection{Special Class Methods}
A Constructor is a method that gets called whenever you create a new instance of
a class. In python you create a constructor by writing a function inside the
method name \_\_init\_\_. It can even accept arguments, e.g. to set attribute values
upon creating a new instance of the class. For instance, we could create a new
car object and set its brand, model, and year attributes on a single line,
rather than expending an additional line for each attribute:
\lstset{basicstyle=\scriptsize, numbers=left, captionpos=b, tabsize=4}
\begin{lstlisting}[caption=Constructor,language={Python},
xleftmargin=15pt, label=lst:constructor]
class New_car(Car):
    def __init__(self,brand, model, year):
        # Sets all the properties
        self.brand = brand
        self.model = model
        self.year = year
 
    def start_car(self):
        """ Start the cars engine """
        print ("vroem vroem")
 
if __name__ == "__main__":
    # Creates two instances of new_car, each with unique properties
    car1 = New_car("Ford","F-150",2001)
    car2 = New_car("Toyota","Corolla",2007)
 
    car1.start_car()
    car2.start_car()
\end{lstlisting}

For more information on classes go to the Class Section in this book.
