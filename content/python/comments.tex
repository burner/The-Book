\section{Comments}
There will always be a time in which you have to return to your code. Perhaps it
is to fix a bug, or to add a new feature. Regardless, looking at your own code
after six months is almost as bad as looking at someone else's code. What one
needs is a means to leave reminders to yourself as to what you were doing.  For
this purpose, you leave comments. Comments are little snippets of text embedded
inside your code that are ignored by the Python interpreter. A comment is
denoted by the hash character (\#) and extends to the end of the line. For
example:
\lstset{basicstyle=\scriptsize, numbers=left, captionpos=b, tabsize=4}
\begin{lstlisting}[caption=Basic Comment,language={Python},
xleftmargin=15pt, label=lst:basiccomment]
#!/usr/bin/env python
# commentexample.py
 
# Display the knights that come after Scene 24
print("The Knights Who Say Ni!")
# print("I will never see the light of day!")
\end{lstlisting}

As you can see, you can also use comments to temporarily remove segments of your
code, like the second print statement.

\section{Comment Guidelines}
The following guidelines are from PEP 8, written by Guido van Rossum.  General
Comments that contradict the code are worse than no comments. Always make a
priority of keeping the comments up-to-date when the code changes!  Comments
should be complete sentences. If a comment is a phrase or sentence, its first
word should be capitalized, unless it is an identifier that begins with a lower
case letter (never alter the case of identifiers!).  If a comment is short, the
period at the end can be omitted. Block comments generally consist of one or
more paragraphs built out of complete sentences, and each sentence should end in
a period.  You should use two spaces after a sentence-ending period.  When
writing English, Strunk and White applies.  Python coders from non-English
speaking countries: please write your comments in English, unless you are 120\%
sure that the code will never be read by people who don't speak your language.
Inline Comments An inline comment is a comment on the same line as a statement.
Inline comments should be separated by at least two spaces from the statement.
They should start with a \# and a single space.  Inline comments are unnecessary
and in fact distracting if they state the obvious. Don't do this:
\lstset{basicstyle=\scriptsize, numbers=left, captionpos=b, tabsize=4}
\begin{lstlisting}[caption=End of Line Comment,language={Python},
xleftmargin=15pt, label=lst:dndoflinecomment]
x = x + 1 # Increment x
But sometimes, this is useful:
x = x + 1 # Compensate for border
\end{lstlisting}

\section{Documentation Strings}
But what if you just want to know how to use a function, class, or method? You
could add comments before the function, but comments are inside the code, so you
would have to pull up a text editor and view them that way. But you can't pull
up comments from a C extension, so that is less than ideal. You could always
write a separate text file with how to call the functions, but that would mean
that you would have to remember to update that file. If only there was a
mechanism for being able to embed the documentation and get at it easily...
Fortunately, Python has such a capability. Documentation strings (or docstrings)
are used to create easily-accessible documentation. You can add a docstring to a
function, class, or module by adding a string as the first indented statement.
For example:
\lstset{basicstyle=\scriptsize, numbers=left, captionpos=b, tabsize=4}
\begin{lstlisting}[caption=Documentation String,language={Python},
xleftmargin=15pt, label=lst:documentationstring]
#!/usr/bin/env python
# docstringexample.py
 
"""Example of using documentation strings."""
 
class Knight:
    """
    An example class.
 
    Call spam to get bacon.
    """
 
    def spam(eggs="bacon"):
        """Prints the argument given."""
        print(eggs)
\end{lstlisting}

The convention is to use triple-quoted strings, because it makes it easier to
add more documentation spanning multiple lines.  To access the documentation,
you can use the help function inside a Python shell with the object you want
help on, or you can use the pydoc command from your system's shell. If we were
in the directory where docstringexample.py lives, one could enter pydoc
docstringexample to get documentation on that module.
