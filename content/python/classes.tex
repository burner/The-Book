\section{Classes}
\subsection{Defining a Class}
To define a class, use the following format:
\lstset{basicstyle=\scriptsize, numbers=left, captionpos=b, tabsize=4}
\begin{lstlisting}[caption=Class Definition,language={Python},
xleftmargin=15pt, label=lst:classdefinition]
class ClassName:
    ...
    ...
\end{lstlisting}

The capitalization in this class definition is the convention, but is not
required by the language.

\subsection{Instance Construction}
The class is a callable object that constructs an instance of the class when
called. To construct an instance of a class, "call" the class object:
\lstset{basicstyle=\scriptsize, numbers=left, captionpos=b, tabsize=4}
\begin{lstlisting}[caption=Instancing a Class,language={Python},
xleftmargin=15pt, label=lst:instancingaclass]
f = Foo()
\end{lstlisting}

This constructs an instance of class Foo and creates a reference to it in f.

\subsection{Class Members}
In order to access the member of an instance of a class, use the syntax <class
instance>.<member>. It is also possible to access the members of the class
definition with <class name>.<member>.

\subsection{Methods}
A method is a function within a class. The first argument (methods must always
take at least one argument) is always the instance of the class on which the
function is invoked. For example
\lstset{basicstyle=\scriptsize, numbers=left, captionpos=b, tabsize=4}
\begin{lstlisting}[caption=Defining member Functions,language={Python},
xleftmargin=15pt, label=lst:definigmemberfunctions]
>>> class Foo:
...     def setx(self, x):
...         self.x = x
...     def bar(self):
...         print(self.x)
\end{lstlisting}

If this code were executed, nothing would happen, at least until an instance of
Foo were constructed, and then bar were called on that instance.

\subsection{Invoking Member Methods}
Calling a method is much like calling a function, but instead of passing the
instance as the first parameter like the list of formal parameters suggests, use
the function as an attribute of the instance.
>>> f.setx(5)
\lstset{basicstyle=\scriptsize, numbers=left, captionpos=b, tabsize=4}
\begin{lstlisting}[caption=Calling Member Functions,language={Python},
xleftmargin=15pt, label=lst:callingmemberfunctions]
>>> f.bar()
\end{lstlisting}

This will output
\scriptsize
\begin{verbatim}
5
\end{verbatim}
\normalsize

It is possible to call the method on an arbitrary object, by using it as an
attribute of the defining class instead of an instance of that class, like so:
\lstset{basicstyle=\scriptsize, numbers=left, captionpos=b, tabsize=4}
\begin{lstlisting}[caption=Call on arbitrary object,language={Python},
xleftmargin=15pt, label=lst:callonarbitraryobject]
>>> Foo.setx(f,5)
>>> Foo.bar(f)
\end{lstlisting}

This will have the same output.

\subsection{Dynamic Class Structure}
As shown by the method setx above, the members of a Python class can change
during runtime, not just their values, unlike classes in languages like C or
Java. We can even delete f.x after running the code above.
\lstset{basicstyle=\scriptsize, numbers=left, captionpos=b, tabsize=4}
\begin{lstlisting}[caption=Delete class Member,language={Python},
xleftmargin=15pt, label=lst:deleteclassmember]
>>> del f.x
>>> f.bar()

Traceback (most recent call last):
  File "<stdin>", line 1, in ?
  File "<stdin>", line 5, in bar
AttributeError: Foo instance has no attribute 'x'
\end{lstlisting}

Another effect of this is that we can change the definition of the Foo class
during program execution. In the code below, we create a member of the Foo class
definition named y. If we then create a new instance of Foo, it will now have
this new member.
\lstset{basicstyle=\scriptsize, numbers=left, captionpos=b, tabsize=4}
\begin{lstlisting}[caption=Add member to Class,language={Python},
xleftmargin=15pt, label=lst:addmembertoclass]
>>> Foo.y = 10
>>> g = Foo()
>>> g.y
10
\end{lstlisting}

\subsection{Viewing Class Dictionaries}
At the heart of all this is a dictionary that can be accessed by
"vars(ClassName)"
\lstset{basicstyle=\scriptsize, numbers=left, captionpos=b, tabsize=4}
\begin{lstlisting}[caption=Dictionary as Heart of all Things,language={Python},
xleftmargin=15pt, label=lst:dictionaryasheartofallthings]
>>> vars(g)
{}
\end{lstlisting}

At first, this output makes no sense. We just saw that g had the member y, so
why isn't it in the member dictionary? If you remember, though, we put y in the
class definition, Foo, not g.
\lstset{basicstyle=\scriptsize, numbers=left, captionpos=b, tabsize=4}
\begin{lstlisting}[caption=Variable of a Class 1,language={Python},
xleftmargin=15pt, label=lst:variableofaclass1]
>>> vars(Foo)
{'y': 10, 'bar': <function bar at 0x4d6a3c>, '__module__': '__main__',
 'setx': <function setx at 0x4d6a04>, '__doc__': None}
\end{lstlisting}

And there we have all the members of the Foo class definition. When Python
checks for g.member, it first checks g's vars dictionary for "member," then Foo.
If we create a new member of g, it will be added to g's dictionary, but not
Foo's.
\lstset{basicstyle=\scriptsize, numbers=left, captionpos=b, tabsize=4}
\begin{lstlisting}[caption=Variable of a Class 2,language={Python},
xleftmargin=15pt, label=lst:variableofaclass2]
>>> g.setx(5)
>>> vars(g)
{'x': 5}
\end{lstlisting}

Note that if we now assign a value to g.y, we are not assigning that value to
Foo.y. Foo.y will still be 10, but g.y will now override Foo.y
\lstset{basicstyle=\scriptsize, numbers=left, captionpos=b, tabsize=4}
\begin{lstlisting}[caption=Variable of a Class 3,language={Python},
xleftmargin=15pt, label=lst:variableofaclass3]
>>> g.y = 9
>>> vars(g)
{'y': 9, 'x': 5}
>>> vars(Foo)
{'y': 10, 'bar': <function bar at 0x4d6a3c>, '__module__': '__main__',
 'setx': <function setx at 0x4d6a04>, '__doc__': None}
Sure enough, if we check the values:
>>> g.y
9
>>> Foo.y
10
\end{lstlisting}

Note that f.y will also be 10, as Python won't find 'y' in vars(f), so it will
get the value of 'y' from vars(Foo).  Some may have also noticed that the
methods in Foo appear in the class dictionary along with the x and y. If you
remember from the section on lambda forms, we can treat functions just like
variables. This means that we can assign methods to a class during runtime in
the same way we assigned variables. If you do this, though, remember that if we
call a method of a class instance, the first parameter passed to the method will
always be the class instance itself.

\subsection{Changing Class Dictionaries}
We can also access the members dictionary of a class using the \_\_dict\_\_
member of the class.
\lstset{basicstyle=\scriptsize, numbers=left, captionpos=b, tabsize=4}
\begin{lstlisting}[caption=Changing Class Dicts,language={Python},
xleftmargin=15pt, label=lst:changingclassdicts]
>>> g.__dict__
{'y': 9, 'x': 5}
\end{lstlisting}

If we add, remove, or change key-value pairs from g.\_\_dict\_\_, this has the same
effect as if we had made those changes to the members of g.
\lstset{basicstyle=\scriptsize, numbers=left, captionpos=b, tabsize=4}
\begin{lstlisting}[caption=Remove class Member,language={Python},
xleftmargin=15pt, label=lst:removeclassmember]
>>> g.__dict__['z'] = -4
>>> g.z
-4
\end{lstlisting}

\subsection{New Style Classes}
New style classes were introduced in python 2.2. A new-style class is a class
that has a built-in as its base, most commonly object. At a low level, a major
difference between old and new classes is their type. Old class instances were
all of type instance. New style class instances will return the same thing as
x.\_\_class\_\_ for their type. This puts user defined classes on a level playing
field with built-ins. Old/Classic classes are slated to disappear in Python
3000. With this in mind all development should use new style classes. New Style
classes also add constructs like properties and static methods familiar to Java
programmers.
Old/Classic Class
\lstset{basicstyle=\scriptsize, numbers=left, captionpos=b, tabsize=4}
\begin{lstlisting}[caption=New Style Classes,language={Python},
xleftmargin=15pt, label=lst:newstyleclasses]
>>> class ClassicFoo:
...     def __init__(self):
...         pass
New Style Class
>>> class NewStyleFoo(object):
...     def __init__(self):
...         pass
\end{lstlisting}

\subsection{Properties}
Properties are attributes with getter and setter methods.
\lstset{basicstyle=\scriptsize, numbers=left, captionpos=b, tabsize=4}
\begin{lstlisting}[caption=Property Example,language={Python},
xleftmargin=15pt, label=lst:propertyexample]
>>> class SpamWithProperties(object):
...     def __init__(self):
...         self.__egg = "MyEgg"
...     def getEgg(self):
...         return self.__egg
...     def setEgg(self,egg):
...         self.__egg = egg
...     egg = property(getEgg,setEgg)
 
>>> sp = SpamWithProperties()
>>> sp.egg
'MyEgg'
>>> sp.egg = "Eggs With Spam"
>>> sp.egg
'Eggs With Spam'
>>>
\end{lstlisting}

\subsection{Static Methods}
Static methods in Python are just like their counterparts in C++ or Java. Static
methods have no "self" argument and don't require you to instantiate the class
before using them. They can be defined using staticmethod()
\lstset{basicstyle=\scriptsize, numbers=left, captionpos=b, tabsize=4}
\begin{lstlisting}[caption=Static Methode,language={Python},
xleftmargin=15pt, label=lst:staticmethode]
>>> class StaticSpam(object):
...     def StaticNoSpam():
...         print("You can't have have the spam, spam, eggs and spam without any
spam... that's disgusting")
...     NoSpam = staticmethod(StaticNoSpam)
 
>>> StaticSpam.NoSpam()
'You can't have have the spam, spam, eggs and spam without any spam... that's disgusting'
\end{lstlisting}

They can also be defined using the function decorator @staticmethod.
\lstset{basicstyle=\scriptsize, numbers=left, captionpos=b, tabsize=4}
\begin{lstlisting}[caption=Another Static Methode,language={Python},
xleftmargin=15pt, label=lst:anotherstaticmethode]
>>> class StaticSpam(object):
...     @staticmethod
...     def StaticNoSpam():
...         print("You can't have have the spam, spam, eggs and spam without any
spam... that's disgusting")
\end{lstlisting}

\subsection{Inheritance}
Like all object oriented languages, Python provides for inheritance. Inheritance
is a simple concept by which a class can extend the facilities of another class,
or in Python's case, multiple other classes. Use the following format for this:
\lstset{basicstyle=\scriptsize, numbers=left, captionpos=b, tabsize=4}
\begin{lstlisting}[caption=Inheritance,language={Python},
xleftmargin=15pt, label=lst:inheritance]
class ClassName(superclass1,superclass2,superclass3,...):
    ...
\end{lstlisting}

The subclass will then have all the members of its superclasses. If a method is
defined in the subclass and in the superclass, the member in the subclass will
override the one in the superclass. In order to use the method defined in the
superclass, it is necessary to call the method as an attribute on the defining
class, as in Foo.setx(f,5) above:
\lstset{basicstyle=\scriptsize, numbers=left, captionpos=b, tabsize=4}
\begin{lstlisting}[caption=Inheritance Example,language={Python},
xleftmargin=15pt, label=lst:inheritanceexample]
>>> class Foo:
...     def bar(self):
...         print("I'm doing Foo.bar()")
...     x = 10
...
>>> class Bar(Foo):
...     def bar(self):
...         print("I'm doing Bar.bar()")
...         Foo.bar(self)
...     y = 9
...
>>> g = Bar()
>>> Bar.bar(g)
I'm doing Bar.bar()
I'm doing Foo.bar()
>>> g.y
9
>>> g.x
10
\end{lstlisting}

Once again, we can see what's going on under the hood by looking at the class
dictionaries.
\lstset{basicstyle=\scriptsize, numbers=left, captionpos=b, tabsize=4}
\begin{lstlisting}[caption=Inherited Member,language={Python},
xleftmargin=15pt, label=lst:]
>>> vars(g)
{}
>>> vars(Bar)
{'y': 9, '__module__': '__main__', 'bar': <function bar at 0x4d6a04>,
 '__doc__': None}
>>> vars(Foo)
{'x': 10, '__module__': '__main__', 'bar': <function bar at 0x4d6994>,
 '__doc__': None}
\end{lstlisting}

When we call g.x, it first looks in the vars(g) dictionary, as usual. Also as
above, it checks vars(Bar) next, since g is an instance of Bar. However, thanks
to inheritance, Python will check vars(Foo) if it doesn't find x in vars(Bar).

\subsection{Special Methods}
There are a number of methods which have reserved names which are used for
special purposes like mimicking numerical or container operations, among other
things. All of these names begin and end with two underscores. It is convention
that methods beginning with a single underscore are 'private' to the scope they
are introduced within.

\subsection{Initialization and Deletion}
\subsubsection{\_\_init\_\_}
One of these purposes is constructing an instance, and the special name for this
is '\_\_init\_\_'. \_\_init\_\_() is called before an instance is returned (it is not
necessary to return the instance manually). As an example,
\lstset{basicstyle=\scriptsize, numbers=left, captionpos=b, tabsize=4}
\begin{lstlisting}[caption=Init Methode,language={Python},
xleftmargin=15pt, label=lst:initmethode]
class A:
    def __init__(self):
        print('A.__init__()')
a = A()
\end{lstlisting}

output:
\scriptsize
\begin{verbatim}
A.__init__()
\end{verbatim}
\normalsize
\_\_init\_\_() can take arguments, in which case it is necessary to pass arguments
to the class in order to create an instance. For example,
\lstset{basicstyle=\scriptsize, numbers=left, captionpos=b, tabsize=4}
\begin{lstlisting}[caption=Init Arguments,language={Python},
xleftmargin=15pt, label=lst:initarguments]
class Foo:
    def __init__ (self, printme):
        print(printme)
foo = Foo('Hi!')
\end{lstlisting}

output:
\scriptsize
\begin{verbatim}
Hi!
\end{verbatim}
\normalsize

Here is an example showing the difference between using \_\_init\_\_() and not using
\_\_init\_\_():
\lstset{basicstyle=\scriptsize, numbers=left, captionpos=b, tabsize=4}
\begin{lstlisting}[caption=Difference,language={Python},
xleftmargin=15pt, label=lst:difference]
class Foo:
    def __init__ (self, x):
         print(x)
foo = Foo('Hi!')
class Foo2:
    def setx(self, x):
        print(x)
f = Foo2()
Foo2.setx(f,'Hi!')
\end{lstlisting}
output:
\scriptsize
\begin{verbatim}
Hi!
Hi!
\end{verbatim}
\normalsize

\subsubsection{\_\_del\_\_}
Similarly, '\_\_del\_\_' is called when an instance is destroyed; e.g. when it
is no longer referenced.

\subsubsection{Representation}
String Representation Override Functions

\begin{tabular}{|c|c|}
\hline
Function&	Operator\\
\hline
\_\_str\_\_&	str(A)\\
\hline
\_\_repr\_\_&	repr(A)\\
\hline
\_\_unicode\_\_&	unicode(x) (2.x only)\\
\hline
\end{tabular}
\paragraph{\_\_str\_\_}
Converting an object to a string, as with the print statement or with the str()
conversion function, can be overridden by overriding \_\_str\_\_. Usually,
\_\_str\_\_ returns a formatted version of the objects content. This will NOT
usually be something that can be executed. For example:
\lstset{basicstyle=\scriptsize, numbers=left, captionpos=b, tabsize=4}
\begin{lstlisting}[caption=str example,language={Python},
xleftmargin=15pt, label=lst:strexample]
class Bar:
    def __init__ (self, iamthis):
        self.iamthis = iamthis
    def __str__ (self):
        return self.iamthis
bar = Bar('apple')
print(bar)
\end{lstlisting}

output:
\scriptsize
\begin{verbatim}
apple
\end{verbatim}
\normalsize

\paragraph{\_\_repr\_\_}
This function is much like \_\_str\_\_(). If \_\_str\_\_ is not present but this
one is, this function's output is used instead for printing. \_\_repr\_\_ is
used to return a representation of the object in string form. In general, it can
be executed to get back the original object. For example:
\lstset{basicstyle=\scriptsize, numbers=left, captionpos=b, tabsize=4}
\begin{lstlisting}[caption=repr example,language={Python},
xleftmargin=15pt, label=lst:reprexample]
class Bar:
    def __init__ (self, iamthis):
        self.iamthis = iamthis
    def __repr__(self):
        return "Bar('%s')" % self.iamthis
bar = Bar('apple')
bar
\end{lstlisting}

outputs (note the difference: now is not necessary to put it inside a print)
\scriptsize
\begin{verbatim}
Bar('apple')
\end{verbatim}
\normalsize

\subsection{Attributes}
\begin{tabular}{|c|c|c|}
\hline
Function&Indirect form & Direct form\\
\hline
\_\_getattr\_\_&	getattr(A)& A.B\\
\hline
\_\_setattr\_\_&	setattr(A)& A.B = C\\
\hline
\_\_delattr\_\_&	delattr(x) (2.x only) & del A.B\\
\hline
\end{tabular}

\subsubsection{\_\_setattr\_\_}
This is the function which is in charge of setting attributes of a class. It is
provided with the name and value of the variables being assigned. Each class, of
course, comes with a default \_\_setattr\_\_ which simply sets the value of the
variable, but we can override it.
\lstset{basicstyle=\scriptsize, numbers=left, captionpos=b, tabsize=4}
\begin{lstlisting}[caption=setattr example,language={Python},
xleftmargin=15pt, label=lst:setattrexample]
>>> class Unchangable:
...    def __setattr__(self, name, value):
...        print "Nice try"
...
>>> u = Unchangable()
>>> u.x = 9
Nice try
>>> u.x
Traceback (most recent call last):
  File "<stdin>", line 1, in ?
AttributeError: Unchangable instance has no attribute 'x'
\end{lstlisting}

\subsubsection{\_\_getattr\_\_}
Similar to \_\_setattr\_\_, except this function is called when we try to access
a class member, and the default simply returns the value.
\lstset{basicstyle=\scriptsize, numbers=left, captionpos=b, tabsize=4}
\begin{lstlisting}[caption=getattr example,language={Python},
xleftmargin=15pt, label=lst:getattrexample]
>>> class HiddenMembers:
...     def __getattr__(self, name):
...         return "You don't get to see " + name
...
>>> h = HiddenMembers()
>>> h.anything
"You don't get to see anything"
\end{lstlisting}

\subsubsection{\_\_delattr\_\_}
This function is called to delete an attribute.
\lstset{basicstyle=\scriptsize, numbers=left, captionpos=b, tabsize=4}
\begin{lstlisting}[caption=delattr example,language={Python},
xleftmargin=15pt, label=lst:delattrexample]
>>> class Permanent:
...     def __delattr__(self, name):
...         print name, "cannot be deleted"
...
>>> p = Permanent()
>>> p.x = 9
>>> del p.x
x cannot be deleted
>>> p.x
9
\end{lstlisting}

\subsection{Operator Overloading}
Operator overloading allows us to use the built-in Python syntax and operators
to call functions which we define.

\begin{tabular}{|c|p{1.5cm}|}
\hline
Function&Operator \\
\hline
\_\_add\_\_&	A + B \\
\hline
\_\_sub\_\_&	A - B \\
\hline
\_\_mul\_\_&	A * B \\
\hline
\_\_div\_\_&	A / B \\
\hline
\_\_floordiv\_\_&	A // B \\
\hline
\_\_mod\_\_&	A \% B \\
\hline
\_\_pow\_\_&	A ** B \\
\hline
\_\_and\_\_&	A \& B \\
\hline
\_\_or\_\_ & A | B \\
\hline
\_\_xor\_\_&	A \^ B \\
\hline
\_\_eq\_\_&	A == B \\
\hline
\_\_ne\_\_&	A != B \\
\hline
\_\_gt\_\_&	A $>$ B \\
\hline
\_\_lt\_\_&	A $<$ B \\
\hline
\_\_ge\_\_&	A $>=$ B \\
\hline
\_\_le\_\_&	A $<=$ B \\
\hline
\_\_lshift\_\_&	A $<<$ B \\
\hline
\_\_rshift\_\_&	A $>>$ B \\
\hline
\_\_contains\_\_&	A in B, A not in B \\
\hline
\end{tabular}

\begin{tabular}{|c|p{2cm}|}
\hline
Function&Operator\\ \hline
\_\_pos\_\_ & $+A$\\ \hline
\_\_neg\_\_ & $-A$\\ \hline
\_\_inv\_\_ & $~A$\\ \hline
\_\_abs\_\_ & abs(A)\\ \hline
\_\_len\_\_ & len(A)\\ \hline
\end{tabular}\\

\begin{tabular}{|c|p{1.5cm}|} \hline
Function & Operator\\ \hline
\_\_getitem\_\_	 & C[i]\\ \hline
\_\_setitem\_\_	 & C[i] = v\\ \hline
\_\_delitem\_\_	 & del C[i]\\ \hline
\_\_getslice\_\_ & C[s:e]\\ \hline
\_\_setslice\_\_ & C[s:e] = v\\ \hline
\_\_delslice\_\_ & del C[s:e]\\ \hline
\end{tabular}

If a class has the \_\_add\_\_ function, we can use the '+' operator to add
instances of the class. This will call \_\_add\_\_ with the two instances of the
class passed as parameters, and the return value will be the result of the
addition.
\lstset{basicstyle=\scriptsize, numbers=left, captionpos=b, tabsize=4}
\begin{lstlisting}[caption=Add Operator,language={Python},
xleftmargin=15pt, label=lst:addoperator]
>>> class FakeNumber:
...     n = 5
...     def __add__(A,B):
...         return A.n + B.n
...
>>> c = FakeNumber()
>>> d = FakeNumber()
>>> d.n = 7
>>> c + d
12
\end{lstlisting}

To override the augmented assignment operators, merely add 'i' in front of the
normal binary operator, i.e. for '+=' use '\_\_iadd\_\_' instead of
'\_\_add\_\_'. The function will be given one argument, which will be the object
on the right side of the augmented assignment operator. The returned value of
the function will then be assigned to the object on the left of the operator.
\lstset{basicstyle=\scriptsize, numbers=left, captionpos=b, tabsize=4}
\begin{lstlisting}[caption=Mul Operator,language={Python},
xleftmargin=15pt, label=lst:muloperator]
>>> c.__imul__ = lambda B: B.n - 6
>>> c *= d
>>> c
1
\end{lstlisting}

It is important to note that the augmented assignment operators will also use
the normal operator functions if the augmented operator function hasn't been set
directly. This will work as expected, with "\_\_add\_\_" being called for "+=" and
so on.
\lstset{basicstyle=\scriptsize, numbers=left, captionpos=b, tabsize=4}
\begin{lstlisting}[caption=Add Assign Operator,language={Python},
xleftmargin=15pt, label=lst:addassignoperator]
>>> c = FakeNumber()
>>> c += d
>>> c
12
\end{lstlisting}

Unary operators will be passed simply the instance of the class that they are
called on.
\lstset{basicstyle=\scriptsize, numbers=left, captionpos=b, tabsize=4}
\begin{lstlisting}[caption=Unary Example,language={Python},
xleftmargin=15pt, label=lst:unaryexample]
>>> FakeNumber.__neg__ = lambda A : A.n + 6
>>> -d
13
\end{lstlisting}

It is also possible in Python to override the indexing and slicing operators.
This allows us to use the class[i] and class[a:b] syntax on our own objects.
The simplest form of item operator is \_\_getitem\_\_. This takes as a parameter the
instance of the class, then the value of the index.
\lstset{basicstyle=\scriptsize, numbers=left, captionpos=b, tabsize=4}
\begin{lstlisting}[caption=Index Operator,language={Python},
xleftmargin=15pt, label=lst:indexoperator]
>>> class FakeList:
...     def __getitem__(self,index):
...         return index * 2
...
>>> f = FakeList()
>>> f['a']
'aa'
\end{lstlisting}

We can also define a function for the syntax associated with assigning a value
to an item. The parameters for this function include the value being assigned,
in addition to the parameters from \_\_getitem\_\_
\lstset{basicstyle=\scriptsize, numbers=left, captionpos=b, tabsize=4}
\begin{lstlisting}[caption=Index Assign,language={Python},
xleftmargin=15pt, label=lst:indexassign]
>>> class FakeList:
...     def __setitem__(self,index,value):
...         self.string = index + " is now " + value
...
>>> f = FakeList()
>>> f['a'] = 'gone'
>>> f.string
'a is now gone'
\end{lstlisting}

We can do the same thing with slices. Once again, each syntax has a different
parameter list associated with it.
\lstset{basicstyle=\scriptsize, numbers=left, captionpos=b, tabsize=4}
\begin{lstlisting}[caption=Slice Operator,language={Python},
xleftmargin=15pt, label=lst:sliceoperator]
>>> class FakeList:
...     def __getslice___(self,start,end):
...         return str(start) + " to " + str(end)
...
>>> f = FakeList()
>>> f[1:4]
'1 to 4'
\end{lstlisting}

Keep in mind that one or both of the start and end parameters can be blank in
slice syntax. Here, Python has default value for both the start and the end, as
show below.
\lstset{basicstyle=\scriptsize, numbers=left, captionpos=b, tabsize=4}
\begin{lstlisting}[caption=Slice Limits,language={Python},
xleftmargin=15pt, label=lst:slicelimits]
>> f[:]
'0 to 2147483647'
\end{lstlisting}

Note that the default value for the end of the slice shown here is simply the
largest possible signed integer on a 32-bit system, and may vary depending on
your system and C compiler.
\_\_setslice\_\_ has the parameters (self,start,end,value)
We also have operators for deleting items and slices.
\_\_delitem\_\_ has the parameters (self,index)
\_\_delslice\_\_ has the parameters (self,start,end)
Note that these are the same as \_\_getitem\_\_ and \_\_getslice\_\_.

\begin{tabular}{|c|c|} \hline
Function&	Operator \\ \hline
\_\_cmp\_\_ &	cmp(x, y) \\ \hline
\_\_hash\_\_ &	hash(x) \\ \hline
\_\_nonzero\_\_ &	bool(x) \\ \hline
\_\_call\_\_ &	f(x) \\ \hline
\_\_iter\_\_ &	iter(x) \\ \hline
\_\_reversed\_\_ &	reversed(x) (2.6+) \\ \hline
\_\_divmod\_\_ &	divmod(x, y) \\ \hline
\_\_int\_\_ &	int(x) \\ \hline
\_\_long\_\_ &	long(x) \\ \hline
\_\_float\_\_ &	float(x) \\ \hline
\_\_complex\_\_ &	complex(x) \\ \hline
\_\_hex\_\_ &	hex(x) \\ \hline
\_\_oct\_\_ &	oct(x) \\ \hline
\_\_index\_\_ &	 \\ \hline
\_\_copy\_\_ &	copy.copy(x) \\ \hline
\_\_deepcopy\_\_ &	copy.deepcopy(x) \\ \hline
\_\_sizeof\_\_ &	sys.getsizeof(x) (2.6+) \\ \hline
\_\_trunc\_\_ &	math.trunc(x) (2.6+) \\ \hline
\_\_format\_\_ &	format(x, ...) (2.6+) \\ \hline
\end{tabular}

\section{Programming Practices}
The flexibility of python classes means that classes can adopt a varied set of
behaviors. For the sake of understandability, however, it's best to use many of
Python's tools sparingly. Try to declare all methods in the class definition,
and always use the <class>.<member> syntax instead of \_\_dict\_\_ whenever
possible. Look at classes in C++ and Java to see what most programmers will
expect from a class.

\subsection{Encapsulation}
Since all python members of a python class are accessible by functions/methods
outside the class, there is no way to enforce encapsulation short of overriding
\_\_getattr\_\_, \_\_setattr\_\_ and \_\_delattr\_\_. General practice, however,
is for the creator of a class or module to simply trust that users will use only
the intended interface and avoid limiting access to the workings of the module
for the sake of users who do need to access it. When using parts of a class or
module other than the intended interface, keep in mind that the those parts may
change in later versions of the module, and you may even cause errors or
undefined behaviors in the module.

\subsection{Doc Strings}
When defining a class, it is convention to document the class using a string
literal at the start of the class definition. This string will then be placed in
the \_\_doc\_\_ attribute of the class definition.
\lstset{basicstyle=\scriptsize, numbers=left, captionpos=b, tabsize=4}
\begin{lstlisting}[caption=Doc String Example,language={Python},
xleftmargin=15pt, label=lst:Doc String Example]
>>> class Documented:
...     """This is a docstring"""
...     def explode(self):
...         """
...         This method is documented, too! The coder is really serious about
...         making this class usable by others who don't know the code as well
...         as he does.
...
...         """
...         print "boom"
>>> d = Documented()
>>> d.__doc__
'This is a docstring'
\end{lstlisting}

Docstrings are a very useful way to document your code. Even if you never write
a single piece of separate documentation (and let's admit it, doing so is the
lowest priority for many coders), including informative docstrings in your
classes will go a long way toward making them usable.  Several tools exist for
turning the docstrings in Python code into readable API documentation, e.g.,
EpyDoc.  Don't just stop at documenting the class definition, either. Each
method in the class should have its own docstring as well. Note that the
docstring for the method explode in the example class Documented above has a
fairly lengthy docstring that spans several lines. Its formatting is in
accordance with the style suggestions of Python's creator, Guido van Rossum.

\section{Adding methods at runtime}
\subsection{To a class}
It is fairly easy to add methods to a class at runtime. Lets assume that we have
a class called Spam and a function cook. We want to be able to use the function
cook on all instances of the class Spam:
\lstset{basicstyle=\scriptsize, numbers=left, captionpos=b, tabsize=4}
\begin{lstlisting}[caption=Reflection,language={Python},
xleftmargin=15pt, label=lst:reflection]
class Spam:
  def __init__(self):
    self.myeggs = 5
 
def cook(self):
  print "cooking %s eggs" % self.myeggs
 
Spam.cook = cook   #add the function to the class Spam
eggs = Spam()      #NOW create a new instance of Spam
eggs.cook()        #and we are ready to cook!
\end{lstlisting}

This will output.
\scriptsize
\begin{verbatim}
cooking 5 eggs
\end{verbatim}
\normalsize

\subsection{To an instance of a class}
It is a bit more tricky to add methods to an instance of a class that has
already been created. Lets assume again that we have a class called Spam and we
have already created eggs. But then we notice that we wanted to cook those eggs,
but we do not want to create a new instance but rather use the already created
one:
\lstset{basicstyle=\scriptsize, numbers=left, captionpos=b, tabsize=4}
\begin{lstlisting}[caption=Add function to Class,language={Python},
xleftmargin=15pt, label=lst:addfunctiontoclass]
class Spam:
  def __init__(self):
    self.myeggs = 5
 
eggs = Spam()
 
def cook(self):
  print "cooking %s eggs" % self.myeggs
 
import types
f = types.MethodType(cook, eggs, Spam)
eggs.cook = f
 
eggs.cook()
\end{lstlisting}

Now we can cook our eggs and the last statement will output:
\scriptsize
\begin{verbatim}
cooking 5 eggs
\end{verbatim}
\normalsize

\subsection{Using a function}
We can also write a function that will make the process of adding methods to an
instance of a class easier.
\lstset{basicstyle=\scriptsize, numbers=left, captionpos=b, tabsize=4}
\begin{lstlisting}[caption=Adding Member Reflection,language={Python},
xleftmargin=15pt, label=lst:addingmemberreflection]
def attach_method(fxn, instance, myclass):
  f = types.MethodType(fxn, instance, myclass)
  setattr(instance, fxn.__name__, f)
\end{lstlisting}

All we now need to do is call the attach\_method with the arguments of the
function we want to attach, the instance we want to attach it to and the class
the instance is derived from. Thus our function call might look like this:
\lstset{basicstyle=\scriptsize, numbers=left, captionpos=b, tabsize=4}
\begin{lstlisting}[caption=Adding Member Usage,language={Python},
xleftmargin=15pt, label=lst:addingmemberusage]
attach_method(cook, eggs, Spam)
\end{lstlisting}
Note that in the function add\_method we cannot write instance.fxn = f since this
would add a function called fxn to the instance.
