\section{Controlflow}
Python, like many other computer programming languages, uses Boolean logic for
its decision control. That is, the Python interpreter compares one or more
values in order to decide whether to execute a piece of code or not, given the
proper syntax and instructions.  Decision control is then divided into two major
categories, conditional and repetition. Conditional logic simply uses the
keyword if and a Boolean expression to decide whether or not to execute a code
block. Repetition builds on the conditional constructs by giving us a simple
method in which to repeat a block of code while a Boolean expression evaluates
to true.
\subsection{Boolean Expressions}
Here is a little example of boolean expressions (you don't have to type it in):
\lstset{basicstyle=\scriptsize, numbers=left, captionpos=b, tabsize=4}
\begin{lstlisting}[caption=Controlflow 1,language={Python},
xleftmargin=15pt, label=lst:controlflow1]
a = 6
b = 7
c = 42
print (1, a == 6)
print (2, a == 7)
print (3, a == 6 and b == 7)
print (4, a == 7 and b == 7)
print (5, not a == 7 and b == 7)
print (6, a == 7 or b == 7)
print (7, a == 7 or b == 6)
print (8, not (a == 7 and b == 6))
print (9, not a == 7 and b == 6)
\end{lstlisting}
With the output being:
\scriptsize
\begin{verbatim}
1 True
2 False
3 True
4 False
5 True
6 True
7 False
8 True
9 False
\end{verbatim}
\normalsize
What is going on? The program consists of a bunch of funny looking print
statements. Each print statement prints a number and an expression. The number
is to help keep track of which statement I am dealing with. Notice how each
expression ends up being either True or False; these are built-in Python values.
The lines:
\lstset{basicstyle=\scriptsize, numbers=left, captionpos=b, tabsize=4}
\begin{lstlisting}[caption=Controlflow 2,language={Python},
xleftmargin=15pt, label=lst:controlflow2]
print (1, a == 6)
print (2, a == 7)
\end{lstlisting}
print out True and False respectively, just as expected, since the first is true
and the second is false. The third print, print (3, a == 6 and b == 7), is a
little different. The operator and means if both the statement before and the
statement after are true then the whole expression is true otherwise the whole
expression is false. The next line, print (4, a == 7 and b == 7), shows how if
part of an and expression is false, the whole thing is false. The behavior of
and can be summarized as follows:

\begin{tabular}{|c|c|}
\hline
expression&	result\\
\hline
true and true	&true\\
\hline
true and false	&false\\
\hline
false and true	&false\\
\hline
false and false	&false\\
\hline
\end{tabular}

Note that if the first expression is false Python does not check the second
expression since it knows the whole expression is false.  The next line, print
(5, not a == 7 and b == 7), uses the not operator. not just gives the opposite
of the expression (The expression could be rewritten as print (5, a != 7 and b
== 7)). Here's the table:

\begin{tabular}{|c|c|}
\hline
expression	&result\\
\hline
not true	&false\\
\hline
not false	&true\\
\hline
\end{tabular}

The two following lines, print (6, a == 7 or b == 7) and print (7, a == 7 or b
== 6), use the or operator. The or operator returns true if the first expression
is true, or if the second expression is true or both are true. If neither are
true it returns false. Here's the table:

\begin{tabular}{|c|c|}
\hline
expression&	result\\
\hline
true or true	&true\\
\hline
true or false	&true\\
\hline
false or true	&true\\
\hline
false or false	&false\\
\hline
\end{tabular}

Note that if the first expression is true Python doesn't check the second
expression since it knows the whole expression is true. This works since or is
true if at least one half of the expression is true. The first part is true so
the second part could be either false or true, but the whole expression is still
true.  The next two lines, print (8, not (a == 7 and b == 6)) and print (9, not
a == 7 and b == 6), show that parentheses can be used to group expressions and
force one part to be evaluated first. Notice that the parentheses changed the
expression from false to true. This occurred since the parentheses forced the
not to apply to the whole expression instead of just the a == 7 portion.
Here is an example of using a boolean expression:
\lstset{basicstyle=\scriptsize, numbers=left, captionpos=b, tabsize=4}
\begin{lstlisting}[caption=Controlflow 3,language={Python},
xleftmargin=15pt, label=lst:controlflow3]
list = ["Life","The Universe","Everything","Jack","Jill","Life","Jill"]
 
# Make a copy of the list.
copy = list[:]
# Sort the copy
copy.sort()
prev = copy[0]
del copy[0]
 
count = 0
 
# Go through the list searching for a match
while count < len(copy) and copy[count] != prev:
    prev = copy[count]
    count = count + 1
 
# If a match was not found then count can't be < len
# since the while loop continues while count is < len
# and no match is found
if count < len(copy):
    print ("First Match:",prev)
\end{lstlisting}
See the Lists chapter to explain what [:] means on the first line.
Here is the output:
\scriptsize
\begin{verbatim}
First Match: Jill
\end{verbatim}
\normalsize
This program works by continuing to check for match while count $<$ len(copy) and
copy[count]. When either count is greater than the last index of copy or a match
has been found the and is no longer true so the loop exits. The if simply checks
to make sure that the while exited because a match was found.  The other 'trick'
of and is used in this example. If you look at the table for and notice that the
third entry is "false and won't check". If count $>=$ len(copy) (in other words
count $<$ len(copy) is false) then copy[count] is never looked at. This is because
Python knows that if the first is false then they both can't be true. This is
known as a short circuit and is useful if the second half of the and will cause
an error if something is wrong. I used the first expression ( count $<$ len(copy))
to check and see if count was a valid index for copy. (If you don't believe me
remove the matches `Jill' and `Life', check that it still works and then reverse
the order of count $<$ len(copy) and copy[count] != prev to copy[count] != prev
and count $<$ len(copy).) Boolean expressions can be used when you need to check
two or more different things at once.

\subsection{Decision}
A Decision is when a program has more than one choice of actions depending on a
variable's value. Think of a traffic light. When it is green, we continue our
drive. When we see the light turn yellow, we proceed to reduce our speed, and
when it is red, we stop. These are logical decisions that depend on the value of
the traffic light. Luckily, Python has a decision statement to help us when our
application needs to make such decision for the user.

\subsubsection{If statement}
Here is a warm-up exercise - a short program to compute the absolute value of a
number:
\lstset{basicstyle=\scriptsize, numbers=left, captionpos=b, tabsize=4}
\begin{lstlisting}[caption=absoval.py,language={Python},
xleftmargin=15pt, label=lst:absoval]
n = raw_input("Integer? ")#Pick an integer.  And remember, if raw_input is not
supported by your OS, use input()
n = int(n)#Defines n as the integer you chose. (Alternatively, you can define n
yourself)
if n < 0:
    print ("The absolute value of",n,"is",-n)
else:
    print ("The absolute value of",n,"is",n)
\end{lstlisting}

Here is the output from the two times that I ran this program:
\scriptsize
\begin{verbatim}
Integer? -34
The absolute value of -34 is 34
Integer? 1
The absolute value of 1 is 1
\end{verbatim}
\normalsize

What does the computer do when it sees this piece of code? First it prompts the
user for a number with the statement "n = raw\_input("Integer? ")". Next it reads
the line "if n $<$ 0:". If n is less than zero Python runs the line "print "The
absolute value of",n,"is",-n". Otherwise python runs the line "print "The
absolute value of",n,"is",n".
More formally, Python looks at whether the expression n $<$ 0 is true or false. An
if statement is followed by an indented block of statements that are run when
the expression is true. After the if statement is an optional else statement and
another indented block of statements. This 2nd block of statements is run if the
expression is false.
Expressions can be tested several different ways. Here is a table of all of
them:
\begin{tabular}{|c|c|}
\hline
operator&	function\\
\hline
$<$	&less than\\
\hline
$<=$	&less than or equal to\\
\hline
$>$	&greater than\\
\hline
$>=$	&greater than or equal to\\
\hline
$==$	&equal\\
\hline
$!=$	&not equal\\
\hline
\end{tabular}

Another feature of the if command is the elif statement. It stands for "else
if," which means that if the original if statement is false and the elif
statement is true, execute the block of code following the elif statement.
Here's an example:
\lstset{basicstyle=\scriptsize, numbers=left, captionpos=b, tabsize=4}
\begin{lstlisting}[caption=ifloop.py,language={Python},
xleftmargin=15pt, label=lst:ifloop]
a = 0
while a < 10:
    a = a + 1
    if a > 5:
        print (a,">",5)
    elif a <= 7:
        print (a,"<=",7)
    else:
        print ("Neither test was true")
\end{lstlisting}
and the output:
\scriptsize
\begin{verbatim}
1 <= 7
2 <= 7
3 <= 7
4 <= 7
5 <= 7
6 > 5
7 > 5
8 > 5
9 > 5
10 > 5
\end{verbatim}
\normalsize

Notice how the elif a <= 7 is only tested when the if statement fails to be
true. elif allows multiple tests to be done in a single if statement.
\subsubsection{If Examples}
\lstset{basicstyle=\scriptsize, numbers=left, captionpos=b, tabsize=4}
\begin{lstlisting}[caption=High\_low.py,language={Python},
xleftmargin=15pt, label=lst:highlow]
#Plays the guessing game higher or lower 
# (originally written by Josh Cogliati, improved by Quique)
 
#This should actually be something that is semi random like the
# last digits of the time or something else, but that will have to
# wait till a later chapter.  (Extra Credit, modify it to be random
# after the Modules chapter)
 
#This is for demonstration purposes only. 
# It is not written to handle invalid input like a full program would.
number = 78
guess = 0
 
while guess != number : 
    guess = raw_input("Guess an integer: ")
    guess = int(guess)
    if guess > number :
        print ("Too high")
 
    elif guess < number :
        print ("Too low")
    else:
        print ("Just right" )
\end{lstlisting}

Sample run:
\scriptsize
\begin{verbatim}
Guess an integer:100
Too high
Guess an integer:50
Too low
Guess an integer:75
Too low
Guess an integer:87
Too high
Guess an integer:81
Too high
Guess an integer:78
Just right
\end{verbatim}
\normalsize

\lstset{basicstyle=\scriptsize, numbers=left, captionpos=b, tabsize=4}
\begin{lstlisting}[caption=even.py,language={Python},
xleftmargin=15pt, label=lst:even]
#Asks for a number.
#Prints if it is even or odd
 
number = raw_input("Tell me a number: ")
number = float(number)
if number % 2 == 0:
    print (number,"is even.")
elif number % 2 == 1:
    print (number,"is odd.")
else:
    print (number,"is very strange.")
\end{lstlisting}

Sample runs.
\scriptsize
\begin{verbatim}
Tell me a number: 3
3 is odd.

Tell me a number: 2
2 is even.

Tell me a number: 3.14159
3.14159 is very strange.
\end{verbatim}
\normalsize

\lstset{basicstyle=\scriptsize, numbers=left, captionpos=b, tabsize=4}
\begin{lstlisting}[caption=average1.py,language={Python},
xleftmargin=15pt, label=lst:average1]
#keeps asking for numbers until 0 is entered.
#Prints the average value.
 
count = 0
sum = 0.0
number = 1.0  # set this to something that will not exit
#               the while loop immediately.
 
print ("Enter 0 to exit the loop")
 
while number != 0:
    number = raw_input("Enter a number: ")
    number = float(number)
    if number != 0:
       count = count + 1
       sum = sum + number
 
print "The average was:",sum/count
\end{lstlisting}

Sample runs
\scriptsize
\begin{verbatim}
Enter 0 to exit the loop
Enter a number:3
Enter a number:5
Enter a number:0
The average was: 4.0

Enter 0 to exit the loop
Enter a number:1
Enter a number:4
Enter a number:3
Enter a number:0
The average was: 2.66666666667
\end{verbatim}
\normalsize
\lstset{basicstyle=\scriptsize, numbers=left, captionpos=b, tabsize=4}
\begin{lstlisting}[caption=average2.py,language={Python},
xleftmargin=15pt, label=lst:average2]
#keeps asking for numbers until count have been entered.
#Prints the average value.
 
sum = 0.0
 
print ("This program will take several numbers, then average them.")
count = raw_input("How many numbers would you like to sum:")
count = int(count)
current_count = 0
 
while current_count < count:
    current_count = current_count + 1
    print ("Number",current_count)
    number = input("Enter a number: ")
    sum = sum + number
 
print "The average was:",sum/count
\end{lstlisting}
Sample runs
\scriptsize
\begin{verbatim}
This program will take several numbers, then average them.
How many numbers would you like to sum:2
Number 1
Enter a number:3
Number 2
Enter a number:5
The average was: 4.0

This program will take several numbers, then average them.
How many numbers would you like to sum:3
Number 1
Enter a number:1
Number 2
Enter a number:4
Number 3
Enter a number:3
The average was: 2.66666666667
\end{verbatim}
\normalsize

\subsubsection{Conditional Statements}
Many languages (like Java and PHP) have the concept of a one-line conditional
(called The Ternary Operator), often used to simplify conditionally accessing a
value. For instance (in Java):
\lstset{basicstyle=\scriptsize, numbers=left, captionpos=b, tabsize=4}
\begin{lstlisting}[caption=Ternary Operator Java,language={Java},
xleftmargin=15pt, label=lst:ternaryoperatorjava]
int in= ; // read from program input
 
// a normal conditional assignment
int res;
if(number < 0)
  res = -number;
else
  res = number;
 
// this can be simplified to
int res2 = (number < 0) ? -number : number;
\end{lstlisting}

For many years Python did not have the same construct natively, however you
could replicate it by constructing a tuple of results and calling the test as
the index of the tuple, like so:
\lstset{basicstyle=\scriptsize, numbers=left, captionpos=b, tabsize=4}
\begin{lstlisting}[caption=Ternary Operator 1,language={Python},
xleftmargin=15pt, label=lst:ternaryoperator1]
number = int(raw_input("Enter a number to get its absolute value:"))
res = (-number, number)[number > 0]
\end{lstlisting}

It is important to note that, unlike a built in conditional statement, both the
true and false branches are evaluated before returning, which can lead to
unexpected results and slower executions if you're not careful. To resolve this
issue, and as a better practice, wrap whatever you put in the tuple in anonymous
function calls (lambda notation) to prevent them from being evaluated until the
desired branch is called:
\lstset{basicstyle=\scriptsize, numbers=left, captionpos=b, tabsize=4}
\begin{lstlisting}[caption=Ternary Operator Lambda,language={Python},
xleftmargin=15pt, label=lst:ternaryoperatorlambda]
number = int(raw_input("Enter a number to get its absolute value:"))
res = (lambda: number, lambda: -number)[number < 0]()
\end{lstlisting}

Since Python 2.5 however, there has been an equivalent operator to The Ternary
Operator (though not called such, and with a totally different syntax):
\lstset{basicstyle=\scriptsize, numbers=left, captionpos=b, tabsize=4}
\begin{lstlisting}[caption=Ternary Operator Python,language={Python},
xleftmargin=15pt, label=lst:ternaryoperatorpython]
number = int(raw_input("Enter a number to get its absolute value:"))
res = -number if number < 0 else number
\end{lstlisting}

\subsubsection{Switch}

A switch is a control statement present in most computer programming languages
to minimize a bunch of If - elif statements. Sadly Python doesn't officially
support this statement, but with the clever use of an array or dictionary, we
can recreate this Switch statement that depends on a value.
\lstset{basicstyle=\scriptsize, numbers=left, captionpos=b, tabsize=4}
\begin{lstlisting}[caption=Switch Case,language={Python},
xleftmargin=15pt, label=lst:switchcase]
x = 1
 
def hello():
  print ("Hello")
 
def bye():
  print ("Bye")
 
def hola():
  print ("Hola is Spanish for Hello")
 
def adios():
  print ("Adios is Spanish for Bye")
 
# Notice that our switch statement is a regular variable, only that we added the
# function's name inside
# and there are no quotes
menu = [hello,bye,hola,adios]
 
# To call our switch statement, we simply make reference to the array with a
# pair of parentheses
# at the end to call the function
menu[3]()   # calls the adios function since is number 3 in our array.
 
menu[0]()   # Calls the hello function being our first element in our array.
 
menu[x]()   # Calls the bye function as is the second element on the array x = 1
\end{lstlisting}
This works because Python stores a reference of the function in the array at its
particular index, and by adding a pair of parentheses we are actually calling
the function. Here the last line is equivalent to:
\lstset{basicstyle=\scriptsize, numbers=left, captionpos=b, tabsize=4}
\begin{lstlisting}[caption=Swicht Case Function,language={Python},
xleftmargin=15pt, label=lst:swichtcasefunction]
if x==0:
    hello()
elif x==1:
    bye()
elif x==2:
    hola()
else:
    adios()
\end{lstlisting}
\subsubsection{Another way}
Another way is to use lambdas. Code pasted here without permissions.
\lstset{basicstyle=\scriptsize, numbers=left, captionpos=b, tabsize=4}
\begin{lstlisting}[caption=Switch Case Lambda,language={Python},
xleftmargin=15pt, label=lst:switchcaselambda]
result = {
  'a': lambda x: x * 5,
  'b': lambda x: x + 7,
  'c': lambda x: x - 2
}[value](x)
\end{lstlisting}

\subsubsection{While loops}
This is our first control structure. Ordinarily the computer starts with the
first line and then goes down from there. Control structures change the order
that statements are executed or decide if a certain statement will be run. As a
side note, decision statements (e.g., if statements) also influence whether or
not a certain statement will run. Here's the source for a program that uses the
while control structure:
\lstset{basicstyle=\scriptsize, numbers=left, captionpos=b, tabsize=4}
\begin{lstlisting}[caption=,language={Python},
xleftmargin=15pt, label=lst:]
a = 0
while a < 10 :
    a += 1
    print (a)
\end{lstlisting}

And here is the output:
\scriptsize
\begin{verbatim}
1
2
3
4
5
6
7
8
9
10
\end{verbatim}
\normalsize

So what does the program do? First it sees the line a = 0 and makes a zero. Then
it sees while a < 10: and so the computer checks to see if a < 10. The first
time the computer sees this statement a is zero so it is less than 10. In other
words while a is less than ten the computer will run the tabbed in statements.
Here is another example of the use of while:
\lstset{basicstyle=\scriptsize, numbers=left, captionpos=b, tabsize=4}
\begin{lstlisting}[caption=,language={Python},
xleftmargin=15pt, label=lst:]
a = 1
s = 0
print ('Enter Numbers to add to the sum.')
print ('Enter 0 to quit.')
while a != 0:
    print ('Current Sum: ', s)
    a = raw_input('Number? ')
    a = float(a)
    s += a
print ('Total Sum = ',s)
\end{lstlisting}

Enter Numbers to add to the sum.
\scriptsize
\begin{verbatim}
Enter 0 to quit.
Current Sum: 0
Number? 200
Current Sum: 200
Number? -15.25
Current Sum: 184.75
Number? -151.85
Current Sum: 32.9
Number? 10.00
Current Sum: 42.9
Number? 0
Total Sum = 42.9
\end{verbatim}
\normalsize

Notice how print 'Total Sum =',s is only run at the end. The while statement
only affects the lines that are tabbed in (a.k.a. indented). The  != means does
not equal so while a != 0 : means until a is zero run the tabbed in statements
that are afterwards.  Now that we have while loops, it is possible to have
programs that run forever. An easy way to do this is to write a program like
this:
\lstset{basicstyle=\scriptsize, numbers=left, captionpos=b, tabsize=4}
\begin{lstlisting}[caption=,language={Python},
xleftmargin=15pt, label=lst:]
while 1 == 1:
    print ("Help, I'm stuck in a loop.")
\end{lstlisting}

This program will output Help, I'm stuck in a loop. until the heat death of the
universe or you stop it. The way to stop it is to hit the Control (or Ctrl)
button and `c' (the letter) at the same time. This will kill the program. (Note:
sometimes you will have to hit enter after the Control C.)

\paragraph{Examples}~

\lstset{basicstyle=\scriptsize, numbers=left, captionpos=b, tabsize=4}
\begin{lstlisting}[caption=Fibonacci.py,language={Python},
xleftmargin=15pt, label=lst:fibonacci]
#This program calculates the Fibonacci sequence
a = 0
b = 1
count = 0
max_count = 20
while count < max_count:
    count = count + 1
    #we need to keep track of a since we change it
    old_a = a
    old_b = b
    a = old_b
    b = old_a + old_b
    #Notice that the , at the end of a print statement keeps it
    # from switching to a new line
    print (old_a,)
print()
\end{lstlisting}

Output:
\scriptsize
\begin{verbatim}
0 1 1 2 3 5 8 13 21 34 55 89 144 233 377 610 987 1597 2584 4181
\end{verbatim}
\normalsize

\lstset{basicstyle=\scriptsize, numbers=left, captionpos=b, tabsize=4}
\begin{lstlisting}[caption=Password.py,language={Python},
xleftmargin=15pt, label=lst:password]
# Waits until a password has been entered.  Use control-C to break out without
# the password.
 
# Note that this must not be the password so that the 
# while loop runs at least once.
password = "foobar"
 
#note that != means not equal
while password != "unicorn":
    password = raw_input("Password: ")
print ("Welcome in")
\end{lstlisting}

Sample run:
\scriptsize
\begin{verbatim}
Password:auo
Password:y22
Password:password
Password:open sesame
Password:unicorn
Welcome in
\end{verbatim}
\normalsize

\subsubsection{For Loops}
This is another way of using loops:
\lstset{basicstyle=\scriptsize, numbers=left, captionpos=b, tabsize=4}
\begin{lstlisting}[caption=For Loop Range 1,language={Python},
xleftmargin=15pt, label=lst:forlooprange1]
onetoten = range(1,11)
for count in onetoten:
    print (count)
\end{lstlisting}

The output:
\scriptsize
\begin{verbatim}
1
2
3
4
5
6
7
8
9
10
\end{verbatim}
\normalsize

The output looks very familiar, but the program code looks different. The first
line uses the range function. The range function uses two arguments like this
range(start,finish). start is the first number that is produced. finish is one
larger than the last number. Note that this program could have been done in a
shorter way:
\lstset{basicstyle=\scriptsize, numbers=left, captionpos=b, tabsize=4}
\begin{lstlisting}[caption=For Loop Range 2,language={Python},
xleftmargin=15pt, label=lst:forlooprange2]
for count in range(1,11):
    print (count)
\end{lstlisting}

Here are some examples to show what happens with the range command:
\lstset{basicstyle=\scriptsize, numbers=left, captionpos=b, tabsize=4}
\begin{lstlisting}[caption=Range Command 1,language={Python},
xleftmargin=15pt, label=lst:rangecommand1]
>>> range(1,10)
[1, 2, 3, 4, 5, 6, 7, 8, 9]
>>> range(-32, -20)
[-32, -31, -30, -29, -28, -27, -26, -25, -24, -23, -22, -21]
>>> range(5,21)
[5, 6, 7, 8, 9, 10, 11, 12, 13, 14, 15, 16, 17, 18, 19, 20]
>>> range(21,5)
[]
\end{lstlisting}

Another way to use the range() function in a for loop is to supply only one argument:
\lstset{basicstyle=\scriptsize, numbers=left, captionpos=b, tabsize=4}
\begin{lstlisting}[caption=Range Command 2,language={Python},
xleftmargin=15pt, label=lst:rangecommand2]
for a in range(10):
    print a,
\end{lstlisting}

The above code acts exactly the same as:
\lstset{basicstyle=\scriptsize, numbers=left, captionpos=b, tabsize=4}
\begin{lstlisting}[caption=Range Command 3,language={Python},
xleftmargin=15pt, label=lst:rangecommand3]
for a in range(0, 10):
    print a,
\end{lstlisting}

with 0 implied as the starting point. The output is
\scriptsize
\begin{verbatim}
0 1 2 3 4 5 6 7 8 9
\end{verbatim}
\normalsize

The code would cycle through the for loop 10 times as expected, but starting
with 0 instead of 1.  The next line for count in onetoten: uses the for control
structure. A for control structure looks like for variable in list:. list is
gone through starting with the first element of the list and going to the last.
As for goes through each element in a list it puts each into variable. That
allows variable to be used in each successive time the for loop is run through.
Here is another example to demonstrate:
\lstset{basicstyle=\scriptsize, numbers=left, captionpos=b, tabsize=4}
\begin{lstlisting}[caption=For Loop List 1,language={Python},
xleftmargin=15pt, label=lst:forlooplist1]
demolist = ['life',42, 'the universe', 6,'and',7,'everything']
for item in demolist:
    print ("The Current item is: %s" % item)
\end{lstlisting}

The output is:
\scriptsize
\begin{verbatim}
The Current item is: life
The Current item is: 42
The Current item is: the universe
The Current item is: 6
The Current item is: and
The Current item is: 7
The Current item is: everything
\end{verbatim}
\normalsize

Notice how the for loop goes through and sets item to each element in the list.
(Notice how if you don't want print to go to the next line add a comma at the
end of the statement (i.e. if you want to print something else on that line). )
So, what is for good for? The first use is to go through all the elements of a
list and do something with each of them. Here a quick way to add up all the
elements:

\lstset{basicstyle=\scriptsize, numbers=left, captionpos=b, tabsize=4}
\begin{lstlisting}[caption=For Loop List 2,language={Python},
xleftmargin=15pt, label=lst:forlooplist2]
list = [2,4,6,8]
sum = 0
for num in list:
    sum = sum + num
print ("The sum is: %d" % sum)
\end{lstlisting}

with the output simply being:
\scriptsize
\begin{verbatim}
The sum is:  20
\end{verbatim}
\normalsize

Or you could write a program to find out if there are any duplicates in a list like this program does:
\lstset{basicstyle=\scriptsize, numbers=left, captionpos=b, tabsize=4}
\begin{lstlisting}[caption=For Loop List 3,language={Python},
xleftmargin=15pt, label=lst:forlooplist3]
list = [4, 5, 7, 8, 9, 1,0,7,10]
list.sort()
prev = list[0]
del list[0]
for item in list:
    if prev == item:
        print ("Duplicate of ",prev," Found")
    prev = item
\end{lstlisting}

and for good measure:
\scriptsize
\begin{verbatim}
Duplicate of  7  Found
\end{verbatim}
\normalsize

How does it work? Here is a special debugging version:
\lstset{basicstyle=\scriptsize, numbers=left, captionpos=b, tabsize=4}
\begin{lstlisting}[caption=For Loop List 3 Debug,language={Python},
xleftmargin=15pt, label=lst:forlooplist3debug]
l = [4, 5, 7, 8, 9, 1,0,7,10]
print ("l = [4, 5, 7, 8, 9, 1,0,7,10]","\tl:",l)
l.sort()
print ("l.sort()","\tl:",l)
prev = l[0]
print ("prev = l[0]","\tprev:",prev)
del l[0]
print ("del l[0]","\tl:",l)
for item in l:
    if prev == item:
        print ("Duplicate of ",prev," Found")
    print ("if prev == item:","\tprev:",prev,"\titem:",item)
    prev = item
    print ("prev = item","\t\tprev:",prev,"\titem:",item)
\end{lstlisting}

with the output being:
\scriptsize
\begin{verbatim}
l = [4, 5, 7, 8, 9, 1,0,7,10]   l: [4, 5, 7, 8, 9, 1, 0, 7, 10]
l.sort()        l: [0, 1, 4, 5, 7, 7, 8, 9, 10]
prev = l[0]     prev: 0
del l[0]        l: [1, 4, 5, 7, 7, 8, 9, 10]
if prev == item:        prev: 0         item: 1
prev = item             prev: 1         item: 1
if prev == item:        prev: 1         item: 4
prev = item             prev: 4         item: 4
if prev == item:        prev: 4         item: 5
prev = item             prev: 5         item: 5
if prev == item:        prev: 5         item: 7
prev = item             prev: 7         item: 7
Duplicate of  7  Found
if prev == item:        prev: 7         item: 7
prev = item             prev: 7         item: 7
if prev == item:        prev: 7         item: 8
prev = item             prev: 8         item: 8
if prev == item:        prev: 8         item: 9
prev = item             prev: 9         item: 9
if prev == item:        prev: 9         item: 10
prev = item             prev: 10        item: 10
\end{verbatim}
\normalsize

Note: The reason there are so many print statements is because print statements
can show the value of each variable at different times, and help debug the
program. First the program starts with a old list. Next the program sorts the
list. This is so that any duplicates get put next to each other. The program
then initializes a prev(ious) variable. Next the first element of the list is
deleted so that the first item is not incorrectly thought to be a duplicate.
Next a for loop is gone into. Each item of the list is checked to see if it is
the same as the previous. If it is a duplicate was found. The value of prev is
then changed so that the next time the for loop is run through prev is the
previous item to the current. Sure enough, the 7 is found to be a duplicate.
The other way to use for loops is to do something a certain number of times.
Here is some code to print out the first 9 numbers of the Fibonacci series:
\lstset{basicstyle=\scriptsize, numbers=left, captionpos=b, tabsize=4}
\begin{lstlisting}[caption=Fibonacci For Loop,language={Python},
xleftmargin=15pt, label=lst:fibonacciforloop]
a = 1
b = 1
for c in range(1,10):
    print (a)
    n = a + b
    a = b
    b = n
print ("")
\end{lstlisting}

with the surprising output:
\scriptsize
\begin{verbatim}
1
1
2
3
5
8
13
21
34
\end{verbatim}
\normalsize

Everything that can be done with for loops can also be done with while loops but
for loops give a easy way to go through all the elements in a list or to do
something a certain number of times.
