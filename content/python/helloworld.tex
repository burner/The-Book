\section{Hello World}
The first program that every programmer writes is called the "Hello, World!"
program. This program simply outputs the phrase "Hello, World!" and then quits.
Let's write "Hello, World!" in Python!  Open up your text editor and create a
new file called hello.py containing just this line (you can copy-paste if you
want): 
\lstset{basicstyle=\scriptsize, numbers=left, captionpos=b, tabsize=4}
\begin{lstlisting}[caption=Fibonacci,language={Python},
xleftmargin=15pt, label=lst:fibonacci]
print("Hello, world!")
\end{lstlisting}
This program uses the print function, which
simply outputs its parameters to the terminal. print ends with a newline
character, which simply moves the cursor to the next line.  Now that you've
written your first program, let's run it in Python! This process differs
slightly depending on your operating system.
\begin{itemize}
	\item Create a folder on your computer to use for your Python programs,
such as ~/pythonpractice, and save your hello.py program in that folder.
	\item Open up the terminal program.
	\item Type cd \textasciitilde /pythonpractice to change directory to your 
pythonpractice folder, and hit Enter.
	\item Type python hello.py to run your program!
\end{itemize}

The program should print: Hello, world! 
