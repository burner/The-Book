\section{File I/O}
Read entire file:
\lstset{basicstyle=\scriptsize, numbers=left, captionpos=b, tabsize=4}
\begin{lstlisting}[caption=Reading a File,language={Python},
xleftmargin=15pt, label=lst:readingafile]
inputFileText = open("testit.txt", "r").read()
print(inputFileText)
\end{lstlisting}

In this case the "r" parameter means the file will be opened in read-only mode.
Read certain amount of bytes from a file:
\lstset{basicstyle=\scriptsize, numbers=left, captionpos=b, tabsize=4}
\begin{lstlisting}[caption=Reading number of bytes,language={Python},
xleftmargin=15pt, label=lst:readingnumberofbytes]
inputFileText = open("testit.txt", "r").read(123)
print(inputFileText)
\end{lstlisting}
When opening a file, one starts reading at the beginning of the file, if one
would want more random access to the file, it is possible to use seek() to
change the current position in a file and tell() to get to know the current
position in the file. This is illustrated in the following example:
\lstset{basicstyle=\scriptsize, numbers=left, captionpos=b, tabsize=4}
\begin{lstlisting}[caption=Tell and seek on File,language={Python},
xleftmargin=15pt, label=lst:tellandseekonfile]
>>> f=open("/proc/cpuinfo","r")
>>> f.tell()
0L
>>> f.read(10)
'processor\t'
>>> f.read(10)
': 0\nvendor'
>>> f.tell()
20L
>>> f.seek(10)
>>> f.tell()
10L
>>> f.read(10)
': 0\nvendor'
>>> f.close()
>>> f
<closed file '/proc/cpuinfo', mode 'r' at 0xb7d79770>
\end{lstlisting}

Here a file is opened, twice ten bytes are read, tell() shows that the current
offset is at position 20, now seek() is used to go back to position 10 (the same
position where the second read was started) and ten bytes are read and printed
again. And when no more operations on a file are needed the close() function is
used to close the file we opened.  Read one line at a time:
\lstset{basicstyle=\scriptsize, numbers=left, captionpos=b, tabsize=4}
\begin{lstlisting}[caption=Line by Line,language={Python},
xleftmargin=15pt, label=lst:linebyline]
for line in open("testit.txt", "r"):
    print line
\end{lstlisting}

In this case readlines() will return an array containing the individual lines of
the file as array entries. Reading a single line can be done using the
readline() function which returns the current line as a string. This example
will output an additional newline between the individual lines of the file, this
is because one is read from the file and print introduces another newline.
Write to a file requires the second parameter of open() to be "w", this will
overwrite the existing contents of the file if it already exists when opening
the file:
\lstset{basicstyle=\scriptsize, numbers=left, captionpos=b, tabsize=4}
\begin{lstlisting}[caption=Write to file,language={Python},
xleftmargin=15pt, label=lst:writetofile]
outputFileText = "Here's some text to save in a file"
open("testit.txt", "w").write(outputFileText)
\end{lstlisting}

Append to a file requires the second parameter of open() to be "a" (from append):
\lstset{basicstyle=\scriptsize, numbers=left, captionpos=b, tabsize=4}
\begin{lstlisting}[caption=Append to file,language={Python},
xleftmargin=15pt, label=lst:appendtofile]
outputFileText = "Here's some text to add to the existing file."
open("testit.txt", "a").write(outputFileText)
\end{lstlisting}

Note that this does not add a line break between the existing file content and
the string to be added.

\section{Testing Files}
Determine whether path exists:
\lstset{basicstyle=\scriptsize, numbers=left, captionpos=b, tabsize=4}
\begin{lstlisting}[caption=Test if file exists,language={Python},
xleftmargin=15pt, label=lst:testiffileexists]
import os
os.path.exists('<path string>')
\end{lstlisting}

There are some other convenient functions in os.path, where path.code.exists()
only confirms whether or not path exists, there are functions which let you know
if the path is a file, a directory, a mount point or a symlink. There is even a
function os.path.realpath() which reveals the true destination of a symlink:
\lstset{basicstyle=\scriptsize, numbers=left, captionpos=b, tabsize=4}
\begin{lstlisting}[caption=Test Path,language={Python},
xleftmargin=15pt, label=lst:testpath]
>>> import os
>>> os.path.isfile("/")
False
>>> os.path.isfile("/proc/cpuinfo")
True
>>> os.path.isdir("/")
True
>>> os.path.isdir("/proc/cpuinfo")
False
>>> os.path.ismount("/")
True
>>> os.path.islink("/")
False
>>> os.path.islink("/vmlinuz")
True
>>> os.path.realpath("/vmlinuz")
'/boot/vmlinuz-2.6.24-21-generic'
\end{lstlisting}

\section{Common File Operations}
To copy or move a file, use the shutil library.
\lstset{basicstyle=\scriptsize, numbers=left, captionpos=b, tabsize=4}
\begin{lstlisting}[caption=Copy File,language={Python},
xleftmargin=15pt, label=lst:copyfile]
import shutil
shutil.move("originallocation.txt","newlocation.txt")
shutil.copy("original.txt","copy.txt")
\end{lstlisting}

To perform a recursive copy it is possible to use copytree(), to perform a
recursive remove it is possible to use rmtree()
\lstset{basicstyle=\scriptsize, numbers=left, captionpos=b, tabsize=4}
\begin{lstlisting}[caption=Recursive File handling,language={Python},
xleftmargin=15pt, label=lst:recursivefilehandling]
import shutil
shutil.copytree("dir1","dir2")
shutil.rmtree("dir1")
\end{lstlisting}

To remove an individual file there exists the remove() function in the os
module:
\lstset{basicstyle=\scriptsize, numbers=left, captionpos=b, tabsize=4}
\begin{lstlisting}[caption=Remove File,language={Python},
xleftmargin=15pt, label=lst:removefile]
import os
os.remove("file.txt")
\end{lstlisting}
