\section{User Interaction}
Scripts don't normally take user input since they are usually designed to
perform small useful tasks. However, there are times where user input is really
important. There are two ways to retrieve user input: the first is using the
console window (Command Line Interface) and the second is using the Graphical
User Interface (GUI).

\subsection{Console Windows}

Python has two built in functions to retrieve users input on a console; the
first is input() and the second one is raw\_input(). These two functions have
different purposes and both accept a string argument which is displayed on the
terminal before accepting the user input.
\textit{These examples are Python 2.5 specific. In Python 3.0, the input()
function will behave as raw\_input(). To use the original behavior in Python 3.0
use eval(input()).}
The input() function expects a python instruction as the user input. This means
that the user response must be python coded - strings must include quotes or
double quotes. Once this function is called, the input entered by the user will
be evaluated, and return the result to the application. This can make it a
security risk in some cases, unless you're planning to be the only user of the
application.  While the raw\_input() function expects any type of input and
returns a python string, this second function fits more of our current needs.
\lstset{basicstyle=\scriptsize, numbers=left, captionpos=b, tabsize=4}
\begin{lstlisting}[caption=raw\_input example,language={Python},
xleftmargin=15pt, label=lst:rawinputexample]
# Ask user for his name and stores
name = raw_input("Please enter your name:")
# Display the name entered by the user.
print name
\end{lstlisting}

If the information entered by the user needs to be numeric to perform some
calculation the return value most be converted using the float() or int()
conversion function.
\lstset{basicstyle=\scriptsize, numbers=left, captionpos=b, tabsize=4}
\begin{lstlisting}[caption=Get Age,language={Python},
xleftmargin=15pt, label=lst:getage]
# Retrieve user age
age = int(raw_input("Please enter your age:"))
print age
\end{lstlisting}

As you may expect coding all the input for a big application and all the
validation will increase the file size. To make steps simpler we can create an
object (using a Class) that handles retrieving the data and validating.
\lstset{basicstyle=\scriptsize, numbers=left, captionpos=b, tabsize=4}
\begin{lstlisting}[caption=Input Validation,language={Python},
xleftmargin=15pt, label=lst:inputvalidation]
class ainput:   # We can't use input as it is a existent function name, so we use AInput for Advance Input
   ''' This class will create a object with a simpler coding interface to retrieve console input'''
   def __init__(self,msg=""):
      ''' This will create a instance of ainput object'''
      self.data = ""   # Initialize a empty data variable
      if not msg == "":
         self.ask(msg)
 
   def ask(self,msg, req=0):
      ''' This will display the prompt and retrieve the user input.'''
      if req == 0:
         self.data = raw_input(msg)   # Save the user input to a local object variable
      else:
         self.data = raw_input(msg + " (Require)")
 
      # Verify that the information was entered and its not empty. This will accept a space character. Better Validation needed
      if req == 1 and self.data == "":
         self.ask(msg,req)
 
   def getString(self):
      ''' Returns the user input as String'''
      return self.data
 
   def getInteger(self):
      ''' Returns the user input as a Integer'''
      return int(self.data)
 
   def getNumber(self):
      ''' Returns the user input as a Float number'''
      return float(self.data)
\end{lstlisting}
\textit{Notice we only validate if the prompt is required and it is not empty.
Also notice that if a prompt is required and the user just hit <ENTER> it can
fall in a infinite loop. You probably need a more advanced validation and error
verification method, but it's up to you.}
With this tool at our disposal displaying, retrieving and validating user input
is a breeze. The use of the code will run like this. For this example we are
using a single line to display the call and retrieve the information.
\lstset{basicstyle=\scriptsize, numbers=left, captionpos=b, tabsize=4}
\begin{lstlisting}[caption=Retrive and Validate Input,language={Python},
xleftmargin=15pt, label=lst:retriveandvalidateinput]
# import is not imported
name = ainput("Please enter your first name:").getString()
age = ainput("Now enter your age:").getInteger()
print name, age
To test this code copy the following to a python script file (e.g. userInteraction.py)
class ainput:   # We can't use input as it is a existent function name, so we use AInput for Advance Input
   ''' This class will create a object with a simpler coding interface to retrieve console input'''
   def __init__(self,msg=""):
      ''' This will create a instance of ainput object'''
      self.data = ""   # Initialize a empty data variable
      if not msg == "":
         self.ask(msg)
 
   def ask(self,msg, req=0):
      ''' This will display the prompt and retrieve the user input.'''
      if req == 0:
         self.data = raw_input(msg)   # Save the user input to a local object variable
      else:
         self.data = raw_input(msg + " (Require)")
 
      # Verify that the information was entered and its not empty. This will accept a space character. Better Validation needed
      if req == 1 and self.data == "":
         self.ask(msg,req)
 
   def getString(self):
      ''' Returns the user input as String'''
      return self.data
 
   def getInteger(self):
      ''' Returns the user input as a Integer'''
      return int(self.data)
 
   def getNumber(self):
      ''' Returns the user input as a Float number'''
      return float(self.data)
 
def main():
	# import is not imported
	name = ainput("Please enter your first name:").getString()
	age = ainput("Now enter your age:").getInteger()
	print name, age
 
if __name__ == '__main__':
	main()
\end{lstlisting}

\subsection{Graphical User Interface (GUI)}
Python has many different GUI toolkits, of which the most standard is Tkinter
which comes with Python. This section will be limited to basic Tkinter
programming fundamentals, a more detailed reference can be found in GUI
Programming Modules of this wikibook.  Lets proceed by creating a classic
developer Hello World example with the following code.
import Tkinter
\lstset{basicstyle=\scriptsize, numbers=left, captionpos=b, tabsize=4}
\begin{lstlisting}[caption=Tkinter Example,language={Python},
xleftmargin=15pt, label=lst:tkinterexample]
# Define input retrieve function for application input
def retrieve_text():
  print app_entry.get()

if __name__ == "__main__":

  # Create Window Object or Window Form
  app_win = Tkinter.Tk()

  # Create label object
  app_label = Tkinter.Label(app_win,text="Hello World")
  app_label.pack()

  # Create User Input Object
  app_entry = Tkinter.Entry(app_win)
  app_entry.pack()

  # Create Button Object
  app_button = Tkinter.Button(app_win,text="Print Value",command=retrieve_text)
  app_button.pack()

  # Initialize Graphical Loop
  app_win.mainloop()
\end{lstlisting}
On the first line we actually import the Tkinter library in use by using import
Tkinter. Next a function is created to retrieve the value from the input object
use in the GUI. For the moment the value will be print to the console.  Next we
create a windows object where all the GUI components (or widgets as the Tkinter
toolkit calls them) are place. Think of it like a painting canvas where we draw
our application interface.  Having define the windows we proceed to create a
Label widget. Notice the first argument on Label object call is the variable
holding the window object this is for binding purpose, the next argument is the
label text. The next line binds the Label object to the Window object by using
the pack method.
\textit{The Tkinter toolkit has two layout system: pack and grid, Pack is the
simpler to use while Grid provides a table like layout. Only one layout can be
in use for Window or Frame.}
Next stop is the User Input Object, the Tkinter library provides the Entry
Object (same as Textbox in other programming languages). Notice that again we
need the Window object as first argument. The next lines bind the entry object
with the window.  Next is the creation of the button object for the application.
Like for all the other objects the Window object is the first argument. The
button text value is filled much like we did with the label and the button's
action is filled using the command argument. Notice that the command argument
doesn't have quote as it is not a string text but a reference to the function.
The next line proceeds to bind the button to the window.  On the last line a
call to app\_win.mainloop() allows the Tkinter application to run.

\subsection{Advanced OOP GUI}
So far it seems tedious to create a GUI using Python, especially with so much
code to write for such a simple program. But when you bring OOP concepts into
the mix it is different. For a moment think of an application that requires a
couple of inputs with labels to identify them. The OOP methodology has the
advantage of letting us create a new object that presents the label and entry
object at the same time, is reusable and can be called with one line of code.
The code below does just that, it wraps a label and a entry object inside a
Tkinter Frame Object. The object require windows object as first argument and
the label text as the second, while the text function returns a string value for
the user input. A more advanced text box can include validation and data
retrieval.
\lstset{basicstyle=\scriptsize, numbers=left, captionpos=b, tabsize=4}
\begin{lstlisting}[caption=OO Tkinter,language={Python},
xleftmargin=15pt, label=lst:ootkinter]
# Import Tkinter Library
import Tkinter
 
# Define our Textbox Object
class textbox(Tkinter.Frame):
 
    # Object Initialize
    def __init__(self,parent, msg):
        # Initialize the Frame properties by explicit calling its Init method
        Tkinter.Frame.__init__(self,parent)
 
        # Create the Textbox Label on the Left side
        self.g_label = Tkinter.Label(self,text=msg)
        self.g_label.pack(side=Tkinter.LEFT,expand=False)
 
        # Create the Textbox Entry on the Right side
        self.g_entry = Tkinter.Entry(self)
        self.g_entry.pack(side=Tkinter.LEFT, fill=Tkinter.X, expand=True)
 
        # Proceed to pack the frame
        self.pack(fill=Tkinter.X, anchor=Tkinter.NW, expand=True)
 
    def text(self):
        # Textbox text value retrieval
        return self.gui['entry'].get()
\end{lstlisting}

Now on our multi-input application we can use this new object with one simple line for each instance:
\lstset{basicstyle=\scriptsize, numbers=left, captionpos=b, tabsize=4}
\begin{lstlisting}[caption=OO Tkinter usage,language={Python},
xleftmargin=15pt, label=lst:ootkinterusage]
name = textbox(win,"Name:")
\end{lstlisting}
Notice that we don't need to call the pack method as the text box object packs
itself on the last line of the \_\_init\_\_. This is not practical on complex layout
application since the object controls this function, but works wonders for small
script GUI.
