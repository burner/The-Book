\section{Regular Expression}
Python includes a module for working with regular expressions on strings. For
more information about writing regular expressions and syntax not specific to
Python, see the regular expressions wikibook. Python's regular expression syntax
is similar to Perl's To start using regular expressions in your Python scripts,
just import the "re" module:

\subsection{Pattern objects}
If you're going to be using the same regexp more than once in a program, or if
you just want to keep the regexps separated somehow, you should create a pattern
object, and refer to it later when searching/replacing.  To create a pattern
object, use the compile function.
\lstset{basicstyle=\scriptsize, numbers=left, captionpos=b, tabsize=4}
\begin{lstlisting}[caption=Pattern objects,language={Python},
xleftmargin=15pt, label=lst:patternobjects]
import re
foo = re.compile(r'foo(.{,5})bar', re.I+re.S)
\end{lstlisting}

The first argument is the pattern, which matches the string "foo", followed by
up to 5 of any character, then the string "bar", storing the middle characters
to a group, which will be discussed later. The second, optional, argument is the
flag or flags to modify the regexp's behavior. The flags themselves are simply
variables referring to an integer used by the regular expression engine. In
other languages, these would be constants, but Python does not have constants.
Some of the regular expression functions do not support adding flags as a
parameter when defining the pattern directly in the function, if you need any of
the flags, it is best to use the compile function to create a pattern object.
The r preceding the expression string indicates that it should be treated as a
raw string. This should normally be used when writing regexps, so that
backslashes are interpreted literally rather than having to be escaped.
The different flags are:\\
\begin{tabular}{|c|c|p{3.6cm}|}
\hline
Abbr& Full name&	Description\\
\hline
re.I	&re.IGNORECASE&	Makes the regexp case-insensitive\\
\hline
re.L	&re.LOCALE&	Makes the behavior of some special sequences (\textbackslash
w, \textbackslash W, \textbackslash b,
\textbackslash B, \textbackslash s, \textbackslash S) dependant on the current locale\\
\hline
re.M	&re.MULTILINE	&Makes the \^ and \$ characters match at the beginning and
end of each line, rather than just the beginning and end of the string\\
\hline
re.S	&re.DOTALL&	Makes the . character match every character including
newlines.\\
\hline
re.U	&re.UNICODE&	Makes \textbackslash w, \textbackslash W, \textbackslash
b, \textbackslash B, \textbackslash d, \textbackslash D, \textbackslash s,
\textbackslash S dependent on
Unicode character properties\\
\hline
re.X	&re.VERBOSE&	Ignores whitespace except when in a character class or
preceded by an non-escaped backslash, and ignores \# (except when in a character
class or preceded by an non-escaped backslash) and everything after it to the
end of a line, so it can be used as a comment. This allows for cleaner-looking
regexps.\\
\hline
\end{tabular}

\subsection{Matching and searching}
One of the most common uses for regular expressions is extracting a part of a
string or testing for the existence of a pattern in a string. Python offers
several functions to do this.  The match and search functions do mostly the same
thing, except that the match function will only return a result if the pattern
matches at the beginning of the string being searched, while search will find a
match anywhere in the string.
\lstset{basicstyle=\scriptsize, numbers=left, captionpos=b, tabsize=4}
\begin{lstlisting}[caption=Matching and searching,language={Python},
xleftmargin=15pt, label=lst:matchingandsearching]
>>> import re
>>> foo = re.compile(r'foo(.{,5})bar', re.I+re.S)
>>> st1 = 'Foo, Bar, Baz'
>>> st2 = '2. foo is bar'
>>> search1 = foo.search(st1)
>>> search2 = foo.search(st2)
>>> match1 = foo.match(st1)
>>> match2 = foo.match(st2)
\end{lstlisting}

In this example, match2 will be None, because the string st2 does not start with
the given pattern. The other 3 results will be Match objects (see below).  You
can also match and search without compiling a regexp:
\lstset{basicstyle=\scriptsize, numbers=left, captionpos=b, tabsize=4}
\begin{lstlisting}[caption=Search without compile 1,language={Python},
xleftmargin=15pt, label=lst:searchwithoutcompile1]
>>> search3 = re.search('oo.*ba', st1, re.I)
\end{lstlisting}

Here we use the search function of the re module, rather than of the pattern
object. For most cases, its best to compile the expression first. Not all of the
re module functions support the flags argument and if the expression is used
more than once, compiling first is more efficient and leads to cleaner looking
code.  The compiled pattern object functions also have parameters for starting
and ending the search, to search in a substring of the given string. In the
first example in this section, match2 returns no result because the pattern does
not start at the beginning of the string, but if we do:
\lstset{basicstyle=\scriptsize, numbers=left, captionpos=b, tabsize=4}
\begin{lstlisting}[caption=Search without compile 2,language={Python},
xleftmargin=15pt, label=lst:seachwithoutcompile2]
>>> match3 = foo.match(st2, 3)
\end{lstlisting}

it works, because we tell it to start searching at character number 3 in the
string.  What if we want to search for multiple instances of the pattern? Then
we have two options. We can use the start and end position parameters of the
search and match function in a loop, getting the position to start at from the
previous match object (see below) or we can use the findall and finditer
functions. The findall function returns a list of matching strings, useful for
simple searching. For anything slightly complex, the finditer function should be
used. This returns an iterator object, that when used in a loop, yields Match
objects. For example:
\lstset{basicstyle=\scriptsize, numbers=left, captionpos=b, tabsize=4}
\begin{lstlisting}[caption=Search return Iterator,language={Python},
xleftmargin=15pt, label=lst:searchreturniterator]
>>> str3 = 'foo, Bar Foo. BAR FoO: bar'
>>> foo.findall(str3)
[', ', '. ', ': ']
>>> for match in foo.finditer(str3):
...     match.group(1)
...
', '
'. '
': '
\end{lstlisting}

If you're going to be iterating over the results of the search, using the
finditer function is almost always a better choice.

\subsection{Match objects}
Match objects are returned by the search and match functions, and include
information about the pattern match.  The group function returns a string
corresponding to a capture group (part of a regexp wrapped in ()) of the
expression, or if no group number is given, the entire match. Using the search1
variable we defined above:
\lstset{basicstyle=\scriptsize, numbers=left, captionpos=b, tabsize=4}
\begin{lstlisting}[caption=Match function,language={Python},
xleftmargin=15pt, label=lst:matchfunction]
>>> search1.group()
'Foo, Bar'
>>> search1.group(1)
', '
\end{lstlisting}
	
Capture groups can also be given string names using a special syntax and
referred to by matchobj.group('name'). For simple expressions this is
unnecessary, but for more complex expressions it can be very useful.  You can
also get the position of a match or a group in a string, using the start and end
functions:
\lstset{basicstyle=\scriptsize, numbers=left, captionpos=b, tabsize=4}
\begin{lstlisting}[caption=Capture groups,language={Python},
xleftmargin=15pt, label=lst:capturegroups]
>>> search1.start()
0
>>> search1.end()
8
>>> search1.start(1)
3
>>> search1.end(1)
5
\end{lstlisting}

This returns the start and end locations of the entire match, and the start and
end of the first (and in this case only) capture group, respectively.

\subsection{Replacing}
Another use for regular expressions is replacing text in a string. To do this in
Python, use the sub function.  sub takes up to 3 arguments: The text to replace
with, the text to replace in, and, optionally, the maximum number of
substitutions to make. Unlike the matching and searching functions, sub returns
a string, consisting of the given text with the substitution(s) made.
\lstset{basicstyle=\scriptsize, numbers=left, captionpos=b, tabsize=4}
\begin{lstlisting}[caption=Replacing,language={Python},
xleftmargin=15pt, label=lst:replacing]
>>> import re
>>> mystring = 'This string has a q in it'
>>> pattern = re.compile(r'(a[n]? )(\w) ')
>>> newstring = pattern.sub(r"\1'\2' ", mystring)
>>> newstring
"This string has a 'q' in it"
\end{lstlisting}

This takes any single alphanumeric character (\\w in regular expression syntax)
preceded by "a" or "an" and wraps in in single quotes. The \\1 and \\2 in the
replacement string are backreferences to the 2 capture groups in the expression;
these would be group(1) and group(2) on a Match object from a search.  The subn
function is similar to sub, except it returns a tuple, consisting of the result
string and the number of replacements made. Using the string and expression from
before:
\lstset{basicstyle=\scriptsize, numbers=left, captionpos=b, tabsize=4}
\begin{lstlisting}[caption=Replace alphanumeric,language={Python},
xleftmargin=15pt, label=lst:replacealphanumeric]
>>> subresult = pattern.subn(r"\1'\2' ", mystring)
>>> subresult
("This string has a 'q' in it", 1)
\end{lstlisting}

\subsection{Other functions}
The re module has a few other functions in addition to those discussed above.
The split function splits a string based on a given regular expression:
\lstset{basicstyle=\scriptsize, numbers=left, captionpos=b, tabsize=4}
\begin{lstlisting}[caption=Split,language={Python},
xleftmargin=15pt, label=lst:split]
>>> import re
>>> mystring = '1. First part 2. Second part 3. Third part'
>>> re.split(r'\d\.', mystring)
['', ' First part ', ' Second part ', ' Third part']
\end{lstlisting}

The escape function escapes all non-alphanumeric characters in a string. This is
useful if you need to take an unknown string that may contain regexp
metacharacters like ( and . and create a regular expression from it.
\lstset{basicstyle=\scriptsize, numbers=left, captionpos=b, tabsize=4}
\begin{lstlisting}[caption=Escape,language={Python},
xleftmargin=15pt, label=lst:escape]
>>> re.escape(r'This text (and this) must be escaped with a "\" to use in a regexp.')
'This\\ text\\ \\(and\\ this\\)\\ must\\ be\\ escaped\\ with\\ a\\ \\"\\\\\\"\\ to\\ use\\ in\\ a\\ regexp\\.'
\end{lstlisting}
