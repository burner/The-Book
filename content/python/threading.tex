\section{Threading}
Threading in python is used to run multiple threads (tasks, function calls) at
the same time. Note that this does not mean, that they are executed on different
CPUs. Python threads will NOT make your program faster if it already uses 100 %
CPU time, probably you then want to look into parallel programming. If you are
interested in parallel progamming with python, please see here.  Python threads
are used in cases where the execution of a task involves some waiting. One
example would be interaction with a service hosted on another computer, such as
a webserver. Threading allows python to execute other code while waiting; this
is easily simulated with the sleep function.

\subsection{Examples}
Make a thread that prints numbers from 1-10, waits for 1 sec between:
\lstset{basicstyle=\scriptsize, numbers=left, captionpos=b, tabsize=4}
\begin{lstlisting}[caption=Sleep Function,language={Python},
xleftmargin=15pt, label=lst:sleepfunction]
import thread
import time
 
def loop1_10():
    for i in range(1, 11):
        time.sleep(1)
        print(i)
 
thread.start_new_thread(loop1_10, ())
\end{lstlisting}

\subsection{A Minimal Example with Object}
\lstset{basicstyle=\scriptsize, numbers=left, captionpos=b, tabsize=4}
\begin{lstlisting}[caption=Class Thread Example,language={Python},
xleftmargin=15pt, label=lst:classthreadexample]
#!/usr/bin/env python
import threading
import time
 
class MyThread(threading.Thread):
    def run(self):
        print("%s started!" % self.getName())
        time.sleep(1)
        print("%s finished!" % self.getName())
 
if __name__ == '__main__':
    for x in range(4):
        mythread = MyThread(name = "Thread-%d" % (x + 1))
        mythread.start()
        time.sleep(.9)
\end{lstlisting}

This should output:
\scriptsize
\begin{verbatim}
Thread-1 started!
Thread-2 started!
Thread-1 finished!
Thread-3 started!
Thread-2 finished!
Thread-4 started!
Thread-3 finished!
Thread-4 finished! 
\end{verbatim}
\normalsize
