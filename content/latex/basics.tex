\section{Document Classes}
The first information LaTeX needs to know when processing an input file is the type of document the author wants to create. This is specified with the \lstinline{\documentclass} command.
\begin{lstlisting}
\documentclass[options]{class}
\end{lstlisting}


Here class specifies the type of document to be created. The LaTeX distribution provides additional classes for other documents, including letters and slides. The options parameter customizes the behavior of the document class. The options have to be separated by commas. 

Example: an input file for a LaTeX document could start with the line
\begin{lstlisting}
\documentclass[11pt,twoside,a4paper]{article}
\end{lstlisting}

which instructs LaTeX to typeset the document as an article with a base font size of 11 points, and to produce a layout suitable for double sided printing on A4 paper.

Here are some document classes that can be used with LaTeX:

\begin{table*}[H]
\begin{tabular}{|c|p{12cm}|}
\hline
\texttt{article } &
 for articles in scientific journals, presentations, short reports, program documentation, invitations, ... 
\\
 \texttt{IEEEtran} &
 for articles with the IEEE Transactions format. 
\\
 \texttt{proc} &
 a class for proceedings based on the article class. 
\\
 \texttt{minimal} &
 is as small as it can get. It only sets a page size and a base font. It is mainly used for debugging purposes. 
\\
 \texttt{report} &
 for longer reports containing several chapters, small books, thesis, ... 
\\
 \texttt{book} &
 for real books 
\\
 \texttt{slides} &
 for slides. The class uses big sans serif letters. 
\\
 \texttt{memoir} &
 for changing sensibly the output of the document. It is based on the
\texttt{book} class, but you can create any kind of document with it
\\
 \texttt{letter} &
 for writing letters.  
\\
 \texttt{beamer} &
 for writing presentations (see ]]).  \\
\hline
\end{tabular}
\end{table*}

The most common options for the standard document classes are listed in the following table:

\begin{table*}[H]
\begin{tabular}{|c|p{9cm}|}
\hline
 \texttt{10pt, 11pt, 12pt} &
Sets the size of the main font in the document. If no option is specified, 10pt is assumed. 
\\
 \texttt{a4paper, letterpaper,}... &
Defines the paper size. The default size is \texttt{letterpaper}; However, many
European distributions of TeX now come pre-set for A4, not Letter, and this
is also true of all distributions of pdfLaTeX. Besides that, \texttt{a5paper,
b5paper, executivepaper}, and \texttt{legalpaper} can be specified.
\\
 \texttt{fleqn} &
Typesets displayed formulas left-aligned instead of centered.
\\
 \texttt{leqno} &
Places the numbering of formulae on the left hand side instead of the right. 
\\
 \texttt{titlepage, notitlepage} &
Specifies whether a new page should be started after the document title or not. The article class does not start a new page by default, while report and book do. 
\\
 \texttt{onecolumn, twocolumn} &
Instructs LaTeX to typeset the document in one column or two columns. 
\\
 \texttt{twoside, oneside} &
Specifies whether double or single sided output should be generated. The
classes \texttt{article} and \texttt{report} are single sided and the
\texttt{book} class is double sided by default. Note that this option concerns
the style of the document only. The option \texttt{twoside} does not tell the
printer you use that it should actually make a two-sided printout.
\\
 \texttt{landscape} &
Changes the layout of the document to print in landscape mode. 
\\
 \texttt{openright, openany} &
Makes chapters begin either only on right hand pages or on the next page
available. This does not work with the \texttt{article} class, as it does not
know about chapters. The \texttt{report} class by default starts chapters on
the next page available and the \texttt{book} class starts them on right hand
pages. 
\\
\texttt{draft} &
makes LaTeX indicate hyphenation and justification problems with a small square in the right-hand margin of the problem line so they can be located quickly by a human. It also suppresses the inclusion of images and shows only a frame where they would normally occur.
\\
\hline
\end{tabular}
 \caption{Document Classes}
\end{table*}

For example, if you want a report to be in 12pt type on A4, but printed one-sided in draft mode, you would use:
	
\begin{lstlisting}
\documentclass[12pt,a4paper,oneside,draft]{report}
\end{lstlisting}


\subsection{Packages}
While writing your document, you will probably find that there are some areas where basic LaTeX cannot solve your problem. If you want to include
graphics, colored text or source code from a file into your document, you need to enhance the capabilities of LaTeX. Such enhancements are called packages. Packages are activated with the

\begin{lstlisting}
\usepackage[options]{package}
\end{lstlisting}

command, where package is the name of the package and options is a list of keywords that trigger special features in the package. Some packages come with the LaTeX base distribution. Others are provided separately.

Modern TeX distributions come with a large number of packages pre-installed. If you are working on a Unix system, use the command \lstinline{texdoc} for accessing package documentation. For more information, see the ]] section. 

\subsection{Files you might Encounter}
When you work with LaTeX you will soon find yourself in a maze of files with various extensions and probably no clue. The following list explains the most common file types you might encounter when working with TeX:

\begin{table*}[H]
\begin{tabular}{|c|p{12cm}|}
\hline
 \texttt{.aux} & A file that transports information from one compiler run to the next. Among
other things, the \texttt{.aux} file is used to store information associated
with cross-references.  \\
 \texttt{.bbl} & Bibliography file output by BiBTeX and used by LaTeX \\
 \texttt{.bib}  & Bibliography database file \\
 \texttt{.blg} & BiBTeX log file.  \\
 \texttt{.bst} & BiBTeX style file.  \\
 \texttt{.cls} & Class files define what your document looks like. They are selected with the 
\begin{lstlisting}
\documentclass 
\end{lstlisting} command.  \\
 \texttt{.dtx} & Documented TeX. This is the main distribution format for LaTeX style files. If you process a \texttt{.dtx} file you get documented macro code of the LaTeX package contained in the \texttt{.dtx} file.  \\
 \texttt{.ins} & The installer for the files contained in the matching \texttt{.dtx} file. If you download a LaTeX package from the net, you will normally get a \texttt{.dtx} and a \texttt{.ins} file. Run LaTeX on the \texttt{.ins} file to unpack the \texttt{.dtx} file.  \\
 \texttt{.fd} & Font description file telling LaTeX about new fonts.  \\
 \texttt{.dvi} & Device Independent File. This is the main result of a LaTeX compile run with \textit{latex}. You can look at its content with a DVI previewer program or you can send it to a printer with dvips or a similar application.  \\
 \texttt{.pdf} & Portable Document Format. This is the main result of a LaTeX compile run with \textit{pdflatex}. You can look at its content or print it with any PDF viewer.  \\
 \texttt{.log} & Gives a detailed account of what happened during the last compiler run.  \\
 \texttt{.toc} & Stores all your section headers. It gets read in for the next compiler run and is used to produce the table of contents.  \\
 \texttt{.lof} & This is like \texttt{.toc} but for the list of figures.  \\
 \texttt{.lot} & And again the same for the list of tables.  \\
 \texttt{.idx} & If your document contains an index. LaTeX stores all the words that go into the index in this file. Process this file with makeindex.  \\
 \texttt{.ind} & The processed .idx file, ready for inclusion into your document on the next compile cycle.  \\
 \texttt{.ilg} & Logfile telling what makeindex did.  \\
 % TODO \texttt{.sty} & LaTeX Macro package. This is a file you can load into your LaTeX document using the \lstinline{\usepackage} command.  \\
 \texttt{.tex} & LaTeX or TeX input file. It can be compiled with latex.  \\
\hline
\end{tabular}
\caption{Common file extensions in LaTeX}
\end{table*}

\subsection{Big Projects}
When working on big documents, you might want to split the input file into several parts. LaTeX has three commands to insert a file into another when building the document.
The simplest is the 
\begin{lstlisting}
\lstinline{\input}
\end{lstlisting}
command:

\begin{lstlisting}
\input{filename}
\end{lstlisting}

\lstinline{\input} inserts the contents of another file, named \textit{filename.tex}; note that the .tex extension is omitted. For all practical purposes, \lstinline{\input} is no more than a simple, automated cut-and-paste of the source code in \textit{filename.tex}.

The other main inclusion command is \include:

\begin{lstlisting}
\include{filename}
\end{lstlisting}


The \lstinline{\include} command is different from \lstinline{\input} in that it's the output that is added instead of the commands from the other files. Therefore a new page will be created at every \lstinline{\include} command, which makes it appropriate to use it for large entities such as book chapters.

Very large documents (that usually include many files) take a very long time to compile, and most users find it convenient to test their last changes by including only the files they have been working on. One option is to hunt down all \lstinline{\include} commands in the inclusion hierarchy and to comment them out:

\begin{lstlisting}
%\include{filename1}
\include{filename2}
\include{filename3}
%\include{filename4}
\end{lstlisting}


In this case, the user wants to include only \textit{filename2.tex} and \textit{filename3.tex}. If the inclusion hierarchy is intricate, commenting can become error-prone: page numbering will change, and any cross references won't work. A better alternative is to retain the include calls and use the \lstinline{\includeonly} command \textit{in the preamble}:

\begin{lstlisting}
\includeonly{filename2,filename3}
\end{lstlisting}


This way, only \lstinline{\include} commands for the specified files will be executed, and inclusion will be handled in only one place. Note that there must be no spaces between the filenames and the commas.

Remember that the input file should omit all the commands referring to the main document structure, which should be kept in the original document file. This includes lines containing usepackages, document class, and everything but the code strictly referring to the section that is to be included. In this way you'll avoid finding characters of your code in the output document, or worse, not finding anything after the included file, in case you forget to erase the 

\begin{lstlisting}
\end{document}
\end{lstlisting}
 

line of your included file.

\subsection{Picking suitable filenames}
Never, ever use directories (folders) or file names that contain spaces. Although your operating system probably supports them, some don't, and they will only cause grief and tears with TeX. Make filenames as short or as long as you wish, but strictly avoid spaces. Stick to lower-case letters without accents (a-z), the digits 0-9, the hyphen (-), and the full point or period (.), (similar to the conventions for a Web URL): it will let you refer to TeX files over the Web more easily and make your files more portable. Some operating systems do not distinguish between upper-case and lower-case letters, others do. Therefore it's best not to mix them.
