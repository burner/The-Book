\section{Errors and Warnings}
LaTeX describes what it's typesetting while it does it, and if it encounters
something it doesn't understand or can't do, it will display a message saying
what's wrong. It may also display warnings for less serious conditions.

\textit{Don't panic if you see error messages}: it's very common to mistype or
mis-spell commands, forget curly braces, type a forward slash instead of a
backslash, or use a special character by mistake. Errors are easily spotted and
easily corrected in your editor, and you can then run LaTeX again to check you
have fixed everything. Some of the most common errors are described in next
sections.

\subsection{Error messages} The format of an error message is always the same.
Error messages begin with an exclamation mark at the start of the line, and
give a description of the error, followed by another line starting with the
number, which refers to the line-number in your document file which LaTeX was
processing when the error was spotted. Here's an example, showing that the user
mistyped the <tt>\tableofcontents</tt>
command:


\begin{verbatim}
! Undefined control sequence.
l.6 \tableofcotnetns

\end{verbatim}

When LaTeX finds an error like this, it displays the error message and pauses. You must type one of the following letters to
continue:

\begin{tabular}{c c} \hline
Key 
& Meaning
\\ \hline
x
& Stop immediately and \texttt{x}it the program.
\\ \hline
q
& Carry on\texttt{q}uietly as best you can and don't bother me with any more error messages.
\\ \hline
e
& Stop the program but re-position the text in my\texttt{e}ditor at the point where you found the error (This only works if you're using an editor which LaTeX can communicate with).
\\ \hline
h
& Try to give me more\texttt{h}elp.
\\ \hline
i
& (followed by a correction) means\texttt{i}nput the correction in place of the error and carry on (This is only a temporary fix to get the file processed. You still have to make that correction in the editor).
\\ \hline
r
&\texttt{r}un in non-stop mode. Plow through any errors, unless too many pile up and it fails (100 errors).
\\ \hline
\end{tabular}

Some systems (Emacs is one example) run LaTeX with a "nonstop" switch turned on, so it will always process through to the end of the file, regardless of errors, or until a limit is reached.

\subsection{Warnings}
Warnings don't begin with an exclamation mark: they are just comments by LaTeX
about things you might want to look into, such as overlong or underrun lines
(often caused by unusual hyphenations, for example), pages running short or
long, and other typographical niceties (most of which you can ignore until
later).  Unlike other systems, which try to hide unevennesses in the text
(usually unsuccessfully) by interfering with the letterspacing, LaTeX takes the
view that the author or editor should be able to contribute. While it is
certainly possible to set LaTeX's parameters so that the spacing is
sufficiently sloppy that you will almost never get a warning about
badly-fitting lines or pages, you will almost certainly just be delaying
matters until you start to get complaints from your readers or publishers.

\subsection{Examples}
Only a few common error messages are given here: those most likely to be
encountered by beginners. If you find another error message not shown here, and
it's not clear what you should do, ask for help.

Most error messages are self-explanatory, but be aware that the place where
LaTeX spots and reports an error may be later in the file than the place where
it actually occurred. For example if you forget to close a curly brace which
encloses, say, italics, LaTeX won't report this until something else occurs
which can't happen until the curly brace is encountered (e.g. the end of the
document!) Some errors can only be righted by humans who can read and
understand what the document is supposed to mean or look like.

Newcomers should remember to check the list of special characters: a very large
number of errors when you are learning LaTeX are due to accidentally typing a
special character when you didn't mean to. This disappears after a few days as
you get used to them.

\subsection{Too many \} }
\begin{verbatim}
! Too many }'s.
l.6 \date December 2004}

\end{verbatim}

The reason LaTeX thinks there are too many \}'s here is that the opening curly
brace is missing after the \verb|\date| control sequence and before the word
December, so the closing curly brace is seen as one too many (which it is!). In
fact, there are other things which can follow the \verb|\date| command apart
from a date in curly braces, so LaTeX cannot possibly guess that you've missed
out the opening curly brace until it finds a closing one!

\subsection{Undefined control sequence}


\begin{verbatim}
! Undefined control sequence.
l.6 \dtae
{December 2004}

\end{verbatim}

In this example, LaTeX is complaining that it has no such command ("control
sequence") as \verb|\data|. Obviously it's been mistyped, but only a human
can detect that fact: all LaTeX knows is that \verb|\data| is not a command
it knows about: it's undefined. Mistypings are the most common source of
errors. If your editor has drop-down menus to insert common commands and
environments, use them!

\subsection{Not in Mathematics Mode}
\begin{verbatim}
! Missing $ inserted
\end{verbatim}

A character that can only be used in the mathematics was inserted in normal
text. Either switch to mathematic mode via\verb|\begin{math}...\end{math}| or
use the 'quick math mode':\verb|\ensuremath{math stuff}|

This can also happen if you use the wrong character encoding, for example using
utf8 without \verb|\usepackage[utf8]{inputenc}|" or using iso8859-1 without
\verb|\usepackage[latin1]{inputenc}|", there are several character encoding
formats, make sure to pick the right one.

\subsection{Runaway argument}
\begin{verbatim}
Runaway argument?
{December 2004 \maketitle
! Paragraph ended before \date was complete.
<to be read again>
\par
l.8

\end{verbatim}

In this error, the closing curly brace has been omitted from the date. It's the
opposite of the error of too many \}'s, and it results in \verb|\maketitle|
trying to format the title page while LaTeX is still expecting more text for
the date! As\verb|\maketitle| creates new paragraphs on the title page, this is
detected and LaTeX complains that the previous paragraph has ended
but\verb|\date| is not yet finished.

\subsection{Underfull hbox}
\begin{verbatim}
Underfull \hbox (badness 1394) in paragraph
at lines 28--30
[][]\LY1/brm/b/n/10 Bull, RJ: \LY1/brm/m/n/10
Ac-count-ing in Busi-
[94]

\end{verbatim}

This is a warning that LaTeX cannot stretch the line wide enough to fit,
without making the spacing bigger than its currently permitted maximum. The
badness (0-10,000) indicates how severe this is (here you can probably ignore a
badness of 1394). It says what lines of your file it was typesetting when it
found this, and the number in square brackets is the number of the page onto
which the offending line was printed. The codes separated by slashes are the
typeface and font style and size used in the line. Ignore them for the moment. 

This comes up if you force a linebreak, e.g.,\verb|\\|, and have a return
before it. Normally TeX ignores linebreaks, providing full paragraphs to ragged
text. In this case it is necessary to pull the linebreak up one line to the end
of the previous sentence.

\subsection{Overfull hbox}
\begin{verbatim}
[101]
Overfull \hbox (9.11617pt too wide) in paragraph
at lines 860--861
[]\LY1/brm/m/n/10 Windows, \LY1/brm/m/it/10 see
\LY1/brm/m/n/10 X Win-
\end{verbatim}

An overfull\verb|\hbox| means that there is a hyphenation or justification
problem: moving the last word on the line to the next line would make the
spaces in the line wider than the current limit; keeping the word on the line
would make the spaces smaller than the current limit, so the word is left on
the line, but with the minimum allowed space between words, and which makes the
line go over the edge.

The warning is given so that you can find the line in the code that originates
the problem (in this case: 860-861) and fix it. The line on this example is too
long by a shade over 9pt. The chosen hyphenation point which minimizes the
error is shown at the end of the line (Win-). Line numbers and page numbers are
given as before. In this case, 9pt is too much to ignore (over 3mm), and a
manual correction needs making (such as a change to the hyphenation), or the
flexibility settings need changing.

If the "overfull" word includes a forward slash, such as "\verb|input/output|",
this should be properly typeset as "\verb|inpup \ output|".
The use of \verb|/| has the same effect as using the
"\verb|/|" character, except that it can form the end of a line (with
the following words appearing at the start of the next line). The
"\verb|/|" character is typically used in units, such as
"\verb|mm/year|" character, which should not be broken over multiple
lines.

\subsection{Missing package}
\begin{verbatim}
! LaTeX Error: File paralisy.sty not found.
Type X to quit or <RETURN> to proceed,
or enter new name. (Default extension: sty)
Enter file name:

\end{verbatim}

When you use the \verb|\usepackage| command to request LaTeX to use a certain
package, it will look for a file with the specified name and the filetype
\verb|.sty|. In this case the user has mistyped the name of the paralist
package, so it's easy to fix. However, if you get the name right, but the
package is not installed on your machine, you will need to download and install
it before continuing. If you don't want to affect the global installation of
the machine, you can simply download from Internet the necessary \verb|.sty|
file and put it in the same folder of the document you are compiling.

\subsection{Package babel Warning: No hyphenation patterns were loaded for the language X}
Although this is a warning from the Babel package and not from LaTeX, this
error is very common and (can) give some strange hyphenation (word breaking)
problems in your document. Wrong hyphenation rules can decrease the neatness of
your document.
\begin{verbatim}
Package babel Warning: No hyphenation patterns were loaded for
(babel)                the language `Latin'
(babel)                I will use the patterns loaded for \language=0 instead.
\end{verbatim}

This can happen after the usage of: 
\begin{lstlisting}
\lstset{language=<source lang="latex" enclose="none">}
\usepackage[latin]{babel}
\end{lstlisting}

