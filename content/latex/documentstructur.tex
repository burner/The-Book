\section{The \texttt{document} environment}
The main point of writing a text is to convey ideas, information, or knowledge
to the reader. The reader will understand the text better if these ideas are
well-structured, and will see and feel this structure much better if the
typographical form reflects the logical and semantic structure of the content.

LaTeX is different from other typesetting systems in that you just have to tell
it the logical and semantical structure of a text. It then derives the
typographical form of the text according to the rules in the document class
file and in various style files. LaTeX allows users to structure their
documents with a variety of hierarchal constructs, including chapters,
sections, subsections and paragraphs. 

After the Document Class Declaration, the text of your document is enclosed
between two commands which identify the beginning and end of the actual
document:
\begin{lstlisting}[language={TeX}, breaklines=true,xleftmargin=15pt]
\documentclass[11pt,a4paper,oneside]{report}

\begin{document}
...
\end{document}
\end{lstlisting}


You would put your text where the dots are. The reason for marking off the
beginning of your text is that LaTeX allows you to insert extra setup
specifications before it (where the blank line is in the example above: we'll
be using this soon). The reason for marking off the end of your text is to
provide a place for LaTeX to be programmed to do extra stuff automatically at
the end of the document, like making an index.

A useful side-effect of marking the end of the document text is that you can
store comments or temporary text underneath the \texttt{\\end\{document\}} in the
knowledge that LaTeX will never try to typeset them:
\begin{lstlisting}[language={TeX}, breaklines=true,xleftmargin=15pt]
...
\end{document}
Don't forget to get the extra chapter from Dorando!
\end{lstlisting}

\subsection{ Preamble }
The \textit{preamble} is everything from the start of the LaTeX source file
until the \texttt{\\begin\{document\}} command. It normally contains commands
that affect the entire document.
\begin{lstlisting}[language={TeX}, breaklines=true,xleftmargin=15pt]
% simple.tex - A simple article to illustrate document structure.
\documentclass{article}
\usepackage{mathptmx}
\begin{document}
\end{lstlisting}


The first line is a comment (as denoted by the \% sign). The
\lstinline{\documentclass} command takes an argument, which in this case is
article, because that's the type of document we want to produce. It is also
possible to create your own, as is often done by journal publishers, who simply
provide you with their own class file, which tells LaTeX how to format your
content. But we'll be happy with the standard article class for now.
\lstinline{\usepackage} is an important command that tells LaTeX to utilize
some external macros. In this instance, we specified \lstinline{mathptmx} which
means LaTeX will use the Postscript Times type 1 font instead of the default
ComputerModern font. And finally, the \texttt{\\begin\{document\}}. This
strictly isn't part of the preamble, but I'll put it here anyway, as it implies
the end of the preamble by nature of stating that the document is now starting.

\subsection{ Top Matter }
At the beginning of most documents there will be information about the document
itself, such as the title and date, and also information about the authors,
such as name, address, email etc. All of this type of information within LaTeX
is collectively referred to as \textit{top matter}. Although never explicitly
specified (there is no \texttt{\\topmatter} command) you are likely
to encounter the term within LaTeX documentation.

A simple example:
\begin{lstlisting}[language={TeX}, breaklines=true,xleftmargin=15pt]
\documentclass[11pt,a4paper,oneside]{report}
\begin{document}
\title{How to Structure a LaTeX Document}
\author{Andrew Roberts}
\date{December 2004}
\maketitle
\end{document}
\end{lstlisting}

The \texttt{\\title}, \texttt{\\author}, and
\texttt{\date} commands are self-explanatory. You put the title,
author name, and date in curly braces after the relevant command. The title and
author are usually compulsory (at least if you want LaTeX to write the title
automatically); if you omit the \texttt{\date} command, LaTeX uses
today's date by default. You always finish the top matter with the
\texttt{\maketitle} command, which tells LaTeX that it's complete
and it can typeset the title according to the information you have provided and
the class (style) you are using. If you omit \texttt{\maketitle}, the titling
will never be typeset (unless you write your own).

Here is a more complicated example:
{{LaTeX/Usagecode= &
\title{How to Structure a \LaTeX{} Document}
\author{Andrew Roberts\\
  School of Computing,\\
  University of Leeds,\\
  Leeds,\\
  United Kingdom,\\
  LS2 1HE\\
  \texttt{andyr@comp.leeds.ac.uk}<!---->}
\date{\today}
\maketitle
}}

as you can see, you can use commands as arguments of {{LaTeX/LaTeXcode=\title
&}} and the others. The double backslash ({{LaTeX/LaTeXcode=\\ &}}) is the
LaTeX command for forced linebreak. LaTeX normally decides by itself where to
break lines, and it's usually right, but sometimes you need to cut a line
short, like here, and start a new one.

If there are two authors separate them with the {{LaTeX/LaTeXcode=\and &}}
command:
{{LaTeX/Usagecode= &
\title{Our Fun Document}
\author{Jane Doe \and John Doe} 
\date{\today}
\maketitle
}}

If you are provided with a class file from a publisher, or if you use the AMS
article class ({{LaTeX/Packageamsart &}}), then you can use several different
commands to enter author information. The email address is at the end, and the
{{LaTeX/LaTeXcode=\texttt &}} commands formats the email address using a
mono-spaced font. The built-in command called {{LaTeX/LaTeXcode=\today &}} will
be replaced with the current date when processed by LaTeX. But you are free to
put whatever you want as a date, in no set order. If braces are left empty,
then the date is omitted. 

Using this approach, you can create only basic output whose layout is very hard
to change. If you want to create your title freely, see the \_CreationTitle
Creation section. &

\subsection{ Abstract }
As most research papers have an abstract, there are predefined commands for
telling LaTeX which part of the content makes up the abstract. This should
appear in its logical order, therefore, after the top matter, but before the
main sections of the body. This command is available for the document classes
\textit{article} and \textit{report}, but not \textit{book}. 

\begin{lstlisting}
\documentclass{article}

\begin{document}

\begin{abstract}
Your abstract goes here...
...
\end{abstract}
...
\end{document}
\end{lstlisting}

By default, LaTeX will use the word "Abstract" as a title for your abstract, if
you want to change it into anything else, e.g. "Executive Summary", add the
following line in the preamble:

\begin{lstlisting}
\renewcommand{\abstractname}{Executive Summary}
\end{lstlisting}


\subsection{ Sectioning Commands }
The commands for inserting sections are fairly intuitive. Of course, certain
commands are appropriate to different document classes. For example, a book has
chapters but an article doesn't. Here is an edited version of some of the
structure commands in use from \textit{simple.tex}.

\begin{lstlisting}
\section{Introduction}
This section's content...

\section{Structure}
This section's content...

\subsection{Top Matter}
This subsection's content...

\subsection{Article Information}
This subsection's content...
\end{lstlisting}


Notice that you do not need to specify section numbers; LaTeX will sort that
out for you. Also, for sections, you do not need to markup which content
belongs to a given block, using \lstinline{\begin} and \lstinline{\end}
commands, for example. LaTeX provides 7 levels of depth for defining sections:

\begin{tabular}
! Command
! Level
! Comment
\\
 \texttt{\part{\textit{part} &}}
 style="text &-align: center"   &-1
 not in letters &
\\
 \texttt{\chapter{\textit{chapter} &}}
 style="text &-align: center"  0 &
 only books and reports &
\\
 \texttt{\section{\textit{section} &}}
 style="text &-align: center"  1 &
 not in letters &
\\
 \texttt{\subsection{\textit{subsection} &}}
 style="text &-align: center"  2 &
 not in letters &
\\
 \texttt{\subsection{\textit{subsection} &}}
 style="text &-align: center"  3 &
 not in letters &
\\
 \texttt{\paragraph{\textit{paragraph} &}}
 style="text &-align: center"  4 &
 not in letters &
\\
 \texttt{\subparagraph{\textit{subparagraph} &}}
 style="text &-align: center"  5 &
 not in letters &
\\
\end{tabular}

All the titles of the sections are added automatically to the table of contents
(if you decide to insert one). But if you make manual styling changes to your
heading, for example a very long title, or some special line-breaks or unusual
font-play, this would appear in the Table of Contents as well, which you almost
certainly don't want. LaTeX allows you to give an optional extra version of the
heading text which only gets used in the Table of Contents and any running
heads, if they are in effect. This optional alternative heading goes in [square
brackets] before the curly braces:

\begin{lstlisting}
\section[Effect on staff turnover]{An analysis of the
effect of the revised recruitment policies on staff
turnover at divisional headquarters}
\end{lstlisting}

\paragraph{ Section numbering }
Numbering of the sections is performed automatically by LaTeX, so don't bother
adding them explicitly, just insert the heading you want between the curly
braces.  Parts get roman numerals (Part I, Part II, etc.); chapters and
sections get decimal numbering like this document, and appendices (which are
just a special case of chapters, and share the same structure) are lettered (A,
B, C, etc.). You can change the depth to which section numbering occurs, so you
can turn it off selectively. By default it is set to 2. If you only want parts,
chapters, and sections numbered, not subsections or subsections etc., you can
change the value of the \texttt{secnumdepth} counter using the
\texttt{\setcounter} command, giving the depth level from the previous table.
For example, if you want to change it to "1":
\begin{lstlisting}
\setcounter{secnumdepth}{1}
\end{lstlisting}

A related counter is \texttt{tocdepth}, which specifies what depth to take the
Table of Contents to. It can be reset in exactly the same way as
\texttt{secnumdepth}. For example:
\begin{lstlisting}
\setcounter{tocdepth}{3}
\end{lstlisting}

To get an unnumbered section heading which does not go into the Table of
Contents, follow the command name with an asterisk before the opening curly
brace:
\begin{lstlisting}
\subsection*{Introduction}
\end{lstlisting}

All the divisional commands from \\part* to \\subparagraph* have this "starred"
version which can be used on special occasions for an unnumbered heading when
the setting of \texttt{secnumdepth} would normally mean it would be numbered.

If you want the unnumbered section to be in the table of contents anyway, use
the \texttt{\\addcontentsline} command like this:
\begin{lstlisting}
\section*{Introduction}
\addcontentsline{toc}{section}{Introduction}
\end{lstlisting}

Note if you use pdf bookmarks you will need to add a phantom section so that
bookmark will lead to the correct place in the document:
\begin{lstlisting}
\phantomsection
\addcontentsline{toc}{section}{Introduction}
\section*{Introduction} 
\end{lstlisting}

For chapters you will also need to clear the page (this will also correct page
numbering in the ToC):
\begin{lstlisting}
\cleardoublepage
\phantomsection
\addcontentsline{toc}{chapter}{Bibliography}
\bibliographystyle{unsrt}
\bibliography{my\_bib\_file}
\end{lstlisting}

The value where the section numbering starts from can be set with the following
command:
\begin{lstlisting}
\setcounter{section}{4}
\end{lstlisting}

The next section after this command will now be numbered 5.  Any counter can be
incremented/decremented with the following command:
\begin{lstlisting}
\addtocounter{counter}{integer}
\end{lstlisting}

The \\phantomsection command is defined in the hyperref package.

\subsection{ Appendices }
The separate numbering of appendices is also supported by LaTeX. The
\texttt{\\appendix} macro can be used to indicate that following sections or
chapters are to be numbered as appendices.

In the report or book classes this gives:
\begin{lstlisting}
\appendix
\chapter{First Appendix}
\end{lstlisting}

For the article class use:
\begin{lstlisting}
\appendix
\section{First Appendix}
\end{lstlisting}

\subsection{ Ordinary paragraphs }
Paragraphs of text come after section headings. Simply type the text and leave
a blank line between paragraphs. The blank line means "start a new paragraph
here": it does \textbf{not} mean you get a blank line in the typeset output.
For formatting paragraph indents and spacing between paragraphs, refer to the
#Paragraph\_IndentsFormatting section. &

\subsection{ Table of contents }
All auto-numbered headings get entered in the Table of Contents (ToC)
automatically. You don't have to print a ToC, but if you want to, just add the
command \texttt{\\tableofcontents} at the point where you want it printed
(usually after the Abstract or Summary).

Entries for the ToC are recorded each time you process your document, and
reproduced the next time you process it, so you need to re-run LaTeX one extra
time to ensure that all ToC pagenumber references are correctly calculated.
We've already seen how to use the optional argument to the sectioning commands
to add text to the ToC which is slightly different from the one printed in the
body of the document. It is also possible to add extra lines to the ToC, to
force extra or unnumbered section headings to be included.

The commands \lstinline{\listoffigures} and \lstinline{\listoftables} work in
exactly the same way as \lstinline{\tableofcontents} to automatically list all
your tables and figures. If you use them, they normally go after the
\lstinline{\tableofcontents} command. The \lstinline{\tableofcontents} command
normally shows only numbered section headings, and only down to the level
defined by the \lstinline{tocdepth} counter, but you can add extra entries with
the \lstinline{\addcontentsline} command. For example if you use an unnumbered
section heading command to start a preliminary piece of text like a Foreword or
Preface, you can write:
\begin{lstlisting}
\subsection*{Preface}
\addcontentsline{toc}{subsection}{Preface}
\end{lstlisting}

This will format an unnumbered ToC entry for "Preface" in the "subsection"
style. You can use the same mechanism to add lines to the List of Figures or
List of Tables by substituting \texttt{lof} or \texttt{lot} for \texttt{toc}.
If the hyperref package is used and the link does not point correct to the
chapter, the command \lstinline{\\phantomsection} in combination with
\lstinline{\clearpage} or \lstinline{\cleardoublepage} can be used (see also
\_and\_Cross-referencing#The\_hyperref\_package\_and\_.5CphantomsectionLabels\_and\_Cross-referencing):
\begin{lstlisting}
\cleardoublepage
\phantomsection
\addcontentsline{toc}{chapter}{List of Figures}
\listoffigures
\end{lstlisting}

To change the title of the TOC, you have to paste this command
\begin{lstlisting}
\renewcommand{\contentsname}{<New table of contents title>}
\end{lstlisting}

in your document preamble.  The List of Figures (LoF) and List of Tables (LoT)
names can be changed by replacing the 
\begin{lstlisting}
\contentsname
\end{lstlisting}
with 
\begin{lstlisting}
\listfigurename
\end{lstlisting}
for LoF and 
\begin{lstlisting}
\listtablename
\end{lstlisting}
for LoT.

\paragraph{ Depth }
The default ToC will list headings of level 3 and above.  To change how deep
the table of contents displays automatically the following command can be used
in the preamble:
\begin{lstlisting}
\setcounter{tocdepth}{4}
\end{lstlisting}

This will make the table of contents include everything down to paragraphs.
The levels are defined above on this page.

\subsection{ The Bibliography }
Any good research paper will have a whole list of references. There are two
ways to insert your references into LaTeX: *you can embed them within the
document itself. It's simpler, but it can be time-consuming if you are writing
several papers about similar subjects so that you often have to cite the same
books.  *you can store them in an external [http://www.bibtex.org BibTeX file ]
and then link them via a command to your current document and use a
[http://www.cs.stir.ac.uk/~kjt/software/latex/showbst.html Bibtex style] to
define how they appear. This way you can create a small database of the
references you might use and simply link them, letting LaTeX work for you.

In order to know how to add the bibliography to your document, see the
\_ManagementBibliography Management section.
