\section{Procedures and functions}
In C programming, all executable code resides within a \textbf{function}. A
function is a named block of code that performs a task and then returns control
to a caller. Note that other programming languages may distinguish between a
function, subroutine, subprogram, procedure, or method --
in C, these are all functions.

A function is often executed (called) several times, from several different
places, during a single execution of the program.  After finishing a
subroutine, the program will branch back (return) to the point after the call.

Functions are a powerful programming tool.

As a basic example, suppose you are writing code to print out the first 5
squares of numbers, do some intermediate processing, then print the first 5
squares again. We could write it like this:
\lstset{basicstyle=\scriptsize, numbers=left, captionpos=b, tabsize=4}
\begin{lstlisting}[caption=,language={C},
breaklines=true,xleftmargin=15pt, label=lst:]
#include <stdio.h>

int main(void) {
	int i;
	for(i=1; i <= 5; i++) {
	   printf("%d ", i*i);
	}
	for(i=1; i <= 5; i++) {
	   printf("%d ", i*i);
	}
return 0;
\end{lstlisting}

We have to write the same loop twice. We may want to somehow put this code in a
separate place and simply jump to this code when we want to use it. This would
look like:
\lstset{basicstyle=\scriptsize, numbers=left, captionpos=b, tabsize=4}
\begin{lstlisting}[caption=,language={C},
breaklines=true,xleftmargin=15pt, label=lst:]
#include <stdio.h>

void Print_Squares(void) {
int i;
for(i=1; i <=5; i++) {
	printf("%d ", i*i);
}

int main(void) {
	Print_Squares();
	Print_Squares();
	return 0;
}
\end{lstlisting}

This is precisely what functions are for.

\subsection{More on functions}
A function is like a black box. It takes in input, does something with it, then
spits out an answer.

Note that a function may not take any inputs at all, or it may not return
anything at all. In the above example, if we were to make a function of that
loop, we may not need any inputs, and we aren't returning anything at all (Text
output doesn't count --- when we speak of \emph{returning} we mean to say
meaningful data that the program can use).

We have some terminology to refer to functions:
\begin{itemize}
\setlength{\itemsep}{0cm}
\setlength{\parskip}{0cm}
	\item A function, call it \emph{f}, that uses another function \emph{g}, is
said to \emph{call} \emph{g}. For example, \emph{f} calls \emph{g} to print the
squares of ten numbers.
	\item A function's inputs are known as its \emph{arguments}
	\item A function \emph{g} that gives some kind of answer back to \emph{f}
is said to \emph{return} that answer. For example, \emph{g} returns the sum of
its arguments.
\end{itemize}

\section{Writing functions in C}
It's always good to learn by example. Let's write a function that will return the square of a number.

\lstset{basicstyle=\scriptsize, numbers=left, captionpos=b, tabsize=4}
\begin{lstlisting}[caption=,language={C},
breaklines=true,xleftmargin=15pt, label=lst:]
int square(int x) {
	  int square_of_x;
	  square_of_x = x * x;
	  return square_of_x;
}
\end{lstlisting}

To understand how to write such a function like this, it may help to look at
what this function does as a whole. It takes in an \texttt{int}, x, and squares
it, storing it in the variable square\_of\_x. Now this value is returned. 

The first int at the beginning of the function declaration is the type of data
that the function returns. In this case when we square an integer we get an
integer, and we are returning this integer, and so we write \texttt{int} as the
return type.

Next is the name of the function. It is good practice to use meaningful and
descriptive names for functions you may write. It may help to name the function
after what it is written to do. In this case we name the function ``square'',
because that's what it does --- it squares a number.

Next is the function's first and only argument, an \texttt{int}, which will be
referred to in the function as x. This is the function's \emph{input}. 

In between the braces is the actual guts of the function. It declares an
integer variable called square\_of\_x that will be used to hold the value of
the square of x. Note that the variable square\_of\_x can \textbf{only} be used
within this function, and not outside. We'll learn more about this sort of
thing later, and we will see that this property is very useful. 

We then assign x multiplied by x, or x squared, to the variable square\_of\_x,
which is what this function is all about. Following this is a \texttt{return}
statement. We want to return the value of the square of x, so we must say that
this function returns the contents of the variable square\_of\_x.

Our brace to close, and we have finished the declaration.

Written in a more concise manner, this code performs the exact same function as
the above:

\lstset{basicstyle=\scriptsize, numbers=left, captionpos=b, tabsize=4}
\begin{lstlisting}[caption=,language={C},
breaklines=true,xleftmargin=15pt, label=lst:]
int square(int x) {
	return x * x;
}
\end{lstlisting}

Note this should look familiar --- you have been writing functions already, in
fact --- main is a function that is always written.

\subsubsection{In general}
In general, if we want to declare a function, we write
\lstset{basicstyle=\scriptsize, numbers=left, captionpos=b, tabsize=4}
\begin{lstlisting}[caption=,language={C},
breaklines=true,xleftmargin=15pt, label=lst:]
{
  /* code */
} 
\end{lstlisting}

We've previously said that a function can take no arguments, or can return
nothing, or both. What do we write if we want the function to return nothing?
We use C's \texttt{void} keyword. \texttt{void} basically means ``nothing'' ---
so if we want to write a function that returns nothing, for example, we write

\lstset{basicstyle=\scriptsize, numbers=left, captionpos=b, tabsize=4}
\begin{lstlisting}[caption=,language={C},
breaklines=true,xleftmargin=15pt, label=lst:]
void sayhello(int number\_of\_times) {
	 int i;
	 for(i=1; i <= number_of_times; i++)
	 {
	    printf("Hello!\n");
	 }
}
\end{lstlisting}

Notice that there is no \texttt{return} statement in the function above. Since
there's none, we write \texttt{void} as the return type.

What about a function that takes no arguments? If we want to do this, we can
write for example

\lstset{basicstyle=\scriptsize, numbers=left, captionpos=b, tabsize=4}
\begin{lstlisting}[caption=,language={C},
breaklines=true,xleftmargin=15pt, label=lst:]
float calculate_number(void) {
	float to_return=1;
	int i;
	for(i=0; i < 100; i++) {
		to_return += 1;
		to_return = 1/to_return;
	}
	return to_return;
}
\end{lstlisting}

Notice this function doesn't take any inputs, but merely returns a number
calculated by this function.

Naturally, you can combine both void return and void in arguments together to
get a valid function, also.

\subsubsection{Recursion}
Here's a simple function that does an infinite loop. It prints a line and calls
itself, which again prints a line and calls itself again, and this continues
until the stack overflows and the program crashes. A function calling itself is
called recursion, and normally you will have a conditional that would stop the
recursion after a small, finite number of steps.
\begin{lstlisting}[caption=,language={C},
breaklines=true,xleftmargin=15pt, label=lst:]
	     // don't run this!
void infinite_recursion() {
	printf("Infinite loop!\n");
	infinite_recursion();
}
\end{lstlisting}

A simple check can be done like this. Note that ++depth is used so the
increment will take place before the value is passed into the function.
Alternatively you can increment on a separate line before the recursion call.
If you say print\_me(3,0); the function will print the line Recursion 3 times.
\lstset{basicstyle=\scriptsize, numbers=left, captionpos=b, tabsize=4}
\begin{lstlisting}[caption=,language={C},
breaklines=true,xleftmargin=15pt, label=lst:]
void print_me(int j, int depth) {
	if(depth < j) {
	    printf("Recursion! depth = %d j = %d\n",depth,j); //j keeps its value
	    print_me(j, ++depth);
	}
}
\end{lstlisting}

Recursion is most often used for jobs such as directory tree scans, seeking for
the end of a linked list, parsing a tree structure in a database and
factorising numbers (and finding primes) among other things.

\subsubsection{Static Functions}
If a function is to be called only from within the file in which it is
declared, it is appropriate to declare it as a static function. When a function
is declared static, the compiler will now compile to an object file in a way
that prevents the function from being called from code in other files. Example:
\lstset{basicstyle=\scriptsize, numbers=left, captionpos=b, tabsize=4}
\begin{lstlisting}[caption=,language={C},
breaklines=true,xleftmargin=15pt, label=lst:]
static short compare( short a, short b ) {
	return (a+4 < b)? a : b;
}
\end{lstlisting}

\subsection{Using C functions}
We can now \emph{write} functions, but how do we use them? When we write main,
we place the function outside the braces that encompass main.

When we want to use that function, say, using our \texttt{calculate\_number}
function above, we can write something like
\lstset{basicstyle=\scriptsize, numbers=left, captionpos=b, tabsize=4}
\begin{lstlisting}[caption=,language={C},
breaklines=true,xleftmargin=15pt, label=lst:]
float f;
f = calculate_number();
\end{lstlisting}

If a function takes in arguments, we can write something like
\lstset{basicstyle=\scriptsize, numbers=left, captionpos=b, tabsize=4}
\begin{lstlisting}[caption=,language={C},
breaklines=true,xleftmargin=15pt, label=lst:]
int square_of_10;
square_of_10 = square(10);
\end{lstlisting}

If a function doesn't return anything, we can just say
\lstset{basicstyle=\scriptsize, numbers=left, captionpos=b, tabsize=4}
\begin{lstlisting}[caption=,language={C},
breaklines=true,xleftmargin=15pt, label=lst:]
say_hello();
\end{lstlisting}

since we don't need a variable to catch its return value.

\subsection{Functions from the C Standard Library}
While the C language doesn't itself contain functions, it is usually linked
with the C Standard Library. To use this library, you need to add an \#include
directive at the top of the C file, which may be one of the following:
\begin{itemize}
\setlength{\itemsep}{0cm}
\setlength{\parskip}{0cm}
\item \texttt{\textless{}assert.h\textgreater{}}
\item \texttt{\textless{}ctype.h\textgreater{}}
\item \texttt{\textless{}errno.h\textgreater{}}
\item \texttt{\textless{}float.h\textgreater{}}
\end{itemize}
\begin{itemize}
\setlength{\itemsep}{0cm}
\setlength{\parskip}{0cm}
\item \texttt{\textless{}limits.h\textgreater{}}
\item \texttt{\textless{}locale.h\textgreater{}}
\item \texttt{\textless{}math.h\textgreater{}}
\item \texttt{\textless{}setjmp.h\textgreater{}}
\end{itemize}
\begin{itemize}
\setlength{\itemsep}{0cm}
\setlength{\parskip}{0cm}
\item \texttt{\textless{}signal.h\textgreater{}}
\item \texttt{\textless{}stdarg.h\textgreater{}}
\item \texttt{\textless{}stddef.h\textgreater{}}
\item \texttt{\textless{}stdio.h\textgreater{}}
\end{itemize}
\begin{itemize}
\setlength{\itemsep}{0cm}
\setlength{\parskip}{0cm}
\item \texttt{\textless{}stdlib.h\textgreater{}}
\item \texttt{\textless{}string.h\textgreater{}}
\item \texttt{\textless{}time.h\textgreater{}}
\end{itemize}

The functions available are:
\begin{table*}[h]
\scriptsize
\begin{tabular}{|p{3cm}|p{3cm}|p{3cm}|p{6.5cm}|}
\hline
\textbf{ \texttt{\textless{}assert.h\textgreater{}}} & \textbf{ \texttt{\textless{}limits.h\textgreater{}}} & \textbf{ \texttt{\textless{}signal.h\textgreater{}}} & \textbf{ \texttt{\textless{}stdlib.h\textgreater{}}} \\ \hline
\begin{itemize}
\setlength{\itemsep}{0cm}
\setlength{\parskip}{0cm}
	\item assert(int)
\end{itemize}
 &  \begin{itemize}
\setlength{\itemsep}{0cm}
\setlength{\parskip}{0cm}
	\item (constants only)
\end{itemize}
 &  \begin{itemize}
\setlength{\itemsep}{0cm}
\setlength{\parskip}{0cm}
\setlength{\itemsep}{0cm}
\setlength{\parskip}{0cm}
	\item int raise(int sig). This
	\item void* signal(int sig, void (*func)(int))
\end{itemize}
 &  \begin{itemize}
\setlength{\itemsep}{0cm}
\setlength{\parskip}{0cm}
\setlength{\itemsep}{0cm}
\setlength{\parskip}{0cm}
	\item atof(char*), atoi(char*), atol(char*)
	\item strtod(char * str, char ** endptr ), strtol(char *str, char **endptr), strtoul(char *str, char **endptr)
	\item rand(), srand()
	\item malloc(size\_t), calloc (size\_t elements, size\_t elementSize), realloc(void*, int)
	\item free (void*)
	\item exit(int), abort()
	\item atexit(void (*func)())
	\item getenv
	\item system
	\item qsort(void *, size\_t number, size\_t size, int (*sortfunc)(void*, void*))
	\item abs, labs
	\item div, ldiv
\end{itemize}
 \\ \hline
		\textbf{ \texttt{\textless{}ctype.h\textgreater{}}} & \textbf{ \texttt{\textless{}locale.h\textgreater{}}} & \textbf{ \texttt{\textless{}stdarg.h\textgreater{}}} & \textbf{ \texttt{\textless{}string.h\textgreater{}}} \\ \hline
		 \begin{itemize}
\setlength{\itemsep}{0cm}
\setlength{\parskip}{0cm}
	\item isalnum, isalpha, isblank
	\item iscntrl, isdigit, isgraph
	\item islower, isprint, ispunct
	\item isspace, isupper, isxdigit
	\item tolower, toupper
\end{itemize}
 &  \begin{itemize}
\setlength{\itemsep}{0cm}
\setlength{\parskip}{0cm}
	\item struct lconv* localeconv(void);
	\item char* setlocale(int, const char*);
\end{itemize}
 &  \begin{itemize}
\setlength{\itemsep}{0cm}
\setlength{\parskip}{0cm}
	\item va\_start (va\_list, ap)
	\item va\_arg (ap, (type))
	\item va\_end (ap)
	\item va\_copy (va\_list, va\_list)
\end{itemize}
 &  \begin{itemize}
\setlength{\itemsep}{0cm}
\setlength{\parskip}{0cm}
	\item memcpy, memmove
	\item memchr, memcmp, memset
	\item strcat, strncat, strchr, strrchr
	\item strcmp, strncmp, strccoll
	\item strcpy, strncpy
	\item strerror
	\item strlen
	\item strspn, strcspn
	\item strpbrk
	\item strstr
	\item strtok
	\item strxfrm
\end{itemize}
 \\ \hline
		\textbf{ errno.h} & \textbf{ math.h} & \textbf{ stddef.h} & \textbf{ time.h} \\ \hline
		 \begin{itemize}
\setlength{\itemsep}{0cm}
\setlength{\parskip}{0cm}
	\item (errno)
\end{itemize}
 &  \begin{itemize}
\setlength{\itemsep}{0cm}
\setlength{\parskip}{0cm}
	\item sin, cos, tan
	\item asin, acos, atan, atan2
	\item sinh, cosh, tanh
	\item ceil
	\item exp
	\item fabs
	\item floor
	\item fmod
	\item frexp
	\item ldexp
	\item log, log10
	\item modf
	\item pow
	\item sqrt
\end{itemize}
 &  \begin{itemize}
	\item offsetof macro
\end{itemize}
 &  \begin{itemize}
\setlength{\itemsep}{0cm}
\setlength{\parskip}{0cm}
	\item asctime (struct tm* tmptr)
	\item clock\_t clock()
	\item char* ctime(const time\_t* timer)
	\item double difftime(time\_t timer2, time\_t timer1)
	\item struct tm* gmtime(const time\_t* timer)
	\item struct tm* gmtime\_r(const time\_t* timer, struct tm* result)
	\item struct tm* localtime(const time\_t* timer)
	\item time\_t mktime(struct tm* ptm)
	\item time\_t time(time\_t* timer)
	\item char * strptime(const char* buf, const char* format, struct tm* tptr)
	\item time\_t timegm(struct tm *brokentime)
\end{itemize}
 \\ \hline
		\textbf{ float.h} & \textbf{ setjmp.h} & \multicolumn{2}{|c|}{\textbf{ stdio.h}} \\ \hline
		 \begin{itemize}
\setlength{\itemsep}{0cm}
\setlength{\parskip}{0cm}
	\item (constants
\end{itemize}
 &  \begin{itemize}
\setlength{\itemsep}{0cm}
\setlength{\parskip}{0cm}
	\item int setjmp(jmp\_buf env)
	\item void longjmp(jmp\_buf env, int value)
\end{itemize}
 &  \begin{itemize}
\setlength{\itemsep}{0cm}
\setlength{\parskip}{0cm}
	\item fclose
	\item fopen, freopen
	\item remove
	\item rename
	\item rewind
	\item tmpfile
	\item clearerr
	\item feof, ferror
	\item fflush
	\item fgetpos, fsetpos
	\item fgetc, fputc
	\item fgets, fputs
	\item ftell, fseek
\end{itemize}
 &  \begin{itemize}
\setlength{\itemsep}{0cm}
\setlength{\parskip}{0cm}
	\item fread, fwrite
	\item getc, putc
	\item getchar, putchar, fputchar
	\item gets, puts
	\item printf, vprintf
	\item fprintf, vfprintf
	\item sprintf, snprintf, vsprintf, vsnprintf
	\item perror
	\item scanf, vscanf
	\item fscanf, vfscanf
	\item sscanf, vsscanf
	\item setbuf, setvbuf
	\item tmpnam
	\item ungetc
\end{itemize}
 \\ \hline
	\end{tabular}
\end{table*}

\subsection{Variable-length argument lists}
Functions with variable-length argument lists are functions that can take a
varying number of arguments. An example in the C standard library is the
\texttt{printf} function, which can take any number of arguments depending on
how the programmer wants to use it.

C programmers rarely find the need to write new functions with variable-length
arguments.  If they want to pass a bunch of things to a function, they
typically define a structure to hold all those things -- perhaps a linked list,
or an array -- and call that function with the data in the arguments.

However, you may occasionally find the need to write a new function that
supports a variable-length argument list.  To create a function that can accept
a variable-length argument list, you must first include the standard library
header \texttt{stdarg.h}. Next, declare the function as you would normally.
Next, add as the last argument an elipsis (``...''). This indicates to the
compiler that a variable list of arguments is to follow. For example, the
following function declaration is for a function that returns the average of a
list of numbers:
\lstset{basicstyle=\scriptsize, numbers=left, captionpos=b, tabsize=4}
\begin{lstlisting}[caption=,language={C},
breaklines=true,xleftmargin=15pt, label=lst:]
	 float average (int n_args, ...);
\end{lstlisting}

Note that because of the way variable-length arguments work, we must somehow,
in the arguments, specify the number of elements in the variable-length part of
the arguments. In the \texttt{average} function here, it's done through an
argument called \texttt{n\_args.} In the \texttt{printf} function, it's done
with the format codes that you specify in that first string in the arguments
you provide.

Now that the function has been declared as using variable-length arguments, we
must next write the code that does the actual work in the function.  To access
the numbers stored in the variable-length argument list for our
\texttt{average} function, we must first declare a variable for the list
itself:
\lstset{basicstyle=\scriptsize, numbers=left, captionpos=b, tabsize=4}
\begin{lstlisting}[caption=,language={C},
breaklines=true,xleftmargin=15pt, label=lst:]
	 va_list myList;
\end{lstlisting}

The \texttt{va\_list} type is a type declared in the \texttt{stdarg.h} header
that basically allows you to keep track of your list. To start actually using
\texttt{myList}, however, we must first assign it a value. After all, simply
declaring it by itself wouldn't do anything. To do this, we must call
\texttt{va\_start}, which is actually a macro defined in \texttt{stdarg.h.} In
the arguments to \texttt{va\_start}, you must provide the \texttt{va\_list}
variable you plan on using, as well as the name of the last variable appearing
before the ellipsis in your function declaration:
\lstset{basicstyle=\scriptsize, numbers=left, captionpos=b, tabsize=4}
\begin{lstlisting}[caption=,language={C},
breaklines=true,xleftmargin=15pt, label=lst:]
float average (int n\_args, ...) {
	va_list myList;
	va_start (myList, n_args);
}
\end{lstlisting}

Now that \texttt{myList} has been prepped for usage, we can finally start
accessing the variables stored in it. To do so, use the \texttt{va\_arg} macro,
which pops off the next argument on the list. In the arguments to
\texttt{va\_arg}, provide the \texttt{va\_list} variable you're using, as well
as the primitive data type (e.g. \texttt{int}, \texttt{char}) that the variable
you're accessing should be:
\lstset{basicstyle=\scriptsize, numbers=left, captionpos=b, tabsize=4}
\begin{lstlisting}[caption=,language={C},
breaklines=true,xleftmargin=15pt, label=lst:]
float average (int n\_args, ...) {
	va_list myList;
	va_start (myList, n_args);
	
	int myNumber = va_arg (myList, int);
}
\end{lstlisting}

By popping \texttt{n\_args} integers off of the variable-length argument list,
we can manage to find the average of the numbers:
\lstset{basicstyle=\scriptsize, numbers=left, captionpos=b, tabsize=4}
\begin{lstlisting}[caption=,language={C},
breaklines=true,xleftmargin=15pt, label=lst:]
float average (int n\_args, ...) {
	va_list myList;
	va_start (myList, n_args);
	
	int numbersAdded = 0;
	int sum = 0;
	 
	while (numbersAdded < n_args) {
		int number = va_arg (myList, int); // Get next number from list
		sum += number;
		numbersAdded += 1;
	}
	 
	float avg = (float)(sum) / (float)(numbersAdded); // Find the average
	return avg;
}
\end{lstlisting}

By calling \texttt{average (2, 10, 20)}, we get the average of \texttt{10} and
\texttt{20}, which is \texttt{15.}
