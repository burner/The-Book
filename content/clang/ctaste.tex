\section{A taste of C}
As with nearly every other programming language learning book, we use the
\emph{Hello world} program to introduce you to C.

\lstset{basicstyle=\scriptsize, numbers=left, captionpos=b, tabsize=4}
\begin{lstlisting}[caption=,language={C},
breaklines=true,xleftmargin=15pt,label=lst:]
#include <stdio.h>

int main(void) {
   printf("Hello, world!\n");
   return 0;
}
\end{lstlisting}

This program prints Hello, world! and then exits.
Enter this code into your text editor or IDE, and save it as hello.c.

Then, presuming you are using GCC, type \texttt{gcc -o hello hello.c}. This
tells gcc to compile your hello.c program into a form the machine can execute.
The '-o hello' tells it to call the compiled program 'hello'.

If you have entered this correctly, you should now see a file called hello.
This file is the binary version of your program, and when run should display
Hello, world!

Here is an example of how compiling and running looks when using a terminal on
a unix system. \texttt{ls} is a common unix command that will list the files in
the current directory, which in this case is the directory \texttt{progs}
inside the home directory (represented with the special tilde, \~{}, symbol).
After running the \texttt{gcc} command, \texttt{ls} will list a new file,
\texttt{hello} in green. Green is the standard color coding of \texttt{ls} for
executable files. 

\begin{verbatim}
~/progs$ ls
hello.c
~/progs$ gcc -o hello hello.c
~/progs$ ls
hello  hello.c
~/progs$ ./hello
Hello, world!
~/progs$
\end{verbatim}

\subsection{Part-by-part explanation}
\texttt{\#include \textless{}stdio.h\textgreater{}} tells the C compiler to
find the standard header called \emph{\textless{stdio.h\textgreater{}}} and add
it to this program. In C, you often have to pull in extra optional components
when you need them. \emph{\textless{}stdio.h\textgreater{}} contains
descriptions of standard input/output functions which you can use to send
messages to a user, or to read input from a user.

\texttt{int main(void)} is something you'll find in every C program. Every
program has a \emph{main} function. Generally, the main function is where a
program begins. However, one C program can be scattered across multiple files,
so you won't always find a main function in every file.
The \emph{int} at the beginning means that main will return an integer to the
operating system when it is finished.

\texttt{printf("Hello, world!\textbackslash{}n");} is the statement that
actually puts the message to the screen. \emph{printf} is the formatted
printing function that is declared in the file \emph{stdio.h} --- which is why
you had to \emph{\#include} that at the start of the program.
\emph{\textbackslash{}n} is an escape sequence which adds a new line at the end
of the printed text.

\texttt{return 0;} will return zero (which is the integer referred to on line
3) to the operating system. When a program runs successfully its return value
is zero (GCC4 complains if it doesn't when compiling). A non-zero value is
returned to indicate a warning or error.

The empty line is there because it is (at least on UNIX) considered good
practice to end a file with a new line. In gcc using the \texttt{-Wall
-pedantic -ansi} options, if the file does not end with a new line this message
is displayed: ``warning: no newline at end of file''. (The newline isn't shown
on the example because MediaWiki automatically removes it)
