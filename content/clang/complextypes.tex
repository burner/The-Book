\section{Complex types}\label{Complex_types}
\newcounter{comtypcnt}
\subsection{Data structures}
A data structure (``struct'') contains multiple pieces of data.  Each piece of
data (called a ``member'') can be accessed by the name of the variable,
followed by a '.', then the name of the member.  (Another way to access a
member is using the member operator '-\textgreater{}').  The member variables
of a struct can be of any data type and can even be an array or a pointer.

\subsection{Pointers}
Pointers are variables that don't hold the actual data. Instead they point to
the memory location of some other variable.  For example,
\lstset{basicstyle=\scriptsize, numbers=left, captionpos=b, tabsize=4}
\begin{lstlisting}[caption=Section \thesection listing \arabic{comtypcnt},language={C},
breaklines=true,xleftmargin=15pt,label=lst:section\thesection listing\arabic{comtypcnt}]
int *pointer = &variable;
\end{lstlisting}
\stepcounter{comtypcnt}

defines a pointer to an \texttt{int}, and also makes it point to the particular
integer contained in \texttt{variable}.

The '*' is what makes this an integer pointer.  To make the pointer point to a
different integer, use the form
\lstset{basicstyle=\scriptsize, numbers=left, captionpos=b, tabsize=4}
\begin{lstlisting}[caption=Section \thesection listing \arabic{comtypcnt},language={C},
breaklines=true,xleftmargin=15pt,label=lst:section\thesection listing\arabic{comtypcnt}]
pointer = &sandwiches;
\end{lstlisting}
\stepcounter{comtypcnt}

Where \& is the \emph{address of} operator.  Often programmers set the value of
the pointer to NULL (a standard macro defined as 0 or (void*)0 ) like this:
\lstset{basicstyle=\scriptsize, numbers=left, captionpos=b, tabsize=4}
\begin{lstlisting}[caption=Section \thesection listing \arabic{comtypcnt},language={C},
breaklines=true,xleftmargin=15pt,label=lst:section\thesection listing\arabic{comtypcnt}]
pointer = NULL;
\end{lstlisting}
\stepcounter{comtypcnt}

This tells us that the pointer isn't currently pointing to any real location.

Additionally, to dereference (access the thing being pointed at) the pointer,
use the form:
\lstset{basicstyle=\scriptsize, numbers=left, captionpos=b, tabsize=4}
\begin{lstlisting}[caption=Section \thesection listing \arabic{comtypcnt},language={C},
breaklines=true,xleftmargin=15pt,label=lst:section\thesection listing\arabic{comtypcnt}]
value = *pointer;
\end{lstlisting}
\stepcounter{comtypcnt}

\subsection{Structs}
A data structure contains multiple pieces of data.  One defines a data
structure using the \texttt{struct} keyword. For example,
\lstset{basicstyle=\scriptsize, numbers=left, captionpos=b, tabsize=4}
\begin{lstlisting}[caption=Section \thesection listing \arabic{comtypcnt},language={C},
breaklines=true,xleftmargin=15pt,label=lst:section\thesection listing\arabic{comtypcnt}]
struct mystruct {
    int int_member;
    double double_member;
    char string_member[25];
} variable;
\end{lstlisting}
\stepcounter{comtypcnt}

\texttt{variable} is an instance of \texttt{mystruct}. You can omit it from the
end of the \texttt{struct} declaration and declare it later using:
\lstset{basicstyle=\scriptsize, numbers=left, captionpos=b, tabsize=4}
\begin{lstlisting}[caption=Section \thesection listing \arabic{comtypcnt},language={C},
breaklines=true,xleftmargin=15pt,label=lst:section\thesection listing\arabic{comtypcnt}]
struct mystruct variable;
\end{lstlisting}
\stepcounter{comtypcnt}

It is often common practice to make a \emph{type synonym} so we don't have to
type ``struct mystruct'' all the time. C allows us the possibility to do so
using a \texttt{typedef} statement, which aliases a type:
\lstset{basicstyle=\scriptsize, numbers=left, captionpos=b, tabsize=4}
\begin{lstlisting}[caption=Section \thesection listing \arabic{comtypcnt},language={C},
breaklines=true,xleftmargin=15pt,label=lst:section\thesection listing\arabic{comtypcnt}]
typedef struct {
  ...
} Mystruct;
\end{lstlisting}
\stepcounter{comtypcnt}

The \texttt{struct} itself has no name (by the absence of a name on the first
line), but it is aliased as \texttt{Mystruct}. Then you can use
\lstset{basicstyle=\scriptsize, numbers=left, captionpos=b, tabsize=4}
\begin{lstlisting}[caption=Section \thesection listing \arabic{comtypcnt},language={C},
breaklines=true,xleftmargin=15pt,label=lst:section\thesection listing\arabic{comtypcnt}]
Mystruct structure;
\end{lstlisting}
\stepcounter{comtypcnt}

Note that it is commonplace, and good style to capitalize the \textbf{first
letter} of a type synonym. However in the actual definition we need to give the
struct a \emph{tag} so we can refer to it: we may have a \emph{recursive data
structure} of some kind.

\subsection{Unions}
The definition of a union is similar to that of a struct. The difference
between the two is that in a struct, the members occupy different areas of
memory, but in a union, the members occupy the same area of memory. Thus, in
the following type, for example:
\lstset{basicstyle=\scriptsize, numbers=left, captionpos=b, tabsize=4}
\begin{lstlisting}[caption=Section \thesection listing \arabic{comtypcnt},language={C},
breaklines=true,xleftmargin=15pt,label=lst:section\thesection listing\arabic{comtypcnt}]
union {
    int i;
    double d;
} u;
\end{lstlisting}
\stepcounter{comtypcnt}

The programmer can access either \texttt{u.i} or \texttt{u.d}, but not both at
the same time. Since \texttt{u.i} and \texttt{u.d} occupy the same area of
memory, modifying one modifies the value of the other, sometimes in
unpredictable ways.

The size of a union is the size of its largest member.

\subsection{Type modifiers}
\textbf{\texttt{register}} is a hint to the compiler to attempt to optimise the
storage of the given variable by storing it in a register of the computer's CPU
when the program is run. Most optimising compilers do this anyway, so use of
this keyword is often unnecessary. In fact, ANSI C states that a compiler can
ignore this keyword if it so desires -- and many do. Microsoft Visual C++ is an
example of an implementation that completely ignores the \emph{register}
keyword.

\textbf{\texttt{volatile}} is a special type modifier which informs the
compiler that the value of the variable may be changed by external entities
other than the program itself. This is necessary for certain programs compiled
with optimisations --- if a variable were not defined \texttt{volatile} then
the compiler may assume that certain operations involving the variable were
safe to optimise away when in fact they aren't. \emph{volatile} is particularly
relevant when working with embedded systems (where a program may not have
complete control of a variable) and multi-threaded applications.

\textbf{\texttt{auto}} is a modifier which specifies an ``automatic'' variable
that is automatically created when you get inside a scope and destroyed when
you get out of a scope. If you think this sounds like pretty much what you've
been doing all along when you declare a variable, you're right: all declared
items within a block are implicitly ``automatic''. For this reason, the
\emph{auto} keyword is more like the answer to a trivia question than a useful
modifier, and there are lots of very competent programmers that are unaware of
its existence.

\textbf{\texttt{extern}} is used when a file needs to access a variable in
another file that it may not have \#included directly. Therefore, \emph{extern}
does not actually carve out space for a new variable, it just provides the
compiler with sufficient information to access the remote variable.

Add \texttt{restrict} and \texttt{static}, and try changing names of variable
to something other than \texttt{variable}.
