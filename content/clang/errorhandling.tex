\section{Error handling}
\newcounter{errorcnt}
C does not provide direct support for error handling (also known as exception
handling). By convention, the programmer is expected to prevent errors from
occurring in the first place, and test return values from functions. For
example, -1 and NULL are used in several functions such as socket() (Unix
socket programming) or malloc() respectively to indicate problems that the
programmer should be aware about. In a worst case scenario where there is an
unavoidable error and no way to recover from it, a C programmer usually tries
to log the error and ``gracefully'' terminate the program. 

There is an external variable called ``errno'', accessible by the programs
after including \textless{}errno.h\textgreater{} --- that file comes from the
definition of the possible errors that can occur in some Operating Systems
(e.g. Linux --- in this case, the definition is in include/asm-generic/errno.h)
when programs ask for resources. Such variable indexes error descriptions, that
is accessible by the function 'strerror( errno )'. 

The following code tests the return value from the library function malloc to
see if dynamic memory allocation completed properly:
\lstset{basicstyle=\scriptsize, numbers=left, captionpos=b, tabsize=4}
\begin{lstlisting}[caption=Section \thesection listing \arabic{errorcnt},language={C},
breaklines=true,xleftmargin=15pt, label=lst:section\thesection listing\arabic{errorcnt}]
#include <stdio.h>        /* fprintf */
#include <errno.h>        /* errno */
#include <stdlib.h>       /* malloc, free, exit */
#include <string.h>       /* strerror */

extern int errno;

int main( void ) {
	/* pointer to char, requesting dynamic allocation of 2,000,000,000
	 * storage elements (declared as an integer constant of type
	 * unsigned long int). (If your system has less than 2GB of memory
	 * available, then this call to malloc will fail) */
	char *ptr = malloc( 2000000000UL );
	
	if( ptr == NULL ) {
	puts("malloc failed");
	puts(strerror(errno));
	} else {
		/* the rest of the code hereafter can assume that 2,000,000,000
		 * chars were successfully allocated... 
		 */
		free( ptr );
	}
	exit(EXIT_SUCCESS); /* exiting program */
}
\end{lstlisting}
\stepcounter{errorcnt}

The code snippet above shows the use of the return value of the library
function malloc to check for errors. Many library functions have return values
that flag errors, and thus should be checked by the astute programmer. In the
snippet above, a NULL pointer returned from malloc signals an error in
allocation, so the program exits. In more complicated implementations, the
program might try to handle the error and try to recover from the failed memory
allocation.

\subsection{Preventing divide by zero errors}
A common pitfall made by C programmers is not checking if a divisor is zero
before a division command. The following code will produce a runtime error and
in most cases, exit.
\lstset{basicstyle=\scriptsize, numbers=left, captionpos=b, tabsize=4}
\begin{lstlisting}[caption=Section \thesection listing \arabic{errorcnt},language={C},
breaklines=true,xleftmargin=15pt, label=lst:section\thesection listing\arabic{errorcnt}]
int dividend = 50;
int divisor = 0;
int quotient;

quotient = (dividend/divisor); /* This will produce a runtime error! */
\end{lstlisting}
\stepcounter{errorcnt}

For reasons beyond the scope of this document, you must check or make sure that
a divisor is never zero. Alternatively, for *nix processes, you can stop the OS
from terminating your process by blocking the SIGFPE signal. 

The code below fixes this by checking if the divisor is zero before dividing.
\lstset{basicstyle=\scriptsize, numbers=left, captionpos=b, tabsize=4}
\begin{lstlisting}[caption=Section \thesection listing \arabic{errorcnt},language={C},
breaklines=true,xleftmargin=15pt, label=lst:section\thesection listing\arabic{errorcnt}]
#include <stdio.h> /* for fprintf and stderr */
#include <stdlib.h> /* for exit */
int main( void ) {
	int dividend = 50;
	int divisor = 0;
	int quotient;
	
	if(divisor == 0) {
		/* Example handling of this error. Writing a message to stderr, and
		 * exiting with failure. */
		fprintf(stderr, "Division by zero! Aborting...\n");
		exit(EXIT_FAILURE); /* indicate failure.*/
	}
	quotient = dividend / divisor;
	exit(EXIT_SUCCESS); /* indicate success.*/
}
\end{lstlisting}
\stepcounter{errorcnt}

\subsection{Signals}
In some cases, the environment may respond to a programming error in C by
raising a signal. Signals are events raised by the host environment or
operating system to indicate that a specific error or critical event has
occurred (e.g. a division by zero, interrupt, and so on.) However, these
signals are not meant to be used as a means of error catching; they usually
indicate a critical event that will interfere with normal program flow. 

To handle signals, a program needs to use the \texttt{signal.h} header file. A
signal handler will need to be defined, and the signal() function is then
called to allow the given signal to be handled. Some signals that are raised to
an exception within your code (e.g. a division by zero) are unlikely to allow
your program to recover. These signal handlers will be required to instead
ensure that some resources are properly cleaned up before the program
terminates.

\subsection{setjmp}
The setjmp function can be used to emulate the exception handling feature of
other programming languages. The first call to setjmp provides a reference
point to returning to a given function, and is valid as long as the function
containing setjmp() doesn't return or exit. A call to longjmp causes the
execution to return to the point of the associated setjmp call.
\lstset{basicstyle=\scriptsize, numbers=left, captionpos=b, tabsize=4}
\begin{lstlisting}[caption=Section \thesection listing \arabic{errorcnt},language={C},
breaklines=true,xleftmargin=15pt, label=lst:section\thesection listing\arabic{errorcnt}]
#include <stdio.h>
#include <setjmp.h>

jmp_buf test1;

void tryjump() {
	longjmp(test1, 3);
}

int main(void) {
	if(setjmp(test1)==0) {
		printf ("setjmp() returned 0.");
		tryjump();
	} else {
		printf ("setjmp returned from a longjmp call."):
	}
}
\end{lstlisting}
\stepcounter{errorcnt}

The values of non-volatile variables may be corrupted when setjmp returns from
a longjmp call. 

While setjmp() and longjmp() may be used for error handling, it is generally
preferred to use the return value of a function to indicate an error, if
possible. 
