\section{History}
The field of computing as we know it today started in 1947 with three
scientists at Bell Telephone Laboratories--William Shockley, Walter Brattain,
and John Bardeen--and their groundbreaking invention of the transistor. In
1956, the first fully transistor-based computer, the TX-0, was completed at
MIT. The first integrated circuit was created in 1958 by Jack Kilby at Texas
Instruments but, the first high-level programming language existed even before
then.

In 1954, The Fortran project, named for it being the Formula Translator, began.
Fortran begot Algol 58, the Algorithmic Language, in 1958. Algol 58 begot Algol
60 in 1960. Algol 60 begot CPL, the Combined Programming Language, in 1963. CPL
begot BCPL, Basic CPL, in 1967. BCPL begot B in 1969. B begot C in 1971.

B was the first language in the C lineage directly. It was created by Ken
Thompson at Bell Labs and was an interpreted language used in early internal
versions of the UNIX operating system.  Thompson and Dennis Ritchie, also of
Bell Labs, improved B and called it NB. Further extensions to NB created its
logical successor, C, a compiled language. Most of UNIX was then rewritten in
NB and then C, which led to a more portable operating system.

The portability of UNIX was the main reason for the initial popularity of both
UNIX and C. Rather than creating a new operating system for each new machine,
system programmers could simply write the few system-dependent parts required
for the machine, and write a C compiler for the new system. Since most of the
system utilities were therefore written in C, it simply made sense to also
write new utilities in that language.

The American National Standards Institute started work on standardizing the C
language in 1983, and completed the standard in 1989. The standard, ANSI
X3.159-1989 ``Programming Language C'', served as a basis for all
implementations of C compilers. The standards were later updated in 1990 and
1999, allowing for features that were either in common use, or were appearing
in C++. 
