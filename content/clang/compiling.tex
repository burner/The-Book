\section{Compiling}
Having covered the basic concepts of C programming, we can now briefly discuss
the process of \emph{compilation}.

Like any programming language, C by itself is completely incomprehensible to a
microprocessor. Its purpose is to provide an intuitive way for humans to
provide instructions that can be easily converted into machine code that
\emph{is} comprehensible to a microprocessor. The \textbf{\emph{compiler}} is
what takes this code, and translates it into the machine code. 

To those new to programming, this seems fairly simple. A naive compiler might
read in every source file, translate everything into machine code, and write
out an executable. This could work, but has two serious problems. First, for a
large project, the computer may not have enough memory to read all of the
source code at once. Second, if you make a change to a single source file, you
would rather not have to recompile the \emph{entire} application. 

To deal with these problems, compilers break their job down into steps; for
each source file (each \texttt{.c} file), the compiler reads the file, reads
the files it references with \texttt{\#include}, and translates it to machine
code. The result of this is an ``object file'' (\texttt{.o}). Once every object
file is made, a ``linker'' collects all of the object files and writes the
actual program. This way, if you change one source file, only that file needs
to be recompiled and then the application needs to be re-linked. 

Without going into the painful details, it can be beneficial to have a
superficial understanding of the compilation process. In brief, here it is: 

\subsection{Preprocessor}
Many times you will need to give special instructions to your compiler. This is
done by inserting \textbf{preprocessor directives} into your code. When you
begin compiling your code, a special program called the preprocessor scans the
source code and preforms simple substitution of tokenized strings for others
according to predefined rules. The preprocessor is not a part of the C
language.

In C language, all preprocessor directives begin with the pound character (\#).
You can see one preprocessor directive in the Hello world program introduced in
A taste of C:

\begin{verbatim}
	#include <stdio.h>
\end{verbatim}

This directive causes the header to be included into your program. Other
directives such as \texttt{\#pragma} control compiler settings and macros. The
result of the preprocessing stage is a text string. You can think of the
preprocessor as a non-interactive text editor that prepares your code for the
compilation step.

\subsection{Syntax Checking}
This step ensures that the code is valid and will sequence into an executable
program. Under some compilers, you may get messages or warnings indicating
potential issues with your program (such as a statement always being true or
false, etc.) 

When an error is detected in the program, the compiler will normally report the
filename and line that is giving trouble. 

\subsection{Object Code}
The compiler produces a machine code equivalent of the source code that can
then be linked into the final program. The code itself can't be executed yet,
as it has to complete the linking stage. 

It's important to note after discussing the basics that compilation is a ``one
way street''. That is, compiling a C source file into machine code is easy, but
``decompiling'' (turning machine code into the C source that creates it) is
not. Decompilers for C do exist, but they rarely create useful code.

\subsection{Linking}
Linking combines the separate object codes into one complete program by
integrating libraries and the code and producing either an executable program
or a library. Linking is performed by a linker, which is often part of a
compiler.

Common errors during this stage are either missing functions, or duplicate
functions.

\subsection{Automation}
For large C projects, many programmers choose to automate compilation, both in
order to reduce user interaction requirements and to speed up the process by
only recompiling modified files.

Most integrated development environments have some kind of project management,
which makes such automation very easy. On UNIX-like systems, make and Makefiles
are often used to accomplish the same.
