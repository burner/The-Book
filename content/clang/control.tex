\section{Control}
\newcounter{controlcnt}
Very few programs follow exactly one control path and have each instruction
stated explicitly. In order to program effectively, it is necessary to
understand how one can alter the steps taken by a program due to user input or
other conditions, how some steps can be executed many times with few lines of
code, and how programs can appear to demonstrate a rudimentary grasp of logic.
C constructs known as conditionals and loops grant this power.

From this point forward, it is necessary to understand what is usually meant by
the word \emph{block}. A block is a group of code statements that are
associated and intended to be executed as a unit. In C, the beginning of a
block of code is denoted with \{ (left curly), and the end of a block is
denoted with \}. It is not necessary to place a semicolon after the end of a
block. Blocks can be empty, as in \{\}. Blocks can also be nested; i.e.\ there
can be blocks of code within larger blocks.

\subsection{Conditionals}
There is likely no meaningful program written in which a computer does not
demonstrate basic decision-making skills. It can actually be argued that there
is no meaningful human activity in which some sort of decision-making,
instinctual or otherwise, does not take place. For example, when driving a car
and approaching a traffic light, one does not think, ``I will continue driving
through the intersection.'' Rather, one thinks, ``I will stop if the light is
red, go if the light is green, and if yellow go only if I am traveling at a
certain speed a certain distance from the intersection.'' These kinds of
processes can be simulated in C using conditionals.

A conditional is a statement that instructs the computer to execute a certain
block of code or alter certain data only if a specific condition has been met.
The most common conditional is the If-Else statement, with conditional
expressions and Switch-Case statements typically used as more shorthanded
methods.

Before one can understand conditional statements, it is first necessary to
understand how C expresses logical relations. C treats logic as being
arithmetic. The value 0 (zero) represents false, and \textbf{\emph{all other
values}} represent true. If you chose some particular value to represent true
and then compare values against it, sooner or later your code will fail when
your assumed value (often 1) turns out to be incorrect. Code written by people
uncomfortable with the C language can often be identified by the usage of
\#define to make a ``TRUE'' value.

Because logic is arithmetic in C, arithmetic operators and logical operators
are one and the same. Nevertheless, there are a number of operators that are
typically associated with logic:

\subsubsection{Relational and Equivalence Expressions:}
\begin{description}
	\item[a \textless{} b] 1 if \textbf{a} is less than \textbf{b}, 0 otherwise.
	\item[a \textgreater{} b] 1 if \textbf{a} is greater than \textbf{b}, 0 otherwise.
	\item[a \textless{}= b] 1 if \textbf{a} is less than or equal to \textbf{b}, 0 otherwise.
	\item[a \textgreater{}= b] 1 if \textbf{a} is greater than or equal to \textbf{b}, 0 otherwise.
	\item[a == b] 1 if \textbf{a} is equal to \textbf{b}, 0 otherwise.
	\item[a != b] 1 if \textbf{a} is not equal to \textbf{b}, 0 otherwise
\end{description}

New programmers should take special note of the fact that the ``equal to''
operator is ==, not =. This is the cause of numerous coding mistakes and is
often a difficult-to-find bug, as the expression \texttt{(a = b)} sets
\texttt{a} equal to \texttt{b} and subsequently evaluates to \texttt{b}; but
the expression \texttt{(a == b)}, which is usually intended, checks if
\texttt{a} is equal to \texttt{b}. It needs to be pointed out that, if you
confuse = with ==, your mistake will often not be brought to your attention by
the compiler. A statement such as \texttt{if ( c = 20) \{\}} is considered
perfectly valid by the language, but will always assign 20 to \texttt{c} and
evaluate as true. A simple technique to avoid this kind of bug (in many, not
all cases) is to put the constant first. This will cause the compiler to issue
an error, if == got mispelled with =.

Note that C does not have a dedicated boolean type as many other languages do.
0 means false and anything else true. So the following are equivalent:
\lstset{basicstyle=\scriptsize, numbers=left, captionpos=b, tabsize=4}
\begin{lstlisting}[caption=Section \thesection listing \arabic{controlcnt},language={C},
breaklines=true,xleftmargin=15pt, label=lst:section\thesection listing\arabic{controlcnt}]
if(foo()) {
	//do something
}
\end{lstlisting}
\stepcounter{controlcnt}

and
\lstset{basicstyle=\scriptsize, numbers=left, captionpos=b, tabsize=4}
\begin{lstlisting}[caption=Section \thesection listing \arabic{controlcnt},language={C},
breaklines=true,xleftmargin=15pt, label=lst:section\thesection listing\arabic{controlcnt}]
if(foo() != 0) {
	//do something
}
\end{lstlisting}
\stepcounter{controlcnt}

Often \texttt{\#define TRUE 1} and \texttt{\#define FALSE 0} are used to work
around the lack of a boolean type. This is bad practice, since it makes
assumptions that do not hold. It is a better idea to indicate what you are
actually expecting as a result from a function call, as there are many
different ways of indicating error conditions, depending on the situation.
\lstset{basicstyle=\scriptsize, numbers=left, captionpos=b, tabsize=4}
\begin{lstlisting}[caption=Section \thesection listing \arabic{controlcnt},language={C},
breaklines=true,xleftmargin=15pt, label=lst:section\thesection listing\arabic{controlcnt}]
if(strstr("foo", bar) >= 0) {
	//bar contains "foo"
}
\end{lstlisting}
\stepcounter{controlcnt}

Here, \texttt{strstr} returns the index where the substring foo is found and -1
if it was not found. Note that this would fail with the \texttt{TRUE}
definition mentioned in the previous paragraph. It would also not produce the
expected results if we omitted the \texttt{\textgreater{}= 0}.

One other thing to note is that the relational expressions do not evaluate as
they would in mathematical texts. That is, an expression \texttt{myMin
\textless{} value \textless{} myMax} does not evaluate as you probably think it
might. Mathematically, this would test whether or not \emph{value} is between
\emph{myMin} and \emph{myMax}. But in C, what happens is that \emph{value} is
first compared with \emph{myMin}. This produces either a 0 or a 1. It is this
value that is compared against myMax. Example:
\lstset{basicstyle=\scriptsize, numbers=left, captionpos=b, tabsize=4}
\begin{lstlisting}[caption=Section \thesection listing \arabic{controlcnt},language={C},
breaklines=true,xleftmargin=15pt, label=lst:section\thesection listing\arabic{controlcnt}]
int value = 20;
/* ... */
if( 0 < value < 10) { // don't do this! it always evaluates to "true"!
	/* do some stuff */
}
\end{lstlisting}
\stepcounter{controlcnt}

Because \emph{value} is greater than 0, the first comparison produces a value
of 1. Now 1 is compared to be less than 10, which is true, so the statements in
the if are executed. This probably is not what the programmer expected. The
appropriate code would be
\lstset{basicstyle=\scriptsize, numbers=left, captionpos=b, tabsize=4}
\begin{lstlisting}[caption=Section \thesection listing \arabic{controlcnt},language={C},
breaklines=true,xleftmargin=15pt, label=lst:section\thesection listing\arabic{controlcnt}]
int value = 20;
/* ... */
if ( 0 < value && value < 20) {   // the && means "and"
	/* do some stuff */
}
\end{lstlisting}
\stepcounter{controlcnt}

\subsubsection{Logical Expressions}
\begin{description}
	\item[a \textbar{}\textbar{} b] when EITHER \textbf{a} or \textbf{b} is true (or both), the result is 1, otherwise the result is 0.
	\item[a \&\& b] when BOTH \textbf{a} and \textbf{b} are true, the result is 1, otherwise the result is 0.
	\item[!a] when \textbf{a} is true, the result is 0, when \textbf{a} is 0, the result is 1.
\end{description}

Here's an example of a larger logical expression. In the statement:
\lstset{basicstyle=\scriptsize, numbers=left, captionpos=b, tabsize=4}
\begin{lstlisting}[caption=Section \thesection listing \arabic{controlcnt},language={C},
breaklines=true,xleftmargin=15pt, label=lst:section\thesection listing\arabic{controlcnt}]
e = ((a && b) || (c > d));
\end{lstlisting}
\stepcounter{controlcnt}
e is set equal to 1 if a and b are non-zero, or if c is greater than d. In all
other cases, e is set to 0.

C uses short circuit evaluation of logical expressions. That is to say, once it
is able to determine the truth of a logical expression, it does no further
evaluation. This is often useful as in the following:
\lstset{basicstyle=\scriptsize, numbers=left, captionpos=b, tabsize=4}
\begin{lstlisting}[caption=Section \thesection listing \arabic{controlcnt},language={C},
breaklines=true,xleftmargin=15pt, label=lst:section\thesection listing\arabic{controlcnt}]
int myArray[12];
....
if ( i < 12 && myArray[i] > 3) { 
....
\end{lstlisting}
\stepcounter{controlcnt}

In the snippet of code, the comparison of i with 12 is done first. If it
evaluates to 0 (false), \textbf{i} would be out of bounds as an index to
\textbf{myArray}. In this case, the program never attempts to access
\textbf{myArray\url{i}} since the truth of the expression is known to be false.
Hence we need not worry here about trying to access an out-of-bounds array
element if it is already known that i is greater than or equal to zero.
A similar thing happens with expressions involving the or \textbar{}\textbar{}
operator.
\lstset{basicstyle=\scriptsize, numbers=left, captionpos=b, tabsize=4}
\begin{lstlisting}[caption=Section \thesection listing \arabic{controlcnt},language={C},
breaklines=true,xleftmargin=15pt, label=lst:section\thesection listing\arabic{controlcnt}]
while( doThis() || doThat()) ...
\end{lstlisting}
\stepcounter{controlcnt}

doThat() is never called if doThis() returns a non-zero (true) value.

\subsubsection{Bitwise Boolean Expressions}
The bitwise operators work bit by bit on the operands. The operands must be of
integral type (one of the types used for integers). The six bitwise operators
are \& (AND), \textbar{} (OR), \^{} (exclusive OR, commonly called XOR), \~{}
(NOT, which changes 1 to 0 and 0 to 1), \textless{}\textless{} (shift left),
and \textgreater{}\textgreater{} (shift right). The negation operator is a
unary operator which preceeds the operand. The others are binary operators
which lie between the two operands. The precedence of these operators is lower
than that of the relational and equivalence operators; it is often required to
parenthesize expressions involving bitwise operators.

For this section, recall that a number starting with \textbf{0x} is
hexadecimal, or hex for short. Unlike the normal decimal system using powers of
10 and digits 0123456789, hex uses powers of 16 and digits 0123456789abcdef.
Hexadecimal is commonly used in C programs because a programmer can quickly
convert it to or from binary (powers of 2 and digits 01). C does not directly
support binary notation, which would be really verbose anyway.

\begin{description}
	\item[a \& b] bitwise boolean and of \textbf{a} and \textbf{b}
\end{description}

\begin{description}
	\item[a \textbar{} b] bitwise boolean or of \textbf{a} and \textbf{b}
\end{description}

\begin{description}
	\item[a \^{} b] bitwise xor of \textbf{a} and \textbf{b}
\end{description}

\begin{description}
	\item[\~{}a] bitwise complement of \textbf{a}.
\end{description}

\begin{description}
	\item[a \textless{}\textless{} b] shift \textbf{a} left by \textbf{b} (multiply a by \(2^b\))
\end{description}

\begin{description}
	\item[a \textgreater{}\textgreater{} b] shift \textbf{a} right by \textbf{b} (divide a by \(2^b\))
\end{description}

\subsubsection{The If-Else statement}
If-Else provides a way to instruct the computer to execute a block of code only
if certain conditions have been met. The syntax of an If-Else construct is:
\lstset{basicstyle=\scriptsize, numbers=left, captionpos=b, tabsize=4}
\begin{lstlisting}[caption=Section \thesection listing \arabic{controlcnt},language={C},
breaklines=true,xleftmargin=15pt, label=lst:section\thesection listing\arabic{controlcnt}]
if(/* condition goes here */) {
	/* if the condition is non-zero (true), this code will execute */
} else {
	/* if the condition is 0 (false), this code will execute */
}
\end{lstlisting}
\stepcounter{controlcnt}

The first block of code executes if the condition in parentheses directly after
the \emph{if} evaluates to non-zero (true); otherwise, the second block
executes.

The \emph{else} and following block of code are completely optional. If there
is no need to execute code if a condition is not true, leave it out.

Also, keep in mind that an \emph{if} can directly follow an \emph{else}
statement. While this can occasionally be useful, chaining more than two or
three if-elses in this fashion is considered bad programming practice. We can
get around this with the Switch-Case construct described later.

Two other general syntax notes need to be made that you will also see in other
control constructs: First, note that there is no semicolon after \emph{if} or
\emph{else}. There could be, but the block (code enclosed in \{ and \}) takes
the place of that. Second, if you only intend to execute one statement as a
result of an \emph{if} or \emph{else}, curly braces are not needed. However,
many programmers believe that inserting curly braces anyway in this case is
good coding practice.

The following code sets a variable c equal to the greater of two variables a
and b, or 0 if a and b are equal.
\lstset{basicstyle=\scriptsize, numbers=left, captionpos=b, tabsize=4}
\begin{lstlisting}[caption=Section \thesection listing \arabic{controlcnt},language={C},
breaklines=true,xleftmargin=15pt, label=lst:section\thesection listing\arabic{controlcnt}]
if(a > b) {
	c = a;
} else if(b > a) {
	c = b;
} else {
	c = 0;
}
\end{lstlisting} 
\stepcounter{controlcnt}

Consider this question: why can't you just forget about \emph{else} and write
the code like:

\lstset{basicstyle=\scriptsize, numbers=left, captionpos=b, tabsize=4}
\begin{lstlisting}[caption=Section \thesection listing \arabic{controlcnt},language={C},
breaklines=true,xleftmargin=15pt, label=lst:section\thesection listing\arabic{controlcnt}]
if(a > b) {
	c = a;
}

if(a < b) {
	c = b;
}

if(a == b) {
	c = 0;
}
\end{lstlisting}
\stepcounter{controlcnt}

There are several answers to this. Most importantly, if your conditionals are
not mutually exclusive, \emph{two} cases could execute instead of only one. If
the code was different and the value of a or b changes somehow (e.g.: you reset
the lesser of a and b to 0 after the comparison) during one of the blocks? You
could end up with multiple \emph{if} statements being invoked, which is not
your intent. Also, evaluating \emph{if} conditionals takes processor time. If
you use \emph{else} to handle these situations, in the case above assuming (a
\textgreater{} b) is non-zero (true), the program is spared the expense of
evaluating additional \emph{if} statements. The bottom line is that it is
usually best to insert an \emph{else} clause for all cases in which a
conditional will not evaluate to non-zero (true).

\paragraph{The conditional expression}
A conditional expression is a way to set values conditionally in a more
shorthand fashion than If-Else. The syntax is:
\lstset{basicstyle=\scriptsize, numbers=left, captionpos=b, tabsize=4}
\begin{lstlisting}[caption=Section \thesection listing \arabic{controlcnt},language={C},
breaklines=true,xleftmargin=15pt, label=lst:section\thesection listing\arabic{controlcnt}]
(/* logical expression goes here */) ? (/* if non-zero (true) */) : (/* if 0 (false) */)
\end{lstlisting}
\stepcounter{controlcnt}

The logical expression is evaluated. If it is non-zero (true), the overall
conditional expression evaluates to the expression placed between the ? and :,
otherwise, it evaluates to the expression after the :. Therefore, the above
example (changing its function slightly such that c is set to b when a and b
are equal) becomes:

\lstset{basicstyle=\scriptsize, numbers=left, captionpos=b, tabsize=4}
\begin{lstlisting}[caption=Section \thesection listing \arabic{controlcnt},language={C},
breaklines=true,xleftmargin=15pt, label=lst:section\thesection listing\arabic{controlcnt}]
c = (a > b) ? a : b;
\end{lstlisting}
\stepcounter{controlcnt}

Conditional expressions can sometimes clarify the intent of the code. Nesting
the conditional operator should usually be avoided. It's best to use
conditional expressions only when the expressions for a and b are simple. Also,
contrary to a common beginner belief, conditional expressions do not make for
faster code. As tempting as it is to assume that fewer lines of code result in
faster execution times, there is no such correlation.

\subsubsection{The Switch-Case statement}
Say you write a program where the user inputs a number 1-5 (corresponding to
student grades, A(represented as 1)-D(4) and F(5)), stores it in a variable
\textbf{grade} and the program responds by printing to the screen the
associated letter grade. If you implemented this using If-Else, your code would
look something like this:

\lstset{basicstyle=\scriptsize, numbers=left, captionpos=b, tabsize=4}
\begin{lstlisting}[caption=Section \thesection listing \arabic{controlcnt},language={C},
breaklines=true,xleftmargin=15pt, label=lst:section\thesection listing\arabic{controlcnt}]
if(grade == 1) {
	printf("A\n");
} else if(grade == 2) {
	printf("B\n");
} else if /* etc. etc. */
\end{lstlisting}
\stepcounter{controlcnt}

Having a long chain of if-else-if-else-if-else can be a pain, both for the
programmer and anyone reading the code. Fortunately, there's a solution: the
Switch-Case construct, of which the basic syntax is:
\lstset{basicstyle=\scriptsize, numbers=left, captionpos=b, tabsize=4}
\begin{lstlisting}[caption=Section \thesection listing \arabic{controlcnt},language={C},
breaklines=true,xleftmargin=15pt, label=lst:section\thesection listing\arabic{controlcnt}]
switch(/* integer or enum goes here */) {
	case /* potential value of the aforementioned int or enum */:
		/* code */
	case /* a different potential value */:
		/* different code */
		/* insert additional cases as needed */
	default: 
		/* more code */
}
\end{lstlisting}
\stepcounter{controlcnt}

The Switch-Case construct takes a variable, usually an int or an enum, placed
after \emph{switch}, and compares it to the value following the \emph{case}
keyword. If the variable is equal to the value specified after \emph{case}, the
construct ``activates'', or begins executing the code after the case statement.
Once the construct has ``activated'', there will be no further evaluation of
\emph{case}s. 

Switch-Case is syntactically ``weird'' in that no braces are required for code
associated with a \emph{case}.

\textbf{\emph{Very important}}: Typically, the last statement for each case is
a break statement. This causes program execution to jump to the statement
following the closing bracket of the switch statement, which is what one would
normally want to happen. However if the break statement is omitted, program
execution continues with the first line of the next case, if any. This is
called a \emph{fall-through}. When a programmer desires this action, a comment
should be placed at the end of the block of statements indicating the desire to
fall through. Otherwise another programmer maintaining the code could consider
the omission of the 'break' to be an error, and inadvertently 'correct' the
problem. Here's an example:
\lstset{basicstyle=\scriptsize, numbers=left, captionpos=b, tabsize=4}
\begin{lstlisting}[caption=Section \thesection listing \arabic{controlcnt},language={C},
breaklines=true,xleftmargin=15pt, label=lst:section\thesection listing\arabic{controlcnt}]
switch ( someVariable ) {
	case 1:
		printf("This code handles case 1\n");
		break;
	case 2:
		printf("This prints when someVariable is 2, along with...\n");
		/* FALL THROUGH */
	case 3:
		printf("This prints when someVariable is either 2 or 3.\n" );
		break;
}
\end{lstlisting}
\stepcounter{controlcnt}

If a \emph{default} case is specified, the associated statements are executed
if none of the other cases match. A \emph{default} case is optional. Here's a
switch statement that corresponds to the sequence of if --- else if statements
above.

Back to our example above. Here's what it would look like as Switch-Case:
\lstset{basicstyle=\scriptsize, numbers=left, captionpos=b, tabsize=4}
\begin{lstlisting}[caption=Section \thesection listing \arabic{controlcnt},language={C},
breaklines=true,xleftmargin=15pt, label=lst:section\thesection listing\arabic{controlcnt}]
switch (grade) {
	case 1:
		printf("A\n");
		break;
	case 2:
		printf("B\n");
		break;
	case 3:
		printf("C\n");
		break;
	case 4:
		printf("D\n");
		break;
	default:
		printf("F\n");
		break;
}
\end{lstlisting}
\stepcounter{controlcnt}

A set of statements to execute can be grouped with more than one value of the
variable as in the following example. (the fall-through comment is not
necessary here because the intended behavior is obvious)

\lstset{basicstyle=\scriptsize, numbers=left, captionpos=b, tabsize=4}
\begin{lstlisting}[caption=Section \thesection listing \arabic{controlcnt},language={C},
breaklines=true,xleftmargin=15pt, label=lst:section\thesection listing\arabic{controlcnt}]
switch (something) {
	case 2:
	case 3:
	case 4:
		/* some statements to execute for 2, 3 or 4 */
		break;
	case 1:
	default:
		/* some statements to execute for 1 or other than 2,3,and 4 */
		break;
}
\end{lstlisting}
\stepcounter{controlcnt}

Switch-Case constructs are particularly useful when used in conjunction with
user defined \emph{enum} data types. Some compilers are capable of warning
about an unhandled enum value, which may be helpful for avoiding bugs.

\subsection{Loops}
Often in computer programming, it is necessary to perform a certain action a
certain number of times or until a certain condition is met. It is impractical
and tedious to simply type a certain statement or group of statements a large
number of times, not to mention that this approach is too inflexible and
unintuitive to be counted on to stop when a certain event has happened. As a
real-world analogy, someone asks a dishwasher at a restaurant what he did all
night. He will respond, ``I washed dishes all night long.'' He is not likely to
respond, ``I washed a dish, then washed a dish, then washed a dish, then...''.
The constructs that enable computers to perform certain repetitive tasks are
called loops.

\subsubsection{While loops}
A while loop is the most basic type of loop. It will run as long as the
condition is non-zero (true). For example, if you try the following, the
program will appear to lock up and you will have to manually close the program
down. A situation where the conditions for exiting the loop will never become
true is called an infinite loop. 
\lstset{basicstyle=\scriptsize, numbers=left, captionpos=b, tabsize=4}
\begin{lstlisting}[caption=Section \thesection listing \arabic{controlcnt},language={C},
breaklines=true,xleftmargin=15pt, label=lst:section\thesection listing\arabic{controlcnt}]
int a=1;
while(42) {
	a = a*2;
}
\end{lstlisting}
\stepcounter{controlcnt}

Here is another example of a while loop. It prints out all the powers of two
less than 100.
\lstset{basicstyle=\scriptsize, numbers=left, captionpos=b, tabsize=4}
\begin{lstlisting}[caption=Section \thesection listing \arabic{controlcnt},language={C},
breaklines=true,xleftmargin=15pt, label=lst:section\thesection listing\arabic{controlcnt}]
int a=1;
while(a<100) {
	printf("a is %d \n",a);
	a = a*2;
}
\end{lstlisting}
\stepcounter{controlcnt}

The flow of all loops can also be controlled by \textbf{break} and
\textbf{continue} statements. A break statement will immediately exit the
enclosing loop. A continue statement will skip the remainder of the block and
start at the controlling conditional statement again. For example:
\lstset{basicstyle=\scriptsize, numbers=left, captionpos=b, tabsize=4}
\begin{lstlisting}[caption=Section \thesection listing \arabic{controlcnt},language={C},
breaklines=true,xleftmargin=15pt, label=lst:section\thesection listing\arabic{controlcnt}]
int a=1;
while (42) { // loops until the break statement in the loop is executed
	printf("a is %d ",a);
	a = a*2;
	if(a>100) {
		break;
	} else if(a==64) {
		continue;  // Immediately restarts at while, skips next step
	}
	printf("a is not 64\n");
}
\end{lstlisting}
\stepcounter{controlcnt}

In this example, the computer prints the value of a as usual, and prints a
notice that a is not 64 (unless it was skipped by the continue statement).

Similar to If above, braces for the block of code associated with a While loop
can be omitted if the code consists of only one statement, for example:
\lstset{basicstyle=\scriptsize, numbers=left, captionpos=b, tabsize=4}
\begin{lstlisting}[caption=Section \thesection listing \arabic{controlcnt},language={C},
breaklines=true,xleftmargin=15pt, label=lst:section\thesection listing\arabic{controlcnt}]
int a=1;
while(a < 100) a = a*2;
\end{lstlisting}
\stepcounter{controlcnt}

This will merely increase a until a is not less than 100.

When the computer reaches the end of the while loop, it always goes back to the
while statement at the top of the loop, where it re-evaluates the controlling
condition.  If that condition is ``true'' at that instant -- even if it was
temporarily 0 for a few statements inside the loop -- then the computer begins
executing the statements inside the loop again; otherwise the computer exits
the loop.  The computer does not ``continuously check'' the controlling
condition of a while loop during the execution of that loop.  It only ``peeks''
at the controlling condition each time it reaches the \texttt{while} at the top
of the loop.

It is very important to note, once the controlling condition of a While loop
becomes 0 (false), the loop will not terminate until the block of code is
finished and it is time to reevaluate the conditional. If you need to terminate
a While loop immediately upon reaching a certain condition, consider using
\textbf{break}.

A common idiom is to write:
\lstset{basicstyle=\scriptsize, numbers=left, captionpos=b, tabsize=4}
\begin{lstlisting}[caption=Section \thesection listing \arabic{controlcnt},language={C},
breaklines=true,xleftmargin=15pt, label=lst:section\thesection listing\arabic{controlcnt}]
int i = 5;
while(i--) {
	printf("java and c# can't do this\n");
}
\end{lstlisting}
\stepcounter{controlcnt}

This executes the code in the while loop 5 times, with i having values that
range from 4 down to 0 (inside the loop). Conveniently, these are the values
needed to access every item of an array containing 5 elements.

\subsubsection{For loops}
For loops generally look something like this:
\lstset{basicstyle=\scriptsize, numbers=left, captionpos=b, tabsize=4}
\begin{lstlisting}[caption=Section \thesection listing \arabic{controlcnt},language={C},
breaklines=true,xleftmargin=15pt, label=lst:section\thesection listing\arabic{controlcnt}]
for(initialization; test; increment) {
	/* code */
}
\end{lstlisting}
\stepcounter{controlcnt}

The \emph{initialization} statement is executed exactly once --- before the
first evaluation of the \emph{test} condition. Typically, it is used to assign
an initial value to some variable, although this is not strictly necessary. The
\emph{initialization} statement can also be used to declare and initialize
variables used in the loop.

The \emph{test} expression is evaluated each time before the code in the
\emph{for} loop executes. If this expression evaluates as 0 (false) when it is
checked (i.e. if the expression is not true), the loop is not (re)entered and
execution continues normally at the code immediately following the FOR-loop. If
the expression is non-zero (true), the code within the braces of the loop is
executed. 

After each iteration of the loop, the \emph{increment} statement is executed.
This often is used to increment the loop index for the loop, the variable
initialized in the initialization expression and tested in the test expression.
Following this statement execution, control returns to the top of the loop,
where the \emph{test} action occurs. If a \emph{continue} statement is executed
within the \emph{for} loop, the increment statement would be the next one
executed. 

Each of these parts of the for statement is optional and may be omitted.
Because of the free-form nature of the for statement, some fairly fancy things
can be done with it. Often a for loop is used to loop through items in an
array, processing each item at a time. 
\lstset{basicstyle=\scriptsize, numbers=left, captionpos=b, tabsize=4}
\begin{lstlisting}[caption=Section \thesection listing \arabic{controlcnt},language={C},
breaklines=true,xleftmargin=15pt, label=lst:section\thesection listing\arabic{controlcnt}]
int  myArray[12];
int ix;
for (ix = 0; ix<12; ix++) {
	myArray[ix] = 5 * ix + 3;
}
\end{lstlisting}
\stepcounter{controlcnt}

The above for loop initializes each of the 12 elements of myArray.  The loop
index can start from any value. In the following case it starts from 1.
\lstset{basicstyle=\scriptsize, numbers=left, captionpos=b, tabsize=4}
\begin{lstlisting}[caption=Section \thesection listing \arabic{controlcnt},language={C},
breaklines=true,xleftmargin=15pt, label=lst:section\thesection listing\arabic{controlcnt}]
for(ix = 1; ix <= 10; ix++) {
	printf("%d ", ix);
}
\end{lstlisting}
\stepcounter{controlcnt}

which will print
\scriptsize
\begin{verbatim}
1 2 3 4 5 6 7 8 9 10 
\end{verbatim}
\normalsize

You will most often use loop indexes that start from 0, since arrays are
indexed at zero, but you will sometimes use other values to initalize a loop
index as well.

The \emph{increment} action can do other things, such as \emph{decrement}. So
this kind of loop is common:

\lstset{basicstyle=\scriptsize, numbers=left, captionpos=b, tabsize=4}
\begin{lstlisting}[caption=Section \thesection listing \arabic{controlcnt},language={C},
breaklines=true,xleftmargin=15pt, label=lst:section\thesection listing\arabic{controlcnt}]
for (i = 5; i > 0; i--) {
	printf("%d ",i);
}
\end{lstlisting}
\stepcounter{controlcnt}

which yields
\scriptsize
\begin{verbatim}
	5 4 3 2 1 
\end{verbatim}
\normalsize

Here's an example where the test condition is simply a variable. If the
variable has a value of 0 or NULL, the loop exits, otherwise the statements in
the body of the loop are executed.
\lstset{basicstyle=\scriptsize, numbers=left, captionpos=b, tabsize=4}
\begin{lstlisting}[caption=Section \thesection listing \arabic{controlcnt},language={C},
breaklines=true,xleftmargin=15pt, label=lst:section\thesection listing\arabic{controlcnt}]
for (t = list_head; t; t = NextItem(t) ) {
	/*body of loop */
}
\end{lstlisting}
\stepcounter{controlcnt}

A WHILE loop can be used to do the same thing as a FOR loop, however a FOR loop
is a more condensed way to perform a set number of repetitions since all of the
necessary information is in a one line statement.

A FOR loop can also be given no conditions, for example:

\lstset{basicstyle=\scriptsize, numbers=left, captionpos=b, tabsize=4}
\begin{lstlisting}[caption=Section \thesection listing \arabic{controlcnt},language={C},
breaklines=true,xleftmargin=15pt, label=lst:section\thesection listing\arabic{controlcnt}]
for(;;) {
	/* block of statements */
}
\end{lstlisting}
\stepcounter{controlcnt}

This is called a infinite loop since it will loop forever unless there is a
break statement within the statements of the for loop. The empty test condition
effectively evaluates as true.

It is also common to use the comma operator in for loops to execute multiple
statements. 
\lstset{basicstyle=\scriptsize, numbers=left, captionpos=b, tabsize=4}
\begin{lstlisting}[caption=Section \thesection listing \arabic{controlcnt},language={C},
breaklines=true,xleftmargin=15pt, label=lst:section\thesection listing\arabic{controlcnt}]
int i, j, n = 10;
for(i = 0, j = 0; i <= n; i++,j+=2) {
	printf("i = %d , j = %d \n",i,j);
}
\end{lstlisting}
\stepcounter{controlcnt}

\subsubsection{Do-While loops}
A DO-WHILE loop is a post-check while loop, which means that it checks the
condition after each run. As a result, even if the condition is zero (false),
it will run at least once. It follows the form of:
\lstset{basicstyle=\scriptsize, numbers=left, captionpos=b, tabsize=4}
\begin{lstlisting}[caption=Section \thesection listing \arabic{controlcnt},language={C},
breaklines=true,xleftmargin=15pt, label=lst:section\thesection listing\arabic{controlcnt}]
do {
	/* do stuff */
} while (condition);
\end{lstlisting}
\stepcounter{controlcnt}

Note the terminating semicolon. This is required for correct syntax. Since this
is also a type of while loop, \textbf{break} and \textbf{continue} statements
within the loop function accordingly. A \textbf{continue} statement causes a
jump to the test of the condition and a \emph{break} statement exits the loop.

It is worth noting that Do-While and While are functionally almost identical,
with one important difference: Do-While loops are always guaranteed to execute
at least once, but While loops will not execute at all if their condition is 0
(false) on the first evaluation.

\subsection{One last thing: goto}
\textbf{goto} is a very simple and traditional control mechanism. It is a
statement used to immediately and unconditionally jump to another line of code.
To use goto, you must place a label at a point in your program. A label
consists of a name followed by a colon (:) on a line by itself. Then, you can
type goto \emph{label}; at the desired point in your program. The code will
then continue executing beginning with \emph{label}. This looks like:
\lstset{basicstyle=\scriptsize, numbers=left, captionpos=b, tabsize=4}
\begin{lstlisting}[caption=Section \thesection listing \arabic{controlcnt},language={C},
breaklines=true,xleftmargin=15pt, label=lst:section\thesection listing\arabic{controlcnt}]
MyLabel:
/* some code */
goto MyLabel;
\end{lstlisting}
\stepcounter{controlcnt}

The ability to transfer the flow of control enabled by gotos is so powerful
that, in addition to the simple if, all other control constructs can be written
using gotos instead. Here, we can let S and T be any arbitrary
statements:
\lstset{basicstyle=\scriptsize, numbers=left, captionpos=b, tabsize=4}
\begin{lstlisting}[caption=Section \thesection listing \arabic{controlcnt},language={C},
breaklines=true,xleftmargin=15pt, label=lst:section\thesection listing\arabic{controlcnt}]
if (''cond'') {
	S;
} else {
	T;
}
/* ... */
\end{lstlisting}
\stepcounter{controlcnt}

The same statement could be accomplished using two gotos and two labels:
\lstset{basicstyle=\scriptsize, numbers=left, captionpos=b, tabsize=4}
\begin{lstlisting}[caption=Section \thesection listing \arabic{controlcnt},language={C},
breaklines=true,xleftmargin=15pt, label=lst:section\thesection listing\arabic{controlcnt}]
if (''cond'') goto Label1;
	T;
	goto Label2;
	Label1:
	S;
	Label2:
/* ... */
\end{lstlisting}
\stepcounter{controlcnt}

Here, the first goto is conditional on the value of ``cond''. The second goto
is unconditional. We can perform the same translation on a loop:
\lstset{basicstyle=\scriptsize, numbers=left, captionpos=b, tabsize=4}
\begin{lstlisting}[caption=Section \thesection listing \arabic{controlcnt},language={C},
breaklines=true,xleftmargin=15pt, label=lst:section\thesection listing\arabic{controlcnt}]
while (''cond1'') {
	S;
	if (''cond2'') break;
		T;
}
/* ... */
\end{lstlisting}
\stepcounter{controlcnt}

Which can be written as:
\lstset{basicstyle=\scriptsize, numbers=left, captionpos=b, tabsize=4}
\begin{lstlisting}[caption=Section \thesection listing \arabic{controlcnt},language={C},
breaklines=true,xleftmargin=15pt, label=lst:section\thesection listing\arabic{controlcnt}]
Start:
if (!''cond1'') goto End;
S;
if (''cond2'') goto End;
T;
goto Start;
End:
/* ... */
\end{lstlisting}
\stepcounter{controlcnt}

As these cases demonstrate, often the structure of what your program is doing
can usually be expressed without using gotos. Undisciplined use of gotos can
create unreadable, unmaintainable code when more idiomatic alternatives (such
as if-elses, or for loops) can better express your structure. Theoretically,
the goto construct does not ever \emph{have} to be used, but there are cases
when it can increase readability, avoid code duplication, or make control
variables unnecessary. You should consider first mastering the idiomatic
solutions, and use goto only when necessary. Keep in mind that many, if not
most, C style guidelines \emph{strictly forbid} use of \textbf{goto}, with the
only common exceptions being the following examples.

One use of goto is to break out of a deeply nested loop. Since \textbf{break}
will not work (it can only escape one loop), \textbf{goto} can be used to jump
completely outside the loop. Breaking outside of deeply nested loops without
the use of the goto is always possible, but often involves the creation and
testing of extra variables that may make the resulting code far less readable
than it would be with \textbf{goto}. The use of \textbf{goto} makes it easy to
undo actions in an orderly fashion, typically to avoid failing to free memory
that had been allocated.

Another accepted use is the creation of a state machine. This is a fairly
advanced topic though, and not commonly needed.
