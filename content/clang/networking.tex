\section{Networking in UNIX}
Network programming under UNIX is relatively simple in C.

This guide assumes you already have a good general idea about C, UNIX and
networks.

\subsection{A simple client}
To start with, we'll look at one of the simplest things you can do, initialize
a stream connection and receive a message from a remote server.
\lstset{basicstyle=\scriptsize, numbers=left, captionpos=b, tabsize=4}
\begin{lstlisting}[caption=,language={C},
breaklines=true,xleftmargin=15pt,label=lst:]
#include <stdio.h>
#include <stdlib.h>
#include <string.h>
#include <unistd.h>
#include <arpa/inet.h>
#include <sys/types.h>
#include <netinet/in.h>
#include <sys/socket.h>

#define MAXRCVLEN 500
#define PORTNUM 2343

int main(int argc, char *argv[]) {
	char buffer[MAXRCVLEN + 1]; /* +1 so we can add null terminator */
	int len, mysocket;
	struct sockaddr_in dest; 

	mysocket = socket(AF_INET, SOCK_STREAM, 0);
 
	memset(&dest, 0, sizeof(dest));				/* zero the struct */
	dest.sin_family = AF_INET;
	dest.sin_addr.s_addr = inet_addr("127.0.0.1"); /* set destination IP number */ 
	dest.sin_port = htons(PORTNUM);				/* set destination port number */

	connect(mysocket, (struct sockaddr *)&dest, sizeof(struct sockaddr));
 
	len = recv(mysocket, buffer, MAXRCVLEN, 0);

	/* We have to null terminate the received data ourselves */
	buffer[len] = '\0';

	printf("Received %s (%d bytes).\n", buffer, len);

	close(mysocket);
	return EXIT_SUCCESS;
}
\end{lstlisting}

This is the very bare bones of a client; in practice, we would check every
function that we called for failure, however for clarity, error checking the
code has been left out for now.

As you can see, the code mainly revolves around ``dest'' which is a struct of
type sockaddr\_in; in this struct we store information about the machine we
want to connect to.
\lstset{basicstyle=\scriptsize, numbers=left, captionpos=b, tabsize=4}
\begin{lstlisting}[caption=,language={C},
breaklines=true,xleftmargin=15pt,label=lst:]
mysocket = socket(AF_INET, SOCK_STREAM, 0);
\end{lstlisting}

The socket function tells our OS that we want a file descriptor for a socket
which we can use for a network stream connection; what the parameters mean is
mostly irrelevent for now.
\lstset{basicstyle=\scriptsize, numbers=left, captionpos=b, tabsize=4}
\begin{lstlisting}[caption=,language={C},
breaklines=true,xleftmargin=15pt,label=lst:]
memset(&dest, 0, sizeof(dest));                /* zero the struct */
dest.sin_family = AF_INET;
dest.sin_addr.s_addr = inet_addr("127.0.0.1"); /* set destination IP number */ 
dest.sin_port = htons(PORTNUM);                /* set destination port number */
\end{lstlisting}

Now we get on to the interesting part:

The first line uses memset to zero the struct.

The second line sets the address family, this should be the same as was used in
the socket function, for most purposes AF\_INET will serve.

The third line is where we set the IP of the machine we need to connect to. The
variable dest.sin\_addr.s\_addr is just an integer stored in Big Endian format,
but we don't have to know that as the inet\_addr function will do the
conversion from string into Big Endian integer for us.

The fourth line sets the destination port number, the htons() function converts
the port number into a Big Endian short integer. If your program is going to be
solely run on machines which use Big Endian numbers as default then
dest.sin\_port = 21; would work just as well, however for portability reasons
htons() should be used.

Now that all of the preliminary work is done, we can actually make the
connection and use it:
\lstset{basicstyle=\scriptsize, numbers=left, captionpos=b, tabsize=4}
\begin{lstlisting}[caption=,language={C},
breaklines=true,xleftmargin=15pt,label=lst:]
connect(mysocket, (struct sockaddr *)&dest, sizeof(struct sockaddr));
\end{lstlisting}

This tells our OS to use the socket mysocket to create a connection to the
machine specified in dest.
\lstset{basicstyle=\scriptsize, numbers=left, captionpos=b, tabsize=4}
\begin{lstlisting}[caption=,language={C},
breaklines=true,xleftmargin=15pt,label=lst:]
len = recv(mysocket, buffer, MAXRCVLEN, 0);
\end{lstlisting}

Now this receives upto MAXRCVLEN bytes of data from the connection and stores
them in the buffer string. The number of characters received is returned by
recv(). It is important to note that the data received will not automatically
be null terminated when stored in buffer hence we need to do it ourselves with
buffer[inputlen] = '\textbackslash{}0'.

And that's about it!

The next step after learning how to receive data is learning how to send it. If
you've understood the previous section then this is quite easy. All you have to
do is use the the send() function, which uses the same parameters as receive.
If in our previous example ``buffer'' had the text we wanted to send and its
length was stored in ``len'' we would write send(mysocket, buffer, len, 0).
send() returns the number of bytes that were sent. It is important to remember
that send(), for various reasons, may not be able to send all of the bytes, so
it is imporant to check that this value is equal to the number of bytes you
were sending. In most cases this can be resolved by resending the unsent data.

\subsection{A simple server}
\lstset{basicstyle=\scriptsize, numbers=left, captionpos=b, tabsize=4}
\begin{lstlisting}[caption=,language={C},
breaklines=true,xleftmargin=15pt,label=lst:]
#include <stdio.h>
#include <stdlib.h>
#include <string.h>
#include <unistd.h>
#include <arpa/inet.h>
#include <sys/types.h>
#include <netinet/in.h>
#include <sys/socket.h>

#define PORTNUM 2343

int main(int argc, char *argv[]) {
	char msg[] = "Hello World !\n";
 
	struct sockaddr_in dest; /* socket info about the machine connecting to us */
	struct sockaddr_in serv; /* socket info about our server */
	int mysocket;			/* socket used to listen for incoming connections */
	int socksize = sizeof(struct sockaddr_in);
 
	memset(&serv, 0, sizeof(serv));	/* zero the struct before filling the fields */
	serv.sin_family = AF_INET;		 /* set the type of connection to TCP/IP */
	serv.sin_addr.s_addr = INADDR_ANY; /* set our address to any interface */
	serv.sin_port = htons(PORTNUM);	/* set the server port number */	
 
	mysocket = socket(AF_INET, SOCK_STREAM, 0);
 
	/* bind serv information to mysocket */
	bind(mysocket, (struct sockaddr *)&serv, sizeof(struct sockaddr));
 
	/* start listening, allowing a queue of up to 1 pending connection */
	listen(mysocket, 1);
	int consocket = accept(mysocket, (struct sockaddr *)&dest, &socksize);
 
	while(consocket)
	{
		
 
		printf("Incoming connection from %s - sending welcome\n", inet_ntoa(dest.sin_addr));
		send(consocket, msg, strlen(msg), 0);
 
		
	}
	close(consocket);
	close(mysocket);
	return EXIT_SUCCESS;
}
\end{lstlisting}

Superficially, this is very similar to the client. The first important
difference is that rather than creating a sockaddr\_in with information about
the machine we're connecting to we create it with information about the server,
and then we ``bind'' it to the socket. This allows the machine to know the data
received on the port specified in the sockaddr\_in should be handled by our
specified socket.

The listen command then tells our program to start listening using that socket,
the second parameter of listen() allows us to specify the maximum number of
connections that can be queued. Each time a connection is made to the server it
is added to the queue, we take connections from the queue using the accept()
function.

Then we have the core of the server code, we use the accept() function to take
a connection from the queue. If there is no connection waiting on the queue the
function causes the program to wait until a connection is received. The
accept() function returns us another socket this is essentially a ``session''
socket solely for communicating with connection we took off the queue. The
original socket (mysocket) continues to listen on the specified port for
further connections.

Once we have ``session'' socket we can handle it in the same way as with the
client using send() and recv() to handle data transfers.

Note that this server can only accept one connection at a time; if you want to
simultaneously handle multiple clients then you'll need to fork() off separate
processes, or use threads, to handle the connections.

\subsection{Useful network functions}
\lstset{basicstyle=\scriptsize, numbers=left, captionpos=b, tabsize=4}
\begin{lstlisting}[caption=,language={C},
breaklines=true,xleftmargin=15pt,label=lst:]
int gethostname(char *hostname, size_t size);
\end{lstlisting}

Parameters are a pointer to the first character of an array of chars and the
size of that array. If possible, it finds the hostname and stores it in the
array. On failure it returns -1.
\lstset{basicstyle=\scriptsize, numbers=left, captionpos=b, tabsize=4}
\begin{lstlisting}[caption=,language={C},
breaklines=true,xleftmargin=15pt,label=lst:]
struct hostent *gethostbyname(const char *name);
\end{lstlisting}

This function obtains information about a domain name and stored it in a
hostent struct, in the hostent structure the most useful part is the (char**)
h\_addr\_list field, this is a null terminated array of IP addresses associated
with that domain. The field h\_addr is a pointer to the first IP address in the
h\_addr\_list array. Returns NULL on failure.

\subsection{What about stateless connections?}
If you don't want to exploit the properties of TCP in your program but rather
just have to use a UDP protocol then you can just switch ``SOCK\_STREAM'' in
your socket call for ``SOCK\_DGRAM'' and use it in the same way. It is
important to remember that UDP does not guarantee delivery of packets and order
of delivery, so checking is important.

If you want to exploit the properties of UDP, then you can use sendto() and
recvfrom() which operate like send() and recv() except you need to provide
extra parameters specifiying who you are communicating with.

\subsection{How to check for errors?}
The functions socket(), recv() and connect() all return -1 on failure and use
errno for further details.
