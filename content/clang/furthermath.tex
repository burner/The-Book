\section{Further math}
The \texttt{\textless{}math.h\textgreater{}} header contains prototypes for
several functions that deal with mathematics. In the 1990 version of the ISO
standard, only the \texttt{double} versions of the functions were specified;
the 1999 version added the \texttt{float} and \texttt{long double} versions. To
use these math functions, you must link your program with the math library. For
some compilers (including GCC), you must specify the additional parameter
\texttt{-lm}.

The functions can be grouped into the following categories:
\subsection{Trigonometric functions}
\subsection{The \texttt{acos} and \texttt{asin} functions}
The \texttt{acos} functions return the arccosine of their arguments in radians,
and the \texttt{asin} functions return the arcsine of their arguments in
radians. All functions expect the argument in the range \url{-1,+1}. The
arccosine returns a value in the range \url{0,&pi;}; the arcsine returns a
value in the range \url{-&pi;/2,+&pi;/2}.

\lstset{basicstyle=\scriptsize, numbers=left, captionpos=b, tabsize=4}
\begin{lstlisting}[caption=,language={C},
breaklines=true,xleftmargin=15pt,label=lst:]
#include <math.h>
float asinf(float x); /* C99 */
float acosf(float x); /* C99 */
double asin(double x);
double acos(double x);
long double asinl(long double x); /* C99 */
long double acosl(long double x); /* C99 */
\end{lstlisting}

\subsubsection{The \texttt{atan} and \texttt{atan2} functions}
The \texttt{atan} functions return the arctangent of their arguments in
radians, and the \texttt{atan2} function return the arctangent of \texttt{y/x}
in radians. The \texttt{atan} functions return a value in the range
\url{-&pi;/2,+&pi;/2} (the reason why \&plusmn;\ensuremath{\pi}/2 are included
in the range is because the floating-point value may represent infinity, and
atan(\&plusmn;\&infin;) = \&plusmn;\ensuremath{\pi}/2); the \texttt{atan2}
functions return a value in the range \url{-&pi;,+&pi;}. For \texttt{atan2}, a
domain error may occur if both arguments are zero.

\lstset{basicstyle=\scriptsize, numbers=left, captionpos=b, tabsize=4}
\begin{lstlisting}[caption=,language={C},
breaklines=true,xleftmargin=15pt,label=lst:]
#include <math.h>
float atanf(float x); /* C99 */
float atan2f(float y, float x); /* C99 */
double atan(double x);
double atan2(double y, double x);
long double atanl(long double x); /* C99 */
long double atan2l(long double y, long double x); /* C99 */
\end{lstlisting}

\subsubsection{The \texttt{cos}, \texttt{sin}, and \texttt{tan} functions}
The \texttt{cos}, \texttt{sin}, and \texttt{tan} functions return the cosine,
sine, and tangent of the argument, expressed in radians.

\lstset{basicstyle=\scriptsize, numbers=left, captionpos=b, tabsize=4}
\begin{lstlisting}[caption=,language={C},
breaklines=true,xleftmargin=15pt,label=lst:]
#include <math.h>
float cosf(float x); /* C99 */
float sinf(float x); /* C99 */
float tanf(float x); /* C99 */
double cos(double x);
double sin(double x);
double tan(double x);
long double cosl(long double x); /* C99 */
long double sinl(long double x); /* C99 */
long double tanl(long double x); /* C99 */
\end{lstlisting}

\subsection{Hyperbolic functions}
The \texttt{cosh}, \texttt{sinh} and \texttt{tanh} functions compute the
hyperbolic cosine, the hyperbolic sine, and the hyperbolic tangent of the
argument respectively. For the hyperbolic sine and cosine functions, a range
error occurs if the magnitude of the argument is too large.

The \texttt{acosh} functions compute the area hyperbolic cosine of the
argument. A domain error occurs for arguments less than 1.

The \texttt{asinh} functions compute the area hyperbolic sine of the argument.

The \texttt{atanh} functions compute the area hyperbolic tangent of the
argument. A domain error occurs if the argument is not in the interval
\href{-1,}{+1}. A range error may occur if the argument equals -1 or +1.

\lstset{basicstyle=\scriptsize, numbers=left, captionpos=b, tabsize=4}
\begin{lstlisting}[caption=,language={C},
breaklines=true,xleftmargin=15pt,label=lst:]
#include <math.h>
float coshf(float x); /* C99 */
float sinhf(float x); /* C99 */
float tanhf(float x); /* C99 */
double cosh(double x); 
double sinh(double x);
double tanh(double x);
long double coshl(long double x); /* C99 */
long double sinhl(long double x); /* C99 */
long double tanhl(long double x); /* C99 */
float acoshf(float x); /* C99 */
float asinhf(float x); /* C99 */
float atanhf(float x); /* C99 */
double acosh(double x); /* C99 */
double asinh(double x); /* C99 */
double atanh(double x); /* C99 */
long double acoshl(long double x); /* C99 */
long double asinhl(long double x); /* C99 */
long double atanhl(long double x); /* C99 */
\end{lstlisting}

\subsection{Exponential and logarithmic functions}
\subsubsection{The \texttt{exp}, \texttt{exp2}, and \texttt{expm1} functions}
The \texttt{exp} functions compute the base-\emph{e} exponential function of
\texttt{x} (\emph{e}\(^{x}\)). A range error occurs if the magnitude of
\texttt{x} is too large.

The \texttt{exp2} functions compute the base-2 exponential function of
\texttt{x} (2\(^{x}\)). A range error occurs if the magnitude of \texttt{x} is
too large.

The \texttt{expm1} functions compute the base-\emph{e} exponential function of
the argument, minus 1. A range error occurs in the magnitude of \texttt{x} is
too large.

\lstset{basicstyle=\scriptsize, numbers=left, captionpos=b, tabsize=4}
\begin{lstlisting}[caption=,language={C},
breaklines=true,xleftmargin=15pt,label=lst:]
#include <math.h>
float expf(float x); /* C99 */
double exp(double x);
long double expl(long double x); /* C99 */
float exp2f(float x); /* C99 */
double exp2(double x); /* C99 */
long double exp2l(long double x); /* C99 */
float expm1f(float x); /* C99 */
double expm1(double x); /* C99 */
long double expm1l(long double x); /* C99 */
\end{lstlisting}

\subsubsection{The \texttt{frexp}, \texttt{ldexp}, \texttt{modf},
\texttt{scalbn}, and \texttt{scalbln} functions}
These functions are heavily used in software floating-point emulators, but are
otherwise rarely directly called.

Inside the computer, each floating point number is represented by two parts:
\begin{itemize}
	\item The significand is either in the range [1/2, 1), or it equals zero.
	\item The exponent is an integer.
\end{itemize}

The value of a floating point number \(v\) is
$(v = significand * 2^{exponent})$.

The \texttt{frexp} functions break the argument floating point number
\texttt{value} into those two parts, the exponent and significand.  After
breaking it apart, it stores the exponent in the \texttt{int} object pointed to
by \texttt{ex}, and returns the significand.  In other words, the value
returned is a copy of the given floating point number but with an exponent
replaced by 0.  If \texttt{value} is zero, both parts of the result are zero.

The \texttt{ldexp} functions multiply a floating-point number by a integral
power of 2 and return the result.  In other words, it returns copy of the given
floating point number with the exponent increased by ex.  A range error may
occur.

The \texttt{modf} functions break the argument \texttt{value} into integer and
fraction parts, each of which has the same sign as the argument. They store the
integer part in the object pointed to by \texttt{*iptr} and return the fraction
part.  The \texttt{*iptr} is a floating-point type, rather than an ``int''
type, because it might be used to store an integer like 1 000 000 000 000 000
000 000 which is too big to fit in an int.

The \texttt{scalbn} and \texttt{scalbln} compute \texttt{x} \&times;
\texttt{FLT\_RADIX\texttt{n}}.
\texttt{FLT\_RADIX} is the base of the floating-point system; if it is 2, the
functions are equivalent to \texttt{ldexp}.

\lstset{basicstyle=\scriptsize, numbers=left, captionpos=b, tabsize=4}
\begin{lstlisting}[caption=,language={C},
breaklines=true,xleftmargin=15pt,label=lst:]
#include <math.h>
float frexpf(float value, int *ex); /* C99 */
double frexp(double value, int *ex);
long double frexpl(long double value, int *ex); /* C99 */
float ldexpf(float x, int ex); /* C99 */
double ldexp(double x, int ex);
long double ldexpl(long double x, int ex); /* C99 */
float modff(float value, float *iptr); /* C99 */
double modf(double value, double *iptr); 
long double modfl(long double value, long double *iptr); /* C99 */
float scalbnf(float x, int ex); /* C99 */
double scalbn(double x, int ex); /* C99 */
long double scalbnl(long double x, int ex); /* C99 */
float scalblnf(float x, long int ex); /* C99 */
double scalbln(double x, long int ex); /* C99 */
long double scalblnl(long double x, long int ex); /* C99 */
\end{lstlisting}

Most C floating point libraries also implement the IEEE754-recommended
nextafter(), nextUp( ), and nextDown( ) functions.

\subsubsection{The \texttt{log}, \texttt{log2}, \texttt{log1p}, and
\texttt{log10} functions}
The \texttt{log} functions compute the base-\emph{e} natural (\textbf{not}
common) logarithm of the argument and return the result. A domain error occurs
if the argument is negative. A range error may occur if the argument is zero.

The \texttt{log1p} functions compute the base-\emph{e} natural (\textbf{not}
common) logarithm of one plus the argument and return the result. A domain
error occurs if the argument is less than -1. A range error may occur if the
argument is -1.

The \texttt{log10} functions compute the common (base-10) logarithm of the
argument and return the result. A domain error occurs if the argument is
negative. A range error may occur if the argument is zero.

The \texttt{log2} functions compute the base-2 logarithm of the argument and
return the result. A domain error occurs if the argument is negative. A range
error may occur if the argument is zero.
\lstset{basicstyle=\scriptsize, numbers=left, captionpos=b, tabsize=4}
\begin{lstlisting}[caption=,language={C},
breaklines=true,xleftmargin=15pt,label=lst:]
#include <math.h>
float logf(float x); /* C99 */
double log(double x);
long double logl(long double x); /* C99 */
float log1pf(float x); /* C99 */
double log1p(double x); /* C99 */
long double log1pl(long double x); /* C99 */
float log10f(float x); /* C99 */
double log10(double x);
long double log10l(long double x); /* C99 */
float log2f(float x); /* C99 */
double log2(double x); /* C99 */
long double log2l(long double x); /* C99 */
\end{lstlisting}

\subsubsection{The \texttt{ilogb} and \texttt{logb} functions}
The \texttt{ilogb} functions extract the exponent of \texttt{x} as a signed int
value. If \texttt{x} is zero, they return the value \texttt{FP\_ILOGB0}; if
\texttt{x} is infinite, they return the value \texttt{INT\_MAX}; if \texttt{x}
is not-a-number they return the value \texttt{FP\_ILOGBNAN}; otherwise, they
are equivalent to calling the corresponding \texttt{logb} function and casting
the returned value to type \texttt{int}. A range error may occur if \texttt{x}
is zero. \texttt{FP\_ILOGB0} and \texttt{FP\_ILOGBNAN} are macros defined in
\texttt{math.h}; \texttt{INT\_MAX} is a macro defined in \texttt{limits.h}.

The \texttt{logb} functions extract the exponent of \texttt{x} as a signed
integer value in floating-point format. If \texttt{x} is subnormal, it is
treated as if it were normalized; thus, for positive finite \texttt{x},
$1 \leq x \times FLT\_RADIX^{-logb(x)} < FLT\_RADIX$.
\texttt{FLT\_RADIX} is the radix for floating-point numbers, defined in the
\texttt{float.h} header.
\lstset{basicstyle=\scriptsize, numbers=left, captionpos=b, tabsize=4}
\begin{lstlisting}[caption=,language={C},
breaklines=true,xleftmargin=15pt, label=lst:]
#include <math.h>
int ilogbf(float x); /* C99 */
int ilogb(double x); /* C99 */
int double ilogbl(long double x); /* C99 */
float logbf(float x); /* C99 */
double logb(double x); /* C99 */
long double logbl(long double x); /* C99 */
\end{lstlisting}

\subsection{Power functions}
\subsubsection{The \texttt{pow} functions}
The \texttt{pow} functions compute \texttt{x} raised to the power \texttt{y}
and return the result. A domain error occurs if \texttt{x} is negative and
\texttt{y} is not an integral value. A domain error occurs if the result cannot
be represented when \texttt{x} is zero and \texttt{y} is less than or equal to
zero. A range error may occur.
\lstset{basicstyle=\scriptsize, numbers=left, captionpos=b, tabsize=4}
\begin{lstlisting}[caption=,language={C},
breaklines=true,xleftmargin=15pt, label=lst:]
#include <math.h>
float powf(float x, float y); /* C99 */
double pow(double x, double y);
long double powl(long double x, long double y); /* C99 */
\end{lstlisting}

\subsubsection{The sqrt functions}
The sqrt functions compute the nonnegative square root of x and return the
result. A domain error occurs if the argument is negative.
\lstset{basicstyle=\scriptsize, numbers=left, captionpos=b, tabsize=4}
\begin{lstlisting}[caption=,language={C},
breaklines=true,xleftmargin=15pt, label=lst:]
#include <math.h>
float sqrtf(float x); /* C99 */
double sqrt(double x);
long double sqrtl(long double x); /* C99 */
\end{lstlisting}

\subsubsection{The cbrt functions}
The cbrt functions compute the cube root of x and return the result.
\lstset{basicstyle=\scriptsize, numbers=left, captionpos=b, tabsize=4}
\begin{lstlisting}[caption=,language={C},
breaklines=true,xleftmargin=15pt, label=lst:]
#include <math.h>
float cbrtf(float x); /* C99 */
double cbrt(double x); /* C99 */
long double cbrtl(long double x); /* C99 */
\end{lstlisting}

\subsubsection{The hypot functions}
The hypot functions compute the square root of the sums of the squares of x and
y, without overflow or underflow, and return the result.
\lstset{basicstyle=\scriptsize, numbers=left, captionpos=b, tabsize=4}
\begin{lstlisting}[caption=,language={C},
breaklines=true,xleftmargin=15pt, label=lst:]
#include <math.h>
float hypotf(float x, float y); /* C99 */
double hypot(double x, double y); /* C99 */
long double hypotl(long double x, long double y); /* C99 */
\end{lstlisting}

\subsection{Nearest integer, absolute value, and remainder functions}
\subsubsection{The ceil and floor functions}
The ceil functions compute the smallest integral value not less than x and
return the result; the floor functions compute the largest integral value not
greater than x and return the result.
\lstset{basicstyle=\scriptsize, numbers=left, captionpos=b, tabsize=4}
\begin{lstlisting}[caption=,language={C},
breaklines=true,xleftmargin=15pt, label=lst:]
#include <math.h>
float ceilf(float x); /* C99 */
double ceil(double x);
long double ceill(long double x); /* C99 */
float floorf(float x); /* C99 */
double floor(double x);
long double floorl(long double x); /* C99 */
\end{lstlisting}

\subsection{The fabs functions}
The fabs functions compute the absolute value of a floating-point number x and
return the result.
\lstset{basicstyle=\scriptsize, numbers=left, captionpos=b, tabsize=4}
\begin{lstlisting}[caption=,language={C},
breaklines=true,xleftmargin=15pt,label=lst:]
#include <math.h>
float fabsf(float x); /* C99 */
double fabs(double x); 
long double fabsl(long double x); /* C99 */
\end{lstlisting}

\subsection{The fmod functions}
The fmod functions compute the floating-point remainder of x/y and return the
value x - i * y, for some integer i such that, if y is nonzero, the result has
the same sign as x and magnitude less than the magnitude of y. If y is zero,
whether a domain error occurs or the fmod functions return zero is
implementation-defined.
\lstset{basicstyle=\scriptsize, numbers=left, captionpos=b, tabsize=4}
\begin{lstlisting}[caption=,language={C},
breaklines=true,xleftmargin=15pt,label=lst:]
#include <math.h>
float fmodf(float x, float y); /* C99 */
double fmod(double x, double y);
long double fmodl(long double x, long double y); /* C99 */
\end{lstlisting}

\subsection{The nearbyint, rint, lrint, and llrint functions}
The nearbyint functions round their argument to an integer value in
floating-point format, using the current rounding direction and without raising
the "inexact" floating-point exception.  The rint functions are similar to the
nearbyint functions except that they can raise the "inexact" floating-point
exception if the result differs in value from the argument.  The lrint and
llrint functions round their arguments to the nearest integer value according
to the current rounding direction. If the result is outside the range of values
of the return type, the numeric result is undefined and a range error may occur
if the magnitude of the argument is too large.
\lstset{basicstyle=\scriptsize, numbers=left, captionpos=b, tabsize=4}
\begin{lstlisting}[caption=,language={C},
breaklines=true,xleftmargin=15pt,label=lst:]
#include <math.h>
float nearbyintf(float x); /* C99 */
double nearbyint(double x); /* C99 */
long double nearbyintl(long double x); /* C99 */
float rintf(float x); /* C99 */
double rint(double x); /* C99 */
long double rintl(long double x); /* C99 */
long int lrintf(float x); /* C99 */
long int lrint(double x); /* C99 */
long int lrintl(long double x); /* C99 */
long long int llrintf(float x); /* C99 */
long long int llrint(double x); /* C99 */
long long int llrintl(long double x); /* C99 */
\end{lstlisting}

\subsection{The round, lround, and llround functions}
The round functions round the argument to the nearest integer value in
floating-point format, rounding halfway cases away from zero, regardless of the
current rounding direction.

The lround and llround functions round the argument to the nearest integer
value, rounding halfway cases away from zero, regardless of the current
rounding direction. If the result is outside the range of values of the return
type, the numeric result is undefined and a range error may occur if the
magnitude of the argument is too large.
\lstset{basicstyle=\scriptsize, numbers=left, captionpos=b, tabsize=4}
\begin{lstlisting}[caption=,language={C},
breaklines=true,xleftmargin=15pt,label=lst:]
#include <math.h>
float roundf(float x); /* C99 */
double round(double x); /* C99 */
long double roundl(long double x); /* C99 */
long int lroundf(float x); /* C99 */
long int lround(double x); /* C99 */
long int lroundl(long double x); /* C99 */
long long int llroundf(float x); /* C99 */
long long int llround(double x); /* C99 */
long long int llroundl(long double x); /* C99 */
\end{lstlisting}

\subsection{The trunc functions}
The functions round their argument to the integer value in floating-point
format that is nearest but no larger in magnitude than the argument.
\lstset{basicstyle=\scriptsize, numbers=left, captionpos=b, tabsize=4}
\begin{lstlisting}[caption=,language={C},
breaklines=true,xleftmargin=15pt,label=lst:]
#include <math.h>
float truncf(float x); /* C99 */
double trunc(double x); /* C99 */
long double truncl(long double x); /* C99 */
\end{lstlisting}

\subsection{The remainders functions}
The functions compute the remainder REM as defined by IEC 60559. The definition
reads, When \emph{y} != 0, the remainder \emph{r} = \emph{x} REM \emph{y} is
defined regardless of the rounding mode by the mathematical reduction \emph{r}
= \emph{x} --- \emph{ny}, where \emph{n} is the integer nearest the exact value
of \emph{x}/\emph{y}; whenever \textbar{}\emph{n} ---
\emph{x}/\emph{y}\textbar{} = $\frac{1}{2}$, then \emph{n} is even.  Thus, the
remainder is always exact. If \emph{r} = 0, its sign shall be that of \emph{x}.
This definition is applicable for all implementations.
\lstset{basicstyle=\scriptsize, numbers=left, captionpos=b, tabsize=4}
\begin{lstlisting}[caption=,language={C},
breaklines=true,xleftmargin=15pt,label=lst:]
#include <math.h>
float remainderf(float x, float y); /* C99 */
double remainder(double x, double y); /* C99 */
long double remainderl(long double x, long double y); /* C99 */
\end{lstlisting}

\subsection{The remquo code functions}
The remquo functions return the same remainder as the remainder functions. In
the object pointed to by quo, they store a value whose sign is the sign of x/y
and whose magnitude is congruent modulo 2n to the magnitude of the integral
quotient of x/y, where n is an implementation-defined integer greater than or
equal to 3.
\lstset{basicstyle=\scriptsize, numbers=left, captionpos=b, tabsize=4}
\begin{lstlisting}[caption=,language={C},
breaklines=true,xleftmargin=15pt,label=lst:]
#include <math.h>
float remquof(float x, float y, int *quo); /* C99 */
double remquo(double x, double y, int *quo); /* C99 */
long double remquol(long double x, long double y, int *quo); /* C99 */
\end{lstlisting}

\section{Error and gamma functions}
The functions compute the error function of the argument $2/(\pi^{frac{1}{2}}
\int 0^{x} e^{-t^{2}} dt)$; the functions compute the complimentary error
function of the argument (that is, 1 --- erf x). For the functions, a range
error may occur if the argument is too large.

The gamma functions compute the natural logarithm of the absolute value of the
gamma of the argument (that is,
log\(_{\emph{e}}\)\textbar{}\&Gamma;(x)\textbar{}). A range error may occur if
the argument is a negative integer or zero.

The tgamma functions compute the gamma of the argument (that is, \&Gamma;(x)).
A domain error occurs if the argument is a negative integer or if the result
cannot be represented when the argument is zero. A range error may occur.
\lstset{basicstyle=\scriptsize, numbers=left, captionpos=b, tabsize=4}
\begin{lstlisting}[caption=,language={C},
breaklines=true,xleftmargin=15pt,label=lst:]
#include <math.h>
float erff(float x); /* C99 */
double erf(double x); /* C99 */
long double erfl(long double x); /* C99 */
float erfcf(float x); /* C99 */
double erfc(double x); /* C99 */
long double erfcl(long double x); /* C99 */
float lgammaf(float x); /* C99 */
double lgamma(double x); /* C99 */
long double lgammal(long double x); /* C99 */
float tgammaf(float x); /* C99 */
double tgamma(double x); /* C99 */
long double tgammal(long double x); /* C99 */
\end{lstlisting}
