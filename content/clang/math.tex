\section{Simple math}
\newcounter{mathcnt}
\subsection{Operators and Assignments}
C has a wide range of operators that make simple math easy to handle. The list
of operators grouped into precedence levels is as follows:

\subsubsection{Primary expressions}
An identifier is a primary expression, provided that it has been declared as
designating an object (in which case it is an lvalue value that can be used as
the left side of an assignment expression) or a function (in which case it is a
function designator).

A constant is a primary expression. Its type depends on its form and value.

A string literal is a primary expression.

A parenthesized expression is a primary expression. Its type and value are
those of the unparenthesized expression.

\subsubsection{Postfix operators}
First, a primary expression is also a postfix expression. The following
expressions are also postfix expressions:

A postfix expression followed by a left square bracket (\texttt{[}), an
expression, and a right square bracket (\texttt{]}) constitutes an invocation
of the array subscript operator. One of the expressions shall have type
``pointer to object \emph{type}'' and the other shall have an integer type; the
result type is \emph{type}. Successive array subscript operators designate an
element of a multidimensional array.

A postfix expression followed by parentheses or an optional parenthesized
argument list indicates an invocation of the function call operator.

A postfix expression followed by a dot (\texttt{.}) followed by an identifier
selects a member from a structure or union; a postfix expression followed by an
arrow (\texttt{-\textgreater{}}) followed by an identifier selects a member
from a structure or union who is pointed to by the pointer on the left-hand
side of the expression.

A postfix expression followed by the increment or decrement operators
(\texttt{++} or \texttt{--}) indicates that the variable is to be incremented
or decremented as a side effect. The value of the expression is the value of
the postfix expression \emph{before} the increment or decrement.

\subsubsection{Unary operators}

First, a unary expression is a postfix expression. The following expressions
are all postfix expressions:

The increment or decrement operators followed by a unary expression is a unary
expression. The value of the expression is the value of the unary expression
\emph{after} the increment or decrement.

The following operators followed by a cast expression are unary expressions:
\begin{verbatim}
	Operator     Meaning
	========     =======
	   &         Address-of; value is the 
				 location of the operand
	   *         Contents-of; value is what 
	   *         is stored at the location
	   -         Negation
	   +         Value-of operator
	   !         Logical negation ( (!E) is 
				 equivalent to (0==E) )
	   ~         Bit-wise complement
\end{verbatim}

The keyword \texttt{sizeof} followed by a unary expression is a unary
expression. The value is the size of the type of the expression in bytes. The
expression is not evaluated.

The keyword \texttt{sizeof} followed by a parenthesized type name is a unary
expression. The value is the size of the type in bytes.

\subsubsection{Cast operators}
A cast expression is a unary expression.

A parenthesized type name followed by a cast expression is a cast expression.
The parenthesized type name has the effect of forcing the cast expression into
the type specified by the type name in parentheses. For arithmetic types, this
either does not change the value of the expression, or truncates the value of
the expression if the expression is an integer and the new type is smaller than
the previous type.

\subsubsection{Multiplicative and additive operators}
In C, simple math is very easy to handle. The following operators exist: +
(addition), --- (subtraction), * (multiplication), / (division), and \%
(modulus); You likely know all of them from your math classes --- except,
perhaps, modulus. It returns the \textbf{remainder} of a division (e.g. 5 \% 2
= 1). 

Care must be taken with the modulus, because it's not the equivalent of the
mathematical modulus: (-5) \% 2 is not 1, but -1. Division of integers will
return an integer, and the division of a negative integer by a positive integer
will round towards zero instead of rounding down (e.g. (-5) / 3 = -1 instead of
-2).

There is no inline operator to do the power (e.g. 5 \^{} 2 is \textbf{not} 25,
and 5 ** 2 is an error), but there is a power function.

The mathematical order of operations does apply. For example (2 + 3) * 2 = 10
while 2 + 3 * 2 = 8. The order of precedence in C is BFDMAS: Brackets (or
parentheses), Functions, Division or Multiplication (from left to right,
whichever comes first), Addition or Subtraction (also from left to right,
whichever comes first).

Actually, there are more operators than the above in C.

Assignment in C is simple. You declare the type of variable, the name of the
variable and what it's equal to. For example, int x = 0; double y = 0.0; char z
= 'a';

\lstset{basicstyle=\scriptsize, numbers=left, captionpos=b, tabsize=4}
\begin{lstlisting}[caption=Section \thesection listing \arabic{mathcnt},language={C},
breaklines=true,xleftmargin=15pt,label=lst:section\thesection listing\arabic{mathcnt}]
#include <stdio.h>

int main() {
	int i = 0, j = 0;
	
	/* while i is less than 5 AND j is less than 5, loop */
	while( (i < 5) && (j < 5) ) {
		/* postfix increment, i++
		 *     the value of i is read and then incremented */
		printf("i: %d\t", i++);
		
		/* prefix increment, ++j 
		 *     the value of j is incremented and then read */
		printf("j: %d\n", ++j);
	}
	
	printf("At the end they have both equal values:\ni: %d\tj: %d\n", i, j);
	
	return 0;
}
\end{lstlisting}
\stepcounter{mathcnt}

will display the following:

\begin{verbatim}
	i: 0    j: 1
	i: 1    j: 2
	i: 2    j: 3
	i: 3    j: 4
	i: 4    j: 5
	At the end they have both equal values:
	i: 5    j: 5
\end{verbatim}

\subsubsection{shift and rotate}
Shift functions are often used in low-level I/O hardware interfacing.  Shift
and rotate functions are heavily used in cryptography and software floating
point emulation.  Other than that, shifts can be used in place of division or
multiplication by a power of two.  Many processors have dedicated function
blocks to make these operations fast -- see Microprocessor Design/Shift and
Rotate Blocks. On processors which have such blocks, most C compilers compile
shift and rotate operators to a single assembly-language instruction -- see X86
Assembly/Shift and Rotate.

\paragraph{shift left}
The \texttt{\textless{}\textless{}} operator shifts the binary representation
to the left, dropping the most significant bits and appending it with zero
bits.  The result is equivalent to multiplying the integer by a power of two.

\paragraph{unsigned shift right}
The unsigned shift right operator, also sometimes called the logical right
shift operator.  It shifts the binary representation to the right, dropping the
least significant bits and prepending it with zeros.  The
\texttt{\textgreater{}\textgreater{}} operator is equivalent to division by a
power of two for unsigned integers.

\paragraph{signed shift right}
The signed shift right operator, also sometimes called the arithmetic right
shift operator.  It shifts the binary representation to the right, dropping the
least significant bit, but prepending it with copies of the original sign bit.
The \texttt{\textgreater{}\textgreater{}} operator is not equivalent to
division for signed integers.

In C, the behavior of the \texttt{\textgreater{}\textgreater{}} operator
depends on the data type it acts on.  Therefore, a signed and an unsigned right
shift looks exactly the same, but produces a different result in some cases.

\paragraph{rotate right}
Contrary to popular belief, it is possible to write C code that compiles down
to the ``rotate'' assembly language instruction (on CPUs that have such an
instruction).

Most compilers recognize this idiom:

\lstset{basicstyle=\scriptsize, numbers=left, captionpos=b, tabsize=4}
\begin{lstlisting}[caption=Section \thesection listing \arabic{mathcnt},language={C},
breaklines=true,xleftmargin=15pt,label=lst:section\thesection listing\arabic{mathcnt}]
	unsigned int x;
	unsigned int y;
	/* ... */
	y = (x >> shift) | (x << (32 - shift));
\end{lstlisting}
\stepcounter{mathcnt}

and compile it to a single 32 bit rotate instruction.

\paragraph{rotate left}
Most compilers recognize this idiom:

\lstset{basicstyle=\scriptsize, numbers=left, captionpos=b, tabsize=4}
\begin{lstlisting}[caption=Section \thesection listing \arabic{mathcnt},language={C},
breaklines=true,xleftmargin=15pt,label=lst:section\thesection listing\arabic{mathcnt}]
	 unsigned int x;
	 unsigned int y;
	 /* ... */
	 y = (x << shift) | (x >> (32 - shift));
\end{lstlisting}
\stepcounter{mathcnt}

and compile it to a single 32 bit rotate instruction.

On some systems, this may be \#defineed as a macro or defined as an inline
function called something like leftrotate32 or rotl32 in a header file
like bitops.h.

\subsubsection{Relational and equality operators}
The relational binary operators \texttt{\textless{}} (less than),
\texttt{\textgreater{}} (greater than), \texttt{\textless{}=} (less than or
equal), and \texttt{\textgreater{}=} (greater than or equal) operators return a
value of 1 if the result of the operation is true, 0 if false.

The equality binary operators \texttt{==} (equals) and \texttt{!=} (not equals)
operators are similar to the relational operators except that their precedence
is lower.

\subsubsection{Bitwise operators}
The bitwise operators are \texttt{\&} (and), \texttt{\^{}} (exclusive or) and
\texttt{\textbar{}} (inclusive or). The \texttt{\&} operator has higher
precedence than \texttt{\^{}}, which has higher precedence than
\texttt{\textbar{}}.

\subsubsection{Logical operators}
The logical operators are \texttt{\&\&} (and), and
\texttt{\textbar{}\textbar{}} (or). Both of these operators produce 1 if the
relationship is true and 0 for false. Both of these operators short-circuit; if
the result of the expression can be determined from the first operand, the
second is ignored.

\subsubsection{Conditional operators}
The ternary \texttt{?:} operator is the conditional operator. The expression
\texttt{(x ? y : z)} has the value of \texttt{y} if \texttt{x} is nonzero,
\texttt{z} otherwise.

\subsubsection{Assignment operators}
The assignment operators are \texttt{=}, \texttt{*=}, \texttt{/=},
\texttt{\%=}, \texttt{+=}, \texttt{-=}, \texttt{\textless{}\textless{}=},
\texttt{\textgreater{}\textgreater{}=}, \texttt{\&=}, \texttt{\^{}=}, and
\texttt{\textbar{}=} . The \texttt{=} operator stores the value of the right
operand into the location determined by the left operand, which must be an
\url{http://en.wikibooks.org.org/wiki/lvalue}{lvalue}. For the others,
\texttt{x op= y} is shorthand for \texttt{x = x op (y)} .

\subsubsection{Comma operator}
The operator with the least precedence is the comma operator. The value of the
expression \texttt{x, y} is the value of \texttt{y}, but \texttt{x} is
evaluated.
