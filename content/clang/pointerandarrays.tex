\section{Pointers and arrays}
\newcounter{pntcnt}
A \textbf{pointer} is a value that designates the address (i.e., the location
in memory), of some value. There are four fundamental things you need to know
about pointers:

\begin{itemize}
	\item How to declare them
	\item How to assign to them
	\item How to reference the value to which the pointer points (known as
\emph{dereferencing}) and
	\item How they relate to arrays
\end{itemize}

We'll also discuss the relationship of pointers with text strings and the more
advanced concept of function pointers.

Pointers are variables that hold a memory location -- the location of some
other variable.  One can access the value of the variable pointed to using the
dereferencing operator '*'. The initial value is usually set to NULL, and when
it is set, the address of the variable is passed in with the '\&' or address
operator.  Pointers can hold any data type, even functions.

The vast majority of arrays in C are simple lists, also called ``1 dimensional
arrays''. We will briefly cover multi-dimensional arrays in a later chapter.

\subsection{Declaring pointers}
Consider the following snippet of code which declares two pointers:
\lstset{basicstyle=\scriptsize, numbers=left, captionpos=b, tabsize=4}
\begin{lstlisting}[caption=Section \thesection listing \arabic{pntcnt},language={C},
breaklines=true,xleftmargin=15pt,label=lst:section\thesection listing\arabic{pntcnt}]
struct MyStruct {
	int   m_aNumber;
	float num2;
};

int *pJ2;
struct MyStruct *pAnItem;
\end{lstlisting}
\stepcounter{pntcnt}

The first four lines define a structure.  The next line declares a variable
which points to an \texttt{int}, and the bottom line declares a variable which
points to something with structure MyStruct. So to declare a variable as
something which points to some type, rather than contains some type, the
asterisk (\texttt{*}) is placed before the variable name. 

In the first of the following lines of code, \texttt{var1} is a pointer to a
long while \texttt{var2} is a long and not a pointer to a long. In the second
line \texttt{p3} is declared as a pointer to a pointer to an int.
\lstset{basicstyle=\scriptsize, numbers=left, captionpos=b, tabsize=4}
\begin{lstlisting}[caption=Section \thesection listing \arabic{pntcnt},language={C},
breaklines=true,xleftmargin=15pt,label=lst:section\thesection listing\arabic{pntcnt}]
long  *var1, var2;
int   **p3;
\end{lstlisting}
\stepcounter{pntcnt}

Pointer types are often used as parameters to function calls. The following
shows how to declare a function which uses a pointer as an argument. Since C
passes function arguments by value, in order to allow a function to modify a
value from the calling routine, a pointer to the value must be passed. Pointers
to structures are also used as function arguments even when nothing in the
struct will be modified in the function. This is done to avoid copying the
complete contents of the structure onto the stack. More about pointers as
function arguments later.
\lstset{basicstyle=\scriptsize, numbers=left, captionpos=b, tabsize=4}
\begin{lstlisting}[caption=Section \thesection listing \arabic{pntcnt},language={C},
breaklines=true,xleftmargin=15pt,label=lst:section\thesection listing\arabic{pntcnt}]
int MyFunction(struct MyStruct *pStruct);
\end{lstlisting}
\stepcounter{pntcnt}

\subsection{Assigning values to pointers}
So far we've discussed how to declare pointers. The process of assigning values
to pointers is next. To assign a pointer the address of a variable, the
\texttt{\&} or 'address of' operator is used. 
\lstset{basicstyle=\scriptsize, numbers=left, captionpos=b, tabsize=4}
\begin{lstlisting}[caption=Section \thesection listing \arabic{pntcnt},language={C},
breaklines=true,xleftmargin=15pt,label=lst:section\thesection listing\arabic{pntcnt}]
int myInt;
int *pPointer;
struct MyStruct dvorak;
struct MyStruct *pKeyboard;
 
pPointer = &myInt;
pKeyboard = &dvorak;
\end{lstlisting}
\stepcounter{pntcnt}

Here, pPointer will now reference myInt and pKeyboard will reference dvorak.

Pointers can also be assigned to reference dynamically allocated memory. The
malloc() and calloc() functions are often what are used to do this.
\lstset{basicstyle=\scriptsize, numbers=left, captionpos=b, tabsize=4}
\begin{lstlisting}[caption=Section \thesection listing \arabic{pntcnt},language={C},
breaklines=true,xleftmargin=15pt,label=lst:section\thesection listing\arabic{pntcnt}]
#include <stdlib.h>
/* ... */
struct MyStruct *pKeyboard;
/* ... */
pKeyboard = malloc(sizeof *pKeyboard);
\end{lstlisting}
\stepcounter{pntcnt}

The malloc function returns a pointer to dynamically allocated memory (or NULL
if unsuccessful). The size of this memory will be appropriately sized to
contain the MyStruct structure.

The following is an example showing one pointer being assigned to another and
of a pointer being assigned a return value from a function.
\lstset{basicstyle=\scriptsize, numbers=left, captionpos=b, tabsize=4}
\begin{lstlisting}[caption=Section \thesection listing \arabic{pntcnt},language={C},
breaklines=true,xleftmargin=15pt,label=lst:section\thesection listing\arabic{pntcnt}]
static struct MyStruct val1, val2, val3, val4;

struct MyStruct *ASillyFunction( int b ) {
	struct MyStruct *myReturn;
	
	if(b == 1) myReturn = &val1;
	else if(b==2) myReturn = &val2;
	else if(b==3) myReturn = &val3;
	else myReturn = &val4;
	
	return myReturn;
}

struct MyStruct *strPointer;
int *c, *d;
int j;

c = &j; /* pointer assigned using & operator */
d = c;  /* assign one pointer to another     */
strPointer = ASillyFunction( 3 ); /* pointer returned from a function. */
\end{lstlisting}
\stepcounter{pntcnt}

When returning a pointer from a function, do not return a pointer that points
to a value that is local to the function or that is a pointer to a function
argument. Pointers to local variables become invalid when the function exits.
In the above function, the value returned points to a static variable.
Returning a pointer to dynamically allocated memory is also valid.

\subsection{Pointer dereferencing}
To access a value to which a pointer points, the \texttt{*} operator is used.
Another operator, the \texttt{-\textgreater{}} operator is used in conjunction
with pointers to structures. Here's a short example.
\lstset{basicstyle=\scriptsize, numbers=left, captionpos=b, tabsize=4}
\begin{lstlisting}[caption=Section \thesection listing \arabic{pntcnt},language={C},
breaklines=true,xleftmargin=15pt,label=lst:section\thesection listing\arabic{pntcnt}]
int c, d;
int *pj;
struct MyStruct astruct;
struct MyStruct *bb;

c = 10;
pj = &c;  /* pj points to c */
d = *pj;  /* d is assigned the value to which pj points, 10 */
pj = &d;  /* now points to d */
*pj = 12; /* d is now 12 */

bb = &astruct;
(*bb).m_aNumber = 3; /* assigns 3 to the m_aNumber member of astruct */
bb->num2 = 44.3;     /* assigns 44.3 to the num2 member of astruct   */
*pj = bb->m_aNumber; /* eqivalent to d = astruct.m_aNumber;          */
\end{lstlisting}
\stepcounter{pntcnt}

The expression \texttt{bb-\textgreater{}mem} is entirely equivalent to
\texttt{(*bb).mem}. They both access the \texttt{mem} element of the structure
pointed to by \texttt{bb}. There is one more way of dereferencing a pointer,
which will be discussed in the following section.

When dereferencing a pointer that points to an invalid memory location, an
error often occurs which results in the program terminating. The error is often
reported as a segmentation error. A common cause of this is failure to
initialize a pointer before trying to dereference it.

C is known for giving you just enough rope to hang yourself, and pointer
dereferencing is a prime example. You are quite free to write code that
accesses memory outside that which you have explicity requested from the
system. And many times, that memory may appear as available to your program due
to the vagaries of system memory allocation. However, even if 99 executions
allow your program to run without fault, that 100th execution may be the time
when your ``memory pilfering'' is caught by the system and the program fails.
Be careful to ensure that your pointer offsets are within the bounds of
allocated memory!

The declaration \texttt{void *somePointer;} is used to declare a pointer of
some nonspecified type. You can assign a value to a void pointer, but you must
cast the variable to point to some specified type before you can dereference
it. Pointer arithmetic is also not valid with \texttt{void *} pointers.

\subsection{Pointers and Arrays}
Up to now, we've carefully been avoiding discussing arrays in the context of
pointers. The interaction of pointers and arrays can be confusing but here are
two fundamental statements about it:
\begin{itemize}
	\item A variable declared as an array of some type acts as a pointer to
that type. When used by itself, it points to the first element of the array.
	\item A pointer can be indexed like an array name.
\end{itemize}

The first case often is seen to occur when an array is passed as an argument to
a function. The function declares the parameter as a pointer, but the actual
argument may be the name of an array. The second case often occurs when
accessing dynamically allocated memory. Let's look at examples of each. In the
following code, the call to calloc() effectively allocates an array of struct
MyStruct items.
\lstset{basicstyle=\scriptsize, numbers=left, captionpos=b, tabsize=4}
\begin{lstlisting}[caption=Section \thesection listing \arabic{pntcnt},language={C},
breaklines=true,xleftmargin=15pt,label=lst:section\thesection listing\arabic{pntcnt}]
float KrazyFunction( struct MyStruct *parm1, int p1size, int bb ) { 
	int ix; //declaring an integer variable//
	for(ix=0; ix<p1size; ix++) {
		if(parm1[ix].m_aNumber == bb )
			return parm1[ix].num2;
	}
	return 0.0f;
}

/* ... */
struct MyStruct myArray[4];
#define MY_ARRAY_SIZE (sizeof(myArray)/sizeof(*myArray))
float v3;
struct MyStruct *secondArray;
int someSize;
int ix;
/* initialization of myArray ... */
v3 = KrazyFunction( myArray, MY_ARRAY_SIZE, 4 );
/* ... */
secondArray = calloc( someSize, sizeof *secondArray);
for(ix=0; ix<someSize; ix++) {
	secondArray[ix].m_aNumber = ix *2;
	secondArray[ix].num2 = .304 * ix * ix;
}
\end{lstlisting}
\stepcounter{pntcnt}

Pointers and array names can pretty much be used interchangeably. There are
exceptions. You cannot assign a new pointer value to an array name. The array
name will always point to the first element of the array. In the function
\texttt{KrazyFunction} above, you could however assign a new value to parm1, as
it is just a pointer to the first element of myArray. It is also valid for a
function to return a pointer to one of the array elements from an array passed
as an argument to a function. A function should never return a pointer to a
local variable, even though the compiler will probably not complain.

When declaring parameters to functions, declaring an array variable without a
size is equivalent to declaring a pointer. Often this is done to emphasize the
fact that the pointer variable will be used in a manner equivalent to an array. 
\lstset{basicstyle=\scriptsize, numbers=left, captionpos=b, tabsize=4}
\begin{lstlisting}[caption=Section \thesection listing \arabic{pntcnt},language={C},
breaklines=true,xleftmargin=15pt,label=lst:section\thesection listing\arabic{pntcnt}]
/* two equivalent function definitions */
int LittleFunction( int *paramN );
int LittleFunction( int paramN[] );
\end{lstlisting}
\stepcounter{pntcnt}

Now we're ready to discuss pointer arithmetic. You can add and subtract integer
values to/from pointers. If myArray is declared to be some type of array, the
expression \texttt{*(myArray+j)}, where j is an integer, is equivalent to
\texttt{myArray[j]}. So for instance in the above example where we had the
expression secondArray[i].num2, we could have written that as
\texttt{*(secondArray+i).num2} or more simply
\texttt{(secondArray+i)-\textgreater{}num2}.

Note that for addition and subtraction of integers and pointers, the value of
the pointer is not adjusted by the integer amount, but is adjusted by the
amount multiplied by the size (in bytes) of the type to which the pointer
refers. One pointer may also be subtracted from another, provided they point to
elements of the same array (or the position just beyond the end of the array).
If you have a pointer that points to an element of an array, the index of the
element is the result when the array name is subtracted from the pointer.
Here's an example.
\lstset{basicstyle=\scriptsize, numbers=left, captionpos=b, tabsize=4}
\begin{lstlisting}[caption=Section \thesection listing \arabic{pntcnt},language={C},
breaklines=true,xleftmargin=15pt,label=lst:section\thesection listing\arabic{pntcnt}]
struct MyStruct someArray[20];
struct MyStruct *p2;
int idx;

.
/* array initialization .. */
. 
for(p2 = someArray; p2 < someArray+20;  ++p2) {
	if (p2->num2 > testValue) break;
}
idx = p2 - someArray;
\end{lstlisting}
\stepcounter{pntcnt}

You may be wondering how pointers and multidimensional arrays interact. Lets
look at this a bit in detail. Suppose A is declared as a two dimensional array
of floats (\texttt{float A[D1][D2];}) and that pf is declared a pointer to a
float. If pf is initialized to point to A[0][0], then *(pf+1) is equivalent to
A[0][1] and *(pf+D2) is equivalent to A[1][0]. The elements of the array are
stored in row-major order.
\lstset{basicstyle=\scriptsize, numbers=left, captionpos=b, tabsize=4}
\begin{lstlisting}[caption=Section \thesection listing \arabic{pntcnt},language={C},
breaklines=true,xleftmargin=15pt,label=lst:section\thesection listing\arabic{pntcnt}]
float A[6][8];
float *pf;
pf = &A[0][0]; 
*(pf+1) = 1.3;   /* assigns 1.3 to A[0][1] */
*(pf+8) = 2.3;   /* assigns 2.3 to A[1][0] */
\end{lstlisting}
\stepcounter{pntcnt}
	
Let's look at a slightly different problem. We want to have a two dimensional
array, but we don't need to have all the rows the same length. What we do is
declare an array of pointers. The second line below declares A as an array of
pointers. Each pointer points to a float. Here's some applicable code:
\lstset{basicstyle=\scriptsize, numbers=left, captionpos=b, tabsize=4}
\begin{lstlisting}[caption=Section \thesection listing \arabic{pntcnt},language={C},
breaklines=true,xleftmargin=15pt,label=lst:section\thesection listing\arabic{pntcnt}]
float  linearA[30];
float *A[6];

A[0] = linearA;      /*  5 - 0 = 5 elements in row  */
A[1] = linearA + 5;  /* 11 - 5 = 6 elements in row  */
A[2] = linearA + 11; /* 15 - 11 = 4 elements in row */
A[3] = linearA + 15; /* 21 - 15 = 6 elements        */
A[4] = linearA + 21; /* 25 - 21 = 4 elements        */
A[5] = linearA + 25; /* 30 - 25 = 5 elements        */

A[3][2] = 3.66;  /* assigns 3.66 to linearA[17];     */
A[3][-3] = 1.44; /* refers to linearA[12];           
negative indices are sometimes useful. But avoid using 
them as much as possible. */
\end{lstlisting}
\stepcounter{pntcnt}

We also note here something curious about array indexing. Suppose myArray is an
array and idx is an integer value. The expression myArray[idx] is equivalent to
idx[myArray]. The first is equivalent to *(myArray+idx), and the second is
equivalent to *(idx+myArray). These turn out to be the same, since the addition
is commutative.

Pointers can be used with preincrement or post decrement, which is sometimes
done within a loop, as in the following example. The increment and decrement
applies to the pointer, not to the object to which the pointer refers.  In
other words, *pArray++ is equivalent to *(pArray++).
\lstset{basicstyle=\scriptsize, numbers=left, captionpos=b, tabsize=4}
\begin{lstlisting}[caption=Section \thesection listing \arabic{pntcnt},language={C},
breaklines=true,xleftmargin=15pt,label=lst:section\thesection listing\arabic{pntcnt}]
long myArray[20];
long *pArray;
int i;

/* Assign values to the entries of myArray */
pArray = myArray;
for(i=0; i<10; ++i) {
	*pArray++ = 5 + 3*i + 12*i*i;
	*pArray++ = 6 + 2*i + 7*i*i;
}
\end{lstlisting}
\stepcounter{pntcnt}

\subsection{Pointers in Function Arguments}
Often we need to invoke a function with an argument that is itself a pointer.
In many instances, the variable is itself a parameter for the current function
and may be a pointer to some type of structure. The ampersand character is not
needed in this circumstance to obtain a pointer value, as the variable is
itself a pointer. In the example below, the variable \texttt{pStruct}, a
pointer, is a parameter to function \texttt{FunctTwo,} and is passed as an
argument to \texttt{FunctOne}.  The second parameter to \texttt{FunctOne} is an
int. Since in function \texttt{FunctTwo, mValue} is a pointer to an int, the
pointer must first be dereferenced using the * operator, hence the second
argument in the call is \texttt{*mValue}. The third parameter to function
\texttt{FunctOne} is a pointer to a long. Since \texttt{pAA} is itself a
pointer to a long, no ampersand is needed when it is used as the third argument
to the function.
\lstset{basicstyle=\scriptsize, numbers=left, captionpos=b, tabsize=4}
\begin{lstlisting}[caption=Section \thesection listing \arabic{pntcnt},language={C},
breaklines=true,xleftmargin=15pt,label=lst:section\thesection listing\arabic{pntcnt}]
int FunctOne( struct SomeStruct *pValue, int iValue, long *lValue ) {
   /*  do some stuff ... */
   return 0;
}

int FunctTwo( struct someStruct *pStruct, int *mValue ) {
	int j;
	long  AnArray[25];
	long *pAA;
	 
	pAA = &AnArray[13];
	j = FunctOne( pStruct, *mValue, pAA );
	return j;
}
\end{lstlisting}
\stepcounter{pntcnt}

\subsection{Pointers and Text Strings}
Historically, text strings in C have been implemented as arrays of characters,
with the last byte in the string being a zero, or the null character
'\textbackslash{}0'. Most C implementations come with a standard library of
functions for manipulating strings. Many of the more commonly used functions
expect the strings to be null terminated strings of characters.  To use these
functions requires the inclusion of the standard C header file ``string.h''. 

A statically declared, initialized string would look similar to the following:
\lstset{basicstyle=\scriptsize, numbers=left, captionpos=b, tabsize=4}
\begin{lstlisting}[caption=Section \thesection listing \arabic{pntcnt},language={C},
breaklines=true,xleftmargin=15pt,label=lst:section\thesection listing\arabic{pntcnt}]
static const char *myFormat = "Total Amount Due: %d";
\end{lstlisting}
\stepcounter{pntcnt}

The variable \texttt{myFormat} can be viewed as an array of 21 characters.
There is an implied null character ('\textbackslash{}0') tacked on to the end
of the string after the 'd' as the 21st item in the array.  You can also
initialize the individual characters of the array as follows:
\lstset{basicstyle=\scriptsize, numbers=left, captionpos=b, tabsize=4}
\begin{lstlisting}[caption=Section \thesection listing \arabic{pntcnt},language={C},
breaklines=true,xleftmargin=15pt,label=lst:section\thesection listing\arabic{pntcnt}]
static const char myFlower[] = { 'P', 'e', 't', 'u', 'n', 'i', 'a', '\0' };
\end{lstlisting}
\stepcounter{pntcnt}

An initialized array of strings would typically be done as follows:
\lstset{basicstyle=\scriptsize, numbers=left, captionpos=b, tabsize=4}
\begin{lstlisting}[caption=Section \thesection listing \arabic{pntcnt},language={C},
breaklines=true,xleftmargin=15pt,label=lst:section\thesection listing\arabic{pntcnt}]
static const char *myColors[] = {
   "Red", "Orange", "Yellow", "Green", "Blue", "Violet" };
\end{lstlisting}
\stepcounter{pntcnt}

The initilization of an especially long string can be split across lines of
source code as follows.
\lstset{basicstyle=\scriptsize, numbers=left, captionpos=b, tabsize=4}
\begin{lstlisting}[caption=Section \thesection listing \arabic{pntcnt},language={C},
breaklines=true,xleftmargin=15pt,label=lst:section\thesection listing\arabic{pntcnt}]
static char *longString = "Hello. My name is Rudolph and I work as a reindeer "
  "around Christmas time up at the North Pole.  My boss is a really swell guy."
  " He likes to give everybody gifts.";
\end{lstlisting}
\stepcounter{pntcnt}

The library functions that are used with strings are discussed in a later chapter.

\subsubsection{Pointers to Functions}
C also allows you to create pointers to functions. Pointers to functions can
get rather messy. Declaring a typedef to a function pointer generally clarifies
the code. Here's an example that uses a function pointer, and a void * pointer
to implement what's known as a callback. The \texttt{DoSomethingNice} function
invokes a caller supplied function \texttt{TalkJive} with caller data. Note
that \texttt{DoSomethingNice} really doesn't know anything about what
\texttt{dataPointer}refers to.
\lstset{basicstyle=\scriptsize, numbers=left, captionpos=b, tabsize=4}
\begin{lstlisting}[caption=Section \thesection listing \arabic{pntcnt},language={C},
breaklines=true,xleftmargin=15pt,label=lst:section\thesection listing\arabic{pntcnt}]
typedef  int (*MyFunctionType)( int, void *);      /* a typedef for a function pointer */

#define THE_BIGGEST 100

int DoSomethingNice( int aVariable, MyFunctionType aFunction, void *dataPointer ) {
	int rv = 0;
	if (aVariable < THE_BIGGEST) {
		/* invoke function through function pointer (old style) */
		rv = (*aFunction)(aVariable, dataPointer );
	} else {
		  /* invoke function through function pointer (new style) */
		rv = aFunction(aVariable, dataPointer );
	};
	return rv;
}

struct sDataINeed {
	int colorSpec;
	char *phrase;
};
typedef struct sDataINeed DataINeed;

int TalkJive( int myNumber, void *someStuff ) {
	/* recast void * to pointer type specifically needed for this function */
	DataINeed *myData = someStuff;
	/* talk jive. */
	return 5;
}

static DataINeed  sillyStuff = { BLUE, "Whatcha talkin 'bout Willis?" };

/* ... */
DoSomethingNice( 41, &TalkJive,  &sillyStuff );
\end{lstlisting}
\stepcounter{pntcnt}

Some versions of C may not require an ampersand preceeding the
\texttt{TalkJive} argument in the \texttt{DoSomethingNice} call. Some
implementations may require specifically casting the argument to the
\texttt{MyFunctionType} type, even though the function signature exacly matches
that of the typedef.

Function pointers can be useful for implementing a form of polymorphism in C.
First one declares a structure having as elements function pointers for the
various operations to that can be specified polymorphically. A second base
object structure containing a pointer to the previous structure is also
declared. A class is defined by extending the second structure with the data
specific for the class, and static variable of the type of the first structure,
containing the addresses of the functions that are associated with the class.
This type of polymorphism is used in the standard library when file I/O
functions are called.

A similar mechanism can also be used for implementing a state machine in C. A
structure is defined which contains function pointers for handling events that
may occur within state, and for functions to be invoked upon entry to and exit
from the state. An instance of this structure corresponds to a state. Each
state is initialized with pointers to functions appropriate for the state. The
current state of the state machine is in effect a pointer to one of these
states. Changing the value of the current state pointer effectively changes the
current state. When some event occurs, the appropriate function is called
through a function pointer in the current state.

\subsubsection{Practical use of function pointer in C}
Function pointers are mainly used to reduce the complexity of switch statement.
Example With Switch statement: 
\lstset{basicstyle=\scriptsize, numbers=left, captionpos=b, tabsize=4}
\begin{lstlisting}[caption=Section \thesection listing \arabic{pntcnt},language={C},
breaklines=true,xleftmargin=15pt,label=lst:section\thesection listing\arabic{pntcnt}]
#include <stdio.h>
int add(int a, int b);
int sub(int a, int b);
int mul(int a, int b);
int div(int a, int b);

int main() {
	int i, result;
	int a=10;
	int b=5;
	scanf("Enter the value between 0 and 3 : %d",&i); 
	switch(i) {
		case 0: result = add(a,b); break;
		case 1: result = sub(a,b); break;
		case 2: result = mul(a,b); break;
		case 3: result = div(a,b); break;
	}
}

int add(int i, int j) {
	return (i+j);
}

int sub(int i, int j) {
	return (i-j);
}

int mul(int i, int j) {
	return (i*j);
}

int div(int i, int j) {
	return (i/j);
}
\end{lstlisting}
\stepcounter{pntcnt}

Without Switch Statement or Use of function pointer: Here you will see the
beautiful use of function pointer.
\lstset{basicstyle=\scriptsize, numbers=left, captionpos=b, tabsize=4}
\begin{lstlisting}[caption=Section \thesection listing \arabic{pntcnt},language={C},
breaklines=true,xleftmargin=15pt,label=lst:section\thesection listing\arabic{pntcnt}]
#include <stdio.h>

int add(int a, int b);
int sub(int a, int b);
int mul(int a, int b);
int div(int a, int b);
int (*oper[4])(int a, int b) = \{add, sub, mul, div\};

int main() {
	int i,result;
	int a=10;
	int b=5;
	scanf("Enter the value between 0 and 3 : %d",&i); 
	result = (*oper[i])(a,b);
}

int add(int i, int j) {
	return (i+j);
}

int sub(int i, int j) {
	return (i-j);
}

int mul(int i, int j) {
	return (i*j);
}

int div(int i, int j) {
	return (i/j);
}
\end{lstlisting}
\stepcounter{pntcnt}

\subsubsection{Examples of pointer constructs}
Below are some example constructs which may aid in creating your pointer.

\begin{tabular}{p{2.0cm} p{6.0cm}}
	int i;         & integer variable 'i'\\
	int *p;        & pointer 'p' to an integer\\
	int a[];       & array 'a' of integers\\
	int f();       & function 'f' with return value of type integer\\
	int **pp;      & pointer 'pp' to a pointer to an integer\\
	int (*pa)[];   & pointer 'pa' to an array of integer\\
	int (*pf)();   & pointer 'pf' to a function with returnvalue integer\\
	int *ap[];     & array 'ap' of pointers to an integer\\
	int *fp();     & function 'fp' which returns a pointer to an integer\\
	int ***ppp;    & pointer 'ppp' to a pointer to a pointer to an integer\\
	int (**ppa)[]; & pointer 'ppa' to a pointer to an array of integers\\
	int (**ppf)(); & pointer 'ppf' to a pointer to a function with return value of type integer\\
	int *(*pap)[]; & pointer 'pap' to an array of pointers to an integer\\
	int *(*pfp)(); & pointer 'pfp' to function with return value of type pointer to an integer\\
	int **app[];   & array of pointers 'app' that point to integer values\\
	int (*apa[])[];& array of pointers 'apa' to arrays of integers\\
	int (*apf[])();& array of pointers 'apf' to functions with return values of type integer\\
	int ***fpp();  & function 'fpp' which returns a pointer to a pointer to a pointer to an int\\
	int (*fpa())[];& function 'fpa' with return value of a pointer to array of integers\\
	int (*fpf())();& function 'fpf' with return value of a pointer to function which returns an integer
\end{tabular}

\subsubsection{sizeof}
The sizeof operator is often used to refer to the size of a static array
declared earlier in the same function.

To find the end of an array.
\lstset{basicstyle=\scriptsize, numbers=left, captionpos=b, tabsize=4}
\begin{lstlisting}[caption=Section \thesection listing \arabic{pntcnt},language={C},
breaklines=true,xleftmargin=15pt,label=lst:section\thesection listing\arabic{pntcnt}]
/* better.c - demonstrates one method of fixing the problem */
 
#include <stdio.h>
#include <string.h>
 
int main(int argc, char *argv[]) {
	char buffer[10];
	if (argc < 2) {
		fprintf(stderr, "USAGE: %s string\n", argv[0]);
		return 1;
	}
	strncpy(buffer, argv[1], sizeof(buffer));
	buffer[sizeof(buffer) - 1] = '\0';
	return 0;
}
\end{lstlisting}
\stepcounter{pntcnt}

To iterate over every element of an array, use
\lstset{basicstyle=\scriptsize, numbers=left, captionpos=b, tabsize=4}
\begin{lstlisting}[caption=Section \thesection listing \arabic{pntcnt},language={C},
breaklines=true,xleftmargin=15pt,label=lst:section\thesection listing\arabic{pntcnt}]
#define NUM_ELEM(x) (sizeof (x) / sizeof (*(x)))

for( i = 0; i < NUM_ELEM(array); i++ ) {
	/* do something with array[i] */
	;
}
\end{lstlisting}
\stepcounter{pntcnt}

Note that the sizeof operator only works on things defined earlier in the same
function.  The compiler replaces it with some fixed constant number.  In this
case, the buffer was declared as an array of 10 char's earlier in the same
function, and the compiler replaces sizeof(buffer) with the number 10 at
compile time (equivalent to us hard-coding 10 into the code in place of
sizeof(buffer)).  The information about the length of buffer is not actually
stored anywhere in memory (unless we keep track of it separately) and cannot be
programmatically obtained at run time from the array/pointer itself.

Often a function needs to know the size of an array it was given -- an array
defined in some other function.
For example,
\lstset{basicstyle=\scriptsize, numbers=left, captionpos=b, tabsize=4}
\begin{lstlisting}[caption=Section \thesection listing \arabic{pntcnt},language={C},
breaklines=true,xleftmargin=15pt,label=lst:section\thesection listing\arabic{pntcnt}]
/* broken.c - demonstrates a flaw */
 
#include <stdio.h>
#include <string.h>
#define NUM_ELEM(x) (sizeof (x) / sizeof (*(x)))
 
int sum( int input_array[] ){
	int sum_so_far = 0;
	// WON'T WORK -- input_array wasn't defined in this function.
	for( i = 0; i < NUM_ELEM(input_array); i++ ) {
	  sum_so_far += input_array[i];
	};
	return( sum_so_far );
}
 
int main(int argc, char *argv[]) {
	int left_array[] = { 1, 2, 3 };
	int right_array[] = { 10, 9, 8, 7, 6, 5, 4, 3, 2, 1 };
	int the_sum = sum( left_array );
	printf( "the sum of left_array is: %d", the_sum );
	the_sum = sum( right_array );
	printf( "the sum of right_array is: %d", the_sum );
	
	return 0;
}
\end{lstlisting}
\stepcounter{pntcnt}

Unfortunately, (in C and C++) the length of the array cannot be obtained from
an array passed in at run time, because (as mentioned above) the size of an
array is not stored anywhere.  The compiler always replaces sizeof with a
constant.  This sum() routine needs to handle more than just one constant
length of an array.

There are some common ways to work around this fact:
\begin{itemize}
	\item Write the function to require, for each array parameter, a ``length''
parameter (which has type ``size\_t''). (Typically we use sizeof at the point
where this function is called).
	\item Use of a convention, such as a
\url{http://en.wikipedia.org/wiki/null-terminated_string}{null-terminated
string} to mark the end of the array.
	\item Instead of passing raw arrays, pass a structure that includes the
length of the array (such as ``.length'') as well as the array (or a pointer to
the first element); similar to the string or vector classes in C++.
\end{itemize}

\lstset{basicstyle=\scriptsize, numbers=left, captionpos=b, tabsize=4}
\begin{lstlisting}[caption=Section \thesection listing \arabic{pntcnt},language={C},
breaklines=true,xleftmargin=15pt,label=lst:section\thesection listing\arabic{pntcnt}]
/* fixed.c - demonstrates one work-around */
 
#include <stdio.h>
#include <string.h>
#define NUM_ELEM(x) (sizeof (x) / sizeof (*(x)))
 
int sum( int input_array[], size_t length ){
	int sum_so_far = 0;
	int i;
	for( i = 0; i < length; i++ ) {
	  sum_so_far += input_array[i];
	};
	return( sum_so_far );
}
 
int main(int argc, char *argv[]) {
	int left_array[] = { 1, 2, 3, 4 };
	int right_array[] = { 10, 9, 8, 7, 6, 5, 4, 3, 2, 1 };
	// works here, because left_array is defined in this function
	int the_sum = sum( left_array, NUM_ELEM(left_array) ); 
	printf( "the sum of left_array is: %d", the_sum );
	// works here, because right_array is defined in this function
	the_sum = sum( right_array, NUM_ELEM(right_array) ); 
	printf( "the sum of right_array is: %d", the_sum );
	
	return 0;
}
\end{lstlisting}
\stepcounter{pntcnt}

It's worth mentioning that sizeof operator has two variations: \texttt{sizeof
(\emph{type})} (for instance: \texttt{sizeof (int)} or \texttt{sizeof (struct
some\_structure)}) and \texttt{sizeof \emph{expression}} (for instance:
\texttt{sizeof some\_variable.some\_field} or \texttt{sizeof 1}).
