\section{Memory management}
In C, you have already considered creating variables for use in the program.
You have created some arrays for use, but you may have already noticed some
limitations:
\begin{itemize}
	\item the size of the array must be known beforehand
	\item the size of the array cannot be changed in the duration of your
program
\end{itemize}

\emph{Dynamic memory allocation} in C is a way of circumventing these problems.

\subsection{Malloc}
\lstset{basicstyle=\scriptsize, numbers=left, captionpos=b, tabsize=4}
\begin{lstlisting}[caption=,language={C},
breaklines=true,xleftmargin=15pt,label=lst:]
#include <stdlib.h>
void *calloc(size_t nmemb, size_t size);
void free(void *ptr);
void *malloc(size_t size);
void *realloc(void *ptr, size_t size);
\end{lstlisting}

The C function \texttt{malloc} is the means of implementing dynamic memory
allocation. It is defined in stdlib.h or malloc.h, depending on what operating
system you may be using. Malloc.h contains only the definitions for the memory
allocation functions and not the rest of the other functions defined in
stdlib.h. Usually you will not need to be so specific in your program, and if
both are supported, you should use \textless{}stdlib.h\textgreater{}, since
that is ANSI C, and what we will use here.

The corresponding call to release allocated memory back to the operating system
is \texttt{free}.

When dynamically allocated memory is no longer needed, \texttt{free} should be
called to release it back to the memory pool. Overwriting a pointer that points
to dynamically allocated memory can result in that data becoming inaccessible.
If this happens frequently, eventually the operating system will no longer be
able to allocate more memory for the process. Once the process exits, the
operating system is able to free all dynamically allocated memory associated
with the process.

Let's look at how dynamic memory allocation can be used for arrays.

Normally when we wish to create an array we use a declaration such as
\lstset{basicstyle=\scriptsize, numbers=left, captionpos=b, tabsize=4}
\begin{lstlisting}[caption=,language={C},
breaklines=true,xleftmargin=15pt,label=lst:]
int array[10];
\end{lstlisting}

Recall \texttt{array} can be considered a pointer which we use as an array. We
specify the length of this array is 10 \texttt{int}s. After array\url{0}, nine
other integers have space to be stored consecutively.

Sometimes it is not known at the time the program is written how much memory
will be needed for some data. In this case we would want to dynamically
allocate required memory after the program has started executing.  To do this
we only need to declare a pointer, and invoke malloc when we wish to make space
for the elements in our array, \emph{or}, we can tell malloc to make space when
we first initialize the array. Either way is acceptable and useful.

We also need to know how much an int takes up in memory in order to make room
for it; fortunately this is not difficult, we can use C's builtin
\texttt{sizeof} operator. For example, if \texttt{sizeof(int)} yields 4, then
one \texttt{int} takes up 4 bytes. Naturally, \texttt{2*sizeof(int)} is how
much memory we need for 2 \texttt{int}s, and so on.

So how do we malloc an array of ten \texttt{int}s like before? If we wish to
declare and make room in one hit, we can simply say
\lstset{basicstyle=\scriptsize, numbers=left, captionpos=b, tabsize=4}
\begin{lstlisting}[caption=,language={C},
breaklines=true,xleftmargin=15pt,label=lst:]
int *array = malloc(10*sizeof(int));
\end{lstlisting}

We only need to declare the pointer; \texttt{malloc} gives us some space to
store the 10 \texttt{int}s, and returns the pointer to the first element, which
is assigned to that pointer.

\textbf{Important note!} \texttt{malloc} does \emph{not} initialize the array;
this means that the array may contain random or unexpected values! Like
creating arrays without dynamic allocation, the programmer must initialize the
array with sensible values before using it. Make sure you do so, too.
(\emph{See later the function \texttt{memset}} for a simple method.

It is not necessary to immediately call malloc after declaring a pointer for
the allocated memory.  Often a number of statements exist between the
declaration and the call to malloc, as follows:

\lstset{basicstyle=\scriptsize, numbers=left, captionpos=b, tabsize=4}
\begin{lstlisting}[caption=,language={C},
breaklines=true,xleftmargin=15pt,label=lst:]
int *array = NULL;
printf(``Hello World!!!'');
/* more statements */
array = malloc(10*sizeof(int)); /* delayed allocation */
/* use the array */
\end{lstlisting}

\subsubsection{Error checking}
When we want to use \texttt{malloc}, we have to be mindful that the pool of
memory available to the programmer is finite''. As such, we can conceivably run
out of memory! In this case, \texttt{malloc} will return \texttt{NULL}. In
order to stop the program crashing from having no more memory to use, one
should always check that malloc has not returned \texttt{NULL} before
attempting to use the memory; we can do this by 
\lstset{basicstyle=\scriptsize, numbers=left, captionpos=b, tabsize=4}
\begin{lstlisting}[caption=,language={C},
breaklines=true,xleftmargin=15pt,label=lst:]
int *pt = malloc(3 * sizeof(int));
if(pt == NULL) {
	fprintf(stderr, "Out of memory, exiting\n");
	exit(1);
}
\end{lstlisting}

Of course, suddenly quitting as in the above example is not always appropriate,
and depends on the problem you are trying to solve and the architecture you are
programming for. For example if program is a small, non critical application
that's running on a desktop quitting may be appropriate. However if the program
is some type of editor running on a desktop, you may want to give the operator
the option of saving his tediously entered information instead of just exiting
the program. A memory allocation failure in an embedded processor, such as
might be in a washing machine, could cause an automatic reset of the machine.
For this reason, many embedded systems designers avoid dynamic memory
allocation altogether.

\subsection{The \texttt{calloc} function}
The \texttt{calloc} function allocates space for an array of items and
initilizes the memory to zeros. The call \texttt{mArray = calloc( count,
sizeof(struct V))} allocates \texttt{count} objects, each of whose size is
sufficient to contain an instance of the structure \texttt{struct V}. The space
is initialized to all bits zero. The function returns either a pointer to the
allocated memory or, if the allocation fails, \texttt{NULL}.

\subsection{The \texttt{realloc} function}
\texttt{void * realloc ( void * ptr, size\_t size );}
The \texttt{realloc} function changes the size of the object pointed to by
\texttt{ptr} to the size specified by \texttt{size}. The contents of the object
shall be unchanged up to the lesser of the new and old sizes. If the new size
is larger, the value of the newly allocated portion of the object is
indeterminate. If \texttt{ptr} is a null pointer, the \texttt{realloc} function
behaves like the \texttt{malloc} function for the specified size. Otherwise, if
\texttt{ptr} does not match a pointer earlier returned by the \texttt{calloc},
\texttt{malloc}, or \texttt{realloc} function, or if the space has been
deallocated by a call to the \texttt{free} or \texttt{realloc} function, the
behavior is undefined. If the space cannot be allocated, the object pointed to
by \texttt{ptr} is unchanged. If \texttt{size} is zero and \texttt{ptr} is not
a null pointer, the object pointed to is freed. The \texttt{realloc} function
returns either a null pointer or a pointer to the possibly moved allocated
object.

\subsection{The \texttt{free} function}
Memory that has been allocated using \texttt{malloc}, \texttt{realloc}, or
\texttt{calloc} must be released back to the system memory pool once it is no
longer needed. This is done to avoid perpetually allocating more and more
memory, which could result in an eventual memory allocation failure. Memory
that is not released with \texttt{free} is however released when the current
program terminates on most operating systems. Calls to \texttt{free} are as in
the following example.
\lstset{basicstyle=\scriptsize, numbers=left, captionpos=b, tabsize=4}
\begin{lstlisting}[caption=,language={C},
breaklines=true,xleftmargin=15pt,label=lst:]
int *myStuff = malloc( 20 * sizeof(int)); 
if (myStuff != NULL) {
	/* more statements here */
	/* time to release myStuff */
	free( myStuff );
}
\end{lstlisting}

\subsubsection{free with recursive data structures}
It should be noted that \texttt{free} is neither intelligent nor recursive. The
following code that depends on the recursive application of free to the
internal variables of a struct does not work.
\lstset{basicstyle=\scriptsize, numbers=left, captionpos=b, tabsize=4}
\begin{lstlisting}[caption=,language={C},
breaklines=true,xleftmargin=15pt,label=lst:]
typedef struct BSTNode {
	int value; 
	struct BSTNode* left;
	struct BSTNode* right;
} BSTNode;
 
// Later: ... 
 
BSTNode* temp = (BSTNode*) calloc(1, sizeof(BSTNode));
temp->left = (BSTNode*) calloc(1, sizeof(BSTNode));
 
// Later: ... 
 
free(temp); // WRONG! don't do this!
\end{lstlisting}

The statement ``\texttt{free(temp);}'' will *not* free temp-\textgreater{}left,
causing a memory leak.

Because C does not have a garbage collector, C programmers are responsible for
making sure there is a \texttt{free()} exactly once for each time there is a
\texttt{malloc()}.  If a tree has been allocated one node at a time, then it
needs to be freed one node at a time.

\subsubsection{Don't free undefined pointers}
Furthermore, using \texttt{free} when the pointer in question was never
allocated in the first place often crashes or leads to mysterious bugs further
along.

To avoid this problem, always initialize pointers when they are declared.
Either use malloc() at the point they are declared (as in most examples in this
chapter), or set them to NULL when they are declared (as in the ``delayed
allocation'' example in this chapter).
