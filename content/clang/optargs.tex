\section{Opt Args}
A \emph{library} in C is a group of functions and declarations, exposed for use
by other programs. The library therefore consists of an \emph{interface}
expressed in a \texttt{.h} file (named the ``header'') and an
\emph{implementation} expressed in a \texttt{.c} file. This \texttt{.c} file
might be precompiled or otherwise inaccessible, or it might be available to the
programmer. (Note: Libraries may call functions in other libraries such as the
Standard C or math libraries to do various tasks.)

We're going to use as example a function to parse arguments from the command
line. Arguments on the command line could be by themselves:
\lstset{basicstyle=\scriptsize, numbers=left, captionpos=b, tabsize=4}
\begin{lstlisting}[caption=,language={C},
breaklines=true,xleftmargin=15pt,label=lst:]
-i
\end{lstlisting}

have an optional argument that is concatenated to the letter:
\lstset{basicstyle=\scriptsize, numbers=left, captionpos=b, tabsize=4}
\begin{lstlisting}[caption=,language={C},
breaklines=true,xleftmargin=15pt,label=lst:]
-ioptarg
\end{lstlisting}

or have the argument in a separate argv-element:
\lstset{basicstyle=\scriptsize, numbers=left, captionpos=b, tabsize=4}
\begin{lstlisting}[caption=,language={C},
breaklines=true,xleftmargin=15pt,label=lst:]
-i optarg
\end{lstlisting}

In order to parse all these types of arguments, we have written the following
``getopt.c'' file:
\lstset{basicstyle=\scriptsize, numbers=left, captionpos=b, tabsize=4}
\begin{lstlisting}[caption=,language={C},
breaklines=true,xleftmargin=15pt,label=lst:]
#include <stdio.h>              /* for fprintf() and EOF */
#include <string.h>             /* for strchr() */
#include "getopt.h"             /* consistency check */
 
/* variables */
int opterr = 1;                 /* getopt prints errors if this is on */
int optind = 1;                 /* token pointer */
int optopt;                     /* option character passed back to user */
char *optarg;                   /* flag argument (or value) */
 
/* function */
/* return option character, EOF if no more or ? if problem.
	The arguments to the function:
	argc, argv - the arguments to the main() function. An argument of "--"
	stops the processing.
	opts - a string containing the valid option characters.
	an option character followed by a colon (:) indicates that
	the option has a required argument.
*/
int getopt (int argc, char **argv, char *opts) {
	static int sp = 1;            /* character index into current token */
	register char *cp;            /* pointer into current token */
 
	if (sp == 1) {
		/* check for more flag-like tokens */
		if (optind >= argc || argv[optind][0] != '-' || argv[optind][1] == '\0')
			return EOF;
		else if (strcmp (argv[optind], "--") == 0) {
			optind++;
			return EOF;
		}
	}
 
	optopt = argv[optind][sp];
 
	if (optopt == ':' || (cp = strchr (opts, optopt)) == NULL) {
		if (opterr)
			fprintf (stderr, "%s: invalid option -- '%c'\n", argv[0], optopt);
 
		/* if no characters left in this token, move to next token */
		if (argv[optind][++sp] == '\0') {
			optind++;
			sp = 1;
		}
 
		return '?';
	}
 
	if (*++cp == ':') {
		/* if a value is expected, get it */
		if (argv[optind][sp + 1] != '\0')
			/* flag value is rest of current token */
			optarg = argv[optind++] + (sp + 1);
		else if (++optind >= argc)
		{
			if (opterr)
				fprintf (stderr, "%s: option requires an argument -- '%c'\n",
							argv[0], optopt);
			sp = 1;
			return '?';
		}
		else
		/* flag value is next token */
		optarg = argv[optind++];
		sp = 1;
	} else {
		/* set up to look at next char in token, next time */
		if (argv[optind][++sp] == '\0')
		{
			/* no more in current token, so setup next token */
			sp = 1;
			optind++;
		}
		optarg = 0;
	}
	return optopt;
} 
\end{lstlisting}

The interface would be the following ``getopt.h'' file:
\lstset{basicstyle=\scriptsize, numbers=left, captionpos=b, tabsize=4}
\begin{lstlisting}[caption=,language={C},
breaklines=true,xleftmargin=15pt,label=lst:]
#ifndef GETOPT_H
	#define GETOPT_H
 
	/* exported variables */
	extern int opterr, optind, optopt;
	extern char *optarg;
 
	/* exported function */
	int getopt(int, char **, char *);
#endif
\end{lstlisting}

At a minimum, a programmer has the interface file to figure out how to use a
library, although, in general, the library programmer also wrote documentation
on how to use the library. In the above case, the documentation should say that
the provided arguments
\textless{}code\textgreater{}**argv\textless{}code\textgreater{} and
\textless{}code\textgreater{}*opts\textless{}code\textgreater{} both shouldn't
be null pointers (or why would you be using the
\textless{}code\textgreater{}getopt\textless{}/code\textgreater{} function
anyway?). Specifically, it typically states what each parameter is for and what
return values can be expected in which conditions. Programmers that use a
library, are normally not interested in the implementation of the library --
unless the implementation has a bug, in which case he would want to complain
somehow.

Both the implementation of the getopts library, and programs that use the
library should state \textless{}code\textgreater{}\#include
``getopt.h''\textless{}/code\textgreater{}, in order to refer to the
corresponding interface. Now the library is ``linked'' to the program -- the
one that contains the main() function. The program may refer to dozens of
interfaces.

In some cases, just placing \textless{}code\textgreater{}\#include
``getopt.h''\textless{}/code\textgreater{} may appear correct but will still
fail to link properly. This indicates that the library is not installed
correctly, or there may be some additional configuration required. You will
have to check either the compiler's documentation or library's documentation on
how to resolve this issue. 
