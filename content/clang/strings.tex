\section{Strings}
\newcounter{stringcnt}
A \textbf{string} in C is merely an array of characters. The length of a string
is determined by a terminating null character: \texttt{'\textbackslash{}0'}.
So, a string with the contents, say, \texttt{"abc"} has four characters:
\texttt{'a'}, \texttt{'b'}, \texttt{'c'}, and the terminating null character.

The terminating null character has the value zero.

\subsection{The \texttt{\textless{}string.h\textgreater{}} Standard Header}
Because programmers find raw strings cumbersome to deal with, they wrote the
code in the \texttt{\textless{}string.h\textgreater{}} library. It represents
not a concerted design effort but rather the accretion of contributions made by
various authors over a span of years.  First, three types of functions exist in
the string library:

\begin{itemize}
	\item the \texttt{mem} functions manipulate sequences of arbitrary
characters without regard to the null character;
	\item the \texttt{str} functions manipulate null-terminated sequences of
characters;
	\item the \texttt{strn} functions manipulate sequences of non-null
characters.
\end{itemize}

\subsubsection{The more commonly-used string functions}
The nine most commonly used functions in the string library are:

\begin{itemize}
	\item \texttt{strcat} --- concatenate two strings
	\item \texttt{strchr} --- string scanning operation
	\item \texttt{strcmp} --- compare two strings
	\item \texttt{strcpy} --- copy a string
	\item \texttt{strlen} --- get string length
	\item \texttt{strncat} --- concatenate one string with part of another
	\item \texttt{strncmp} --- compare parts of two strings
	\item \texttt{strncpy} --- copy part of a string
	\item \texttt{strrchr} --- string scanning operation
\end{itemize}

\paragraph{The \texttt{strcat} function}
\texttt{char *strcat(char * restrict s1, const char * restrict s2);}

\emph{Some people recommend using} \texttt{strncat()} \emph{or}
\texttt{strlcat()} \emph{instead of strcat, in order to avoid buffer overflow.}

The \texttt{strcat()} function shall append a copy of the string pointed to by
\texttt{s2} (including the terminating null byte) to the end of the string
pointed to by \texttt{s1}. The initial byte of \texttt{s2} overwrites the null
byte at the end of \texttt{s1}. If copying takes place between objects that
overlap, the behavior is undefined. The function returns \texttt{s1}.

This function is used to attach one string to the end of another string. It is
imperative that the first string (\texttt{s1}) have the space needed to store
both strings.

Example:
\lstset{basicstyle=\scriptsize, numbers=left, captionpos=b, tabsize=4}
\begin{lstlisting}[caption=Section \thesection listing \arabic{stringcnt},language={C},
breaklines=true,xleftmargin=15pt,label=lst:section\thesection listing\arabic{stringcnt}]
#include <stdio.h>
#include <string.h>
...
static const char *colors[] = {"Red","Orange","Yellow","Green","Blue","Purple" };
static const char *widths[] = {"Thin","Medium","Thick","Bold" };
...
char penText[20];
...
int penColor = 3, penThickness = 2;
strcpy(penText, colors[penColor]);
strcat(penText, widths[penThickness]);
printf("My pen is %s\n", penText); // prints 'My pen is GreenThick'
\end{lstlisting}
\stepcounter{stringcnt}

Before calling \texttt{strcat()}, the destination must currently contain a null
terminated string or the first character must have been initialized with the
null character (e.g. \texttt{penText[0] = '\textbackslash{}0';}).

The following is a public-domain implementation of \texttt{strcat}:
\lstset{basicstyle=\scriptsize, numbers=left, captionpos=b, tabsize=4}
\begin{lstlisting}[caption=Section \thesection listing \arabic{stringcnt},language={C},
breaklines=true,xleftmargin=15pt,label=lst:section\thesection listing\arabic{stringcnt}]
#include <string.h>
/* strcat */
char *(strcat)(char *restrict s1, const char *restrict s2) {
	char *s = s1;
	/* Move s so that it points to the end of s1.  */
	while (*s != '\0')
	    s++;
	/* Copy the contents of s2 into the space at the end of s1.  */
	strcpy(s, s2);
	return s1;
}
\end{lstlisting}
\stepcounter{stringcnt}

\paragraph{The \texttt{strchr} function}
\texttt{char *strchr(const char *s, int c);}

The \texttt{strchr()} function shall locate the first occurrence of \texttt{c}
(converted to a \texttt{char}) in the string pointed to by \texttt{s}. The
terminating null byte is considered to be part of the string. The function
returns the location of the found character, or a null pointer if the character
was not found.

This function is used to find certain characters in strings.

At one point in history, this function was named \texttt{index}. The
\texttt{strchr} name, however cryptic, fits the general pattern for naming.

The following is a public-domain implementation of \texttt{strchr}:
\lstset{basicstyle=\scriptsize, numbers=left, captionpos=b, tabsize=4}
\begin{lstlisting}[caption=Section \thesection listing \arabic{stringcnt},language={C},
breaklines=true,xleftmargin=15pt,label=lst:section\thesection listing\arabic{stringcnt}]
#include <string.h>
/* strchr */
char *(strchr)(const char *s, int c) {
	/* Scan s for the character.  When this loop is finished,
	   s will either point to the end of the string or the
	   character we were looking for.  */
	while (*s != '\0' && *s != (char)c)
		s++;
	return ( (*s == c) ? (char *) s : NULL );
}
\end{lstlisting}
\stepcounter{stringcnt}

\paragraph{The \texttt{strcmp} function}
\texttt{int strcmp(const char *s1, const char *s2);}
A rudimentary form of string comparison is done with the strcmp() function. It
takes two strings as arguments and returns a value less than zero if the first
is lexographically less than the second, a value greater than zero if the first
is lexographically greater than the second, or zero if the two strings are
equal. The comparison is done by comparing the coded (ascii) value of the
chararacters, character by character.

This simple type of string comparison is nowadays generally considered
unacceptable when sorting lists of strings.  More advanced algorithms exist
that are capable of producing lists in dictionary sorted order. They can also
fix problems such as strcmp() considering the string ``Alpha2'' greater than
``Alpha12''. (In the previous example, ``Alpha2'' compares greater than
``Alpha12'' because '2' comes after '1' in the character set.) What we're
saying is, don't use this \texttt{strcmp()} alone for general string sorting in
any commercial or professional code.

The \texttt{strcmp()} function shall compare the string pointed to by
\texttt{s1} to the string pointed to by \texttt{s2}. The sign of a non-zero
return value shall be determined by the sign of the difference between the
values of the first pair of bytes (both interpreted as type \texttt{unsigned
char}) that differ in the strings being compared. Upon completion,
\texttt{strcmp()} shall return an integer greater than, equal to, or less than
0, if the string pointed to by \texttt{s1} is greater than, equal to, or less
than the string pointed to by \texttt{s2}, respectively.

Since comparing pointers by themselves is not practically useful unless one is
comparing pointers within the same array, this function lexically compares the
strings that two pointers point to.

This function is useful in comparisons, e.g.
\lstset{basicstyle=\scriptsize, numbers=left, captionpos=b, tabsize=4}
\begin{lstlisting}[caption=Section \thesection listing \arabic{stringcnt},language={C},
breaklines=true,xleftmargin=15pt,label=lst:section\thesection listing\arabic{stringcnt}]
if (strcmp(s, "whatever") == 0) /* do something */
    ;
\end{lstlisting}
\stepcounter{stringcnt}

The collating sequence used by \texttt{strcmp()} is equivalent to the machine's
native character set. The only guarantee about the order is that the digits
from \texttt{'0'} to \texttt{'9'} are in consecutive order.

The following is a public-domain implementation of \texttt{strcmp}:
\lstset{basicstyle=\scriptsize, numbers=left, captionpos=b, tabsize=4}
\begin{lstlisting}[caption=Section \thesection listing \arabic{stringcnt},language={C},
breaklines=true,xleftmargin=15pt,label=lst:section\thesection listing\arabic{stringcnt}]
#include <string.h>
/* strcmp */
int (strcmp)(const char *s1, const char *s2) {
	unsigned char uc1, uc2;
	/* Move s1 and s2 to the first differing characters 
	   in each string, or the ends of the strings if they
	   are identical.  */
	while (*s1 != '\0' && *s1 == *s2) {
		s1++;
		s2++;
	}
	/* Compare the characters as unsigned char and
	   return the difference.  */
	uc1 = (*(unsigned char *) s1);
	uc2 = (*(unsigned char *) s2);
	return ((uc1 < uc2) ? -1 : (uc1 > uc2));
}
\end{lstlisting}
\stepcounter{stringcnt}

\paragraph{The \texttt{strcpy} function}
\texttt{char *strcpy(char *restrict s1, const char *restrict s2);}

\emph{Some people recommend always using} \texttt{strncpy()} \emph{instead of
strcpy, to avoid buffer overflow.}

The \texttt{strcpy()} function shall copy the C string pointed to by
\texttt{s2} (including the terminating null byte) into the array pointed to by
\texttt{s1}. If copying takes place between objects that overlap, the behavior
is undefined. The function returns \texttt{s1}. There is no value used to
indicate an error: if the arguments to \texttt{strcpy()} are correct, and the
destination buffer is large enough, the function will never fail.

Example:
\lstset{basicstyle=\scriptsize, numbers=left, captionpos=b, tabsize=4}
\begin{lstlisting}[caption=Section \thesection listing \arabic{stringcnt},language={C},
breaklines=true,xleftmargin=15pt,label=lst:section\thesection listing\arabic{stringcnt}]
#include <stdio.h>
#include <string.h>
/* ... */
static const char *penType="round";
/* ... */
char penText[20];
/* ... */
strcpy(penText, penType);
\end{lstlisting}
\stepcounter{stringcnt}

Important: You must ensure that the destination buffer (\texttt{s1}) is able to
contain all the characters in the source array, including the terminating null
byte. Otherwise, \texttt{strcpy()} will overwrite memory past the end of the
buffer, causing a buffer overflow, which can cause the program to crash, or can
be exploited by hackers to compromise the security of the computer.

The following is a public-domain implementation of \texttt{strcpy}:
\lstset{basicstyle=\scriptsize, numbers=left, captionpos=b, tabsize=4}
\begin{lstlisting}[caption=Section \thesection listing \arabic{stringcnt},language={C},
breaklines=true,xleftmargin=15pt,label=lst:section\thesection listing\arabic{stringcnt}]
#include <string.h>
/* strcpy */
char *(strcpy)(char *restrict s1, const char *restrict s2) {
	char *dst = s1;
	const char *src = s2;
	/* Do the copying in a loop.  */
	while ((*dst++ = *src++) != '\0')
	    ;               /* The body of this loop is left empty. */
	/* Return the destination string.  */
	return s1;
}
\end{lstlisting}
\stepcounter{stringcnt}

\paragraph{The \texttt{strlen} function}
\texttt{size\_t strlen(const char *s);}

The \texttt{strlen()} function shall compute the number of bytes in the string
to which \texttt{s} points, not including the terminating null byte.  It
returns the number of bytes in the string. No value is used to indicate an
error.

The following is a public-domain implementation of \texttt{strlen}:

\lstset{basicstyle=\scriptsize, numbers=left, captionpos=b, tabsize=4}
\begin{lstlisting}[caption=Section \thesection listing \arabic{stringcnt},language={C},
breaklines=true,xleftmargin=15pt,label=lst:section\thesection listing\arabic{stringcnt}]
#include <string.h>
/* strlen */
size_t (strlen)(const char *s) {
	const char *p = s;
	/* Loop over the data in s.  */
	while (*p != '\0')
		p++;
	return (size_t)(p - s);
}
\end{lstlisting}
\stepcounter{stringcnt}

\paragraph{The \texttt{strncat} function}
\texttt{char *strncat(char *restrict s1, const char *restrict s2, size\_t n);}

The \texttt{strncat()} function shall append not more than \texttt{n} bytes (a
null byte and bytes that follow it are not appended) from the array pointed to
by \texttt{s2} to the end of the string pointed to by \texttt{s1}. The initial
byte of \texttt{s2} overwrites the null byte at the end of \texttt{s1}. A
terminating null byte is always appended to the result. If copying takes place
between objects that overlap, the behavior is undefined. The function returns
\texttt{s1}.

The following is a public-domain implementation of \texttt{strncat}:
\lstset{basicstyle=\scriptsize, numbers=left, captionpos=b, tabsize=4}
\begin{lstlisting}[caption=Section \thesection listing \arabic{stringcnt},language={C},
breaklines=true,xleftmargin=15pt,label=lst:section\thesection listing\arabic{stringcnt}]
#include <string.h>
/* strncat */
char *(strncat)(char *restrict s1, const char *restrict s2, size_t n) {
	char *s = s1;
	/* Loop over the data in s1.  */
	while (*s != '\0')
		s++;
	/* s now points to s1's trailing null character, now copy
	   up to n bytes from s1 into s stopping if a null character
	   is encountered in s2.
	   It is not safe to use strncpy here since it copies EXACTLY n
	   characters, NULL padding if necessary.  */
	while (n != 0 && (*s = *s2++) != '\0') {
		n--;
		s++;
	}
	if (*s != '\0')
		*s = '\0';
	return s1;
}
\end{lstlisting}
\stepcounter{stringcnt}

\paragraph{The \texttt{strncmp} function}
\texttt{int strncmp(const char *s1, const char *s2, size\_t n);}

The \texttt{strncmp()} function shall compare not more than \texttt{n} bytes
(bytes that follow a null byte are not compared) from the array pointed to by
\texttt{s1} to the array pointed to by \texttt{s2}. The sign of a non-zero
return value is determined by the sign of the difference between the values of
the first pair of bytes (both interpreted as type \texttt{unsigned char}) that
differ in the strings being compared. See \texttt{strcmp} for an explanation of
the return value.

This function is useful in comparisons, as the \texttt{strcmp} function is.

The following is a public-domain implementation of \texttt{strncmp}:

\lstset{basicstyle=\scriptsize, numbers=left, captionpos=b, tabsize=4}
\begin{lstlisting}[caption=Section \thesection listing \arabic{stringcnt},language={C},
breaklines=true,xleftmargin=15pt,label=lst:section\thesection listing\arabic{stringcnt}]
#include <string.h>
/* strncmp */
int (strncmp)(const char *s1, const char *s2, size_t n) {
	unsigned char uc1, uc2;
	/* Nothing to compare?  Return zero.  */
	if (n == 0)
		return 0;
	/* Loop, comparing bytes.  */
	while (n-- > 0 && *s1 == *s2) {
	    /* If we've run out of bytes or hit a null, return zero
	        since we already know *s1 == *s2.  */
	    if (n == 0 || *s1 == '\0')
	      	return 0;
	    s1++;
	    s2++;
	}
	uc1 = (*(unsigned char *) s1);
	uc2 = (*(unsigned char *) s2);
	return ((uc1 < uc2) ? -1 : (uc1 > uc2));
}
\end{lstlisting}
\stepcounter{stringcnt}

\paragraph{The \texttt{strncpy} function}
\texttt{char *strncpy(char *restrict s1, const char *restrict s2, size\_t n);}

The \texttt{strncpy()} function shall copy not more than \texttt{n} bytes
(bytes that follow a null byte are not copied) from the array pointed to by
\texttt{s2} to the array pointed to by \texttt{s1}. If copying takes place
between objects that overlap, the behavior is undefined. If the array pointed
to by \texttt{s2} is a string that is shorter than \texttt{n} bytes, null bytes
shall be appended to the copy in the array pointed to by \texttt{s1}, until
\texttt{n} bytes in all are written. The function shall return s1; no return
value is reserved to indicate an error.

It is possible that the function will \textbf{not} return a null-terminated
string, which happens if the \texttt{s2} string is longer than \texttt{n}
bytes.

The following is a public-domain version of \texttt{strncpy}:
\lstset{basicstyle=\scriptsize, numbers=left, captionpos=b, tabsize=4}
\begin{lstlisting}[caption=Section \thesection listing \arabic{stringcnt},language={C},
breaklines=true,xleftmargin=15pt,label=lst:section\thesection listing\arabic{stringcnt}]
#include <string.h>
/* strncpy */
char *(strncpy)(char *restrict s1, const char *restrict s2, size_t n) {
	char *dst = s1;
	const char *src = s2;
	/* Copy bytes, one at a time.  */
	while (n > 0) {
		n--;
		if ((*dst++ = *src++) == '\0') {
			/* If we get here, we found a null character at the end
			   of s2, so use memset to put null bytes at the end of
			   s1.  */
			memset(dst, '\0', n);
			break;
		}
	}
	return s1;
}
\end{lstlisting}
\stepcounter{stringcnt}

\paragraph{The \texttt{strrchr} function}
\texttt{char *strrchr(const char *s, int c);}

\texttt{strrchr} is similar to \texttt{strchr}, except the string is searched
right to left.

The \texttt{strrchr()} function shall locate the last occurrence of \texttt{c}
(converted to a \texttt{char}) in the string pointed to by \texttt{s}. The
terminating null byte is considered to be part of the string. Its return value
is similar to \texttt{strchr}'s return value.

At one point in history, this function was named \texttt{rindex}. The
\texttt{strrchr} name, however cryptic, fits the general pattern for naming.

The following is a public-domain implementation of \texttt{strrchr}:
\lstset{basicstyle=\scriptsize, numbers=left, captionpos=b, tabsize=4}
\begin{lstlisting}[caption=Section \thesection listing \arabic{stringcnt},language={C},
breaklines=true,xleftmargin=15pt,label=lst:section\thesection listing\arabic{stringcnt}]
#include <string.h>
/* strrchr */
char *(strrchr)(const char *s, int c) {
	const char *last = NULL;
	/* If the character we're looking for is the terminating null,
	   we just need to look for that character as there's only one
	   of them in the string.  */
	if (c == '\0')
		return strchr(s, c);
	/* Loop through, finding the last match before hitting NULL.  */
	while ((s = strchr(s, c)) != NULL) {
		last = s;
		s++;
	}
	return (char *) last;
}
\end{lstlisting}
\stepcounter{stringcnt}

\subsubsection{The less commonly-used string functions}
The less-used functions are:

\begin{itemize}
	\item \texttt{memchr} --- Find a byte in memory
	\item \texttt{memcmp} --- Compare bytes in memory
	\item \texttt{memcpy} --- Copy bytes in memory
	\item \texttt{memmove} --- Copy bytes in memory with overlapping areas
	\item \texttt{memset} --- Set bytes in memory
	\item \texttt{strcoll} --- Compare bytes according to a locale-specific collating sequence
	\item \texttt{strcspn} --- Get the length of a complementary substring
	\item \texttt{strerror} --- Get error message
	\item \texttt{strpbrk} --- Scan a string for a byte
	\item \texttt{strspn} --- Get the length of a substring
	\item \texttt{strstr} --- Find a substring
	\item \texttt{strtok} --- Split a string into tokens
	\item \texttt{strxfrm} --- Transform string
\end{itemize}

\paragraph{Copying functions}
\subparagraph{The \texttt{memcpy} function}
\texttt{void *memcpy(void * restrict s1, const void * restrict s2, size\_t n);}

The \texttt{memcpy()} function shall copy \texttt{n} bytes from the object
pointed to by \texttt{s2} into the object pointed to by \texttt{s1}. If copying
takes place between objects that overlap, the behavior is undefined. The
function returns \texttt{s1}.

Because the function does not have to worry about overlap, it can do the
simplest copy it can.

The following is a public-domain implementation of \texttt{memcpy}:
\lstset{basicstyle=\scriptsize, numbers=left, captionpos=b, tabsize=4}
\begin{lstlisting}[caption=Section \thesection listing \arabic{stringcnt},language={C},
breaklines=true,xleftmargin=15pt,label=lst:section\thesection listing\arabic{stringcnt}]
#include <string.h>
/* memcpy */
void *(memcpy)(void * restrict s1, const void * restrict s2, size_t n) {
	char *dst = s1;
	const char *src = s2;
	/* Loop and copy.  */
	while (n-- != 0)
		*dst++ = *src++;
	return s1;
}
\end{lstlisting}
\stepcounter{stringcnt}

\subparagraph{The \texttt{memmove} function}
\texttt{void *memmove(void *s1, const void *s2, size\_t n);}

The \texttt{memmove()} function shall copy \texttt{n} bytes from the object
pointed to by \texttt{s2} into the object pointed to by \texttt{s1}. Copying
takes place as if the \texttt{n} bytes from the object pointed to by
\texttt{s2} are first copied into a temporary array of \texttt{n} bytes that
does not overlap the objects pointed to by \texttt{s1} and \texttt{s2}, and
then the \texttt{n} bytes from the temporary array are copied into the object
pointed to by \texttt{s1}. The function returns the value of \texttt{s1}.

The easy way to implement this without using a temporary array is to check for
a condition that would prevent an ascending copy, and if found, do a descending
copy.

The following is a public-domain, though not completely portable,
implementation of \texttt{memmove}:
\lstset{basicstyle=\scriptsize, numbers=left, captionpos=b, tabsize=4}
\begin{lstlisting}[caption=Section \thesection listing \arabic{stringcnt},language={C},
breaklines=true,xleftmargin=15pt,label=lst:section\thesection listing\arabic{stringcnt}]
#include <string.h>
/* memmove */
void *(memmove)(void *s1, const void *s2, size_t n) {
   /* note: these don't have to point to unsigned chars */
   char *p1 = s1;
   const char *p2 = s2;
   /* test for overlap that prevents an ascending copy */
   if (p2 < p1 && p1 < p2 + n) {
	   /* do a descending copy */
	   p2 += n;
	   p1 += n;
	   while (n-- != 0) 
		   *--p1 = *--p2;
   } else 
	   while (n-- != 0) 
		   *p1++ = *p2++;
   return s1; 
}
\end{lstlisting}
\stepcounter{stringcnt}

\subsubsection{Comparison functions}

\paragraph{The \texttt{memcmp} function}
\texttt{int memcmp(const void *s1, const void *s2, size\_t n);}

The \texttt{memcmp()} function shall compare the first \texttt{n} bytes (each
interpreted as \texttt{unsigned char}) of the object pointed to by \texttt{s1}
to the first \texttt{n} bytes of the object pointed to by \texttt{s2}. The sign
of a non-zero return value shall be determined by the sign of the difference
between the values of the first pair of bytes (both interpreted as type
\texttt{unsigned char}) that differ in the objects being compared.

The following is a public-domain implementation of \texttt{memcmp}:
\lstset{basicstyle=\scriptsize, numbers=left, captionpos=b, tabsize=4}
\begin{lstlisting}[caption=Section \thesection listing \arabic{stringcnt},language={C},
breaklines=true,xleftmargin=15pt,label=lst:section\thesection listing\arabic{stringcnt}]
#include <string.h>
/* memcmp */
int (memcmp)(const void *s1, const void *s2, size_t n) {
	const unsigned char *us1 = (const unsigned char *) s1;
	const unsigned char *us2 = (const unsigned char *) s2;
	while (n-- != 0) {
		if (*us1 != *us2)
			return (*us1 < *us2) ? -1 : +1;
		us1++;
		us2++;
	}
	return 0;
}
\end{lstlisting}
\stepcounter{stringcnt}

\paragraph{The \texttt{strcoll} and \texttt{strxfrm} functions}
\texttt{int strcoll(const char *s1, const char *s2);}

\texttt{size\_t strxfrm(char *s1, const char *s2, size\_t n);}

The ANSI C Standard specifies two locale-specific comparison functions.

The \texttt{strcoll} function compares the string pointed to by \texttt{s1} to
the string pointed to by \texttt{s2}, both interpreted as appropriate to the
\texttt{LC\_COLLATE} category of the current locale. The return value is
similar to \texttt{strcmp}.

The \texttt{strxfrm} function transforms the string pointed to by \texttt{s2}
and places the resulting string into the array pointed to by \texttt{s1}. The
transformation is such that if the \texttt{strcmp} function is applied to the
two transformed strings, it returns a value greater than, equal to, or less
than zero, corresponding to the result of the \texttt{strcoll} function applied
to the same two original strings. No more than \texttt{n} characters are placed
into the resulting array pointed to by \texttt{s1}, including the terminating
null character. If \texttt{n} is zero, \texttt{s1} is permitted to be a null
pointer. If copying takes place between objects that overlap, the behavior is
undefined. The function returns the length of the transformed string.

These functions are rarely used and nontrivial to code, so there is no code for
this section.

\subsubsection{Search functions}

\paragraph{The \texttt{memchr} function}
\texttt{void *memchr(const void *s, int c, size\_t n);}

The \texttt{memchr()} function shall locate the first occurrence of \texttt{c}
(converted to an \texttt{unsigned char}) in the initial \texttt{n} bytes (each
interpreted as \texttt{unsigned char}) of the object pointed to by \texttt{s}.
If \texttt{c} is not found, \texttt{memchr} returns a null pointer.

The following is a public-domain implementation of \texttt{memchr}:
\lstset{basicstyle=\scriptsize, numbers=left, captionpos=b, tabsize=4}
\begin{lstlisting}[caption=Section \thesection listing \arabic{stringcnt},language={C},
breaklines=true,xleftmargin=15pt,label=lst:section\thesection listing\arabic{stringcnt}]
#include <string.h>
/* memchr */
void *(memchr)(const void *s, int c, size_t n) {
	const unsigned char *src = s;
	unsigned char uc = c;
	while (n-- != 0) {
		if (*src == uc)
			return (void *) src;
		src++;
	}
	return NULL;
}
\end{lstlisting}
\stepcounter{stringcnt}

\paragraph{The \texttt{strcspn}, \texttt{strpbrk}, and \texttt{strspn} functions}
\texttt{size\_t strcspn(const char *s1, const char *s2);}

\texttt{char *strpbrk(const char *s1, const char *s2);}

\texttt{size\_t strspn(const char *s1, const char *s2);}

The \texttt{strcspn} function computes the length of the maximum initial
segment of the string pointed to by \texttt{s1} which consists entirely of
characters \textbf{not} from the string pointed to by \texttt{s2}.

The \texttt{strpbrk} function locates the first occurrence in the string
pointed to by \texttt{s1} of any character from the string pointed to by
\texttt{s2}, returning a pointer to that character or a null pointer if not
found.

The \texttt{strspn} function computes the length of the maximum initial segment
of the string pointed to by \texttt{s1} which consists entirely of characters
from the string pointed to by \texttt{s2}.

All of these functions are similar except in the test and the return value.

The following are public-domain implementations of \texttt{strcspn},
\texttt{strpbrk}, and \texttt{strspn}:
\lstset{basicstyle=\scriptsize, numbers=left, captionpos=b, tabsize=4}
\begin{lstlisting}[caption=Section \thesection listing \arabic{stringcnt},language={C},
breaklines=true,xleftmargin=15pt,label=lst:section\thesection listing\arabic{stringcnt}]
#include <string.h>
/* strcspn */
size_t (strcspn)(const char *s1, const char *s2) {
	const char *sc1;
	for (sc1 = s1; *sc1 != '\0'; sc1++)
		if (strchr(s2, *sc1) != NULL)
			return (sc1 - s1);
	return sc1 - s1;			/* terminating nulls match */
}
\end{lstlisting}
\stepcounter{stringcnt}

\lstset{basicstyle=\scriptsize, numbers=left, captionpos=b, tabsize=4}
\begin{lstlisting}[caption=Section \thesection listing \arabic{stringcnt},language={C},
breaklines=true,xleftmargin=15pt,label=lst:section\thesection listing\arabic{stringcnt}]
#include <string.h>
/* strpbrk */
char *(strpbrk)(const char *s1, const char *s2) {
	const char *sc1;
	for (sc1 = s1; *sc1 != '\0'; sc1++)
		if (strchr(s2, *sc1) != NULL)
			return (char *)sc1;
	return NULL;				/* terminating nulls match */
}
\end{lstlisting}
\stepcounter{stringcnt}

\lstset{basicstyle=\scriptsize, numbers=left, captionpos=b, tabsize=4}
\begin{lstlisting}[caption=Section \thesection listing \arabic{stringcnt},language={C},
breaklines=true,xleftmargin=15pt,label=lst:section\thesection listing\arabic{stringcnt}]
#include <string.h>
/* strspn */
size_t (strspn)(const char *s1, const char *s2) {
	const char *sc1;
	for (sc1 = s1; *sc1 != '\0'; sc1++)
		if (strchr(s2, *sc1) == NULL)
			return (sc1 - s1);
	return sc1 - s1;			/* terminating nulls don't match */
}
\end{lstlisting}
\stepcounter{stringcnt}

\paragraph{The \texttt{strstr} function}
\texttt{char *strstr(const char *haystack, const char *needle);}

The \texttt{strstr()} function shall locate the first occurrence in the string
pointed to by \texttt{haystack} of the sequence of bytes (excluding the
terminating null byte) in the string pointed to by \texttt{needle}. The
function returns the pointer to the matching string in \texttt{haystack} or a
null pointer if a match is not found. If \texttt{needle} is an empty string,
the function returns \texttt{haystack}.

The following is a public-domain implementation of \texttt{strstr}:
\lstset{basicstyle=\scriptsize, numbers=left, captionpos=b, tabsize=4}
\begin{lstlisting}[caption=Section \thesection listing \arabic{stringcnt},language={C},
breaklines=true,xleftmargin=15pt,label=lst:section\thesection listing\arabic{stringcnt}]
#include <string.h>
/* strstr */
char *(strstr)(const char *haystack, const char *needle) {
	size_t needlelen;
	/* Check for the null needle case.  */
	if (*needle == '\0')
		return (char *) haystack;
	needlelen = strlen(needle);
	for (; (haystack = strchr(haystack, *needle)) != NULL; haystack++)
		if (strncmp(haystack, needle, needlelen) == 0)
			return (char *) haystack;
	return NULL;
}
\end{lstlisting}
\stepcounter{stringcnt}

\paragraph{The \texttt{strtok} function}
\texttt{char *strtok(char *restrict s1, const char *restrict delimiters);}

A sequence of calls to \texttt{strtok()} breaks the string pointed to by
\texttt{s1} into a sequence of tokens, each of which is delimited by a byte
from the string pointed to by \texttt{delimiters}. The first call in the
sequence has \texttt{s1} as its first argument, and is followed by calls with a
null pointer as their first argument. The separator string pointed to by
\texttt{delimiters} may be different from call to call.

The first call in the sequence searches the string pointed to by \texttt{s1}
for the first byte that is not contained in the current separator string
pointed to by \texttt{delimiters}. If no such byte is found, then there are no
tokens in the string pointed to by \texttt{s1} and \texttt{strtok()} shall
return a null pointer. If such a byte is found, it is the start of the first
token.

The \texttt{strtok()} function then searches from there for a byte (or
multiple, consecutive bytes) that is contained in the current separator string.
If no such byte is found, the current token extends to the end of the string
pointed to by \texttt{s1}, and subsequent searches for a token shall return a
null pointer. If such a byte is found, it is overwritten by a null byte, which
terminates the current token. The \texttt{strtok()} function saves a pointer to
the following byte, from which the next search for a token shall start.

Each subsequent call, with a null pointer as the value of the first argument,
starts searching from the saved pointer and behaves as described above.

The \texttt{strtok()} function need not be reentrant. A function that is not
required to be reentrant is not required to be thread-safe.

Because the \texttt{strtok()} function must save state between calls, and you
could not have two tokenizers going at the same time, the Single Unix Standard
defined a similar function, \texttt{strtok\_r()}, that does not need to save
state. Its prototype is this:

\texttt{char *strtok\_r(char *s, const char *delimiters, char **lasts);}

The \texttt{strtok\_r()} function considers the null-terminated string
\texttt{s} as a sequence of zero or more text tokens separated by spans of one
or more characters from the separator string \texttt{delimiters}. The argument
lasts points to a user-provided pointer which points to stored information
necessary for \texttt{strtok\_r()} to continue scanning the same string.

In the first call to \texttt{strtok\_r()}, \texttt{s} points to a
null-terminated string, \texttt{delimiters} to a null-terminated string of
separator characters, and the value pointed to by \texttt{lasts} is ignored.
The \texttt{strtok\_r()} function shall return a pointer to the first character
of the first token, write a null character into \texttt{s} immediately
following the returned token, and update the pointer to which \texttt{lasts}
points.

In subsequent calls, \texttt{s} is a null pointer and \texttt{lasts} shall be
unchanged from the previous call so that subsequent calls shall move through
the string \texttt{s}, returning successive tokens until no tokens remain. The
separator string \texttt{delimiters} may be different from call to call. When
no token remains in \texttt{s}, a NULL pointer shall be returned.

The following public-domain code for \texttt{strtok} and \texttt{strtok\_r}
codes the former as a special case of the latter:
\lstset{basicstyle=\scriptsize, numbers=left, captionpos=b, tabsize=4}
\begin{lstlisting}[caption=Section \thesection listing \arabic{stringcnt},language={C},
breaklines=true,xleftmargin=15pt,label=lst:section\thesection listing\arabic{stringcnt}]
#include <string.h>
/* strtok_r */
char *(strtok_r)(char *s, const char *delimiters, char **lasts) {
	char *sbegin, *send;
	sbegin = s ? s : *lasts;
	sbegin += strspn(sbegin, delimiters);
	if (*sbegin == '\0') {
		*lasts = "";
		return NULL;
	}
	send = sbegin + strcspn(sbegin, delimiters);
	if (*send != '\0')
		*send++ = '\0';
	*lasts = send;
	return sbegin;
}
/* strtok */
char *(strtok)(char *restrict s1, const char *restrict delimiters) {
	static char *ssave = "";
	return strtok_r(s1, delimiters, &ssave);
}
\end{lstlisting}
\stepcounter{stringcnt}

\subsubsection{Miscellaneous functions}
These functions do not fit into one of the above categories.

\paragraph{The \texttt{memset} function}
\texttt{void *memset(void *s, int c, size\_t n);}

The \texttt{memset()} function converts \texttt{c} into \texttt{unsigned char},
then stores the character into the first \texttt{n} bytes of memory pointed to
by \texttt{s}.

The following is a public-domain implementation of \texttt{memset}:
\lstset{basicstyle=\scriptsize, numbers=left, captionpos=b, tabsize=4}
\begin{lstlisting}[caption=Section \thesection listing \arabic{stringcnt},language={C},
breaklines=true,xleftmargin=15pt,label=lst:section\thesection listing\arabic{stringcnt}]
#include <string.h>
/* memset */
void *(memset)(void *s, int c, size_t n) {
	unsigned char *us = s;
	unsigned char uc = c;
	while (n-- != 0)
		*us++ = uc;
	return s;
}
\end{lstlisting}
\stepcounter{stringcnt}

\paragraph{The \texttt{strerror} function}
\texttt{char *strerror(int errorcode);}

This function returns a locale-specific error message corresponding to the
parameter. Depending on the circumstances, this function could be trivial to
implement, but this author will not do that as it varies.

The Single Unix System Version 3 has a variant, \texttt{strerror\_r}, with this
prototype:

\texttt{int strerror\_r(int errcode, char *buf, size\_t buflen);}

This function stores the message in \texttt{buf}, which has a length of size
\texttt{buflen}.
