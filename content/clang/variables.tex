\section{Variables}
\newcounter{varcnt}
Like most programming languages, C is able to use and process named variables
and their contents. \textbf{Variables} are simply names used to refer to some
location in memory --- a location that holds a value with which we are working. 

It may help to think of variables as a placeholder for a value. You can think
of a variable as being equivalent to its assigned value. So, if you have a
variable \emph{i} that is initialized (set equal) to 4, then it follows that
\emph{i+1} will equal \emph{5}.

Since C is a relatively low-level programming language, before a C program can
utilize memory to store a variable it must claim the memory needed to store the
values for a variable. This is done by \textbf{declaring} variables. Declaring
variables is the way in which a C program shows the number of variables it
needs, what they are going to be named, and how much memory they will need.

Within the C programming language, when managing and working with variables, it
is important to know the \emph{type} of variables and the \emph{size} of these
types. Since C is a fairly low-level programming language, these aspects of its
working can be hardware specific --- that is, how the language is made to work
on one type of machine can be different from how it is made to work on another.

\textbf{All} variables in C are ``typed''. That is, every variable declared
must be assigned as a certain type of variable.

\subsection{Declaring, Initializing, and Assigning Variables}
Here is an example of declaring an integer, which we've called
\texttt{some\_number}. (Note the semicolon at the end of the line; that is how
your compiler separates one program \emph{statement} from another.)
\lstset{basicstyle=\scriptsize, numbers=left, captionpos=b, tabsize=4}
\begin{lstlisting}[caption=Section \thesection listing \arabic{varcnt},language={C},
breaklines=true,xleftmargin=15pt,label=lst:section\thesection listing\arabic{varcnt}]
int some_number;
\end{lstlisting}

This statement means we're declaring some space for a variable called
some\_number, which will be used to store \texttt{int}eger data. Note that we
must specify the type of data that a variable will store. There are specific
keywords to do this --- we'll look at them in the next section.

Multiple variables can be declared with one statement, like this:
\lstset{basicstyle=\scriptsize, numbers=left, captionpos=b, tabsize=4}
\begin{lstlisting}[caption=Section \thesection listing \arabic{varcnt},language={C},
breaklines=true,xleftmargin=15pt,label=lst:section\thesection listing\arabic{varcnt}]
int anumber, anothernumber, yetanothernumber;
\end{lstlisting}
\stepcounter{varcnt}

We can also declare \emph{and} assign some content to a variable at the same time.
\lstset{basicstyle=\scriptsize, numbers=left, captionpos=b, tabsize=4}
\begin{lstlisting}[caption=Section \thesection listing \arabic{varcnt},language={C},
breaklines=true,xleftmargin=15pt,label=lst:section\thesection listing\arabic{varcnt}]
int some_number=3;
\end{lstlisting}
\stepcounter{varcnt}

This is called \emph{initialization}.

In early versions of C, variables must be declared at the beginning of a block.
In C99 it is allowed to mix declarations and statements arbitrarily—but doing
so is not usual, because it is rarely necessary, some compilers still don’t
support C99 (portability), and it may, because it is uncommon yet, irritate
fellow programmers (maintainability).

After declaring variables, you can assign a value to a variable later on using a statement like this:
\lstset{basicstyle=\scriptsize, numbers=left, captionpos=b, tabsize=4}
\begin{lstlisting}[caption=Section \thesection listing \arabic{varcnt},language={C},
breaklines=true,xleftmargin=15pt,label=lst:section\thesection listing\arabic{varcnt}]
some_number=3;
\end{lstlisting}
\stepcounter{varcnt}

You can also assign a variable the value of another variable, like so:
\lstset{basicstyle=\scriptsize, numbers=left, captionpos=b, tabsize=4}
\begin{lstlisting}[caption=Section \thesection listing \arabic{varcnt},language={C},
breaklines=true,xleftmargin=15pt,label=lst:section\thesection listing\arabic{varcnt}]
anumber = anothernumber;
\end{lstlisting}
\stepcounter{varcnt}

Or assign multiple variables the same value with one statement:
\lstset{basicstyle=\scriptsize, numbers=left, captionpos=b, tabsize=4}
\begin{lstlisting}[caption=Section \thesection listing \arabic{varcnt},language={C},
breaklines=true,xleftmargin=15pt,label=lst:section\thesection listing\arabic{varcnt}]
anumber = anothernumber = yetanothernumber = 3;
\end{lstlisting}
\stepcounter{varcnt}

This is because the assignment \texttt{x = y} returns the value of the
assignment. \texttt{x = y = z} is really shorthand for \texttt{x = (y = z)}.

\subsubsection{Naming Variables}
Variable names in C are made up of letters (upper and lower case) and digits.
The underscore character (``\_'') is also permitted. Names must not begin with
a digit. Unlike some languages (such as Perl and some BASIC dialects), C does
not use any special prefix characters on variable names.

Some examples of valid (but not very descriptive) C variable names:
\lstset{basicstyle=\scriptsize, numbers=left, captionpos=b, tabsize=4}
\begin{lstlisting}[caption=Section \thesection listing \arabic{varcnt},language={C},
breaklines=true,xleftmargin=15pt,label=lst:section\thesection listing\arabic{varcnt}]
foo
Bar
BAZ
foo_bar
_foo42
_
QuUx
\end{lstlisting}
\stepcounter{varcnt}

Some examples of invalid C variable names:
\lstset{basicstyle=\scriptsize, numbers=left, captionpos=b, tabsize=4}
\begin{lstlisting}[caption=Section \thesection listing \arabic{varcnt},language={C},
breaklines=true,xleftmargin=15pt,label=lst:section\thesection listing\arabic{varcnt}]
2foo (must not begin with a digit)
my foo (spaces not allowed in names)
$foo ($ not allowed -- only letters, digits, and _)
while (language keywords cannot be used as names)
\end{lstlisting}
\stepcounter{varcnt}

As the last example suggests, certain words are reserved as keywords in the
language, and these cannot be used as variable names.

In addition there are certain sets of names that, while not language keywords,
are reserved for one reason or another. For example, a C compiler might use
certain names ``behind the scenes'', and this might cause problems for a
program that attempts to use them. Also, some names are reserved for possible
future use in the C standard library. The rules for determining exactly what
names are reserved (and in what contexts they are reserved) are too complicated
to describe here, and as a beginner you don't need to worry about them much
anyway. For now, just avoid using names that begin with an underscore
character.

The naming rules for C variables also apply to naming other language constructs
such as function names, struct tags, and macros, all of which will be covered
later.

\subsection{Literals}
Anytime within a program in which you specify a value explicitly instead of
referring to a variable or some other form of data, that value is referred to
as a \textbf{literal}. In the initialization example above, 3 is a literal.
Literals can either take a form defined by their type (more on that soon), or
one can use hexadecimal (hex) notation to directly insert data into a variable
regardless of its type
. Hex numbers are always preceded with \emph{0x}. For now, though, you probably
shouldn't be too concerned with hex.

\subsection{The Four Basic Types}
In Standard C there are four basic data types. They are \texttt{int},
\texttt{char}, \texttt{float}, and \texttt{double}.

We will briefly describe them here, then go into more detail in C
Programming/Types.

\subsubsection{The \texttt{int} type}
The \texttt{int} type stores integers in the form of ``whole numbers''. An
integer is typically the size of one machine word, which on most modern home
PCs is 32 bits (4 octets). Examples of literals are whole numbers (integers)
such as 1,2,3, 10, 100... When \texttt{int} is 32 bits (4 octets), it can store
any whole number (integer) between -2147483648 and 2147483647. A 32 bit word
(number) has the possibility of representing any one number out of 4294967296
possibilities (2 to the power of 32).

If you want to declare a new int variable, use the \texttt{int} keyword. For
example:

\lstset{basicstyle=\scriptsize, numbers=left, captionpos=b, tabsize=4}
\begin{lstlisting}[caption=Section \thesection listing \arabic{varcnt},language={C},
breaklines=true,xleftmargin=15pt,label=lst:section\thesection listing\arabic{varcnt}]
int numberOfStudents, i, j=5;
\end{lstlisting}
\stepcounter{varcnt}

In this declaration we declare 3 variables, numberOfStudents, i and j, j here
is assigned the literal 5.

\subsubsection{The \texttt{char} type}
The \texttt{char} type is capable of holding any member of the execution
character set. It stores the same kind of data as an \texttt{int} (i.e.
integers), but always has a size of one byte. The size of a byte is specified
by the macro \texttt{CHAR\_BIT} which specifies the number of bits in a char
(byte). In standard C it never can be less than 8 bits. A variable of type
\texttt{char} is most often used to store character data, hence its name. Most
implementations use the ASCII character set as the execution character set, but
it's best not to know or care about that unless the actual values are
important.

Examples of character literals are a, b, 1, etc., as well as some special
characters such as \texttt{\textbackslash{}0} (the null character) and
\texttt{\textbackslash{}n} (newline, recall Hello, World). Note that the
\texttt{char} value must be enclosed within single quotations.

When we initialize a character variable, we can do it two ways. One is
preferred, the other way is \textbf{\emph{bad}} programming practice.

The first way is to write
\lstset{basicstyle=\scriptsize, numbers=left, captionpos=b, tabsize=4}
\begin{lstlisting}[caption=Section \thesection listing \arabic{varcnt},language={C},
breaklines=true,xleftmargin=15pt,label=lst:section\thesection listing\arabic{varcnt}]
char letter1 = 'a';
\end{lstlisting}
\stepcounter{varcnt}

This is \emph{good} programming practice in that it allows a person reading
your code to understand that letter1 is being initialized with the letter 'a'
to start off with.

The second way, which should \emph{not} be used when you are coding letter
characters, is to write
\lstset{basicstyle=\scriptsize, numbers=left, captionpos=b, tabsize=4}
\begin{lstlisting}[caption=Section \thesection listing \arabic{varcnt},language={C},
breaklines=true,xleftmargin=15pt,label=lst:section\thesection listing\arabic{varcnt}]
char letter2 = 97; /* in ASCII, 97 = 'a' */
\end{lstlisting}
\stepcounter{varcnt}

This is considered by some to be extremely \textbf{\emph{bad}} practice, if we
are using it to store a character, not a small number, in that if someone reads
your code, most readers are forced to look up what character corresponds with
the number 97 in the encoding scheme. In the end, \texttt{letter1} and
\texttt{letter2} store both the same thing -- the letter a, but the first
method is clearer, easier to debug, and much more straightforward. 

One important thing to mention is that characters for numerals are represented
differently from their corresponding number, i.e.\ '1' is not equal to 1. 

There is one more kind of literal that needs to be explained in connection with
chars: the \textbf{string literal}. A string is a series of characters, usually
intended to be displayed. They are surrounded by double quotations (`` '', not
' '). An example of a string literal is the ``Hello, world!\textbackslash{}n''
in the \"Hello, World\" example.

\subsubsection{The \texttt{float} type}
\texttt{float} is short for Floating Point. It stores real numbers also, but is
only one machine word in size. Therefore, it is used when less precision than a
double provides is required. \texttt{float} literals must be suffixed with F or
f, otherwise they will be interpreted as doubles. Examples are: 3.1415926f,
4.0f, 6.022e+23f. float variables can be declared using the \texttt{float}
keyword.

\subsubsection{The \texttt{double} type}
The \texttt{double} and \texttt{float} types are very similar. The
\texttt{float} type allows you to store single-precision floating point
numbers, while the \texttt{double} keyword allows you to store double-precision
floating point numbers --- real numbers, in other words, both integer and
non-integer values. Its size is typically two machine words, or 8 bytes on most
machines. Examples of \texttt{double} literals are 3.1415926535897932, 4.0,
6.022e+23 
%(\href{http://en.wikipedia.org/wiki/Scientific\_notation}{scientific notation}). 
If you use 4 instead of 4.0, the 4 will be interpreted as an
\texttt{int}.

The distinction between floats and doubles was made because of the differing
sizes of the two types. When C was first used, space was at a minimum and so
the judicious use of a float instead of a double saved some memory. Nowadays,
with memory more freely available, you do not really need to conserve memory
like this --- it may be better to use doubles consistently. Indeed, some C
implementations use doubles instead of floats when you declare a float
variable.

If you want to use a double variable, use the \texttt{double} keyword.

\subsection{\texttt{sizeof}}
If you have any doubts as to the amount of memory actually used by any type
(and this goes for types we'll discuss later, also), you can use the
\texttt{sizeof} operator to find out for sure. (For completeness, it is
important to mention that \texttt{sizeof} is a compile-time unary operator, not
a function.) Its syntax is:

\lstset{basicstyle=\scriptsize, numbers=left, captionpos=b, tabsize=4}
\begin{lstlisting}[caption=Section \thesection listing \arabic{varcnt},language={C},
breaklines=true,xleftmargin=15pt,label=lst:section\thesection listing\arabic{varcnt}]
sizeof object
sizeof(type)
\end{lstlisting}
\stepcounter{varcnt}

The two expressions above return the size of the object and type specified, in
bytes. The return type is \texttt{size\_t} (defined in the header
\texttt{\textless{}stddef.h\textgreater{}}) which is an unsigned value. Here's
an example usage:

\lstset{basicstyle=\scriptsize, numbers=left, captionpos=b, tabsize=4}
\begin{lstlisting}[caption=Section \thesection listing \arabic{varcnt},language={C},
breaklines=true,xleftmargin=15pt,label=lst:section\thesection listing\arabic{varcnt}]
size_t size;
int i;
size = sizeof(i);
\end{lstlisting}
\stepcounter{varcnt}

\texttt{size} will be set to 4, assuming \texttt{CHAR\_BIT} is defined as 8,
and an integer is 32 bits wide. The value of \texttt{sizeof}'s result is the
number of bytes.

Note that when \texttt{sizeof} is applied to a \texttt{char}, the result is 1;
that is:

\lstset{basicstyle=\scriptsize, numbers=left, captionpos=b, tabsize=4}
\begin{lstlisting}[caption=Section \thesection listing \arabic{varcnt},language={C},
breaklines=true,xleftmargin=15pt,label=lst:section\thesection listing\arabic{varcnt}]
sizeof(char)
\end{lstlisting}
\stepcounter{varcnt}

always returns 1.

\subsection{Data type modifiers}
One can alter the data storage of any data type by preceding it with certain
modifiers.

\texttt{long} and \texttt{short} are modifiers that make it possible for a data
type to use either more or less memory. The \texttt{int} keyword need not
follow the \texttt{short} and \texttt{long} keywords. This is most commonly the
case. A \texttt{short} can be used where the values fall within a lesser range
than that of an \texttt{int}, typically -32768 to 32767. A \texttt{long} can be
used to contain an extended range of values. It is not guaranteed that a
\texttt{short} uses less memory than an \texttt{int}, nor is it guaranteed that
a \texttt{long} takes up more memory than an \texttt{int}. It is only
guaranteed that sizeof(short) \textless{}= sizeof(int) \textless{}=
sizeof(long). Typically a \texttt{short} is 2 bytes, an \texttt{int} is 4
bytes, and a \texttt{long} either 4 or 8 bytes. Modern C compilers also provide
\texttt{long long} which is
typically an 8 byte integer. 

In all of the types described above, one bit is used to indicate the sign
(positive or negative) of a value. If you decide that a variable will never
hold a negative value, you may use the \texttt{unsigned} modifier to use that
one bit for storing other data, effectively doubling the range of values while
mandating that those values be positive. The \texttt{unsigned} specifier also
may be used without a trailing \texttt{int}, in which case the size defaults to
that of an \texttt{int}. There is also a \textbf{signed} modifier which is the
opposite, but it is not necessary, except for certain uses of \texttt{char},
and seldom used since all types (except \texttt{char}) are signed by default.

To use a modifier, just declare a variable with the data type and relevant modifiers:
\lstset{basicstyle=\scriptsize, numbers=left, captionpos=b, tabsize=4}
\begin{lstlisting}[caption=Section \thesection listing \arabic{varcnt},language={C},
breaklines=true,xleftmargin=15pt,label=lst:section\thesection listing\arabic{varcnt}]
unsigned short int usi; /* fully qualified --- unsigned short int */
short si; /* short int */
unsigned long uli; /* unsigned long int */
\end{lstlisting}
\stepcounter{varcnt}

\subsection{\texttt{const} qualifier}
When the \textbf{const} qualifier is used, the declared variable must be
initialized at declaration. It is then not allowed to be changed.

While the idea of a variable that never changes may not seem useful, there are
good reasons to use \texttt{const}. For one thing, many compilers can perform
some small optimizations on data when it knows that data will never change. For
example, if you need the value of \ensuremath{\pi} in your calculations, you
can declare a const variable of \texttt{pi}, so a program or another function
written by someone else cannot change the value of \texttt{pi}.

Note that a Standard conforming compiler must issue a warning if an attempt is
made to change a \texttt{cons}t variable --- but after doing so the compiler is
free to ignore the \\texttt{onst} qualifier.

\subsection{Magic numbers}
When you write C programs, you may be tempted to write code that will depend on
certain numbers. For example, you may be writing a program for a grocery store.
This complex program has thousands upon thousands of lines of code. The
programmer decides to represent the cost of a can of corn, currently 99 cents,
as a literal throughout the code. Now, assume the cost of a can of corn changes
to 89 cents. The programmer must now go in and manually change each entry of 99
cents to 89. While this is not that big of a problem, considering the ``global
find-replace'' function of many text editors, consider another problem: the
cost of a can of green beans is also initially 99 cents. To reliably change the
price, you have to look at every occurrence of the number 99.

C possesses certain functionality to avoid this. This functionality is
approximately equivalent, though one method can be useful in one circumstance,
over another.

\subsection{Using the \texttt{const} keyword}
The \texttt{const} keyword helps eradicate \textbf{magic numbers}. By declaring
a variable \texttt{const corn} at the beginning of a block, a programmer can
simply change that const and not have to worry about setting the value
elsewhere.

There is also another method for avoiding magic numbers. It is much more
flexible than \texttt{const}, and also much more problematic in many ways. It
also involves the preprocessor, as opposed to the compiler. Behold...

\subsubsection{\#define}
When you write programs, you can create what is known as a \emph{macro}, so
when the computer is reading your code, it will replace all instances of a word
with the specified expression.

Here's an example. If you write
\begin{enumerate}
\setlength{\leftmargin}{0pt}
\setlength{\itemsep}{0pt}
\setlength{\parsep}{0pt}
\setlength{\parskip}{0pt}
	\item define PRICE\_OF\_CORN 0.99
\end{enumerate}

when you want to, for example, print the price of corn, you use the word
\texttt{PRICE\_OF\_CORN} instead of the number 0.99 --- the preprocessor will
replace all instances of \texttt{PRICE\_OF\_CORN} with 0.99, which the compiler
will interpret as the literal \texttt{double} 0.99. The preprocessor performs
substitution, that is, \texttt{PRICE\_OF\_CORN} is replaced by 0.99 so this
means there is no need for a semicolon.

It is important to note that \texttt{\#define} has basically the same
functionality as the ``find-and-replace'' function in a lot of text
editors/word processors. 

For some purposes, \texttt{\#define} can be harmfully used, and it is usually
preferable to use \texttt{const} if \texttt{\#define} is unnecessary. It is
possible, for instance, to \texttt{\#define}, say, a macro \texttt{DOG} as the
number 3, but if you try to print the macro, thinking that \texttt{DOG}
represents a string that you can show on the screen, the program will have an
error. \texttt{\#define} also has no regard for type. It disregards the
structure of your program, replacing the text \emph{everywhere} (in effect,
disregarding scope), which could be advantageous in some circumstances, but can
be the source of problematic bugs.

You will see further instances of the \texttt{\#define} directive later in the
text. It is good convention to write \texttt{\#define}d words in all capitals,
so a programmer will know that this is not a variable that you have declared
but a \texttt{\#define}d macro.

\subsection{Scope}
In the Basic Concepts section, the concept of scope was introduced. It is
important to revisit the distinction between local types and global types, and
how to declare variables of each. To declare a local variable, you place the
declaration at the beginning (i.e. before any non-declarative statements) of
the block to which the variable is intended to be local. To declare a global
variable, declare the variable outside of any block. If a variable is global,
it can be read, and written, from anywhere in your program.

Global variables are not considered good programming practice, and should be
avoided whenever possible. They inhibit code readability, create naming
conflicts, waste memory, and can create difficult-to-trace bugs. Excessive
usage of globals is usually a sign of laziness and/or poor design. However, if
there is a situation where local variables may create more obtuse and
unreadable code, there's no shame in using globals.

\subsection{Other Modifiers}
Included here, for completeness, are more of the modifiers that standard C
provides. For the beginning programmer, \emph{static} and \emph{extern} may be
useful. \emph{volatile} is more of interest to advanced programmers.
\emph{register} and \emph{auto} are largely deprecated and are generally not of
interest to either beginning or advanced programmers.

\textbf{static} is sometimes a useful keyword.  It is a common misbelief that
the only purpose is to make a variable stay in memory. When you declare a
function or global variable as \emph{static} it will become internal. You
cannot access the function or variable through the extern (see below) keyword
from other files in your project. When you declare a local variable as
\emph{static}, it is created just like any other variable. However, when the
variable goes out of scope (i.e. the block it was local to is finished) the
variable stays in memory, retaining its value. The variable stays in memory
until the program ends. While this behaviour resembles that of global
variables, static variables still obey scope rules and therefore cannot be
accessed outside of their scope. Variables declared static are initialized to
zero (or for pointers, NULL) by default. 

You can use static in (at least) two different ways. Consider this code, and
imagine it is in a file called jfile.c:


\lstset{basicstyle=\scriptsize, numbers=left, captionpos=b, tabsize=4}
\begin{lstlisting}[caption=Section \thesection listing \arabic{varcnt},language={C},
breaklines=true,xleftmargin=15pt,label=lst:section\thesection listing\arabic{varcnt}]
include <stdio.h>
static int j = 0;
	
void up(void) {
	/* k is set to 0 when the program starts. The line is then "ignored"
	 * for the rest of the program (i.e. k is not set to 0 every time up()
	 * is called)
	 */
	static int k = 0;
	j++;
	k++;
	printf("up() called.   k= %2d, j= %2d\n", k , j);
}
	
void down(void) {
	static int k = 0;
	j--;
	k--;
	printf("down() called. k= %2d, j= %2d\n", k , j);
}
	
int main(void) {
	int i;
	  
	/* call the up function 3 times, then the down function 2 times */
	for (i= 0; i < 3; i++)
	   up();
	for (i= 0; i < 2; i++)
	   down();
	 
	return 0;
}
\end{lstlisting}
\stepcounter{varcnt}

The j var is accessible by both up and down and retains its value. the k vars
also retain their value, but they are two different variables in each their
scopes. static vars are a good way to implement encapsulation, a term from the
object-oriented way of thinking that effectively means not allowing changes to
be made to a variable except through function calls.

Running the program above will produce the following output:
\lstset{basicstyle=\scriptsize, numbers=left, captionpos=b, tabsize=4}
\begin{lstlisting}[caption=Section \thesection listing \arabic{varcnt},language={C},
breaklines=true,xleftmargin=15pt,label=lst:section\thesection listing\arabic{varcnt}]
up() called. k= 1, j= 1
up() called. k= 2, j= 2
up() called. k= 3, j= 3
down() called. k= -1, j= 2
down() called. k= -2, j= 1
\end{lstlisting}
\stepcounter{varcnt}

\textbf{extern} is used when a file needs to access a variable in another file
that it may not have \textit{\#included} directly.  Therefore, \emph{extern}
does not actually carve out space for a new variable, it just provides the
compiler with sufficient information to access the remote variable.

\textbf{\texttt{volatile}} is a special type modifier which informs the
compiler that the value of the variable may be changed by external entities
other than the program itself. This is necessary for certain programs compiled
with optimizations --- if a variable were not defined \texttt{volatile} then
the compiler may assume that certain operations involving the variable are safe
to optimize away when in fact they aren't. \emph{volatile} is particularly
relevant when working with embedded systems (where a program may not have
complete control of a variable) and multi-threaded applications.

\textbf{auto} is a modifier which specifies an ``automatic'' variable that is
automatically created when in scope and destroyed when out of scope. If you
think this sounds like pretty much what you've been doing all along when you
declare a variable, you're right: all declared items within a block are
implicitly ``automatic''. For this reason, the \emph{auto} keyword is more like
the answer to a trivia question than a useful modifier, and there are lots of
very competent programmers that are unaware of its existence.

\textbf{\texttt{register}} is a hint to the compiler to attempt to optimize the
storage of the given variable by storing it in a register of the computer's CPU
when the program is run. Most optimizing compilers do this anyway, so use of
this keyword is often unnecessary. In fact, ANSI C states that a compiler can
ignore this keyword if it so desires -- and many do. Microsoft Visual C++ is an
example of an implementation that completely ignores the \emph{register}
keyword.
