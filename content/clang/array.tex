\section{Arrays}
Arrays in C act to store related data under a single variable name with an
index, also known as a \emph{subscript}. It is easiest to think of an array as
simply a list or ordered grouping for variables of the same type. As such,
arrays often help a programmer organize collections of data efficiently and
intuitively.

Later we will consider the concept of a \emph{pointer}, fundamental to C, which
extends the nature of the array (array can be termed as a constant pointer).
For now, we will consider just their declaration and their use.

\subsection{Arrays}
If we want an array of six integers (or numbers), we write in C:
\lstset{basicstyle=\scriptsize, numbers=left, captionpos=b, tabsize=4}
\begin{lstlisting}[caption=,language={C},
breaklines=true,xleftmargin=15pt,label=lst:]
int numbers[6];
\end{lstlisting}

For a character array called letters,
\lstset{basicstyle=\scriptsize, numbers=left, captionpos=b, tabsize=4}
\begin{lstlisting}[caption=,language={C},
breaklines=true,xleftmargin=15pt,label=lst:]
char letters[6];
\end{lstlisting}

and so on.

If we wish to initialize as we declare, we write:
\lstset{basicstyle=\scriptsize, numbers=left, captionpos=b, tabsize=4}
\begin{lstlisting}[caption=,language={C},
breaklines=true,xleftmargin=15pt,label=lst:]
int point[6]={0,0,1,0,0,0};
\end{lstlisting}

If not all elements in the array are initialized, the remaining elements will
contain a value of 0.

If we want to access a variable stored in an array, for example with the above
declaration, the following code will store a 1 in the variable \texttt{x}
\lstset{basicstyle=\scriptsize, numbers=left, captionpos=b, tabsize=4}
\begin{lstlisting}[caption=,language={C},
breaklines=true,xleftmargin=15pt,label=lst:]
int x;
x = point[2];
\end{lstlisting}

Arrays in C are indexed starting at 0, as opposed to starting at 1. The first
element of the array above is \texttt{point[0]}. The index to the last value in
the array is the array size minus one.  In the example above the subscripts run
from 0 through 5. C does not guarantee bounds checking on array accesses. The
compiler may not complain about the following (though the best compilers do):
\lstset{basicstyle=\scriptsize, numbers=left, captionpos=b, tabsize=4}
\begin{lstlisting}[caption=,language={C},
breaklines=true,xleftmargin=15pt,label=lst:]
char y;
int z = 9;
char point[6] = { 1, 2, 3, 4, 5, 6 };
//examples of accessing outside the array. A compile error is not always raised
y = point[15];
y = point[-4];
y = point[z];
\end{lstlisting}

During program execution, an out of bounds array access does not always cause a
run time error. Your program may happily continue after retrieving a value from
point[-1]. To alleviate indexing problems, the sizeof() expression is commonly
used when coding loops that process arrays.
\lstset{basicstyle=\scriptsize, numbers=left, captionpos=b, tabsize=4}
\begin{lstlisting}[caption=,language={C},
breaklines=true,xleftmargin=15pt,label=lst:]
int ix;
short anArray[]= { 3, 6, 9, 12, 15 };
 
for (ix=0; ix< (sizeof(anArray)/sizeof(short)); ++ix) {
	DoSomethingWith( anArray[ix] );
}
\end{lstlisting}

Notice in the above example, the size of the array was not explicitly
specified. The compiler knows to size it at 5 because of the five values in the
initializer list. Adding an additional value to the list will cause it to be
sized to six, and because of the sizeof expression in the \texttt{for} loop,
the code automatically adjusts to this change. Good programming practice is
declare a variable \emph{size } and store the size of the array.
\lstset{basicstyle=\scriptsize, numbers=left, captionpos=b, tabsize=4}
\begin{lstlisting}[caption=,language={C},
breaklines=true,xleftmargin=15pt,label=lst:]
size = sizeof(anArray)/sizeof(short)
\end{lstlisting}

C also supports multi dimensional arrays (or, rather, arrays of arrays). The
simplest type is a two dimensional array. This creates a rectangular array ---
each row has the same number of columns. To get a char array with 3 rows and 5
columns we write in C
\lstset{basicstyle=\scriptsize, numbers=left, captionpos=b, tabsize=4}
\begin{lstlisting}[caption=,language={C},
breaklines=true,xleftmargin=15pt,label=lst:]
char two_d[3][5];
\end{lstlisting}

To access/modify a value in this array we need two subscripts:
\lstset{basicstyle=\scriptsize, numbers=left, captionpos=b, tabsize=4}
\begin{lstlisting}[caption=,language={C},
breaklines=true,xleftmargin=15pt,label=lst:]
char ch;
ch = two_d[2][4];
\end{lstlisting}

or
\lstset{basicstyle=\scriptsize, numbers=left, captionpos=b, tabsize=4}
\begin{lstlisting}[caption=,language={C},
breaklines=true,xleftmargin=15pt,label=lst:]
two_d[0][0] = 'x';
\end{lstlisting}

Similarly, a multi-dimensional array can be initialized like this:
\lstset{basicstyle=\scriptsize, numbers=left, captionpos=b, tabsize=4}
\begin{lstlisting}[caption=,language={C},
breaklines=true,xleftmargin=15pt,label=lst:]
int two_d[2][3] = {{ 5, 2, 1 },
                   { 6, 7, 8 }};
\end{lstlisting}

The amount of columns must be explicitly stated; however, the compiler will
find the appropriate amount of rows based on the initializer list.  There are
also weird notations possible:
\lstset{basicstyle=\scriptsize, numbers=left, captionpos=b, tabsize=4}
\begin{lstlisting}[caption=,language={C},
breaklines=true,xleftmargin=15pt,label=lst:]
int a[100];
int i = 0;
if(a[i]==i[a]) {
	printf("Hello World!\n");
}
\end{lstlisting}

a[i] and i[a] refer to the same location. (This is explained later in the next
Chapter.)

\subsection{Strings}
C has no string handling facilities built in; consequently, strings are defined
as arrays of characters. C allows a character array to be represented by a
character string rather than a list of characters, with the null terminating
character automatically added to the end. For example, to store the string
``Merkkijono'', we would write
\lstset{basicstyle=\scriptsize, numbers=left, captionpos=b, tabsize=4}
\begin{lstlisting}[caption=,language={C},
breaklines=true,xleftmargin=15pt,label=lst:]
char string[] = "Merkkijono";
\end{lstlisting}

or
\lstset{basicstyle=\scriptsize, numbers=left, captionpos=b, tabsize=4}
\begin{lstlisting}[caption=,language={C},
breaklines=true,xleftmargin=15pt,label=lst:]
char string[] = {'M', 'e', 'r', 'k', 'k', 'i', 'j', 'o', 'n', 'o', '\0'};
\end{lstlisting}

In the first example, the string will have a null character automatically
appended to the end by the compiler; by convention, library functions expect
strings to be terminated by a null character. The latter declaration indicates
individual elements, and as such the null terminator needs to be added
manually. 

Strings do not always have to be linked to an explicit variable. As you have
seen already, a string of characters can be created directly as an unnamed
string that is used directly (as with the printf functions.) 

To create an extra long string, you will have to split the string into multiple
sections, by closing the first section with a quote, and recommencing the
string on the next line (also starting and ending in a quote):
\lstset{basicstyle=\scriptsize, numbers=left, captionpos=b, tabsize=4}
\begin{lstlisting}[caption=,language={C},
breaklines=true,xleftmargin=15pt,label=lst:]
char string[] = "This is a very, very long "
                "string that requires two lines.";
\end{lstlisting}

While strings may also span multiple lines by putting the backslash character
at the end of the line, this method is deprecated. 

There is a useful library of string handling routines which you can use by
including another header file.
\lstset{basicstyle=\scriptsize, numbers=left, captionpos=b, tabsize=4}
\begin{lstlisting}[caption=,language={C},
breaklines=true,xleftmargin=15pt,label=lst:]
#include <string.h>  //new header file
\end{lstlisting}
