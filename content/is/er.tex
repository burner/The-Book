\section{Entity-Relationship-Model}
\begin{tikzpicture}
	\node[entity] (emp) {employee};
	\node[attribute] (essn) [above left of=emp,node distance=1.5cm] {\key{Sse}} edge (emp);
	\node[attribute] (bday) [left of=emp,node distance=2.0cm] {BDay} edge (emp);
	\node[attribute] (name) [above of=emp,node distance=1.5cm] {Name} edge (emp);
	\node[attribute] (sName) [above left of=name,node distance=1.1cm] {SName} edge (name);
	\node[attribute] (mName) [above of=name,node distance=1.3cm] {MName} edge (name);
	\node[attribute] (lName) [above right of=name,node distance=1.1cm] {LName} edge (name);
	\node[relationship] (control) [below of=emp,node distance=1.5cm] {Controls} 
		edge node[auto,swap] {1} (emp.-140)
		edge node[auto,swap] {n} (emp.-40);

	\node[entity] (section) [right of=emp,node distance=5.0cm] {section};
	\node[relationship] (works) [right of=emp,node distance=2.5cm] {Works For}
		edge [total] node[auto,swap] {n} (emp)
		edge [total] node[auto,swap] {1} (section);
	\node[relationship] (leads) [below of=works,node distance=1.6cm] {leads}
		edge node[auto,swap] {1} (emp)
		edge [total] node[auto,swap] {1} (section);
	\node[attribute] (name) [above left of=section,node distance=1.5cm] {\key{Name}}
		edge (section);
	\node[attribute] (num) [above of=section,node distance=1.6cm] {\key{Number}}
		edge (section);
	\node[multi attribute] (location) [below of=section,node distance=1.5cm]
		{Location} edge (section);
\end{tikzpicture}

\subsection{Entities}
Entities describe things of the displayed world. Entities can live on there own.
There are two kinds of entities. Strong which have attributes for
identification. Weak entities need extra entities for identification.
\begin{figure}[h]
\scriptsize
\begin{tikzpicture}
	\node[entity] (emp) {Employ};
	\node[ident relationship] (from) [right of=emp, node distance=2.5cm] {belongsTo}
		edge [total] node[auto,swap] {1} (emp);
	\node[weak entity] (fam) [right of=from, node distance=2.5cm] {Family}
		edge [total] node[auto,swap] {n} (from);
\end{tikzpicture}
\caption{Entity example}
\label{fig:is:EntityExample}
\end{figure}
The employ is a strong entity. It doesn't need any other entity make make itself
identifiable. The family needs an employ entity to be even considered a part of
the model. See figure \ref{fig:is:EntityExample}

\subsection{Attributes}
Attributes describe entities. There are different kinds of attributes.
\begin{itemize}
\setlength{\itemsep}{1pt}
	\item atomar - these attributes can be devided
	\item combined - consist of more than one attribute
	\item univalent - can only appear once
	\item multivalent - can appear more than one time
	\item complex - can consist of different kind of attributes.
	\item key - makes an entity identifiable
	\item partiell key - is part of the key that makes the entity idenfiable
\end{itemize}

\subsection{Relationships}
There are two kinds of relationships \textbf{normal relationships} as well as
\textbf{identifying relationships}. These relationships have a cardinality.
Cardinalities are defines as (min,max), common cardinalities are 1:1, 1:n, n:m.
A relationship is called total if all entities participate. Relationships can be
binary as well as ternary. Ternary relationships can be map on binary with the
lose of semantics. Similar to entities relationships can have attributes.

\subsection{Notation}
Notation uses throughout the The-Book.
\begin{figure}[h]
\scriptsize
\begin{tikzpicture}
	\node[entity] (ent) {Entity};
	\node[weak entity] (ient) [right of=ent, node distance=2.5cm] {Identifing Entity};
	\node[relationship] (relation) [right of=ient, node distance=2.5cm] {Relationship};
	\node[ident relationship] (irelation) [below of=ent, node distance=1.5cm] {Ident Rela};
	\node[attribute] (attr) [below of=ient, node distance=1.5cm] {Attribute};
	\node[attribute] (pattr) [below of=relation, node distance=1.5cm] {\discriminator{Partial Key}};
	\node[attribute] (kattr) [below of=irelation, node distance=1.5cm] {\underline{Primary Key}};
\end{tikzpicture}
\caption{Notation}
\label{fig:is:Notation}
\end{figure}
