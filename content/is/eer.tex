\section{Enhanced ER-Modell}
The enhanced ER-Modell extends the ER-Modell with specialization and
generalization. Specialization defines a set of Subentities to a given (Super-)
Entity. Generalization builds a generalizing (Super)-Entity. This is used to
represent object orientierend programs.\\
\subsection{Generalization}
\begin{figure}[h]
\scriptsize
\begin{tikzpicture}
	\node[entity] (employ) {Employee};
	\node[attribute] (d) [below of=employ, node distance=1cm] {d}
		edge node[auto,swap] { } (employ);
	\node[entity] (tec) [below of=d, node distance=1cm] {Technician}
		edge [(-] (d);
	\node[entity] (sec) [left of=tec, node distance=2cm] {Secretary}
		edge [(-] (d);
	\node[attribute] (ks) [above of=sec, node distance=1cm] {Keystrokes}
		edge (sec);
	\node[entity] (eng) [right of=tec, node distance=2cm] {Engineer}
		edge [(-] (d);
	\node[attribute] (field) [above of=eng, node distance=1cm] {Field}
		edge (eng);
	\node[attribute] (name) [right of=employ, node distance=2cm] {Name}
		edge [-] (employ);
\end{tikzpicture}
\caption{Generalization example}
\label{fig:is:GeneralizationExample}
\end{figure}
The (d) means disjunct.

\subsection{Specialization}
\begin{figure}[h]
\scriptsize
\begin{tikzpicture}
	\node[entity] (employ) {Employee};
	\node[attribute] (d) [below left of=employ, node distance=2.0cm] {d}
		edge node[auto,swap] { } (employ);
	\node[entity] (eng) [below of=d, node distance=1.5cm] {Engineer}
		edge [(-] (d);
	\node[entity] (tec) [left of=eng, node distance=1.8cm] {Technician}
		edge [(-] (d);
	\node[entity] (sec) [above left of=d, node distance=1.8cm] {Secretary}
		edge [(-] (d);
	\node[entity] (man) [right of=eng, node distance=1.8cm] {Manager}
		edge [(-] (employ);
	\node[attribute] (d2) [right of=d, node distance=3.5cm] {d}
		edge node[auto,swap] { } (employ);
	\node[entity] (hourem) [below right of=d2, node distance=1.5cm] 
		{Hours\_Employee} edge [(-] (d2);
	\node[entity] (nem) [below left of=hourem, node distance=2.0cm] 
		{Normal\_Employee} edge [(-] (d2);
	\node[entity] (mancon) [below of=man, node distance=2.0cm] 
		{Manager\_Construction} edge [(-] (man) edge [(-] (eng) 
		edge [(-] (nem);
\end{tikzpicture}
\caption{Specialization network}
\label{fig:is:SpecializationNetwork}
\end{figure}

\subsection{From Specialization to Generalization}
Figure \ref{fig:is:FormSpecializationToGeneralization:Specialization} is transformed into
Figure \ref{fig:is:FormSpecializationToGeneralization:Generalization}.
\begin{figure}[h]
\scriptsize
\centering
\begin{minipage}{0.2\textwidth}
	\begin{tikzpicture}
	\node[entity] (car) {Car};
	\node[attribute] (price) [above right of=car, node distance=1.3cm] {Price}
		edge [-] (car);
	\node[attribute] (passanger) [above of=car, node distance=1.5cm] {Passanger}
		edge [-] (car);
	\node[attribute] (speed) [above left of=car, node distance=1.3cm] {MaxSpeed}
		edge [-] (car);
	\node[attribute] (VehicId) [left of=car, node distance=2.0cm] {\underline{VehicleId}}
		edge [-] (car);
	\node[attribute] (plade) [below of=car, node distance=1.0cm] {\underline{Plade}}
		edge [-] (car);
	\end{tikzpicture}
\end{minipage} \hspace{1cm}
\begin{minipage}{0.2\textwidth}
	\begin{tikzpicture}
	\node[entity] (truck) {Truck};
	\node[attribute] (price) [above left of=truck, node distance=1.3cm] {Price}
		edge [-] (truck);
	\node[attribute] (ax) [above of=truck, node distance=1.5cm] {Axles}
		edge [-] (truck);
	\node[attribute] (ccm) [above right of=truck, node distance=1.3cm] {ccm}
		edge [-] (truck);
	\node[attribute] (VehicId) [right of=truck, node distance=2.0cm] {\underline{VehicleId}}
		edge [-] (truck);
	\node[attribute] (plade) [below of=truck, node distance=1.0cm] {\underline{Plade}}
		edge [-] (truck);
	\end{tikzpicture}
\end{minipage}
\caption{Form specialization to generalization: Specialization}
\label{fig:is:FormSpecializationToGeneralization:Specialization}
\end{figure}

\begin{figure}[h]
\scriptsize
\centering
\begin{tikzpicture}
	\node[entity] (vehicle) {Vehicle};
	\node[attribute] (price) [above of=vehicle, node distance=1.8cm] {Price}
		edge [-] (vehicle);
	\node[attribute] (VehicId) [above right of=vehicle, node distance=1.5cm] {\underline{VehicleId}}
		edge [-] (vehicle);
	\node[attribute] (plade) [above left of=vehicle, node distance=1.5cm] {\underline{Plade}}
		edge [-] (vehicle);
	\node[attribute] (d) [below of=vehicle, node distance=1.0cm] {d}
		edge [total] (vehicle);
	\node[entity] (car) [below left of=d, node distance=2.5cm] {Car}
		edge [(-] (d);
	\node[attribute] (passanger) [above of=car, node distance=1.5cm] {Passanger}
		edge [-] (car);
	\node[attribute] (speed) [right of=car, node distance=1.7cm] {Speed}
		edge [-] (car);
	\node[entity] (truck) [below right of=d, node distance=2.5cm] {Truck}
		edge [(-] (d);
	\node[attribute] (axles) [above of=truck, node distance=1.5cm] {Axles}
		edge [-] (truck);
	\node[attribute] (ccm) [right of=truck, node distance=1.7cm] {ccm}
		edge [-] (truck);
\end{tikzpicture}
\caption{Form specialization to generalization: Generalization}
\label{fig:is:FormSpecializationToGeneralization:Generalization}
\end{figure}

\subsection{Documentation and Business Rules}
The (E)ER-Scheme is usually not enough to display all interesting details of the
miniworld. A example would be: The projectbudget is never under \euro{50k}. To
fix this problem \textit{business rules} are used. A business rule is basically
just a textual description of the problem. See table
\ref{tab:is:buisnessRuleExample}
\ref{tab:is:buisnessRuleConstraintExample}.

\begin{table}[h]
\begin{tabular}{|c|c|c|}
\hline
Entity & Description & Attribute\\
\hline
Employee & Company Employee & Ssn, Bday, Name\\
\hline
\end{tabular}
\caption{Business Rule Example}
\label{tab:is:buisnessRuleExample}
\end{table}

\begin{table}[h]
\begin{tabular}{|l|c|}
\hline
\multicolumn{2}{|c|}{Constraints}\\
\hline
\textbf{(BR1} & The projectleader must be member of the project\\
\hline
\multicolumn{2}{|c|}{Derivations}\\
\hline
\textbf{(BR2)} & The number of employees of an branch computes \\
&through the relationship WORKS\_FOR.\\
\hline
\end{tabular}
\caption{Business Rule Constraint Example}
\label{tab:is:buisnessRuleConstraintExample}
\end{table}

Criteria for good business rules are:
\begin{itemize}
	\item correctness
	\item completeness
	\item minimality
	\item readability
	\item easy to understand
\end{itemize}
