\section{Relational Model}
The relational model presents a database as a collection of relations
respectively tables. See the figure \ref{tab:is:rm:uniexample} as for example.
There are tables for students, courses, sections, grade reports and
prerequisite. Data elements have different types. A logical connection exists
between the tables. Compare Student.Student\-Number with
Grade\_Report.Student\-Number for example.

\begin{table}[h]
Student\\
\begin{tabular}{|c|c|c|c|}
\hline
Name & StudentNumber & Class & Major \\
\hline
Smith & 17 & 1 & CS\\ 
\hline
Brown & 8 & 2 & CS\\ 
\hline
\end{tabular}

Course\\
\begin{tabular}{|c|c|c|c|}
\hline
CourseName & CNumber & CHours & Department\\
\hline
Intro into CS & CS1310 & 5 & CS\\
\hline
Data Structures & CS3320 & 4 & CS\\
\hline
Discrete Math & MATH2410 & 3 & MATH\\
\hline
Database & CS3380 & 3 & CS\\
\hline
\end{tabular}

Section\\
\begin{tabular}{|c|c|c|c|c|}
\hline
SectionId & CNumber & Semester & Year & Instructor\\
\hline
85 & MATH2410 & Fall & 98 & King\\
\hline
92 & CS1310 & Fall & 98 & Anderson\\
\hline
102 & CS3320 & Spring & 99 & Knuth\\
\hline
112 & MATH2410 & Fall & 99 & Chang\\
\hline
\end{tabular}

Grade\_Report\\
\begin{tabular}{|c|c|c|}
\hline
StudentNumber & SectionId & Grade \\
\hline
17 & 122 & B\\
\hline
17 & 119 & C\\
\hline
8 & 85 & A\\
\hline
8 & 92 & A\\
\hline
\end{tabular}

Prerequisite\\
\begin{tabular}{|c|c|c|}
\hline
CNumber & PrerequisiteNumber\\
\hline
CS3380 & CS3320\\
\hline
CS3380 & MATH2410\\
\hline
CS3320 & CS1310\\
\hline
\end{tabular}

\caption{Relational Model University example}
\label{tab:is:rm:uniexample}
\end{table}

\subsection{From Entity Relation to Relational Model}
Entities and relationships of the ER-Model are displayed as relations. No
information must be lost. Usually every row of a table represents a entity of a
relationship of the mini-world.

\subsection{Terminologie}
The first row of a table defines the attribute names. Every following row
is called a tuple. To every attribute there is a useful value-type. (Integer,
String, timestamp \dots). The value-type is also called Domain. The name of the
relation as well as the attribute-name are freely choose able. A
Relation-schema $R(A_1, \dots , A_n)$ splits up to R the relation-name A the
attribute names. The degree of a relation is the number of attributes.

\section{Super-Keys and Keys}
