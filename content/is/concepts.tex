\section{Concepts}
\begin{keywords}
Database, ACID
\end{keywords}
Databases work under a closed world assumption, that means that not all Details
of the actual world are took into consideration. Only attributes relevant to the
task are considered. Another imported aspect is that database give every user
the illusion that he is working alone. Therefor four attributes are defined to
make databases work the way the do. These attributes are called ACID.
\begin{itemize}
\setlength{\itemsep}{1pt}
	\item \textbf{Atomicity}: this means that only one operation at a time will change
the database
	\item \textbf{Consistency}: if the database was in a correct state start before 
any change happend it will be correct state afterwords
	\item \textbf{Isolation}: everyone using the database thinks he is working alone
	\item \textbf{Durability}: once the transaction is done it is save
\end{itemize}

\section{Modelling-Level of Data models}
\begin{itemize}
\setlength{\itemsep}{1pt}
	\item \textbf{conceptual data models} defines the mini-world that is
represented by the database
	\item \textbf{logical data models} defines a specification that is easily
implemented
	\item \textbf{physical model} defines the concrete datastorage
\end{itemize}

\section{Hierarchie Model}
The Hierarchie model defines one dataset and all depending datasets as a
hierarchy. Relation between datasets can have different characteristics.
Think unix-file-system-hierarchy.
\begin{itemize}
\setlength{\itemsep}{1pt}
	\item 1:1 parent-element is bound the child-element
	\item 1:n parent-element is bound to n child-elements
	\item m:n m parent-element are bound to n child-elements
(m $\in [1 \dots n]$)
	\item no cycles are allowed
\end{itemize}

\section{Network Model}
The network-model builds upon the Hierarchie Model. One element can be part of
more than one group.

\section{Flat relational model}
The flat relational model is build from tables hat hab attributes and
column-header. These tables are container for complex datatypes.\\
\begin{tabular}{|c|c|c|c|}
\hline
\multicolumn{4}{|c|}{Participants}\\
\hline
StudentNum & Surname & Name & Semester \\
\hline
4711 & Martha & Mustermann & 3 \\
\hline
42 & Arthur & Dent & 23\\
\hline
\end{tabular}\\
Relations are created through grouping attributes in tables or through
value-based associating.\\
\begin{tabular}{|c|c|}
\hline
\multicolumn{2}{|c|}{Lecture}\\
\hline
LectureNum & Title \\
\hline
48 & Informationsystems \\
\hline
43 & Operation Systems\\
\hline
\end{tabular}\\

\begin{tabular}{|c|c|}
\hline
\multicolumn{2}{|c|}{Selection}\\
\hline
LectureNum & StudentNum \\
\hline
42 & 4711 \\
\hline
43 & 42\\
\hline
\end{tabular}

\section{Nested relational Model}
This allows attribute-values to be relation itself.\\

\scriptsize \begin{tabular}{|c|c|c|c|} 
\hline
GRADE\_REPORT & StudentNum & \multicolumn{2}{|c|}{SectionGrade(SectionID,
Grade)}\\
\hline
& \multirow{2}{*}{17} & 122 & B\\
\cline{3-4}
& & 119 & C\\
\hline
& \multirow{4}{*}{8} & 85 & A\\
& & 92 & A\\
\cline{3-4}
& & 102 & B\\
\cline{3-4}
& & 135 & A\\
\hline
\end{tabular}
