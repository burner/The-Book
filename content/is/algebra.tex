\section{Relational Algebra}
\begin{keywords}
SQL commands, Relational databases, Relational Algebra, Databases, Minimality
\end{keywords}
\subsection{Requirements}
To perform set operationen on two relations two requirements must be satisfied.
\begin{itemize}
	\item R and S must have the same degree (think polygon degree)
	\item The value-range of all attributes both both relations R and S must be
		identical
\end{itemize}

\subsection{Join}
The join R $\cup$ S combines all tuple of relation R with Relation S. To perform
this operation, both relation-schemes must be identical. That means the have the
same attributes and attribute-types. Duplications will be deleted.\\
Definition: $R \cup S = \{t | t \in R \lor t \in S\}$\\

\begin{tabular}{ c c c}
	R: & S: & R $\cup$ S \\
	\begin{tabular}{|c|c|c|}
		\hline
		A & B & C\\
		\hline
		1 & 2 & 3\\
		\hline
		4 & 5 & 6\\
		\hline
	\end{tabular} &

	\begin{tabular}{|c|c|c|}
		\hline
		A & B & C\\
		\hline
		7 & 8 & 9\\
		\hline
		4 & 5 & 6\\
		\hline
	\end{tabular} &

	\begin{tabular}{|c|c|c|}
		\hline
		A & B & C\\
		\hline
		7 & 8 & 9\\
		\hline
		4 & 5 & 6\\
		\hline
		1 & 2 & 3\\
		\hline
	\end{tabular}
\end{tabular}

\lstset{language=SQL,tabsize=4,captionpos=b,frame=single,
basicstyle=\footnotesize}
\begin{lstlisting}[caption=SQL Join]
SELECT * FROM R UNION SELECT * FROM S;
\end{lstlisting}

\subsection{Difference}
The difference-operation remove all tuple for relation S that are present in
R.\\
Definition: $R-S= R\backslash S = \{t | t \in R \land t \not\in S \}$\\

\begin{tabular}{ c c c}
	R: & S: & R $\backslash$ S \\
	\begin{tabular}{|c|c|c|}
		\hline
		A & B & C\\
		\hline
		1 & 2 & 3\\
		\hline
		4 & 5 & 6\\
		\hline
	\end{tabular} &

	\begin{tabular}{|c|c|c|}
		\hline
		A & B & C\\
		\hline
		7 & 8 & 9\\
		\hline
		4 & 5 & 6\\
		\hline
	\end{tabular} &

	\begin{tabular}{|c|c|c|}
		\hline
		A & B & C\\
		\hline
		1 & 2 & 3\\
		\hline
	\end{tabular}
\end{tabular}\\

\lstset{language=SQL,tabsize=4,captionpos=b,frame=single,
basicstyle=\footnotesize}
\begin{lstlisting}[caption=SQL Difference]
SELECT * FROM R EXCEPT SELECT * FROM S;
\end{lstlisting}

\subsection{Symmetric Difference}
The symmetric difference creates a set of tuple how are just in R or in S not in
both. This could be called Xor Difference.\\
Definition: $R \bigtriangleup S = (R \backslash S ) \cup (S \backslash R) = (R
\cup S) \backslash (S \pm R)$

\begin{tabular}{ c c c}
	R: & S: & R $\bigtriangleup$ S \\
	\begin{tabular}{|c|c|c|}
		\hline
		A & B & C\\
		\hline
		1 & 2 & 3\\
		\hline
		4 & 5 & 6\\
		\hline
	\end{tabular} &

	\begin{tabular}{|c|c|c|}
		\hline
		A & B & C\\
		\hline
		7 & 8 & 9\\
		\hline
		4 & 5 & 6\\
		\hline
	\end{tabular} &

	\begin{tabular}{|c|c|c|}
		\hline
		A & B & C\\
		\hline
		1 & 2 & 3\\
		\hline
		7 & 8 & 9 \\
		\hline
	\end{tabular}
\end{tabular}\\

\lstset{language=SQL,tabsize=4,captionpos=b,frame=single,
basicstyle=\footnotesize}
\begin{lstlisting}[caption=SQL Symetic Difference]
(SELECT * FROM R UNION SELECT * FROM S) 
EXCEPT
(SELECT * FROM R INTERSECT SELECT * FROM S);
\end{lstlisting}

\subsection{Mean}
The result is the set of tuple that is in R as well S.\\
Definition: $R \cap S = \{t | t \in R \land t \in S \}$

\begin{tabular}{ c c c}
	R: & S: & R $\cap$ S \\
	\begin{tabular}{|c|c|c|}
		\hline
		A & B & C\\
		\hline
		1 & 2 & 3\\
		\hline
		4 & 5 & 6\\
		\hline
	\end{tabular} &

	\begin{tabular}{|c|c|c|}
		\hline
		A & B & C\\
		\hline
		7 & 8 & 9\\
		\hline
		4 & 5 & 6\\
		\hline
	\end{tabular} &

	\begin{tabular}{|c|c|c|}
		\hline
		A & B & C\\
		\hline
		4 & 5 & 6 \\
		\hline
	\end{tabular}
\end{tabular}\\

\lstset{language=SQL,tabsize=4,captionpos=b,frame=single,
basicstyle=\footnotesize}
\begin{lstlisting}[caption=SQL Mean]
SELECT * FROM R INTERSECT SELECT * FROM S;
\end{lstlisting}

\subsection{Crossproduct}
The crossproduct creates a set of tuple that contains all combinations of all
tuple of relation R and S.\\
Definition: $R \times S = \{a_1, a_2, \dots, a_n, b_1, b_2, \dots, b_m) |\\
(a_1, a_2, \dots, a_m) \in \mathcal{R} \land (b_1, b_2, \dots, b_m) \in
\mathcal{S} \}$

\begin{tabular}{ c c}
	R: & S:\\
	\begin{tabular}{|c|c|c|c|}
		\hline
		A & B & C & D\\
		\hline
		1 & 2 & 3 & 4\\
		\hline
		4 & 5 & 6 & 7\\
		\hline
		7 & 8 & 9 & 0\\
		\hline
	\end{tabular} &

	\begin{tabular}{|c|c|c|}
		\hline
		A & B & C\\
		\hline
		1 & 2 & 3\\
		\hline
		7 & 8 & 9\\
		\hline
	\end{tabular}
\end{tabular}\\

\hspace{1cm}$R \times S$\\
\begin{tabular}{|c|c|c|c|c|c|c|}
	\hline
	A & B & C & D & E & F & G\\
	\hline
	1 & 2 & 3 & 4 & 1 & 2 & 3\\
	\hline
	4 & 5 & 6 & 7 & 1 & 2 & 3\\
	\hline
	7 & 8 & 9 & 0 & 1 & 2 & 3\\
	\hline
	1 & 2 & 3 & 4 & 7 & 8 & 3\\
	\hline
	4 & 5 & 6 & 7 & 7 & 8 & 3\\
	\hline
	7 & 8 & 9 & 0 & 7 & 8 & 3\\
	\hline
\end{tabular}\\

\lstset{language=SQL,tabsize=4,captionpos=b,frame=single,
basicstyle=\footnotesize}
\begin{lstlisting}[caption=SQL Mean]
SELECT * FROM R S;
\end{lstlisting}
\begin{lstlisting}[caption=SQL Mean]
SELECT * FROM R CROSS JOIN S;
\end{lstlisting}

\subsection{Projection}
A projection extract defined attributes from a given attributes set.\\
Definition: $\pi_\beta (R) = \{t_\beta | t \in R\}$

\begin{tabular}{ c c c}
	R: & R[A,B] & R[A]\\
	\begin{tabular}{|c|c|c|}
		\hline
		A & B & C \\
		\hline
		1 & 2 & 3\\
		\hline
		4 & 5 & 6\\
		\hline
	\end{tabular} &

	\begin{tabular}{|c|c|}
		\hline
		A & B \\
		\hline
		1 & 2 \\
		\hline
		7 & 8 \\
		\hline
	\end{tabular} &

	\begin{tabular}{|c|}
		\hline
		A\\
		\hline
		1\\
		\hline
		7\\
		\hline
	\end{tabular}
\end{tabular}\\

\lstset{language=SQL,tabsize=4,captionpos=b,frame=single,
basicstyle=\footnotesize}
\begin{lstlisting}[caption=SQL Select]
SELECT A, B FROM R;
\end{lstlisting}
\begin{lstlisting}[caption=SQL Select]
SELECT A FROM R;
\end{lstlisting}

\subsection{Selection}
Select tuple depending on a given condition.\\
Definition: $\sigma_{Expression}(R) = \{t | t \in R \land$satisfies Expression $\}$

\begin{tabular}{ c c c}
	R: & R[A=1] & R[C $>$ 6]\\
	\begin{tabular}{|c|c|c|}
		\hline
		A & B & C \\
		\hline
		1 & 2 & 3\\
		\hline
		4 & 5 & 8\\
		\hline
		1 & 6 & 7\\
		\hline
		8 & 6 & 1\\
		\hline
	\end{tabular} &

	\begin{tabular}{|c|c|c|}
		\hline
		A & B & C \\
		\hline
		1 & 2 & 3\\
		\hline
		1 & 6 & 7\\
		\hline
	\end{tabular} &

	\begin{tabular}{|c|c|c|}
		\hline
		A & B & C \\
		\hline
		4 & 5 & 8\\
		\hline
		1 & 6 & 7\\
		\hline
	\end{tabular} 
\end{tabular}\\

\lstset{language=SQL,tabsize=4,captionpos=b,frame=single,
basicstyle=\footnotesize}
\begin{lstlisting}[caption=SQL Select]
SELECT * FROM R WHERE A=1;
\end{lstlisting}
\begin{lstlisting}[caption=SQL Select]
SELECT * FROM R WHERE C>6;
\end{lstlisting}

\subsection{Join}
\begin{keywords}
Join, Equi-Join, NoEqui-Join, Semi-Join, Outer-Join
\end{keywords}
The join operation describes the combination of a crossproduct followed by a
selection.\\
Definition: $R \bowtie_{Expression} S = \{r \cup s | r \in R \land s \in S \land
Expression\}$\\
\subsubsection{NoEqui-Join}
NoEqui-Join means first crossproduct R and S than select every tuple where the
defined attributes are not equal.\\
$R \bowtie_{A != B} S = \sigma_{A != B} (R \times S) $

\begin{tabular}{ c c}
	R: & S \\
	\begin{tabular}{|c|c|c|c|}
		\hline
		A & B & C & D\\
		\hline
		1 & 2 & 3 & 4\\
		\hline
		4 & 5 & 6 & 7\\
		\hline
		7 & 8 & 9 & 0\\
		\hline
	\end{tabular} &

	\begin{tabular}{|c|c|c|}
		\hline
		E & F & G \\
		\hline
		1 & 2 & 3\\
		\hline
		7 & 8 & 9\\
		\hline
	\end{tabular}
\end{tabular}\\

R $\times$ \\
\begin{tabular}{|c|c|c|c|c|c|c|}
	\hline
	A & B & C & D & E & F & G\\
	\hline
	1 & 2 & 3 & 4 & 1 & 2 & 3\\
	\hline
	4 & 5 & 6 & 7 & 7 & 8 & 9\\
	\hline
	7 & 8 & 9 & 0 & 1 & 2 & 3\\
	\hline                     
	1 & 2 & 3 & 4 & 7 & 8 & 9\\
	\hline
	4 & 5 & 6 & 7 & 1 & 2 & 3\\
	\hline                     
	7 & 8 & 9 & 0 & 7 & 8 & 9\\
	\hline
\end{tabular}\\

JOIN(R, R.a $<>$ S.e, S)\\
\begin{tabular}{|c|c|c|c|c|c|c|}
	\hline
	A & B & C & D & E & F & G\\
	\hline
	4 & 5 & 6 & 7 & 7 & 8 & 9\\
	\hline
	7 & 8 & 9 & 0 & 1 & 2 & 3\\
	\hline                     
	1 & 2 & 3 & 4 & 7 & 8 & 9\\
	\hline
	4 & 5 & 6 & 7 & 1 & 2 & 3\\
	\hline
\end{tabular}\\

\lstset{language=SQL,tabsize=4,captionpos=b,frame=single,
basicstyle=\footnotesize}
\begin{lstlisting}[caption=SQL NoEqui-Join]
SELECT * FROM R, S WHERE R.A <> S.E;
\end{lstlisting}

\subsubsection{Equi-Join}
Equi-Join means first crossproduct R and S than select every tuple where the
defined attributes are equal.\\
$R \bowtie_{A = B} S = \sigma_{A = B} (R \times S) $

\begin{tabular}{ c c}
	R: & S: \\
	\begin{tabular}{|c|c|c|c|}
		\hline
		A & B & C & D\\
		\hline
		1 & 2 & 3 & 4\\
		\hline
		4 & 5 & 6 & 7\\
		\hline
		7 & 8 & 9 & 0\\
		\hline
	\end{tabular} &

	\begin{tabular}{|c|c|c|}
		\hline
		E & F & G \\
		\hline
		1 & 2 & 3\\
		\hline
		7 & 8 & 9\\
		\hline
	\end{tabular}
\end{tabular}\\

R $\times$ S \\
\begin{tabular}{|c|c|c|c|c|c|c|}
	\hline
	A & B & C & D & E & F & G\\
	\hline
	1 & 2 & 3 & 4 & 1 & 2 & 3\\
	\hline
	4 & 5 & 6 & 7 & 7 & 8 & 9\\
	\hline
	7 & 8 & 9 & 0 & 1 & 2 & 3\\
	\hline                     
	1 & 2 & 3 & 4 & 7 & 8 & 9\\
	\hline
	4 & 5 & 6 & 7 & 1 & 2 & 3\\
	\hline                     
	7 & 8 & 9 & 0 & 7 & 8 & 9\\
	\hline
\end{tabular}\\

JOIN(R, R.a = S.e, S)\\
\begin{tabular}{|c|c|c|c|c|c|c|}
	\hline
	A & B & C & D & E & F & G\\
	\hline
	1 & 2 & 3 & 4 & 1 & 2 & 3\\
	\hline
	7 & 8 & 9 & 0 & 7 & 8 & 9\\
	\hline
\end{tabular}\\

\lstset{language=SQL,tabsize=4,captionpos=b,frame=single,
basicstyle=\footnotesize}
\begin{lstlisting}[caption=SQL Equi-Join]
SELECT * FROM R, S WHERE R.A = S.E;
\end{lstlisting}
\begin{lstlisting}[caption=SQL Equi-Join]
SELECT * FROM R INNER JOIN S ON R.A = S.E;
\end{lstlisting}

\subsubsection{Natural Join}
The natural join equals the equal-join but extends it by displaying just one row
of any attribute, even if R and S define two.\\
Definition: $R \bowtie S = \{r \cup s_{[C_1, \dots, C_N]} | r \in R \land s \in S
\land r_{[B_1, \dots, B_N]} = s_{[B_1, \dots, B_N]}$

\begin{tabular}{ c c}
	R: & S \\
	\begin{tabular}{|c|c|c|c|}
		\hline
		A & B & C & D\\
		\hline
		1 & 2 & 3 & 4\\
		\hline
		4 & 5 & 6 & 7\\
		\hline
		7 & 8 & 9 & 0\\
		\hline
	\end{tabular} &

	\begin{tabular}{|c|c|c|}
		\hline
		A & F & G \\
		\hline
		1 & 2 & 3\\
		\hline
		7 & 8 & 9\\
		\hline
	\end{tabular}
\end{tabular}\\

$NATURAL\_JOIN(R,S)$ \\
\begin{tabular}{|c|c|c|c|c|c|}
	\hline
	A & B & C & D & F & G \\
	\hline
	1 & 2 & 3 & 4 & 2 & 3\\
	\hline
	7 & 8 & 9 & 0 & 8 & 9\\
	\hline
\end{tabular}\\

\lstset{language=SQL,tabsize=4,captionpos=b,frame=single,
basicstyle=\footnotesize}
\begin{lstlisting}[caption=SQL Natural-Join]
SELECT R.A, B, C, D, E, F, G FROM 
R INNER JOIN S ON R.A = S.A;
\end{lstlisting}

\subsubsection{Semi Join}
The semi-join is a natual-join where only the left part remains.\\
Definition: $R \propto S = \{r | r \in R \land s \in S
\land r_{[B_1, \dots, B_N]} = s_{[B_1, \dots, B_N]}$

\hspace{-0.5cm}
\begin{tabular}{ c c}
	R: & S \\
	\begin{tabular}{|c|c|c|c|}
		\hline
		A & B & C & D\\
		\hline
		1 & 2 & 3 & 4\\
		\hline
		4 & 5 & 6 & 7\\
		\hline
		7 & 8 & 9 & 0\\
		\hline
	\end{tabular} &

	\begin{tabular}{|c|c|c|}
		\hline
		A & F & G \\
		\hline
		1 & 2 & 3\\
		\hline
		7 & 8 & 9\\
		\hline
	\end{tabular}
\end{tabular}\\

$SEMI\_JOIN(R,S)$\\
\begin{tabular}{|c|c|c|c|}
	\hline
	A & B & C & D\\
	\hline
	1 & 2 & 3 & 4\\
	\hline
	7 & 8 & 9 & 0\\
	\hline
\end{tabular}\\

\lstset{language=SQL,tabsize=4,captionpos=b,frame=single,
basicstyle=\footnotesize}
\begin{lstlisting}[caption=SQL Semi-Join]
SELECT A, B, C, D, FROM NATURAL JOIN S
\end{lstlisting}
\begin{lstlisting}[caption=SQL Semi-Join]
SELECT R.A, R.B, R.C, R.D, FROM 
R INNER JOIN S ON R.A = S.A;
\end{lstlisting}

\subsubsection{Outer-Join}
In contrast to the Equi-Join even tuple how don't find partner in the join are
inserted in the result. These attributes of the result tuple are filled with
NULL values.

\paragraph{Natural-Left-Outer-Join} See example.\\
\begin{tabular}{ c c}
	R: & S\\
	\begin{tabular}{|c|c|c|c|}
		\hline
		A & B & C & D\\
		\hline
		1 & 2 & 3 & 4\\
		\hline
		4 & 5 & 6 & 7\\
		\hline
		7 & 8 & 9 & 0\\
		\hline
	\end{tabular} &

	\begin{tabular}{|c|c|c|}
		\hline
		A & F & G \\
		\hline
		1 & 2 & 3\\
		\hline
		7 & 8 & 9\\
		\hline
	\end{tabular}
\end{tabular}\\

NATURAL-LEFT-OUTER-JOIN\\
\begin{tabular}{|c|c|c|c|c|c|}
	\hline
	A & B & C & D & F & G\\
	\hline
	1 & 2 & 3 & 4 & 2 & 3\\
	\hline
	4 & 5 & 6 & 7 & NULL & NULL\\
	\hline
	7 & 8 & 9 & 0 & 8 & 9\\
	\hline
\end{tabular}\\

\lstset{language=SQL,tabsize=4,captionpos=b,frame=single,
basicstyle=\footnotesize}
\begin{lstlisting}[caption=Natural-Left-Outer-Join]
SELECT * FROM R NATURAL LEFT OUTER JOIN S;
\end{lstlisting}

\paragraph{Left-Outer-Join} See example.\\
\begin{tabular}{ c c}
	R: & S\\
	\begin{tabular}{|c|c|c|c|}
		\hline
		A & B & C & D\\
		\hline
		1 & 2 & 3 & 4\\
		\hline
		4 & 5 & 6 & 7\\
		\hline
		7 & 8 & 9 & 0\\
		\hline
	\end{tabular} &

	\begin{tabular}{|c|c|c|}
		\hline
		A & F & G \\
		\hline
		1 & 2 & 3\\
		\hline
		7 & 8 & 9\\
		\hline
	\end{tabular}
\end{tabular}\\

LEFT-OUTER-JOIN(R, R.A = S.A, S)\\
\begin{tabular}{|c|c|c|c|c|c|c|}
	\hline
	A & B & C & D & A & F & G\\
	\hline
	1 & 2 & 3 & 4 & 1 & 2 & 3\\
	\hline
	4 & 5 & 6 & 7 & NULL & NULL & NULL\\
	\hline
	7 & 8 & 9 & 0 & 7 & 8 & 9\\
	\hline
\end{tabular}\\

\lstset{language=SQL,tabsize=4,captionpos=b,frame=single,
basicstyle=\footnotesize}
\begin{lstlisting}[caption=Left-Outer-Join]
SELECT * FROM R LEFT OUTER JOIN S ON R.A = S.A;
\end{lstlisting}

\paragraph{Natural-RIGHT-Outer-Join} See example.\\
\begin{tabular}{ c c}
	R: & S\\
	\begin{tabular}{|c|c|c|c|}
		\hline
		A & B & C & D\\
		\hline
		1 & 2 & 3 & 4\\
		\hline
		4 & 5 & 6 & 7\\
		\hline
		7 & 8 & 9 & 0\\
		\hline
	\end{tabular} &

	\begin{tabular}{|c|c|c|}
		\hline
		A & F & G \\
		\hline
		1 & 2 & 3\\
		\hline
		7 & 8 & 9\\
		\hline
	\end{tabular}
\end{tabular}\\

NATURAL-RIGHT-OUTER-JOIN\\
\begin{tabular}{|c|c|c|c|c|c|}
	\hline
	A & B & C & D & F & G\\
	\hline
	1 & 2 & 3 & 4 & 2 & 3\\
	\hline
	7 & 8 & 9 & 0 & 8 & 9\\
	\hline
\end{tabular}\\

\lstset{language=SQL,tabsize=4,captionpos=b,frame=single,
basicstyle=\footnotesize}
\begin{lstlisting}[caption=Natural-Right-Outer-Join]
SELECT * FROM R NATURAL RIGHT OUTER JOIN S;
\end{lstlisting}

\paragraph{RIGHT-Outer-Join} See example.\\
\begin{tabular}{ c c}
	R: & S\\
	\begin{tabular}{|c|c|c|c|}
		\hline
		A & B & C & D\\
		\hline
		1 & 2 & 3 & 4\\
		\hline
		4 & 5 & 6 & 7\\
		\hline
		7 & 8 & 9 & 0\\
		\hline
	\end{tabular} &

	\begin{tabular}{|c|c|c|}
		\hline
		A & F & G \\
		\hline
		1 & 2 & 3\\
		\hline
		7 & 8 & 9\\
		\hline
	\end{tabular}
\end{tabular}\\

RIGHT-OUTER-JOIN\\
\begin{tabular}{|c|c|c|c|c|c|}
	\hline
	A & B & C & D & F & G\\
	\hline
	1 & 2 & 3 & 4 & 2 & 3\\
	\hline
	7 & 8 & 9 & 0 & 8 & 9\\
	\hline
\end{tabular}\\

\lstset{language=SQL,tabsize=4,captionpos=b,frame=single,
basicstyle=\footnotesize}
\begin{lstlisting}[caption=Right-Outer-Join]
SELECT * FROM R RIGHT OUTER JOIN S ON R.A=S.A;
\end{lstlisting}

\paragraph{Full-Outer-Join} See example.\\
\begin{tabular}{ c c}
	R: & S\\
	\begin{tabular}{|c|c|c|c|}
		\hline
		A & B & C & D\\
		\hline
		1 & 2 & 3 & 4\\
		\hline
		4 & 5 & 6 & 7\\
		\hline
		7 & 8 & 9 & 0\\
		\hline
	\end{tabular} &

	\begin{tabular}{|c|c|c|}
		\hline
		A & F & G \\
		\hline
		1 & 2 & 3\\
		\hline
		7 & 8 & 9\\
		\hline
		9 & 4 & 6\\
		\hline
	\end{tabular}
\end{tabular}\\

FULL-OUTER-JOIN\\
\begin{tabular}{|c|c|c|c|c|c|}
	\hline
	A & F & G & B & C & D\\
	\hline
	1 & 2 & 3 & 2 & 3 & 4\\
	\hline
	4 & NULL & NULL & 5 & 6 & 7\\
	\hline
	7 & 8 & 9 & 8 & 9 & 0\\
	\hline
	9 & 4 & 6 & NULL & NULL & NULL\\
	\hline
\end{tabular}\\

\lstset{language=SQL,tabsize=4,captionpos=b,frame=single,
basicstyle=\footnotesize}
\begin{lstlisting}[caption=Right-Outer-Join]
SELECT * FROM R RIGHT OUTER JOIN S ON R.A=S.A;
\end{lstlisting}

\subsection{Rename}
Through the rename operation you can rename attributes as well as relations.\\
Definition: $\beta_{[new \leftarrow old]}(R) = \{t' | t' (R - new) \land t'(new)
= t(old)\}$\\

\begin{tabular}{ c c}
	R: & R[B $\rightarrow$ X]:\\
	\begin{tabular}{|c|c|c|}
		\hline
		A & B & C\\
		\hline
		1 & 2 & 3\\
		\hline
		4 & 5 & 6\\
		\hline
	\end{tabular} &

	\begin{tabular}{|c|c|c|}
		\hline
		A & X & C\\
		\hline
		1 & 2 & 3\\
		\hline
		4 & 5 & 6\\
		\hline
	\end{tabular}
\end{tabular}\\

\lstset{language=SQL,tabsize=4,captionpos=b,frame=single,
basicstyle=\footnotesize}
\begin{lstlisting}[caption=Right-Outer-Join]
SELECT A, B AS X, C FROM R;
\end{lstlisting}

\subsection{Devision Operation}
The devision operation can be seen as the logical opposite to the
crossproduct.\\
Definition: $R \div S = \pi_{R'}(R) - \pi_{R'}((\pi_{R'}(R) \times S) - R)$

The examples uses the division operation to get the tuple attributes father and
mother who have the child's defined in S. It also removes the attributes from
the tuple who are defined in S.\\

R:\\
\begin{tabular}{|c|c|c|c|}
	\hline
	Father & Mother & Child & Age\\
	\hline
	Hans & Helga & Harald & 5\\
	\hline
	Hans & Helga & Maria & 2\\
	\hline
	Hans & Ursula & Sabina & 7\\
	\hline
	Martin & Melanie & Maria & 4\\
	\hline
	Martin & Melanie & Gertrud & 7\\
	\hline
	Martin & Melanie & Sabine & 2\\
	\hline
	Peter & Christina & Robert & 9\\
	\hline
\end{tabular}\\

\begin{tabular}{c c}
S: & R $\div$ S\\
	\begin{tabular}{|c|c|}
		\hline
		Child & Age\\
		\hline
		Maria & 4\\
		\hline
		Sabine & 2\\
		\hline
	\end{tabular} & 

	\begin{tabular}{|c|c|}
		\hline
		Father & Mother\\
		\hline
		Martin & Melanie\\
		\hline
	\end{tabular} 
\end{tabular}\\

\subsection{Minimality}
Minimality defines a set of operations that can be used to perform all possible
operations. The set goes as followed.
\begin{itemize}
	\setlength{\itemsep}{1pt}
	\item projection
	\item selection
	\item crossproduct
	\item join
	\item difference
	\item rename
\end{itemize}
