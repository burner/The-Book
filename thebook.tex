\documentclass[a4paper,9pt]{book}
\usepackage[utf8]{inputenc}
\usepackage{amsmath}
\usepackage{wrapfig}
\usepackage{subfloat}
\usepackage{subfig}
\usepackage{listings}
\usepackage{geometry}
\usepackage{fancyhdr}
\pagestyle{plain}
\geometry{a4paper, left=2.25cm, right=1.25cm, top=1.5cm, bottom=1.5cm}
\newenvironment{keywords}{
       \list{}{\advance\topsep by0.35cm\relax\small
       \leftmargin=1cm
       \labelwidth=0.35cm
       \listparindent=0.35cm
       \itemindent\listparindent
       \rightmargin\leftmargin}\item[\hskip\labelsep
                                     \bfseries Keywords:]}
     {\endlist}

\title{IS 1}
\author{Robert Schadek}
\date{\today}

\begin{document}
\maketitle
\setcounter{tocdepth}{5}
\setcounter{secnumdepth}{5}
\tableofcontents

\setlength{\columnsep}{20pt}
\twocolumn

\section{Relational Algebra}
\begin{keywords}
SQL commands, Relational databases, Relational Algebra, Databases
\end{keywords}
\subsection{Requirements}
To perform set operationen on two relations two requirements must be satisfied.
\begin{itemize}
	\item R and S must have the same degree (think polygon degree)
	\item The value-range of all attributes both both relations R and S must be
		identical
\end{itemize}

\subsection{Join}
The join R $\cup$ S combines all tuple of relation R with Relation S. To perform
this operation, both relation-schemes must be identical. That means the have the
same attributes and attribute-types. Duplications will be deleted.\\
Definition: $R \cup S = \{t | t \in R \lor t \in S\}$\\

\begin{tabular}{ c c c}
	R: & S: & R $\cup$ S \\
	\begin{tabular}{|c|c|c|}
		\hline
		A & B & C\\
		\hline
		1 & 2 & 3\\
		\hline
		4 & 5 & 6\\
		\hline
	\end{tabular} &

	\begin{tabular}{|c|c|c|}
		\hline
		A & B & C\\
		\hline
		7 & 8 & 9\\
		\hline
		4 & 5 & 6\\
		\hline
	\end{tabular} &

	\begin{tabular}{|c|c|c|}
		\hline
		A & B & C\\
		\hline
		7 & 8 & 9\\
		\hline
		4 & 5 & 6\\
		\hline
		1 & 2 & 3\\
		\hline
	\end{tabular}
\end{tabular}

\lstset{language=SQL,tabsize=4,captionpos=b,frame=single,
basicstyle=\footnotesize}
\begin{lstlisting}[caption=SQL Join]
SELECT * FROM R UNION SELECT * FROM S;
\end{lstlisting}

\subsection{Difference}
The difference-operation remove all tuple for relation S that are present in
R.\\
Definition: $R-S= R\backslash S = \{t | t \in R \land t \not\in S \}$\\

\begin{tabular}{ c c c}
	R: & S: & R $\backslash$ S \\
	\begin{tabular}{|c|c|c|}
		\hline
		A & B & C\\
		\hline
		1 & 2 & 3\\
		\hline
		4 & 5 & 6\\
		\hline
	\end{tabular} &

	\begin{tabular}{|c|c|c|}
		\hline
		A & B & C\\
		\hline
		7 & 8 & 9\\
		\hline
		4 & 5 & 6\\
		\hline
	\end{tabular} &

	\begin{tabular}{|c|c|c|}
		\hline
		A & B & C\\
		\hline
		1 & 2 & 3\\
		\hline
	\end{tabular}
\end{tabular}

\lstset{language=SQL,tabsize=4,captionpos=b,frame=single,
basicstyle=\footnotesize}
\begin{lstlisting}[caption=SQL Difference]
SELECT * FROM R EXCEPT SELECT * FROM S;
\end{lstlisting}

\subsection{Symmetric Difference}
The symmetric difference creates a set of tuple how are just in R or in S not in
both. This could be called Xor Difference.\\
Definition: $R \bigtriangleup S = (R \backslash S ) \cup (S \backslash R) = (R
\cup S) \backslash (S \pm R)$

\begin{tabular}{ c c c}
	R: & S: & R $\bigtriangleup$ S \\
	\begin{tabular}{|c|c|c|}
		\hline
		A & B & C\\
		\hline
		1 & 2 & 3\\
		\hline
		4 & 5 & 6\\
		\hline
	\end{tabular} &

	\begin{tabular}{|c|c|c|}
		\hline
		A & B & C\\
		\hline
		7 & 8 & 9\\
		\hline
		4 & 5 & 6\\
		\hline
	\end{tabular} &

	\begin{tabular}{|c|c|c|}
		\hline
		A & B & C\\
		\hline
		1 & 2 & 3\\
		\hline
		7 & 8 & 9 \\
		\hline
	\end{tabular}
\end{tabular}

\lstset{language=SQL,tabsize=4,captionpos=b,frame=single,
basicstyle=\footnotesize}
\begin{lstlisting}[caption=SQL Symetic Difference]
(SELECT * FROM R UNION SELECT * FROM S) 
EXCEPT
(SELECT * FROM R INTERSECT SELECT * FROM S);
\end{lstlisting}

\subsection{Mean}
The result is the set of tuple that is in R as well S.\\
Definition: $R \cap S = \{t | t \in R \land t \in S \}$

\begin{tabular}{ c c c}
	R: & S: & R $\cap$ S \\
	\begin{tabular}{|c|c|c|}
		\hline
		A & B & C\\
		\hline
		1 & 2 & 3\\
		\hline
		4 & 5 & 6\\
		\hline
	\end{tabular} &

	\begin{tabular}{|c|c|c|}
		\hline
		A & B & C\\
		\hline
		7 & 8 & 9\\
		\hline
		4 & 5 & 6\\
		\hline
	\end{tabular} &

	\begin{tabular}{|c|c|c|}
		\hline
		A & B & C\\
		\hline
		4 & 5 & 6 \\
		\hline
	\end{tabular}
\end{tabular}

\lstset{language=SQL,tabsize=4,captionpos=b,frame=single,
basicstyle=\footnotesize}
\begin{lstlisting}[caption=SQL Mean]
SELECT * FROM R INTERSECT SELECT * FROM S;
\end{lstlisting}

\subsection{Crossproduct}
The crossproduct creates a set of tuple that contains all combinations of all
tuple of relation R and S.\\
Definition: $R \times S = \{a_1, a_2, \dots, a_n, b_1, b_2, \dots, b_m) |\\
(a_1, a_2, \dots, a_m) \in \mathcal{R} \land (b_1, b_2, \dots, b_m) \in
\mathcal{S} \}$

\begin{tabular}{ c c}
	R: & S:\\
	\begin{tabular}{|c|c|c|c|}
		\hline
		A & B & C & D\\
		\hline
		1 & 2 & 3 & 4\\
		\hline
		4 & 5 & 6 & 7\\
		\hline
		7 & 8 & 9 & 0\\
		\hline
	\end{tabular} &

	\begin{tabular}{|c|c|c|}
		\hline
		A & B & C\\
		\hline
		1 & 2 & 3\\
		\hline
		7 & 8 & 9\\
		\hline
	\end{tabular}
\end{tabular}\\

\hspace{1cm}$R \times S$\\
\begin{tabular}{|c|c|c|c|c|c|c|}
	\hline
	A & B & C & D & E & F & G\\
	\hline
	1 & 2 & 3 & 4 & 1 & 2 & 3\\
	\hline
	4 & 5 & 6 & 7 & 1 & 2 & 3\\
	\hline
	7 & 8 & 9 & 0 & 1 & 2 & 3\\
	\hline
	1 & 2 & 3 & 4 & 7 & 8 & 3\\
	\hline
	4 & 5 & 6 & 7 & 7 & 8 & 3\\
	\hline
	7 & 8 & 9 & 0 & 7 & 8 & 3\\
	\hline
\end{tabular}

\lstset{language=SQL,tabsize=4,captionpos=b,frame=single,
basicstyle=\footnotesize}
\begin{lstlisting}[caption=SQL Mean]
SELECT * FROM R S;
\end{lstlisting}
\begin{lstlisting}[caption=SQL Mean]
SELECT * FROM R CROSS JOIN S;
\end{lstlisting}

\subsection{Projection}
A projection extract defined attributes from a given attributes set.\\
Definition: $\pi_\beta (R) = \{t_\beta | t \in R\}$

\begin{tabular}{ c c c}
	R: & R[A,B] & R[A]\\
	\begin{tabular}{|c|c|c|}
		\hline
		A & B & C \\
		\hline
		1 & 2 & 3\\
		\hline
		4 & 5 & 6\\
		\hline
	\end{tabular} &

	\begin{tabular}{|c|c|}
		\hline
		A & B \\
		\hline
		1 & 2 \\
		\hline
		7 & 8 \\
		\hline
	\end{tabular} &

	\begin{tabular}{|c|}
		\hline
		A\\
		\hline
		1\\
		\hline
		7\\
		\hline
	\end{tabular}
\end{tabular}\\

\lstset{language=SQL,tabsize=4,captionpos=b,frame=single,
basicstyle=\footnotesize}
\begin{lstlisting}[caption=SQL Select]
SELECT A, B FROM R;
\end{lstlisting}
\begin{lstlisting}[caption=SQL Select]
SELECT A FROM R;
\end{lstlisting}

\subsection{Selection}
Select tuple depending on a given condition.\\
Definition: $\sigma_{Expression}(R) = \{t | t \in R \land$satisfies Expression $\}$

\begin{tabular}{ c c c}
	R: & R[A=1] & R[C > 6]\\
	\begin{tabular}{|c|c|c|}
		\hline
		A & B & C \\
		\hline
		1 & 2 & 3\\
		\hline
		4 & 5 & 8\\
		\hline
		1 & 6 & 7\\
		\hline
		8 & 6 & 1\\
		\hline
	\end{tabular} &

	\begin{tabular}{|c|c|c|}
		\hline
		A & B & C \\
		\hline
		1 & 2 & 3\\
		\hline
		1 & 6 & 7\\
		\hline
	\end{tabular} &

	\begin{tabular}{|c|c|c|}
		\hline
		A & B & C \\
		\hline
		4 & 5 & 8\\
		\hline
		1 & 6 & 7\\
		\hline
	\end{tabular} 
\end{tabular}

\begin{lstlisting}[caption=SQL Select]
SELECT * FROM R WHERE A=1;
\end{lstlisting}
\begin{lstlisting}[caption=SQL Select]
SELECT * FROM R WHERE C>6;
\end{lstlisting}

\subsection{Join}
\begin{keywords}
Join, Equi-Join, NoEqui-Join, Semi-Join, Outer-Join
\end{keywords}
The join operation describes the combination of a crossproduct followed by a
selection.\\
Definition: $R \bowtie_{Expression} S = \{r \cup s | r \in R \land s \in S \land
Expression\}$\\
\subsubsection{NoEqui-Join}
$R \bowtie_{Expression} S = \sigma_{Expression} (R \times S) $

\subsubsection{Equi-Join}
\begin{tabular}{ c c c c}
	R: & S & R \times S & JOIN(R, R.a $<>$ S.e, S)\\
	\begin{tabular}{|c|c|c|c|}
		\hline
		A & B & C & D\\
		\hline
		1 & 2 & 3 & 4\\
		\hline
		4 & 5 & 6 & 7\\
		\hline
		7 & 8 & 9 & 0\\
		\hline
	\end{tabular} &

	\begin{tabular}{|c|c|c|}
		\hline
		A & B & C \\
		\hline
		1 & 2 & 3\\
		\hline
		1 & 6 & 7\\
		\hline
	\end{tabular} &

	\begin{tabular}{|c|c|c|}
		\hline
		A & B & C \\
		\hline
		4 & 5 & 8\\
		\hline
		1 & 6 & 7\\
		\hline
	\end{tabular} 
\end{tabular}

\end{document}
